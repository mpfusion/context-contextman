\startcomponent co-typography

\environment contextref-env
\product contextref

\startbuffer[kerning]

\startlinecorrection
\vbox
  {\forgetall
   \vfill
   \definefont[test][ComputerModern at 48pt]\test
   \hbox{box}
   \vskip-36pt
   \def\\#1%
     {\toprulefalse
      \bottomrulefalse
      \ruledhbox{\vrule width 0pt height 72pt#1}}
   \hbox{\\b\\o\\x}}
\stoplinecorrection

\stopbuffer

\startbuffer[styles]

\vskip1ex
\startcombination[3*3]
  {\rmtfc\setupinterlinespace   Serif\end}  {}
  {\sstfc\setupinterlinespace    Sans\end}   {}
  {\tttfc\setupinterlinespace    Mono\end}   {}
  {\rmtfc\setupinterlinespace Regular\end}{}
  {\sstfc\setupinterlinespace Support\end}{}
  {\tttfc\setupinterlinespace    Mono\end}   {}
  {\rmtfc\setupinterlinespace   Roman\end}  {}
  {\sstfc\setupinterlinespace    Sans\end}   {}
  {\tttfc\setupinterlinespace    Type\end}   {}
\stopcombination
\vskip1ex

\stopbuffer

\startbuffer[em-ex-1]

\def\jump{\vl\hskip1em\vl}%
\def\mmmm{\vl M\vl}%
\def\dash{\vl---\vl}%
\def\numb{\vl12\vl}%

\starttable[|c|c|c|c|c|c|]
\HL
\VL \type{\tf} \VL \type{\bf} \VL \type{\sl}  \VL
    \type{\tt} \VL \type{\ss} \VL \type{\tfx} \VL\SR
\HL
\VL \tf\numb   \VL \bf\numb   \VL \sl \numb   \VL
    \tt\numb   \VL \ss\numb   \VL \tfx\numb   \VL\FR
\VL \tf\mmmm   \VL \bf\mmmm   \VL \sl \mmmm   \VL
    \tt\mmmm   \VL \ss\mmmm   \VL \tfx\mmmm   \VL\MR
\VL \tf\jump   \VL \bf\jump   \VL \sl \jump   \VL
    \tt\jump   \VL \ss\jump   \VL \tfx\jump   \VL\MR
\VL \tf\dash   \VL \bf\dash   \VL \sl \dash   \VL
    \tt\dash   \VL \ss\dash   \VL \tfx\dash   \VL\LR
\HL
\stoptable

\stopbuffer

\startbuffer[em-ex-2]

\def\show
  {\hbox
    {\forgetall
     \offinterlineskip
     \vbox
       {\hsize1em\hl[1]\endgraf\vskip1ex\hl[1]}%
     \hskip.25em
     \vbox
       {\hsize.5em \vskip\linewidth x\vskip\linewidth}}}%

\starttable[|c|c|c|c|c|c|]
\HL
\VL \type{\tf} \VL \type{\bf} \VL \type{\sl}  \VL
    \type{\tt} \VL \type{\ss} \VL \type{\tfx} \VL\SR
\HL
\VL \tf\show   \VL \bf\show   \VL \sl\show    \VL
    \tt\show   \VL \ss\show   \VL \tfx\show   \VL\SR
\HL
\stoptable

\stopbuffer


\startbuffer[font-1]
\definefontsynonym [Sans]            [Helvetica]
\definefontsynonym [SansBold]        [Helvetica-Bold]
\definefontsynonym [SansItalic]      [Helvetica-Oblique]
\definefontsynonym [SansSlanted]     [Helvetica-Oblique]
\definefontsynonym [SansBoldItalic]  [Helvetica-BoldOblique]
\definefontsynonym [SansBoldSlanted] [Helvetica-BoldOblique]
\definefontsynonym [SansCaps]        [Helvetica]

\definebodyfont [14.4pt,12pt,11pt,10pt,9pt,8pt,7pt,6pt,5pt] [ss] [default]
\stopbuffer

\startbuffer[font-2]
\definefontsynonym [Helvetica-Bold] [hvb] [encoding=texnansi]
\stopbuffer

\startbuffer[font-3]
\definefontsynonym [Helvetica-Bold] [phvb] [encoding=ec]
\stopbuffer

\startbuffer[font-4]
\definefontsynonym [Regular] [Serif]
\definefontsynonym [Roman]   [Serif]
\stopbuffer

\startbuffer[font-5]
\definebodyfont [default] [rm]
  [ tf=Serif        sa 1,
   tfa=Serif        sa a,
      ...
    sl=SerifSlanted sa 1,
   sla=SerifSlanted sa a,
      ...]
\stopbuffer

\startbuffer[font-6]
\definebodyfont [12pt] [rm]
  [ tf=cmr12,
   tfa=cmr12 scaled \magstep1,
   tfb=cmr12 scaled \magstep2,
   tfc=cmr12 scaled \magstep3,
   tfd=cmr12 scaled \magstep4,
    bf=cmbx12,
    it=cmti12,
    sl=cmsl12,
    bi=cmbxti10 at 12pt,
    bs=cmbxsl10 at 12pt,
    sc=cmcsc10 at 12pt]
\stopbuffer

\startbuffer[font-7]
\definebodyfont [12pt,11pt,10pt,9pt,8pt] [ss]
  [tf=hv  sa 1.000,
   bf=hvb sa 1.000,
   it=hvo sa 1.000,
   sl=hvo sa 1.000,
  tfa=hv  sa 1.200,
  tfb=hv  sa 1.440,
  tfc=hv  sa 1.728,
  tfd=hv  sa 2.074,
   sc=hv  sa 1.000]
\stopbuffer

\startbuffer[font-8]
\definebodyfont [12pt,11pt,10pt,9pt,8pt] [ss]
  [tf=hv sa 1, tfa=hv sa a, tfb=hv sa b, tfc=hv sa c, tfd=hv sa d]
\stopbuffer

\startbuffer[enco-1]
\startmapping[texnansi]
  \definecasemap 228 228 196  \definecasemap 196 228 196
  \definecasemap 235 235 203  \definecasemap 203 235 203
  \definecasemap 239 239 207  \definecasemap 207 239 207
  \definecasemap 246 246 214  \definecasemap 214 246 214
  \definecasemap 252 252 220  \definecasemap 220 252 220
  \definecasemap 255 255 159  \definecasemap 159 255 159
\stopmapping
\stopbuffer

\startbuffer[enco-2]
\startencoding[texnansi]
  \defineaccent " a 228
  \defineaccent " e 235
  \defineaccent " i 239
  \defineaccent " o 246
  \defineaccent " u 252
  \defineaccent " y 255
\stopencoding
\stopbuffer

\startbuffer[enco-3]
\startencoding[texnansi]
  \definecharacter ae 230
  \definecharacter oe 156
  \definecharacter o  248
  \definecharacter AE 198
\stopencoding
\stopbuffer

\startbuffer[font-10]
\definefontsynonym [twelvepoint] [12pt]
\definefontsynonym [xii]         [12pt]
\stopbuffer

\startbuffer[font-11]
\definefontstyle [rm,roman,serif,regular]    [rm]
\definefontstyle [ss,sansserif,sans,support] [ss]
\definefontstyle [tt,teletype,type,mono]     [tt]
\definefontstyle [hw,handwritten]            [hw]
\definefontstyle [cg,calligraphic]           [cg]
\stopbuffer

\startbuffer[font-12]
\definestyle [normal]                  [\tf]  []
\definestyle [bold]                    [\bf]  []
\definestyle [type]                    [\tt]  []
\definestyle [italic]                  [\it]  []
\definestyle [slanted]                 [\sl]  []
\definestyle [bolditalic,italicbold]   [\bs]  []
\definestyle [boldslanted,slantedbold] [\bs]  []
\definestyle [small,smallnormal]       [\tfx] []
\stopbuffer

\startbuffer[font-13]
\definefontsynonym [OldStyle] [MathItalic]
\stopbuffer

\startbuffer[math-1]
$\tf x^2+\bf x^2+\sl x^2+\it x^2+\bs x^2+ \bi x^2 =\rm 6x^2$
$\tf x^2+\bf x^2+\sl x^2+\it x^2+\bs x^2+ \bi x^2 =\tf 6x^2$
$\tf x^2+\bf x^2+\sl x^2+\it x^2+\bs x^2+ \bi x^2 =\bf 6x^2$
$\tf x^2+\bf x^2+\sl x^2+\it x^2+\bs x^2+ \bi x^2 =\sl 6x^2$
\stopbuffer

\startbuffer[math-2]
$\tf\mf x^2 + x^2 + x^2 + x^2 + x^2 + x^2 = 6x^2$
$\bf\mf x^2 + x^2 + x^2 + x^2 + x^2 + x^2 = 6x^2$
$\sl\mf x^2 + x^2 + x^2 + x^2 + x^2 + x^2 = 6x^2$
$\bs\mf x^2 + x^2 + x^2 + x^2 + x^2 + x^2 = 6x^2$
$\it\mf x^2 + x^2 + x^2 + x^2 + x^2 + x^2 = 6x^2$
$\bi\mf x^2 + x^2 + x^2 + x^2 + x^2 + x^2 = 6x^2$
\stopbuffer

\startbuffer[math-3]
$\bf x^2 + x^2 + x^2 + x^2 + x^2 + x^2 = \mf 6x^2$
\stopbuffer

\startbuffer[math-4]
$\bf x^2 + \hbox{whatever} + \sin(2x)$
\stopbuffer

\startbuffer[math-5]
\definebodyfont [12pt] [mm]
  [ex=cmex10 at 12pt,
   mi=cmmi12,
   sy=cmsy10 at 12pt]
\stopbuffer

\startbuffer[math-6]
\definebodyfont [10pt,11pt,12pt] [mm]
  [tf=Sans          sa 1,
   bf=SansBold      sa 1,
   sl=SansItalic    sa 1,
   ex=MathExtension sa 1,
   mi=MathItalic    sa 1,
   sy=MathSymbol    sa 1]

\setupbodyfont
\stopbuffer

\startbuffer[fontfil]
\starttable[|Tl|l|]
\HL
\NC font-cmr \NC Computer Modern Roman        \NC\AR
\NC font-csr \NC Computer Slavik Roman (?)    \NC\AR
\NC font-con \NC Concrete Roman               \NC\AR
\NC font-eul \NC Euler                        \NC\AR
\NC font-ams \NC American Mathematics Society \NC\AR
\HL
\NC font-ant \NC Antykwa Torunska             \NC\AR
\HL
\NC font-lbr \NC Lucida Bright                \NC\AR
\HL
\NC font-pos \NC Base PostScript Fonts        \NC\AR
\NC font-ptm \NC Times Roman                  \NC\AR
\NC font-phv \NC Helvetica                    \NC\AR
\NC font-pcr \NC Courier                      \NC\AR
\HL
\NC font-fil \NC Standard Filenames           \NC\AR
\NC font-ber \NC Karl Berry FileNames         \NC\AR
\HL
\stoptable
\stopbuffer

\startbuffer[encofil]
\starttable[|Tl|l|]
\HL
\NC enco-ans \NC TeXnansi                     \NC\AR
\NC enco-ec  \NC European Computer            \NC\AR
\NC enco-il2 \NC ISO Latin 2                  \NC\AR
\NC enco-plr \NC Polish Roman                 \NC\AR
\HL
\NC enco-ibm \NC default IBM PC code page     \NC\AR
\NC enco-win \NC default MS Windows code page \NC\AR
\HL
\stoptable
\stopbuffer


\chapter[typography]{Typography}

\todo{The situation of Latin-Modern vs. Computer-Modern and TeX-Gyre vs. URW
needs explaining}

\section{Introduction}
\index{typography}

Through the millennia we have developed and adapted methods
for storing facts and thoughts on a variety of different
medium. A very efficient way of doing this is using
logograms, like Chinese have done for ages. Another method
is to represent each syllable in a word by a symbol, like
the Japanese do when writing telegrams. However, the most
familiar way of storing information is using a limited set
of pictures representing so called phonemes. Such a
collection is called an alphabet, and often the same glyph
is used for different sounds.

Although \TEX\ is primarily meant for typesetting languages
that use this third method, in principle the other two can
also be dealt with. In this manual we will focus on the
languages that use such alphabets.

The little pictures representing the characters that make up
an alphabet are more or less standardized, and thereby can
be recognized by readers, even if their details differ. Such
a collection of pictures, often called glyphs, make up a
font.

\usetypescriptfile[type-buy]
\usetypescriptfile[type-gyr]
\usetypescript[map,sans,mono,handwriting,calligraphy,serif][helvetica,schoolbook,times,antykwa-torunska]

\startlinecorrection
\ruledhbox to \hsize
  {\hss
   \definefont[test][ComputerModern          at 60pt]\test gap\setstrut\strut\hss
   \definefont[test][Helvetica               at 60pt]\test gap\hss
   \definefont[test][Times-Roman             at 60pt]\test gap\hss
   \definefont[test][AntykwaTorunska-Regular at 60pt]\test gap\hss}
\stoplinecorrection

From left to right we see the Computer Modern, a Helvetica, 
a Times Roman and an Antiqua Torunska font, all
scaled to 60pt. Fonts colections are designed in such a way
that the overall appearance of a page looks good and that
reading is as comfortable as possible.

\startlinecorrection
\ruledhbox to \hsize
  {\hss
   \definefont[test][Schoolbook-Roman      at 48pt]\test lap\setstrut\strut\hss
   \definefont[test][Schoolbook-Bold       at 48pt]\test lap\hss
   \definefont[test][Schoolbook-Italic     at 48pt]\test lap\hss
   \definefont[test][Schoolbook-BoldItalic at 48pt]\test lap\hss}
\stoplinecorrection

Within a font design there can be variations. In the example
above we see a light, a bold, a light italic, and a bold italic
alternative of the Schoolbook font.

The distance between the individual glyphs in a word depend
on the combinations of these glyphs. In the next sample, the
gap between the~b and the~o as well as the distance between
the~o and the~x is slightly altered. This is called kerning.

\getbuffer[kerning]

Here we showed a Computer Modern, the default \TEX\ font. This
font is designed by Donald Knuth and is a variation on a
Monotype \quote{Modern} font. The Computer Modern has many kerning
pairs, while the Palatino font used in this manual has only a few.

This kind of micro||typography is not to be altered by the
user. It is part of the font design. However the user can
alter fonts and interline spacing and some more aspects on
the level of macro||typography. The choice of font is the
main topic of this chapter.

There are different ways to classify fonts. There are
classification systems based on times of development, the
characteristics of the fonts or the font application, for
example in a newspaper or a book.

\startlinecorrection
\ruledhbox to \hsize
  {\hss
   \definefont[test][LMRoman-Regular           at 48pt]\test ok\setstrut\strut\hss
   \definefont[test][LMSans-Regular            at 48pt]\test ok\hss
   \definefont[test][LMTypewriter-Regular      at 48pt]\test ok\hss
   \definefont[test][LMRoman-CapsRegular       at 48pt]\test ok\hss
   \definefont[test][LMTypewriterVarWd-Regular at 48pt]\test ok\hss}
\stoplinecorrection

In this example we see five font styles of Computer Modern: the
Roman, Sans, Typewriter, Smallcaps and Variable Typewriter. It is
one of the few fonts that comes with dedicated design sizes.
The example below shows the differences of a 5, 7, 9, 12 and
17~point design scaled up to 48 points. Such nuances in font
size are seldom seen these days.

\startlinecorrection
\ruledhbox to \hsize
  {\hss
   \definefont[test][cmr5  at 48pt]\test ok\setstrut\strut\hss
   \definefont[test][cmr7  at 48pt]\test ok\hss
   \definefont[test][cmr9  at 48pt]\test ok\hss
   \definefont[test][cmr12 at 48pt]\test ok\hss
   \definefont[test][cmr17 at 48pt]\test ok\hss}
\stoplinecorrection

The general appearance of a style can be classified
according to many schemes. In \in {table} [tab:font
triplets] we see some examples of the naming of styles.

\placetable
  [here][tab:font triplets]
  {Some ways of classifying the styles in a font.}
  {\getbuffer[styles]}

The first two series are used by typographers, however in
\CONTEXT\ we rather use the last series because it is
traditionally used in plain \TEX. The command \type {\rm} is
used to switch to a roman|/|serif|/|regular style, and \type
{\tt} for switching to mono spaced or typewriter style.

In the next sections we will go into switching of font
styles and fonts in your documents. Note that the font
switching mechanism is rather complex. This is caused by the
different modes like math mode and text mode in \CONTEXT. If
you want to be able to understand the mechanism you will
have to acquaint yourself with the concept of the encoding
vector and obtain some knowledge on fonts and their
peculiarities.

\section[bodyfont]{The mechanism}
\index{bodyfont}
\index{font size}
\index{roman}
\index{sans serif}
\index{typewriter}
\macro{\tex{setupbodyfont}}
\macro{\tex{switchtobodyfont}}
\macro{\tex{x}}
\macro{\tex{xi}}
\macro{\tex{xii}}
\macro{\tex{ix}}
\macro{\tex{viii}}

Font switching is one of the eldest features of \CONTEXT\
because font switching is indispensable in a macropackage.
The last few years extensions to the font switching
mechanism were inevitable. We have chosen the following
starting points during the development of this mechanism:

\startitemize[packed]
\item To change a {\em style} must be easy, this
      means switching to: roman (serif, regular), sans serif
      (support), teletype (or monospaced) etc. (\type {\rm},
      \type {\ss}, \type {\tt} etc.)
\item More than one {\em variations} of character must be
      available like slanted and bold (\type {\sl} and \type
      {\bf}).
\item Different font {\em families} like Computer Modern
      Roman and Lucida Bright must be supported.
\item Changing the bodyfont must also be easy, and so
      font size between 8pt and 12pt must be available by
      default.
\item Within a font different sub|| and superscripts must be
      available. The script sizes can be used during
      switching of family, style and alternative.
\item Specific characteristics of a {\em body font} like
      font definition (encoding vector) must be taken into
      account.
\stopitemize

Text can be typeset in different font sizes. We often use
the unit \type {pt} to specify the size. The availability of
these font sizes are defined in definition files.
Traditionally font designers used to design a glyph
collection for each font size, but nowadays most fonts have
a design size of 10 points. An exception to this rule is the
Computer Modern Roman that comes with most \TEX\
distributions.

The most frequently used font sizes are predefined: 8, 9, 10,
11, 12 and 14.4 points. When you use another size |<|for
example for a titlepage|>| \CONTEXT\ will define this font
itself within the constraints of the used typeface.
\CONTEXT\ works with a precision of 1~digit which prevents
unnecessary loading of fontsizes with small size
differences. When a fontsize is not available \CONTEXT\
prefers to use a somewhat smaller font size. We consider
this to be more tolerable than a somewhat bigger font size.

The bodyfont (main font), font style and size is set up with:

\showsetup{setupbodyfont}

In a running text a temporary font switch is done with the
command:

\showsetup{switchtobodyfont}

This command doesn't change the bodyfont in headers and
footers. With \type {small} and \type {big} you switch to a
smaller or larger font.

In most cases, the command \type {\setupbodyfont} is only
used once: in the styledefinition. Fontswitching is done
with \type {\switchtobodyfont}. Don't mix these two up
because this may lead to some rather strange but legitimate
effects.

\TEX\ searches for font information in the file with the
extension \type {tfm}. Pre||loading is possible but
\CONTEXT\ will only load these files when necessarry. The
reason is that filenames can differ per distribution.

The font used in headers, footers and footnotes are adapted
automatically. This includes the interline space and
vertical whitespaces. Font switches with \type {\vi}, \type
{\vii}, \type {\viii}, \type {\ix}, \type {\x}, \type {\xi}
and \type {\xii} have only local effects.

The commands:

\startbuffer
{\xii  with these commands \par}
{\xi   for font switching  \par}
{\x    it is possible to   \par}
{\ix   produce an eyetest: \par}
{\viii a x c e u i w m q p \par}
\stopbuffer

\startexample
\typebuffer
\stopexample

When changing the size of the bodyfont, the interline space
is adapted automatically. This is shown on the left. On the
right we see what happens when the interline space is not
adapted.

\startlinecorrection
\startcombination
  {\vtop{\forgetall\hsize.4\textwidth                \getbuffer}} {}
  {\vtop{\forgetall\hsize.4\textwidth\everybodyfont{}\getbuffer}} {}
\stopcombination
\stoplinecorrection

\section[font switching]{Font switching}
\index{fonts}
\index{roman}
\index{slanted}
\index{boldface}
\index{italic}
\index{typewriter}
\index{sans serif}
\index{old style}
\index{medaeval numbers}

The mechanism to switch from one style to another is rather
complex and therefore hard to explain. To begin with, the
terminology is a bit fuzzy. We call a collection of font
shapes, like Lucida or Computer Modern Roman a family.
Within such a family, the members can be grouped according
to characteristics. Such a group is called a style. Examples
of styles within a family are: {\rmtf roman}, {\sstf sans
serif} and {\tttf teletype}. We already saw that there can
be alternative classifications, but they all refer to the
pressence of serifs and the glyphs having equal widths. In
some cases handwritten and|/|or calligraphic styles are also
available. Within a style there can be alternatives, like
{\bf boldface} and {\sl slanted}.

There are different ways to change into a new a style or
alternative. You can use \type {\ss} to switch to a sans
serif font style and \type {\bf} to get a bold alternative.
When a different style is chosen, the alternatives adapt
themselves to this style. Often we will typeset the document
in one family and style. This is called the bodyfont.

A consequent use of commands like \type {\bf} and \type
{\sl} in the text will automatically result in the desired
bold and slanted altermatives when you change the family or
style in the setup area of your input file. A somewhat
faster way of style switching is done by \type {\ssbf},
\type {\sssl}, etc. but this should be used with care, since
far less housekeeping takes place.

The alternatives within a style are given below. The
abbreviation \type {\sl} means {\sl slanted}, \type {\it}
means {\it italic} and \type {\bf} means {\bf boldface}.
Sometimes \type {\bs} and \type {\bi} are also available,
meaning {\bs bold slanted} and {\bi bold italic}. When an
alternative is not known, \CONTEXT\ will choose a suitable
replacement automatically.

With \type {\os} we tell \CONTEXT\ that we prefer mediaeval or
old||style numbers {\os 139} over {\rm 139}. The \type {\sc}
generates {\sc Small Caps}. With an \type {x} we switch to
smaller font size, with \type {a}, \type {b}, \type {c} and
\type {d} to a bigger one. The actual font style is stated
by \type {\tf} or typeface.

\startexample
\starttyping
\tfa \tfb \tfc \tfd
\tfx \bfx \slx \itx
\bf \sl \it \bs \bi \sc \os
\stoptyping
\stopexample

It depends on the completeness of the font definition files
whether alternatives like \type {\bfa}, \type {\bfb}, etc.
are available. Not all fonts have for instance italic and
slanted or both their bold alternatives. In such situations,
slanted and italic are threated as equivalents.

Switching to a smaller font is accomplished by \type {\tfx},
\type {\bfx}, \type {\slx}, etc., which adapt themselves to
the actual alternative. An even more general downscaling is
achieved by \type {\tx}, which adapts itself to the style
and alernative. This command is rather handy when one wants
to write macros that act like a chameleon. Going one more
step smaller, is possible too: \type {\txx}. Using \type
{\tx} when \type {\tx} is already given, is equivalent to
\type {\txx}.

Frequent font switching leads to longer processing times.
When no sub- or superscripts are used and you are very
certain what font you want to use, you can perform fast font
switches with: \type {\rmsl}, \type {\ssbf}, \type {\tttf},
etc.

Switching to another font style is done by:

\startexample
\starttyping
\rm \ss \tt \hw \cg
\stoptyping
\stopexample

When \type {\rm} is chosen \CONTEXT\ will interpret the
command \type {\tfd} as \type {\rmd}. All default font
setups use \type {tf}||setups and will adapt automatically.

The various commands will adapt themselves to the actual
setup of font and size. For example:

\startbuffer
{\rm test {\sl test} {\bf test} \tfc test {\tx test} {\bf test}}
{\ss test {\sl test \tx test} {\bf test \tx test}}
\stopbuffer

\startexample
\typebuffer
\stopexample

will result in:

\startreality
\startlines
\getbuffer
\stoplines
\stopreality

When a character is not available the most acceptable
alternative is chosen.

We will not go into the typographical sins of underlining.
These commands are discussed in \in {section} [underline]
(\over [underline]).

\section[character]{Characters}
\index{\type{character}}

A number of commands use the parameter \type {style} to set
up the font style and size. You can use commands like
\type {\sl} or \type {\rma} or keywords like:

\startexample
\starttyping
normal  bold  slanted  boldslanted  italic  bolditalic  type
small  smallbold  smallslanted  ...  smallitalic  ...  smalltype
capital
\stoptyping
\stopexample

The parameter mechanism is rather flexible so with the
parameter \type {style} you can type \type {bold} and \type
{\bf} or \type {bf}. Even the most low level kind of font
switching commands like \type {12ptrmbf} are permitted. This
is fast but requires some insight in macros behind this
mechanism.

\section[fonts]{Available alternatives}
\index{\type{cmr}}
\index{\type{eul}}
\index{\type{con}}
\index{\type{lbr}}

There are only a few font families that can handle math. There
is the Computer Modern Roman, the very beautiful Lucida
Bright that we prefer in electronic documents, and of course
one can use the \quote {prefered by publishers font} Times.
These fonts carry a complete set of characters and symbols
for mathematical typesetting. Among these, the Computer
Modern Roman distinguishes itself by its many design sizes,
which pays off when typesetting complicate math. On this
design there are a few variations called Euler and Concrete.
\footnote {See Concrete Mathematics by Knuth cs., an
outstanding book from the perspective of typography and
didactically.}

The Computer Modern Roman contains 70~charactertypes and
sizes. Because a number of charactersizes are not defined
the 11~point characters are defined as scaled 9~and
10~point characters under the option \type {cmr}. With
\type {eul} and \type {con} we obtain a Computer Modern.

\showsetup{showbodyfont}

With the command \type {\showbodyfont} an overview is
generated of the available characters. Below the 12pt||body
font Computer Modern Roman (\type {cmr}) is shown. The close
reader will note that not all alternatives are available by
default.

\showbodyfont[cmr,12pt]

We can see that the 12pt Lucida Bright (\type {lbr}) is
somewhat bigger than the 12pt Computer Modern Roman. An
\type {x}||character for example \type {\bfx} is 2pts smaller
than the actual typeface. The bigger characters are scaled
by \TEX's \type {\magstep}.

\showbodyfont[lbr,12pt]

A last remark. When you have chosen a larger charactersize,
for example \type {\tfb}, then \type {\tf} equals \type {\tfb},
\type {\bf} equals \type {\bfb}, etc. This method is
preferable over returning to the original character size.

\section[emphasize]{Emphasize}
\index{emphasize}
\index{slanted}
\index{italic}
\macro{\tex{em}}

Within most macropackages the command \type {\em} is
available. This command behaves like a chameleon which means
that it will adapt to the actual typeface. In \CONTEXT\
\type {\em} has the following characteristics:

\startitemize[packed]
\item a switch to {\sl slanted} or {\it
      italic} is possible
\item a switch within \type {\bf} results in {\bs bold
      slanted} or {\bi bold italic} (when available)
\item a so called {\em italic correction} is performed
      automatically (\type {\/})
\stopitemize

The bold italic or bold slanted characters are supported only
when \type {\bs} and \type {\bi} are available.

\startbuffer
The mnemonic {\em em} means {\em emphasis}.
{\em The mnemonic {\em em} means {\em emphasis}.}
{\bf The mnemonic {\em em} means {\em emphasis}.}
{\em \bf The mnemonic {\em em} {\em emphasis}.}
{\it The mnemonic em {\em means \bf emphasis}.}
{\sl The mnemonic em {\em means \bf emphasis}.}
\stopbuffer

\startexample
\typebuffer
\stopexample

This results in:

\startlines
\getbuffer
\stoplines

The advantage of the use of \type {\em} over \type {\it}
and|/|or \type {\sl} is that consistent typesetting is
enforced.

By default emphasis is set at {\em slanted}, but in this text
it is set at {\em italic}. The setting is made by:

\startexample
\starttyping
\setupbodyfontenvironment[default][em=italic]
\stoptyping
\stopexample

\section[capitals]{Capitals}
\index{capital characters}
\index{capitals}
\index{small capitals}
\index{small||caps}
\macro{\tex{setupcapitals}}
\macro{\tex{CAP}}
\macro{\tex{Cap}}
\macro{\tex{cap}}
\macro{\tex{nocap}}
\macro{\tex{Caps}}
\macro{\tex{Words}}
\macro{\tex{WORDS}}
\macro{\tex{Word}}
\macro{\tex{characters}}

Words and abbreviations can be typeset in capitals. Both
small and big characters are converted into capitals. When
\type {\cap} is used to typeset a capital the size is that
of an \type {\tx}. When we switch to slanted (\type {\sl}),
bold (\type {\bf}), etc. the capital letter will also
change. Since \type {\cap} has a specific meaning in math
mode, the format implementation is called \type {\cap}.
However in text mode one can use \type {\cap}.

\showsetup{kap}

\showsetup{Cap}

\showsetup{CAP}

\showsetup{Caps}

The first command converts all letters to a capital. We
advise you not to type capital letters in your source file
because real small caps distinguishes between small and big
letters.

\startbuffer
Capitals for \cap {UK} are \cap {OK} and capitals for \cap {USA} are
okay. But what about capitals in \cap {Y2K}.
\stopbuffer

\startexample
\typebuffer
\stopexample

this results in:

\startreality
\getbuffer
\stopreality

A \type {\cap} within a \type {\cap} will not lead to any
problems:

\startbuffer
\cap {People that have gathered their \cap {capital} at the cost of other
people are not seldom \nocap {decapitated} in revolutionary times.}
\stopbuffer

\startexample
\typebuffer
\stopexample

or:

\startreality
\getbuffer
\stopreality

In this example we see that \type {\cap} can be temporarily
revoked by \type {\nocap}.

\showsetup{nocap}

The command \type {\Cap} changes the first character of a
word into a capital and \type {\CAP} changes letters that
are preceded by \type {\\} into capital letters. With \type
{\Caps} you can change the first character of several words
into a capital letter.

\showsetup{setupcapitals}

With this command the capital mechanism can be set up. The
key \type {sc=yes} switches to real {\sc Small Caps}. With
\type {title} we determine whether capitals in titles are
changed.

Next to the former \type {\cap}||commands we have:

\showsetup{Word}

and

\showsetup{Words}

These commands switch the first characters of words into
capitals. All characters in a word are changed with:

\showsetup{WORD}

We end this section with real small capitals. When these are
available the real small caps \type {\sc} are preferred over
the pseudo||capital in abbreviations and logos.

\startbuffer
In a manual on \TeX\ and Con\TeX t  there is always the question whether to
type \cap{\TeX} and \cap{Con\TeX t} or {\sc \TeX} and {\sc Con\TeX t}. Both
are defined as a logo in the style definition so we type \type {\TEX} and
\type {\CONTEXT}, which come out as \TEX\ and \CONTEXT.
\stopbuffer

\startexample
\typebuffer
\stopexample

Results in:

\startreality
\getbuffer
\stopreality

{\sc It is always possible to typeset text in small
capitals. However, realize that lower case characters
discriminate more and make for an easier read.}

An important difference between \type {\cap} and \type {\sc}
is that the last command is used for a specific designed
font type. The command \type {\cap} on the other hand adapts
itself to the actual typeface: {\sl \cap {kap}}, {\bf \cap
{kap}}, {\bs \cap {kap}}, etc.

Some typesetting packages stretch words (inter character
spacing) to reach an acceptable alignment. In \CONTEXT\
this not supported. On purpose! Words in titles can be
stretched by:

\showsetup{stretched}

\startbuffer
\hbox to \hsize {\stretched{there\\is\\much\\stretch\\in ...}}
\hbox to 20em   {\stretched{... and\\here\\somewhat\\less}}
\stopbuffer

\startexample
\typebuffer
\stopexample

With \type {\\} we enforce a space (\type {{}} is also
allowed).

\startreality
\leavevmode\getbuffer
\stopreality

These typographically non permitted actions are only allowed
in heads. The macros that take care of stretching do this
by processing the text character by character.

\section[verbatim]{Verbatim text}
\index{verbatim text}
\index{typed text}
\index{typing}
\index{verbatim}
\macro{\tex{starttyping}}
\macro{\tex{setuptyping}}
\macro{\tex{setuptype}}
\macro{\tex{type}}
\macro{\tex{typ}}
\macro{\tex{tex}}
\macro{\tex{typefile}}

Text can be displayed in verbatim (typed) form. The text is
typed between the commands:

\showsetup{starttyping}

Like in:

\startbuffer
\starttyping
In this text there are enough examples of verbatim text. The command
definitions and examples are typeset with the mentioned commands. Like in
this example.
\stoptyping
\stopbuffer

\startreality
\typebuffer
\stopreality

For in||line typed text the command \type {\type} is
available.

\showsetup{type}

A complete file can be added to the text with the command:

\showsetup{typefile}

The style of typing is set with:

\showsetup{setuptyping}

This setup influences the display verbatim (\type
{\starttyping}) and the verbatim typesetting of files (\type
{\typefile}) and buffers (\type {\typebuffer}). The first
optional argument can be used to define a specific verbatim
environment.

\starttyping
\setuptyping[file][margin=default]
\stoptyping

When the key \type {space=on}, the spaces are shown:

\startreality
\setuptyping[space=on]
\starttyping
No alignment is to be preferred
over   aligning   by   means  of
spaces or the s t r e t c h i n g of words
\stoptyping
\stopreality

A very special case is:

\startbuffer
\definetyping
  [broadtyping]

\setuptyping
  [broadtyping]
  [oddmargin=-1.5cm,evenmargin=-.75cm]
\stopbuffer

\getbuffer

\typebuffer

This can be used in:

\startbuffer
\startbroadtyping
A verbatim line can be very long and when we don't want to hyphenate we
typeset it in the margin on the uneven pages.
\stopbroadtyping
\stopbuffer

\typebuffer

At a left hand side page the verbatim text is set in the
margin.

\getbuffer

An in||line verbatim is set up by:

\showsetup{setuptype}

When the parameter \type {option} is set at \type {slanted}
all text between \type{<}\type{<} and \type{>}\type{>} is
typeset in {\ttsl a slanted letter}. This feature can be
used with all parameters. In this way \type
+\type{aa<+\type+<bb>+\type+>cc}+ will result in: \type
{aa<<bb>>cc}.

% Het gekruk met de gesplitste << voorkomt ongewenste
% ligaturen in lucida fonts.

For reasons of readability you can also use other characters
than \type+{+ and \type+}+ as {\em outer} parenthesis. You
can choose your own non||active (a non||special) character,
for example: \type{\type+like this+} or \type{\type-like
that-}. Furthermore you can use the mentioned
\type{<}\type{<} and \type{>}\type{>}, as in
\type+\type<+\type+<like this>+\type+>+ or even
\type{\type<like that>}.

The parameter \type {option=commands} enables you to process
commands in a typed text. In this option \texescape\ is
replaced by \type {/}. This option is used for typesetting
manuals like this one. For example:

\startexample
\setuptyping[option=none]
\starttyping
\seethis <</rm : this command has no effect>>
 /vdots
\sihtees <</sl : neither has this one>>
\stoptyping
\stopexample

The double \type{<}\type{<} and \type{>}\type{>} overtake
the function of \type+{}+.

Within the type||commands we are using \type {\tttf}. When
we would have used \type {\tt}, the \type {\sl} would have
produced a slanted and \type {\bf} a bold typeletter. Now
this will not happen:

\startexample
\setuptyping[option=commands]
\starttyping
\seethis <</rm : this command has no effect>>
 /vdots
\sihtees <</sl : neither has this one>>
\stoptyping
\stopexample

One of the most interesting options of typesetting verbatim
is a program source code. We will limit the information on
this topic and refer readers to the documentation in the
files \type {verb-<<xxx>>.tex} and \type {cont-ver.tex}. In
that last file you can find the following lines:

\starttyping
\definetyping [MP]  [option=MP]
\definetyping [PL]  [option=PL]
\definetyping [JS]  [option=JS]
\definetyping [TEX] [option=TEX]
\stoptyping

Here we see that it is possible to define your own verbatim
environment. For that purpose we use the command:

\showsetup{definetyping}

The definitions above couple such an environment to an
option.

\startbuffer
\startMP
beginfig (12) ;
  MyScale = 1.23 ;
  draw unitsquare scaled MyScale shifted (10,20) ;
endfig ;
\stopMP
\stopbuffer

\typebuffer

In color (or reduced gray) this will come out as:

\startreality
\getbuffer
\stopreality

These environments take care of typesetting the text in such
a way that the typographics match the chosen language. It is
possible to write several filters. Languages like \METAPOST,
\METAFONT, \PERL, \JAVASCRIPT, \SQL, and off course \TEX\
are supported. By default color is used to display these
sources, where several palettes take care of the different
commands. That is why you see the parameter \type {palet} in
\type {\setuptyping}. One can use font changes or even own
commands instead, by assigning the appropriate values to the
\type {icommand} (for identifiers), \type {vcommand} (for
variables) and \type {ccommand} parameters (for the rest). By
default we have:

\starttyping
\setuptyping [icommand=\ttsl, vcommand=, ccommand=\tf]
\stoptyping

We have some alternatives for \type {\type}. When
typesetting text with this command the words are not
hyphenated. Hyphenation is performed however when one uses:

\showsetup{typ}

When you are thinking of producing a manual on \TEX\ you
have two commands that may serve you well:

\showsetup{tex}

\showsetup{arg}

The first command places a \tex{} in front of typed text and
the second command encloses the text with \arg{}.

\section{Math}
\index{math}
\macro{\tex{mf}}
\macro{\tex{enablembox}}
\macro{\tex{definebodyfont}}

Many \TEX\ users have chosen \TEX\ for its superb math type
setting. The math oriented character of \TEX\ has also
influenced the font mechanism. We will not go into any
details but the central key is the {\em family}. There is a
font family for \type {\bf}, \type {\it}, etc. Within a
family we distinguish three members: text, script and
scriptscript, or a normal, smaller and smallest font. The
normal font size is used for running text and the smaller
ones for sub and superscripts. The next example will show
what the members of a font family can do.

\typebuffer[math-1]

When this is typeset you see this:

\startlines
\getbuffer[math-1]
\stoplines

We can see that the characters adapt but that the
symbols are typeset in the same font. Technically this
means that the symbols are set in font family~0 (there are
16~families) and in this case that is default \type {\tf}.

It can also be done somewhat differently as we will see in
the next example. A new command is used: \type {\mf}, which
stands for {\em math font}. This command takes care of the
symbols in such a way that they are set in the actual
font.\footnote{We also see a strange visual effect. It seems
as if the lines are sloped.}

\startlines
\getbuffer[math-2]
\stoplines

You should take into account that \TEX\ typesets a formula
as a whole. In some cases this means that setups at the end
of the formula have effect at the beginning.

\typebuffer[math-2]

The exact location of \type {\mf} is not that important.
We also could have typed:

\typebuffer[math-3]

One other aspect of fonts in math mode is the way reserved
names like \type {\sin} and \type {\cos} are typeset.

\typebuffer[math-4]

Unlike plain \TEX, the $\bf\sin$ is also set bold.

\getbuffer[math-4]

In \CONTEXT\ the 12pt math (Computer Modern) fonts are
defined with:

\typebuffer[math-5]

It is possible to use \type {\tf}, \type {\bf}, etc. within
math mode.

\typebuffer[math-6]

The example we used before would become:

{\enablembox\getbuffer[math-6]\getbuffer[math-4]}

\section{Em and Ex}
\index[em]{\type{em}}
\index[ex]{\type{ex}}

In specifying dimensions we can distinguish physical units
like \type {pt} and \type {cm} and internal units like \type
{em} and \type {ex}. These last units are related to the
actual fontsize. When you use these internal units in
specifying for example horizontal and vertical spacing you
don't have to do any recalculating when fonts are switched
in the style definition.

Some insight in these units does not hurt. The width of an
\type {em} is not the with of an M, but that of an --- (an
em||dash). When this glyph is not available in the font
another value is used. \in {Table} [ems] shows some
examples. We see that the width of a digit is about \type
{.5em}. In Computer Modern Roman a digit is excactly half
an em wide.

\placetable
  [here][ems]
  {The width of an \type{em}.}
  {\getbuffer[em-ex-1]}

In most cases we use \type{em} for specifying width and and
\type {ex} for height. \in {Table} [exes] shows some
examples. We see that the height equals the height of a
lowercase~x.

\placetable
  [here][exes]
  {The height of an \type{ex}.}
  {\getbuffer[em-ex-2]}

\section[encoding]{Definitions}
\index{font+definition}
\macro{\tex{startencoding}}
\macro{\tex{startmapping}}
\macro{\tex{definecharacter}}
\macro{\tex{defineaccent}}
\macro{\tex{definecommand}}
\macro{\tex{definecasemap}}
\macro{\tex{definefontsynonym}}
\macro{\tex{definebodyfont}}
\macro{\tex{definebodyfontenvironment}}
\macro{\tex{definefont}}
\macro{\tex{definebodyfontenvironment}}
\macro{\tex{setupbodyfontenvironment}}
\macro{\tex{showbodyfontenvironment}}
\macro{\tex{definefontsynonym}}
\macro{\tex{definestyle}}

{\em This section is meant for curious users or those users
that want to do some experimenting on defining fonts. We
will not discuss precise definitions of accents and
encodings. For these issues we refer to the examples in the
source code and the files \type {font-<<xxx>>} and \type
{enco-<<xxx>>}}.

Earlier we have seen that within a font family there are
different font sizes. The relations between these sizes
are defined with:

\startexample
\setuptyping[option=commands]
\starttyping
\definebodyfontenvironment
  [12pt]
  [        text=12pt,    <</Roman Math dimensions: normal dimensions,>>
         script=9pt,     <</Roman super- and subscripts and>>
   scriptscript=7pt,     <</Roman supersuper- and subsubscripts.>>
              x=10pt,    <</Roman Pseudo caps and >>
             xx=8pt,     <</Roman nested pseudo caps.>>
            big=12pt,    <</Roman In case we switch to/tt big>>
          small=10pt]    <</Roman or/tt small.>>
\stoptyping
\stopexample

When we use a fontsize that is not predefined in this way
\CONTEXT\ applies the same proportions anyhow. You can alter
this definition by specifying the parameter \type {default}.
When you want to have a somewhat bigger fontsize you can
type:

\startexample
\starttyping
\definebodyfontenvironment [24pt]
\stoptyping
\stopexample

You can switch to a 12.4 environment, without any specific
actions. Within a group these fontdefinitions are temporary.
When you use the definitions several times in your document
you should type the definitions in the setup area of your
source file (style definition) since this can save much
runtime.

An overview of the different fontsizes within a family can be
summoned with:

\showsetup{showbodyfontenvironment}

For the \type {lbr} family of fonts this is:

\showbodyfontenvironment[lbr]

For all regular fontsizes environments are predefined
that fulfill their purpose adequately. However when you want
to do some extra defining yourself there is:

\showsetup{setupbodyfontenvironment}

The real definitions, i.e. the coupling of commands to the
font files, can be done in different ways. The most
transparant is the font file \type {font-phv}.

\typebuffer[font-1]

With \type {\definefontsynonym} we couple a logical name,
like \typ {SansBold} to a font name, like \typ
{Helvetica-Bold}. The real coupling is done somewhere else,
by default in the file \type {font-fil}. There you will see:

\typebuffer[font-2]

This is the only location where a system dependent setup is
made. When we work under the naming regime of Karl Berry,
the next setup would be more obvious (see \type
{font-ber}):

\typebuffer[font-3]

Coupling fonts in this way has no real limits. It is
interesting to look in \type {font-unk} where different
styles are coupled in such a way that they be used
interchangeably.

\typebuffer[font-4]

We see that the basic specification is \type {Serif}. The
default serif fonts are defined with:

\typebuffer[font-5]

We saw that \type {\tf} is the default font. Here \type
{\tf} is defined as \typ{Serif sa 1} which means that it is
a serif font, scaled to a normal font size. This \type
{Serif} is projected elsewhere on for example \typ
{LucidaBright} which in turn is projected on the filename
\type {lbr}.

The kind of all||in||one definitions as shown previously for
Helvetica use the \type {default} settings and enable easy
font definitions. This is okay for fonts that come in one
design size.

We, like other \TEX\ users, started with the use of Computer
Modern Roman fonts. Since these fonts have specific design
sizes \CONTEXT\ supports accurate definitions. See the file
\type {font-cmr}:

\typebuffer[font-6]

We use here the available \TEX||specifications \type
{scaled} and \type {at}, but \CONTEXT\ also supports a
combination of both: \type {sa} (scaled at). For example if
we do not want to use the default Helvetica definition we
define:

\typebuffer[font-7]

The scaling is done in relation to the bodyfont size. In
analogy with \TEX's \type {\magstep} we can use \type
{\magfactor}: instead of \type {sa 1.440} we specify \type
{sa \magfactor2}. Because typing all these numbers is rather
tiresome so we replace \type {1.200} by \type {a}, etc. The
relations between \type {a} and \type {1.200} can be set up
in the bodyfont environment.

\typebuffer[font-8]

Since font files are used in all interfaces we use English
commands. The definitions take place in files with the name
\type {font-<<xxx>>.tex}, see for example the file \type
{font-cmr.tex}.

\showsetup{definebodyfont}

The setups \type {ex}, \type {mi}, \type {sy}, \type {ms},
\type {mb} and \type {mc} relate to the math charactersets.
The first three we can also find in plain \TEX, the last
three are necessary in other font families. The symbols and
characters in \AMSTEX\ can also be used in \CONTEXT: \type
{\definebodyfont [ams]}. These can be found in \type {ma}
and \type {mb}.

The \type {a}||\type {d} are not mandatory. As an example we
will define a bigger fontsize of \type {\tf}:

\startbuffer
\definebodyfont [10pt,11pt,12pt] [rm] [tfe=Regular at 48pt]
\tfe Big Words.
\stopbuffer

\startexample
\typebuffer
\stopexample

This becomes:

\startreality
\startlinecorrection
\getbuffer
\stoplinecorrection
\stopreality

This definition brings us to other definitions. It is
possible to define a bodyfont in a several ways. We can use
classifications like \type {Regular}, or abstract names like
\type {TimesRoman}, or filenames, like \type {tir}, or even
fancy names like \type {HeadLetter}.

\starttyping
\definebodyfont[HeadLetter][Regular sa 1.2]
\stoptyping

After these definitions we can use \type {\HeadLetter} to
switch fonts. It may be necessary to adapt the interline
spacing with \type {\setupinterlinespace} like this:

\starttyping
\HeadLetter \setupinterlinespace text \par
\stoptyping

For advanced \TEX\ users there is the dimension||register
\type {\bodyfontsize}. This variable can be used to set
fontwidths. The number (rounded) points is available in
\type {\bodyfontpoints}.

Until now we assumed that an~\type {a} will become an~a
during type setting. However, this is not always the case.
Take for example~\"a or~\ae. This character is not available
in every font and certainly not in the Computer Modern
Typefaces. Often a combination of characters \type {\"a} or a
command \type {\ae} will be used to produce such a character.
In some situation \TEX\ will combine characters
automatically, like in \type {fl} that is combined to fl and
not \hbox{f}\hbox{l}. Another problem occurs in converting
small print to capital print and vice versa.

Below you see an example of the \type {texnansi} mapping:

\typebuffer[enco-1]

This means so much as: in case of a capital the character
with code 228 becomes character 228 and in case of small
print the character becomes character 196.

These definitions can be found in \type {enco-ans}. In this
file we can also see:

\typebuffer[enco-2]

and

\typebuffer[enco-3]

As a result of the way accents are placed over characters we
have to approach accented characters different from normal
characters. There are two methods: \TEX\ does the accenting
itself {\em or} prebuild accentd glyphs are used. The
definitions above take care of both methods. Other
definitions are sometimes needed. In the documentation of
the file \type {enco-ini} more information on this can be
found.

We once again return to font definitions. Fast fontswitching
is done with commands like \type {\xii} or \type
{\twelvepoint}, which is comparable to the way it is done in
plain \TEX. These commands are defined with:

\typebuffer[font-10]

The keys in \type {\setupbodyfont} are defined in terms of:

\typebuffer[font-11]

In many command setups we encounter the parameter \type
{style}. In those situations we can specify a key. These
keys are defined with \type {\definestyle}. The third
argument is only of importance in chapter and section
titles, where, apart from \type {\cap}, we want to obey the
font used there.

\typebuffer[font-12]

In \in{section}[emphasize] we have already explained how
{\em emphasizing} is defined. With oldstyle digits this is
somewhat different. We cannot on the forehand in what font
these can be found. By default we have the setup:

\typebuffer[font-13]

As we see they are obtained from the same font as the math
italic characters.

In addition to these commands there are others, for example
macros for manipulating accents. These commands are
discussed in the file \type {font-ini}. More information can
also be found in the file \type {core-fnt} and specific
gimmicks in the file \type {supp-fun}. So enjoy yourself.

\section{Page texts}
\index{headers}
\index{footers}
\index{menus}

Page texts are texts that are placed in the headers,
footers, margins and edges of the so called pagebody. This
sentence is for instance typeset in the bodyfont in the
running text. The fonts of the page texts are set up by
means of different commands. The values of the parameters
may be something like \type {style=bold} but
\type {style=\ss\bf} is also allowed. Setups like
\type {style=\ssbf} are less obvious because commands like
\type {\cap} will not behave the way you expect.

Switching to a new font style (\type {\ss}) will cost some
time. Usually this is no problem but in interactive documents
where we may use interactive menus with dozens of
items and related font switches the effect can be
considerable. In that case a more efficient font switching
is:

\startexample
\starttyping
\setuplayout[style=\ss]
\stoptyping
\stopexample

Border texts are setup by its command and the related key.
For example footers may be set up with the key
\type {letter}:

\startexample
\starttyping
\setupfooter[style=bold]
\stoptyping
\stopexample

\section{Files}
\index{font files}

A number of font definition files that are standard in most
distributions are mentioned in \in{table}[tab:fontfiles].
These fonts can be recalled by their last three letters.

\placetable
  [here][tab:fontfiles]
  {Some standard font definition files
   ($\type{pos} = \type{ptm} + \type{phv} + \type{pcr}$).}
  {\getbuffer[fontfil]}

The most commonly used encoding vectors, like \type {ans},
\type {ec} and \type {il2}, are preloaded. Extra encoding
files are loaded by \type {\useencoding}, but this is seldom
needed. The last two files mentioned in \in {table}
[tab:encofiles] relate to the support of the non||standard
keyboard styles. These should be loaded explicitly.

\placetable
  [here][tab:encofiles]
  {Some standard encoding definition files.}
  {\getbuffer[encofil]}

\section{Figures}
\index{figures+fonts}

When you use figures in your document they may contain text.
Most of time the \TEX||fonts are not available. When you use
a serif in your document you can best use a Helvetica in the
figures. In \in{figure}[fig:helvetica] we use a Helvetica,
while we use Knuth's Sans Serif in the caption.

\placefigure
  [here]
  [fig:helvetica]
  {\ss The use of fonts in pictures.}
  {\externalfigure[bor0129]}

\stopcomponent
