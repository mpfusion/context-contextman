% interface=en 

\startcomponent co-en-01

\environment contextref-env
\product contextref

\chapter{Preface}

This manual is about \CONTEXT, a system for typesetting 
documents. Central element in this name is the word \TEX\
because the typographical programming language \TEX\ is 
the base for \CONTEXT. 

People who are used to \TEX\ will probably identify this
manual as a \TEX\ document. They recognise the use of
\texescape. One may also notice that the way pararaphs are
broken into lines is often better than in the avarage
typesetting system. 

In this manual we will not discuss \TEX\ in depth because 
highly recommended books on \TEX\ already exist. We would 
like to mention:

\startitemize[n]

\item[texbook] the unsurpassed {\em The \TeX Book} by Donald
E.~Knuth, the source of all knowledge and \TEX nical
inspiration, 

\item[bytopic] the convenient {\em \TeX\ by Topic} by Victor
Eijkhout, the reference manual for \TEX\ programmers, and 

\item[beginners] the recommended {\em The Beginners Book
of \TeX} by Silvio Levy and Raymond Seroul, the book that
turns every beginner into an expert 

\stopitemize

For newcomers we advise (\in[beginners]), for the curious
(\in[texbook]), and for the impatient (\in[bytopic]). \CONTEXT\
users will not necessarly need this literature, unless one
wants to program in \TEX, uses special characters, or has to
typeset math. Again, we would advise (\in[beginners]). 

You may ask yourself if \TEX\ is not just one of the many
typesetting systems to produce documents. That is not so.
While many systems in eighties and nineties pretended to
deliver perfect typographical output, \TEX\ still does a 
pretty good job compared to others. 

\TEX\ is not easy to work with, but when one gets accustomed
to it, we hope you will appreciate its features, 

\blank[big]

Hans Hagen, 1996||2002

\stopcomponent
