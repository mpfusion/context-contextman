% interface=en

\startcomponent co-layout

\environment contextref-env
\product contextref

\startbuffer[baseline]

\def\next#1 #2 #3 #4 #5\\% plus minus gaat hier niet (catcodes)
  {\advance\dimen0 by #1
   \advance\dimen0 by #3}

\dimen0=4\linewidth
\expandafter\next\the\bigskipamount\\
\expandafter\next\the\medskipamount\\
\expandafter\next\the\smallskipamount\\

\startlinecorrection
\midaligned{\startcombination[3]
  {\vbox to \dimen0
     {\hsize=.3\hsize
      \hrule height \linewidth width \hsize \vskip\bigskipamount
      \hrule height \linewidth width \hsize \vskip\medskipamount
      \hrule height \linewidth width \hsize \vskip\smallskipamount
      \hrule height \linewidth width \hsize}}%
  {\translate[nl=maximale rek,en=maximum stretch]}
  {\vbox to \dimen0
     {\hsize=.3\hsize
      \hrule height \linewidth width \hsize \vskip1\bigskipamount
      \hrule height \linewidth width \hsize \vskip1\medskipamount
      \hrule height \linewidth width \hsize \vskip1\smallskipamount
      \hrule height \linewidth width \hsize}}%
  {\translate[nl=geen rek,en=no stretch]}
  {\vbox to \dimen0
     {\hsize=.3\hsize
      \hrule height \linewidth width \hsize \vskip\bigskipamount
      \hrule height \linewidth width \hsize \vskip\medskipamount
      \hrule height \linewidth width \hsize \vskip\smallskipamount
      \hrule height \linewidth width \hsize
      \vskip 3\bigskipamount}}
  {\translate[nl=minimale rek,en=minimal stretch]}
\stopcombination}
\stoplinecorrection

\stopbuffer

\definetabulate[sometable][|c|c|c|c|c|c|c|c|c|]

\startbuffer[boxes-1]
\startsometable
\noalign
  {\vskip\lineheight
   \color[gray]{\hrule height \ht\strutbox depth \dp\strutbox}
   \vskip-2\lineheight
   \nobreak}%
\NC \showstruts\hbox{\hsize1cm \strut a\par a\strut} \NC
    \showstruts\vbox{\hsize1cm \strut a\par a\strut} \NC
    \showstruts\vtop{\hsize1cm \strut a\par a\strut} \NC
    \showstruts\lbox{\hsize1cm \strut a\par a\strut} \NC
    \showstruts\cbox{\hsize1cm \strut a\par a\strut} \NC
    \showstruts\rbox{\hsize1cm \strut a\par a\strut} \NC
    \showstruts\sbox{\hsize1cm \strut a\par a\strut} \NC
    \showstruts\tbox{\hsize1cm \strut a\par a\strut} \NC
    \showstruts\bbox{\hsize1cm \strut a\par a\strut} \NC \NR
\noalign{\nobreak}
\NC \type {\hbox} \NC
    \type {\vbox} \NC
    \type {\vtop} \NC
    \type {\lbox} \NC
    \type {\cbox} \NC
    \type {\rbox} \NC
    \type {\sbox} \NC
    \type {\tbox} \NC
    \type {\bbox} \NC \NR
\stopsometable
\stopbuffer

\startbuffer[boxes-2]

\definetabulate[sometable][|c|c|c|c|c|]

\unprotect

\startsometable
\noalign
  {\vskip\lineheight
   \color[gray]{\hrule height \ht\strutbox depth \dp\strutbox}
   \vskip-2\lineheight
   \nobreak}
\NC \hbox{\externalfigure[cow][\c!height=1cm,\c!frame=\v!on]} \NC
    \vbox{\externalfigure[cow][\c!height=1cm,\c!frame=\v!on]} \NC
    \sbox{\externalfigure[cow][\c!height=1cm,\c!frame=\v!on]} \NC
    \tbox{\externalfigure[cow][\c!height=1cm,\c!frame=\v!on]} \NC
    \bbox{\externalfigure[cow][\c!height=1cm,\c!frame=\v!on]} \NC \NR
\noalign{\nobreak}
\NC \type {\hbox} \NC
    \type {\vbox} \NC
    \type {\sbox} \NC
    \type {\tbox} \NC
    \type {\bbox} \NC \NR
\stopsometable

\protect

\stopbuffer

\unprotect

\startbuffer[boxes-3]
\setbox0=\vbox{The Final Cut\par --- \em Pink Floyd}
\makecutbox0 \box0
\stopbuffer

\startbuffer[boxes-4]
\startbuffer[water]
Drink geen water \crlf direct uit de kraan! \blank

\start
  \tfx \setupinterlinespace Het drinkwater is tijdelijk niet betrouwbaar.
  Kook het water voor consumptie ten minste 2~minuten. Zodra het water
  weer betrouwbaar is, krijgt u bericht. \par
\stop

\blank[2*big]

\language[en] Do not drink water \crlf directly from the tap! \blank

\start
  \tfx \setupinterlinespace The water is temporarily unfit for drinking.
  Boil the water during at least 2~minutes before consumption. As soon
  as the water is reliable again, you will be notified. \par
\stop
\stopbuffer
\stopbuffer


\startbuffer[sdsided]

\unprotect

\switchtobodyfont[\v!small]

\def\somepage#1#2#3%
  {\bgroup
   \doif{#2#3}{}{\setupframed[\c!background=\v!screen]}%
   \framed[\c!height=6em]%
     {\vbox to 5em
        {\hbox to #1{#2\hss#3}%
         \vss
         \hbox to #1{#2\hss#3}}}%
   \egroup}%

\protect

\stopbuffer

\startbuffer[sdsided]

\unprotect

\switchtobodyfont[\v!small]

\def\somepage#1#2#3%
  {\bgroup
   \doif{#2#3}{}{\setupframed[\c!background=\v!screen]}%
   \framed[\c!height=6em]%
     {\vbox to 5em
        {\hbox to #1{#2\hss#3}%
         \vss
         \hbox to #1{#2\hss#3}}}%
   \egroup}%

\startcombination[3*1]
  {\hbox
     {\somepage{1em}{}{}%
      \somepage{4em}{l}{r}%
      \hskip.5em%
      \somepage{1em}{}{}%
      \somepage{4em}{l}{r}}}
  {\tttf\translate[en=singlesided,nl=enkelzijdig]}
  {\hbox
     {\somepage{1em}{}{}%
      \somepage{4em}{l}{r}%
      \hskip.5em%
      \somepage{1em}{}{}%
      \somepage{4em}{r}{l}}}
  {\tttf\translate[en={single...,double...},nl={enkel...,dubbel...}]}
  {\hbox
     {\somepage{4em}{l}{r}%
      \somepage{1em}{}{}%
      \hskip.5em%
      \somepage{1em}{}{}%
      \somepage{4em}{r}{l}}}
  {\tttf\translate[en=doublesided,nl=doublesided]}
\stopcombination

\protect

\stopbuffer

\chapter[layout]{Layout}

\todo{Split this chapter, it is much too large even in it's current
incomplete state}

\section{Introduction}

The look of a publication is determined by the page design,
the chosen fonts and other aspects like vertical spacing.
In this chapter we will explore the latter. Sometimes we
will go into detail but a novice user can skip such parts.
In normal applications, the default setups are most
adequate, because they will adapt to the different
situations. For the impatient reader we will just mention a
few setups. Spacing between paragraphs is defined by:

\starttyping
\setupwhitespace[big]
\stoptyping

In your source file you can best use an empty line between
paragraphs. This increases readability and it makes the
typing of \type {\par} at the end of each paragraph
obsolete. Indentation at every new paragraph is obtained
by:

\starttyping
\setupindenting[medium]
\stoptyping

A doublesided publication is generated when you type:

\starttyping
\setuppagenumbering[alternative=doublesided]
\stoptyping

As you might expect this might generate page numbering on the
right and left hand side of a paper and the margins will be
mirrored automatically.

As we have said before only the curious have to read on.

\section[paragraphs]{Paragraphs}
\index{layout}
\index{paragraphs}
\macro{\tex{paragraph}}
\macro{\tex{par}}

The most important unit in \TEX\ is paragraph. A new
paragraph is forced by:

\startitemize[n,packed]
\item an empty line
\item the \TEX||command\type {\par} or \type {\endgraf}
\item the \CONTEXT||command \type {\paragraph}
\stopitemize

The first alternative is the most obvious. You will obtain a
readable input file (\ASCII\ file) and errors are minimized.
The second alternative is chosen when it is mandatory to the
used command. For example in definitions (see \in
[definitions]).


\section[indentation]{Indentation}
\index{indentation}
\index{smaller layout}
\index{paragraphs+indentation}
\macro{\tex{setupindenting}}
\macro{\tex{indenting}}
\macro{\tex{noindenting}}
\macro{\tex{startnarrower}}
\macro{\tex{setupnarrower}}

When a text has little whitespacing, for example in a novel,
it is a custom to indent each new paragraph. Indentation is
setup with:

\showsetup{setupindenting}

By default there is \quote {no} indentation. When
indentation is turned on, when possible the commands will
determine whether indentation is necessary. For example, it
doesn't look good to indent after a vertical whitespace. In
a number of cases it is even undesirable to indent. Think
for example of headers and itemizations.

This manual is typeset without indentation. The great
quantity of short sentences and examples would result in a
very messy page layout.

When indentation is used, we may have to tell \TEX\ in some
cases {\em not} to indent. This is done by:

\showsetup{noindenting}

We can set up indenting by:

\showsetup{indenting}

The meaning of the setups is described in \in {table}
[tab:indenting]. Next to the commands described above we
could use the \TEX||commands \type {\indent} and
\type {\noindent}.

\startbuffer
\starttable[|l|l|]
\HL
\VL \bf setup      \VL \bf result                      \VL\SR
\HL
\VL \type {no} /
    \type {not}    \VL don't indent the next paragraph \VL\FR
\VL \type {yes} /
    \type {always} \VL turn on indentation             \VL\MR
\VL \type {never}  \VL turn off indentation            \VL\MR
\VL \type {first}  \VL indent first paragraphs too     \VL\MR
\VL \type {next}   \VL don't indent first paragraphs   \VL\LR
\HL
\stoptable
\stopbuffer

\placetable
  [here][tab:indenting]
  {The way of indenting.}
  {\getbuffer}

The settings \type {first} and \type {next} determine if
paragraphs following whitespace should be indented or not. It
is a sort of custom not to indent these.

A text may be typeset smaller than the default textwidth. In
that case the complete text will be indented on both sides.

\showsetup{startnarrower}

For example:

\startbuffer
\startnarrower[3*left,2*right]
The relatively small revolution in in Russia in 1917 had big consequences for
this country as well as the rest of the world. It is interesting to see that
some 80~years later a just as small revolution was needed to undo the 1917
one. In both cases, the main reason for the revolutions was to prevent
democracy from arising.
\stopnarrower
\stopbuffer

\startexample
\typebuffer
\stopexample

Will become:

\startreality
\getbuffer
\stopreality

Next to using \type {left}, \type {right} and \type {middle}
also combinations and manifolds are possible. Indentation in
the example above could have obtained by typing \type
{2*middle,left}. So, \type {middle} is equivalent to \type
{left,right}.

The value of indentation is set up by:

\showsetup{setupnarrower}

\section[whitespacing]{Vertical spacing (whitespacing)}
\index{vertical spacing}
\index{paragraphs+vertical spacing}
\index{spacing}
\macro{\tex{blank}}
\macro{\tex{setupblank}}
\macro{\tex{godown}}
\macro{\tex{startpacked}}
\macro{\tex{setupwhitespace}}
\macro{\tex{nowhitespace}}
\macro{\tex{whitespace}}
\macro{\tex{startlinecorrection}}
\macro{\tex{correctwhitespace}}

Vertical spacing between paragraphs is set up by:

\showsetup{setupwhitespace}

Instead of a random value it is better to use one of the
pre defined dimension. Default there is no vertical spacing.
Without any set up values the vertical spacing is related to
the actual fontsize.

Vertical spacing can be forced by either:

\showsetup{whitespace}

\showsetup{nowhitespace}

These commands have only effect when vertical spacing is set
up. In fact these commands will not be necessary for
\CONTEXT\ takes care of most situations.

\TEX\ handles vertical spacing around lines quite different
from that around text. In case these problematic situations
occur one can use the following commands. Spacing around
figures and tables is dealt with by \CONTEXT, so only use
these commands when the typeset text looks really bad.

\showsetup{startlinecorrection}

For example:

\startexample
\starttyping
\startlinecorrection
\framed{To boxit or not, that's a delicate question!}
\stoplinecorrection
\stoptyping
\stopexample

One can add vertical spacing with the \TEX\ command \type
{\vskip}, but please don't. We advise you to use:

\showsetup{blank}

We can use a value of one of the keywords \type {small},
\type {medium} or \type {big}. A big jump is twice a medium
jump which is four times a small jump. A value however can
be left out (\type {\blank}) when the default vertical
space is desired. It is advisable to set up the vertical
spacing only once in the setup area of your document. Local
alterations throughout your document will result in a
badly||spaced document.

Normally there is some stretch in the vertical spacing. This
enables \TEX\ to fill out a page optimally. In the next
example we see what happens when we add stretch to
whitespace. Each sample shows from top to bottom three \type
{\blank}'s of \type {big}, \type {medium} and \type {small}.
The left and right sample show the range of the stretch. The
rightmost sample shows that adding stretch can result in
shrink.

\getbuffer[baseline]

The last vertical space can be undone by typing \type
{\blank[back]} and the next blank can be blocked by \type
{disable}. With \type {reset} a \type {disable} is ignored.

The command \type {\blank} is one of the more advanced
commands. The next call is allowed:

\startexample
\starttyping
\blank[2*big,medium,disable]
\stoptyping
\stopexample

Since \type {medium} is half the amount of \type {big}, this
results in adding a vertical spaces of 2.5 times \type
{big}. The previous vertical space will be undone
automatically and the \type {disable} suppressed the next
\type {\blank}.

A lasting vertical space can be sustained by \type {force}.
For example, if you want some extra spacing at the top of a
page you will have to type \type {force}.

The default vertical spaces are set up with:

\showsetup{setupblank}

An example of such a definition is:

\startexample
\starttyping
\setupblank[big]
\stoptyping
\stopexample

The vertical spaces will be automatically adapted to the
fontsize and they are flexible. Changing the default set up
locally is therefore not advisable. Without an argument
\type {\setupblank} adapts to the actual fontsize!

The keywords \type {fixed} and \type {flexible} are used to
end or reinstate this adaptive characteristic. In columns it
is recommended to use the setup \type {[fixed,line]} or the
opposite setup \type {[flexible,standard]}.

This text is typeset a bodyfont of \the \bodyfontpoints pt
and is downscaled by a few percent. The setup that is used
in this document is shown in \in {table} [tab:skips]. We see
some stretch in the vertical spacing. The stretching enables
\TEX\ to fill out a page satisfactorily. Default the maximal
vertical space is 75\% of the line space and the stretch
maximal of 25\%.

\edef\globalbaselineskip{\the\baselineskip}

\placetable
  [here]
  [tab:skips]
  {The whitespace values to a \the\bodyfontpoints pt bodyfont.}
\starttable[|l|l|]
\HL
\VL \bf setup       \VL \bf value            \VL\SR
\HL
\VL \type {small}   \VL \the\smallskipamount \VL\FR
\VL \type {medium}  \VL \the\medskipamount   \VL\MR
\VL \type {big}     \VL \the\bigskipamount   \VL\MR
\VL \type {line}    \VL \globalbaselineskip  \VL\LR
\HL
\stoptable

In \in {paragraph} [paragraph] it was said that the vertical
spacing can be set up with the command \type
{\setupwhitespace}. Default there is no whitespace between
paragraphs. The setup of vertical spacing and line spacing
are related to each other.

Instead of direct setup you can use an indirect way. This
has the advantage that you can change the layout more easily.
In that case we use:

\showsetup{defineblank}

If we type for example:

\starttyping
\defineblank[aroundverbatim][medium]
\stoptyping

than \type {aroundverbatim} is equal to \type {medium}, which
can be used, for example around verbatim, as in:

\starttyping
\setuptyping
  [before={\blank[aroundverbatim]},
    after={\blank[aroundverbatim]}]
\stoptyping

If we want some more whitespacing we only have to change the
definition of \type {aroundverbatim}:

\starttyping
\defineblank[aroundverbatim][big]
\stoptyping

The vertical spacing between two lines can be suppressed with
the command:

\showsetup{packed}

Vertical spacing between more than one line is suppressed by:

\showsetup{startpacked}

The spacing around \quote {packed} text is automatically
corrected. Opposed to this command is:

\showsetup{startunpacked}

Skipping more than one vertical space is done with:

\showsetup{godown}

One of the most important lessons to be learned is to avoid
using \type {\vskip} in running text. This can interfere
with some hidden mechanisms of \CONTEXT.

Sometimes \TEX\ is not able to sort out spacing on its own.
In such situations one can insert the next command at the
troublesome location.

\showsetup{correctwhitespace}

Normally one will not need this command, although sometimes
when writing macros, it can be added to make sure that the
spacing is okay. Use this kind of tweaking with care!

\section[word spacing]{Word spacing}
\index{word spacing}
\index{french spacing}
\index{spacing after colon}
\macro{\tex{setupspacing}}
\macro{\tex{space}}
\macro{\tex{fixedspaces}}
\macro{\tex{nospace}}

Default a space is placed after a period that ends a
sentence. In some countries it is custom to stretch the
space after a period. Especially documents typeset in small
columns will look better that way. Because this is a language
specific feature. the default depends on the language. One
can however (temporarily) change this spacing.

\showsetup{setupspacing}

In many cases we combine words and numbers that should not
be separated at linebreaking, for example number~12. These
combinations can be connected by a tight space: \type
{number~12}. Word and number will never be separated at
linebreaking on that spot. A space can be made visible by:

\showsetup{space}

Undesired spaces can be suppressed by:

\showsetup{nospace}

When you want to align a row of numbers you can use
tight spaces with the width of a number. Tight spaces are
activated by:

\showsetup{fixedspaces}

After this command the \type {~} (tilde) generates a tight
space with the width of a number.

\section{Struts}
\index{struts}
\index{alignment}
\macro{\tex{strut}}
\macro{\tex{setstrut}}
\macro{\tex{setnostrut}}
\macro{\tex{showstruts}}

A strut is a little invisible block without width but with
the maximal height and depth of a character or line.
If you want to force these maximal dimensions, for example
when you are using boxes in your own commands, than you can
use the command \type {\strut}:

\startexample
\starttyping
\hbox{\strut test}
\stoptyping
\stopexample

If we leave out the strut in this example the box has no
depth. The characters in the word test don't reach under the
baseline. Compare for example {\showstruts \ruledhbox
{\strut test}} (with strut) with \ruledhbox {test}.

Many commands use struts automatically. If for some reason
you don't want struts you can try to suppress them by \type
{\setnostrut}. However take care that this command works
only locally. A strut can be set by \type {\setstrut}.

The struts that are used by \CONTEXT\ can be made visible
with the command:

\showsetup{showstruts}

\section[text in margin]{Text in the margin}
\index{margin+text}
\index{titles+margins}
\macro{\tex{setupinmargin}}
\macro{\tex{inmarge}}
\macro{\tex{inothermargin}}
\macro{\tex{inleft}}
\macro{\tex{inright}}
\macro{\tex{margintext}}

Texts can be place in the margins with:

\showsetup{inmargin}

A new line in a margin text is forced with \type {\\}. An
example of a margin text is:

\startbuffer
\inmargin{the marginal\\influence of\\advertisement}It would be great
if the recent reduction in washing powder needed to get your wash
perfectly clean had resulted in an equal reduction of time needed to
advertise this kind of products.
\stopbuffer

\startexample
\typebuffer
\stopexample

or:

\startreality
\getbuffer
\stopreality


When this command is used in the middle of a paragraph the
margin text will appear on the same line in the\inmargin
{over here} margin. The command \type {\inmargin} puts the
text in the left or right margin. The location where the
text will show up depends on the character of the document:
single||sided or double||sided. You can also force the text
into a specific margin, using:

\showsetup{inleft}

\showsetup{inright}

There is also:

\showsetup{inothermargin}

Some examples of the use of margin text appear below:

\startbuffer
\startlines
\inleft{to be}\quotation{To be or not to be} to me
\inright{or not}is rather famous english
\inmargin{to be}And just as it is meant to be
that quote will never perish
\stoplines
\stopbuffer

\startexample
\typebuffer
\stopexample

This will become:

\startreality
\getbuffer
\stopreality

\inmargin [low] {\ssd 123}The mechanism of margin texts is
rather complex. If you think of multiline margin texts and
the alignment of these lines with the lines in the textbody
you can imagine a few typographic problems. The number {\ss
123} next to this paragraph is not aligned but is typeset
somewhat lower. This is done by adding the keyword \type
{low}:

\startexample
\starttyping
\inmargin[low]{\ssd 123}The mechanism of margin texts ...
\stoptyping
\stopexample

It is possible to set up the way margin texts are typeset by
means of the command:

\showsetup{setupinmargin}

\start
\setupinmargin[align=middle]

\inleft{a rather\\marginal\\effect\\}With \type {align} we
define the left or right alignment of the margin text.
Default margin texts are right aligned. In this example
alignment is \type {middle}.

\stop

We can also align on the left or right side automatically.
In a double sided document design optimisation of the margin
text may ask for more than one processing step. In the
example below you see some of the possible setups.

\startlines
\setupinmargin [align=left]
\inmargin{left}   This is \type {left} aligned
\setupinmargin [align=middle]
\inmargin{middle} but this goes in the \type {middle}. Don't forget that
\setupinmargin [align=right]
\inmargin{right} \type {right} in this sense, align means a ragged right margin.
\setupinmargin [align=yes]
\inmargin{yes}   Just to be complete, there is \type {yes}
\setupinmargin [align=no]
\inmargin{no}    and \type {no}.
\setupinmargin [align=inner]
\inmargin{inner} The outsiders \type {inner} and
\setupinmargin [align=outer]
\inmargin{outer} \type {outer} adapt themselvs to a doublesided design.
\stoplines

The left and right margin can be set up separately by adding
\type {[left]} or \type {[right]} as the first argument.

\start

\startbuffer
\setupinmargin
  [style=\bfx\setupinterlinespace]
\stopbuffer

\getbuffer

\inleft {that way we can\\move quite some\\text into the
margin}With \type {before} and \type {after} we can
influence margin texts. Bij default the same line spacing is
used as in the textbody. But when a narrower fontsize is
used we can also adapt the interline spacing. For example:

\stop

\startexample
\typebuffer
\stopexample

Page breaking and margin text are in conflict with each
other. The reason is that \TEX\ first typesets a complete
page in order to be able to determine the right spot for
page breaking. However the margin text is already typeset at
that moment. In a next processing stage the margin texts are
typeset correctly. If you want to force margin texts in a
margin you can type \type {\inmargin[+]}.

The next command can be compared with the command like \type
{\section}. Before the command is placed in the margin \TEX\
looks if it can be placed on the actual page. If not, it is
moved to the following page.

\showsetup{margintext}

The layout of your \ASCII||file will not interfere with the
function of this command. This may seem obvious, but \TEX\
programmers know that it is not the case. For example even
commands that take care of index entries can be typed close
to the margin texts.

The layout of your \ASCII||file will not interfere with the
function of this command. You might not expect it to, but
\TEX\ programmers know that with \TEX, the layout of the
source usually interferes with for instance margin texts
and index entries. In \CONTEXT\ commands that take care of
margin texts take care of this situation, so that index
entries can be typed close to the margin texts and margin
texts can be separated from the next paragraph by an empty
line. The same cannot be said for other \TEX\ macropackages.

\startexample
\starttyping
\margintext{text in themargin}
\index{margintexts}

After experimenting a long time I have succeeded to filter
empty lines and commands that stand between body texts and
margin texts. It is amazing but the index entry really works.
\stoptyping
\stopexample

Because of the close relation with the page design the margin
width is set up by means of: \type {\setuplayout} (see \in
{section} [margins]).

\startbuffer[voorbeeld]
\margintext{Isn't}
\margintext{this}
\margintext{cute?}
\stopbuffer

\startbuffer[settings]
\setupinmargin[1][align=right, line=1,style=slanted]
\setupinmargin[2][align=middle,line=2,style=boldslanted]
\setupinmargin[3][align=left,  line=3,style=bold]
\stopbuffer

\start

\getbuffer[settings,voorbeeld]

The command \type {\margintext} enables you to put texts in
the margin that show completely different characteristics
than that of the text body. You can typeset different
margin texts with different characteristics like bodyfont,
line spacing and offset.

\stop

\typebuffer[voorbeeld]

In the setup we see an optional argument. The number is
determined by the order of definition.

\typebuffer[settings]

This means that the second margintext in a row will start on
line~2, and be typeset in a bold slanted font. One can
explicitly force a margintext to go some place, by saying for
instance:

\starttyping
\margintext[2]{this is the second one}
\stoptyping

\section[highlow]{Subscript and superscript}
\index{high text}
\index{low text}
\index{superscript}
\index{subscript}
\macro{\tex{low}}
\macro{\tex{high}}
\macro{\tex{lohi}}

There are three commands to create superscript and subscript
outside the math mode:

\showsetup{high}

\showsetup{low}

\showsetup{lohi}

The next example illustrates the use of these commands:

\startbuffer
You can walk on \high {high} heels or \low {low} heels but your height
is still the same.
\stopbuffer

\startexample
\typebuffer
\stopexample

This results in:

\startreality
\getbuffer
\stopreality

These commands relate to the \type {^} and \type {_} in math
mode. In case of larger fontsizes like \type {\tfc}, the
\type {^} and \type {_} will not create the desired output.
Compare the examples below:

\startbuffer
test\high{test} test test$^{\rm test}$ test
{\bf test\high{test} test test$^{\bf test}$ test}
{\tfb test\high{test} test test$^{\tfb test}$ test}
\stopbuffer

\startexample
\typebuffer
\stopexample

This becomes:

\startreality
\startlines
\getbuffer
\stoplines
\stopreality

\section[columns]{Columns}
\index{columns}
\index{tabulation}
\index{alignment+columns}
\macro{\tex{setupcolumns}}
\macro{\tex{startcolumns}}
\macro{\tex{column}}

The \TEX\ programmer knows that it is not easy to put
text in columns. Gratefully a \CONTEXT\ user is not bothered
with the implementation of extensive macros.

\start
\switchtobodyfont[small]
\startcolumns[n=3,tolerance=tolerant,align=no]
You can typeset text in columns. Most commands can be used
in a normal way without any problems. The floating object
like tables or figures are somewhat limited. This is caused
by the fact that \TEX\ has limited capabilities for
typesetting columns. For insiders: columns are produced with
the primitives: \type {\output} and \type {\vsplit}.

The number of columns is unlimited, however \TEX s memory
can only handle upto about twenty to thirty or fourty
columns.

The number of columns and the type setting of a vertical line
as a column separator is set up by:
\stopcolumns
\stop

\showsetup{setupcolumns}

The \type {n} indicates the number of columns. The column
text is enclosed by:

\showsetup{startcolumns}

The local setup of columns can be added directly after this
command. A new column is forced by:

\showsetup{column}

The text below is typeset in two columns with a \type
{verytolerant} alignment.

\startexample
\starttyping
\startcolumns[rule=on,n=2,tolerance=verytolerant]
Thus, I came to the conclusion that the designer of a new
system must not only be the implementer and first
.
.
\bf D.E. Knuth
\stopcolumns
\stoptyping
\stopexample

\startbuffer
\startcolumns[rule=on,n=2,tolerance=tolerant]
Thus, I came to the conclusion that the designer of a new
system must not only be the implementer and first
large||scale user; the designer should also write the first
user manual.

The separation of any of these four components would have
hurt \TeX\ significantly. If I had not participated fully in
all these activities, literally hundreds of improvements
would never have been made, because I would never have
thought of them or perceived why they were important.

But a system cannot be successful if it is too strongly
influenced by a single person. Once the initial design is
complete and fairly robust, the real test begins as people
with many different viewpoints undertake their own
experiments.

\bf D.E. Knuth
\stopcolumns
\stopbuffer

\start
\setupcolumns[blank={flexible,big}]\getbuffer
\stop

This example makes it painfully obvious that spacing
between lines is not on forehand equal. By default the line
spacing in this document is \type {big}, which equals
$.75\times$\type {\lineheight}. Furthermore, the allowable
stretch in line spacing makes vertical alignment
practically impossible.

For this reason the default line spacing is equal to the
lineskip and stretching is not allowed. When a switch in
fontsize is desirable you should do so before starting the
column mechanism. Font switches within columns will have a
poor result. The next example shows a line spacing equal to
the lineskip.

\start
\setupcolumns[blank={fixed,line}]\getbuffer
\stop

This effect is reached by the (default) setup:

\startexample
\starttyping
\setupcolumns[blank={fixed,line}]
\stoptyping
\stopexample

In \in {section} [grids] typesetting on a grid is explained.
This mechanism works quite well within columns.

\startbuffer[column-1]
\def\FramedColumn#1{\ruledhbox{\box#1}}

\setupcolumns[command=\FramedColumn]
\stopbuffer

\startbuffer[column-2]
\startcolumns
\TEX\ is not an easy to learn typesetting system or program.
The problem is that \quotation {knowing everything is possible}
leads to \quotation {wanting everything that is possible}.
However using \CONTEXT\ or \TEX\ takes considerable learning
time. And it is not feasible to explain every single detail
in this manual. Therefore \quotation {doing} is the answer.
\stopcolumns
\stopbuffer

\start
\let\footnote\gobbleoneargument
\getbuffer[column-1,column-2]
\stop

This text shows that one can do some tricks with columns.
The frame is created by:

\typebuffer[column-1]

\startbuffer[column-3]
\def\FramedColumn#1%
  {\hbox to \hsize
     {\ifodd\currentcolumn\unhbox\hss#1\else\unhbox#1\hss\fi}}
\stopbuffer

A less senseless display is:

\typebuffer[column-3]

This time the columns will look like:

\start
\let\footnote\gobbleoneargument
\getbuffer[column-3,column-2]
\stop

A column can be manipulated as a whole. For example to create
a background:

\startbuffer[column-4]
\setupfootnotes
  [location=columns,
   background=color,
   backgroundcolor=white]

\setuplayout
  [grid=yes]
\stopbuffer

\typebuffer[column-4]

This time the column will be typeset on a grid:

\start
\nowhitespace\getbuffer[column-4,column-2]
\stop

\section[alineas]{Paragraphs in columns}
\index{columns}
\index{paragraphs}

\macro{\tex{defineparagraphs}}
\macro{\tex{setupparagraphs}}
\macro[startparagraph]{\tex{start<<paragraph>>}}
\macro[paragraph]{\tex{<<paragraph>>}}

In some cases you want to typeset a paragraph in columns.
For example in a definition where you have a first column
containing meaningful text and a second column containing
meaningful text. In these cases you can use:

\showsetup{defineparagraphs}

This command defines a column layout that is recalled by its
name.

\showsetup{start<<paragraph>>}

The layout can be set up by:

\showsetup{setupparagraphs}

The width of non||specified columns is determined
automatically. Distance relates to horizontal white space in
front of a column. The next column is specified by:

\showsetup{<<paragraph>>}

We show a simple example of the use of paragraphs in columns.

\startbuffer
\defineparagraphs[TwoColumns][n=2]
\setupparagraphs[TwoColumns][1][width=5cm]

\startTwoColumns
  This is the top left corner.
\TwoColumns
  In graphic environments the top right corner is also called the upper
  right corner.
\stopTwoColumns

\startTwoColumns
  In a similar way, the bottom left corner is called the lower left corner.
\TwoColumns
  Which leaves the bottom right corner, that is also known as lower right
  corner. Now what is the alternative name for the top left corner?
\stopTwoColumns
\stopbuffer

\startexample
\typebuffer
\stopexample

Here the \type {\TwoColumns} separates the columns. With a
default setup this results in:

\startreality
\getbuffer
\stopreality

We also could have used \type {\nextTwoColumns} instead of
\type {\TwoColumns}. Sometimes this is more readable in your
\ASCII\ text. An alternative specification is:

\startexample
\starttyping
\TwoColumns first text \\ second text \\
\stoptyping
\stopexample

You can add a command to the keywords \type {bottom} and \type
{top}. These commands will be executed before or after the
text. For example a column can be forced down by \type
{[top=\vfill]}.

This is the right place to show a more complex example. The
use of \type {paragraphs} is preferred over the use of columns
because the text is kept together. If we want to score an
item on two dimensions we need three columns:

\startbuffer
\defineparagraphs [CombinedItem]     [n=3,rule=on]
\setupparagraphs  [CombinedItem] [2] [width=3em]
\setupparagraphs  [CombinedItem] [3] [width=7em]
\stopbuffer

\getbuffer

\startexample
\typebuffer
\stopexample

The item itself is defined with \type {\defineenumeration}
(see \in {section} [enummeration]):

\startbuffer
\defineenumeration
  [SomeItem]
  [location=left,text=,width=3em,before=,after=]
\stopbuffer

\getbuffer

\startexample
\typebuffer
\stopexample

The scoring is done on a scale that is typeset as an
itemization (see \in {section} [itemizing]). An item might
look like this in \ASCII:

\startbuffer
\startCombinedItem
  \startSomeItem
    The student is able to write a detailed planning for the
    design and construction of a water purification plant.
  \stopSomeItem
\nextCombinedItem
  \startitemize[5,packed]
    \item yes \item no
  \stopitemize
\nextCombinedItem
  \startitemize[5,packed]
    \item self study \item class room \item simulation
  \stopitemize
\stopCombinedItem
\stopbuffer

\startexample
\typebuffer
\stopexample

And will result in:

\getbuffer

When the scoring scales are identical over all items we can
use macros:

\startbuffer
\def\firstscale%
  {\startitemize[5,packed]
     \item yes \item no
   \stopitemize}

\def\secondscale%
  {\startitemize[5,packed]
   \item self study \item class room \item simulation
   \stopitemize}

\startCombinedItem
  \startSomeItem
    The student is able to write a detailed planning for the
    design and construction of a water purification plant.
  \stopSomeItem
\nextCombinedItem
  \firstscale
\nextCombinedItem
  \secondscale
\stopCombinedItem
\stopbuffer

\startexample
\typebuffer
\stopexample

Or even more sophisticated:

\startbuffer
\def\startItem%
  {\startCombinedItem
   \startSomeItem}

\def\stopItem%
  {\stopSomeItem
   \nextCombinedItem \firstscale
   \nextCombinedItem \secondscale
   \stopCombinedItem}

\startItem
  The student is able to write a detailed planning for the
  design and construction of a water purification plant.
\stopItem
\stopbuffer

\startexample
\typebuffer
\stopexample

A definition like the one above can be very surprising. The
commands in such a definition can interfere and result in
undesirable output. We think of \type {\vtop}'s that align
on the baseline and \type {\vbox} s that align under the
baseline. Another example with framed texts show that
\CONTEXT\ takes care of most of the problems.

\startbuffer
\defineparagraphs
  [Three]
  [n=3,
   line=on]

\setupframedtexts
  [align=middle]

\startThree
  \startframedtext[left]   left   \stopframedtext
\nextThree
  \startframedtext[middle] middle \stopframedtext
\nextThree
  \startframedtext[right]  right  \stopframedtext
\stopThree
\stopbuffer

%\startreality
\getbuffer
%\stopreality

\section[tabulate 1]{Tabulate}
\index{tables}
\index{lists}
\index{tabulate}
\macro{\tex{starttabulate}}

In a later chapter we will go into detail on typesetting
tables. Consider this paragraph to be an appetizer. We use
the term tabulate when a table is part of the running text.
A simple tabulation looks like this:

\startbuffer
\starttabulate[|l|p|]
\NC question \NC Sometimes it is surprising to notice that writers,
independently of each other, explore the same theme along similar lines.
Three of the four books mentioned here fall into this category. Which
books do not belong in this list? \NC \NR
\stoptabulate

\starttabulate[|l|l|l|]
\NC A. \NC This Perfect Day           \NC Ira Levin           \NC \NR
\NC B. \NC Opstaan op Zaterdag        \NC Jan Gerhart Toonder \NC \NR
\NC C. \NC Tot waar zal ik je brengen \NC Anton Koolhaas      \NC \NR
\NC D. \NC The City And The Stars     \NC Arthur Clarke       \NC \NR
\stoptabulate
\stopbuffer

\startexample
\typebuffer
\stopexample

This results in:

\startreality
\getbuffer
\stopreality

With \type {\NC} we go to the next column and with \type
{\NR} to the next row. Definitions like \type {[|l|p|]} and
\type {[|l|l|l|]} are called a template. The set ups are
similar to those of \type {\starttable} (see \in {in}
[tables]).

The default template looks like this: \type {[|l|p|]}. The
second column is typeset as a normal paragraph and with a
width that is calculated automatically by \TEX.

\startbuffer
\starttabulate
\NC d: \NC avond, afond, avend, afend \NC \NR
\NC t: \NC avont, afont, avent, afent \NC \NR
\stoptabulate
\stopbuffer

\startexample
\typebuffer
\stopexample

This quotation from \quotation{Spellingsverandering van zin
naar onzin} by G.C. Molewijk (1992) will look like this:
\footnote {For the non||dutch readers: this book \quotation
{Change of spelling, from sense to nonsense} is one of the
most humorous books on the developments in a language one
can imagine. If you ever come to studying dutch, you should
give this book a try.}

\getbuffer

\section[align]{Alignment}
\index{align}
\macro{\tex{setupalign}}
\macro{\tex{startalignment}}
\macro{\tex{leftaligned}}
\macro{\tex{rightaligned}}
\macro{\tex{midaligned}}
\macro{\tex{setuptolerance}}

Horizontal and vertical alignment is set up by:

\showsetup{setupalign}

The keys \type {left}, \type {middle} and \type {right},
\type {inner} and \type {outer} apply to horizontal
alignment and \type {bottom}, \type {height} and \type
{line} to vertical alignment.

The key \type {right} results in the text being typeset
ragged right. The keyword \type {broad} can be combined with
\type {left}, \type {middle} and \type {right} which results
in somewhat more rough alignments.

The option \type {line} lets the last line touch the bottom
of the page while \type {height} aligns the baseline to the
bottom.

Individual lines can be aligned with the commands:

\showsetup{leftaligned}

\showsetup{midaligned}

\showsetup{rightaligned}

alignment over a number of lines is done by:

\showsetup{startalignment}

The text below shows a number of examples of horizontal
alignment.

\startbuffer
\leftaligned{The Brittish stubbornly stick to}
\leftaligned{driving at the left side of the road.}
\blank[medium]
\rightaligned{This can be considered a form conservatism,}
\rightaligned{or alternatively phrased: right||wing thinking.}
\blank[medium]
\midaligned{However, a political drive||in||the||middle}
\midaligned{compromise would definitely lead to accidents.}
\stopbuffer

\startreality
\getbuffer
\stopreality

We done this with:

\startexample
\typebuffer
\stopexample

The last words of a paragraph can be placed on the right
hand side by the command \type {\wordright}, \wordright
{\bf so with:}

\showsetup{wordright}

When typesetting a paragraph, \TEX\ tries several
alternatives and decides which one to choose based on a
system, of penalties. Normally \TEX\ is very strict, but we
can instruct \TEX\ to be a bit more tolerant. This means
that, instead of letting problematic situations remain
unsolved |<|i.e.\ let words that cannot be hyphenated stick
into the margin|>| \TEX\ will add a bit more stretch and
apply different penalties for successive hyphens.

Alignment can be set up by:

\showsetup{setuptolerance}

By default we use \type {[horizontal,verystrict]} for
horizontal alignment and \type {[vertical,strict]} for
vertical alignment. \footnote {If you want a real ugly
result, you should set the \TEX\ variable \type
{\pretolerance} to 10.000. It is up to you.} A last resort
is provided by the keyword \type {stretch}, which in
unsolvable situations will stretch spaces, extending the
ugliness even further.

In double sided typesetting, alignment can be coupled to
the left or right pages.

\startbuffer
\startalignment[inner]
\quotation {Out of nowhere} is a rather normal way of saying that it is
not clear where something originates. It is typically a phrase that has
no counterpart, in the sense that nobody would comprehend the remark
\quotation {Into somewhere}.
\stopalignment

\startalignment[outer]
\quotation {Out of bounds} is a similar quote. There is no counterpart
\quotation {In of bounds}. Both examples demonstrate that in(ner) and
out(er) are not always counterparts.
\stopalignment
\stopbuffer

\startexample
\typebuffer
\stopexample

Results of the commands above depend on the location of the
page (left or right). The commands lead to:

\startreality
\getbuffer
\stopreality

\section[lines]{New lines}
\index{new lines}
\index{new+lines}
\index{numbering+lines}
\macro{\tex{crlf}}
\macro{\tex{startlines}}
\macro{\tex{startlinenumbering}}
\macro{\tex{setuplinenumbering}}
\macro{\tex{setuplines}}

A new line is forced by: \footnote {In titles, headers and
margin texts \type {\\} is available for introducing a new
line.}

\showsetup{crlf}

If you want to have lines show up the way you typed
them in your source file you can use:

\showsetup{startlines}

Default indenting is off. You can set up lines by:

\showsetup{setuplines}

If we set up \type {indenting=odd} for example we will
obtain:

\startbuffer
\setupindenting[medium]
\setuplines[indenting=even]
\startlines
Come on, he said, give me a while,
and I will typeset you this text
with rivers like the river Nile
\stoplines
\stopbuffer

\start
\getbuffer
\stop

This was typed in the source file as:

\typebuffer

Lines can be numbered with:

\showsetup{startlinenumbering}

A simple example of numbered lines might look like this:

\startbuffer
\startlinenumbering
There is of course no problem with trying to prevent illegal copying of
\cap {cd}'s and records. However, why should artists benefit from these
measures, who themselves have no problems with copying themes, lyrics
and melodies?
\stoplinenumbering
\stopbuffer

\startexample
\typebuffer
\stopexample

this becomes:

%\startreality
\getbuffer
%\stopreality

We can influence line numbering by:

\showsetup{setuplinenumbering}

With the variable \type {conversion} you set up the type of
numbering. You may even use your own character, for example
an em||dash (keyed in as \type {---}). In that case this
character is set in front of each line.

In \in {chapter} [cross references] we will explain how we
can refer to a linenumber. The parameters \type {prefix} and
\type {referencing} can be used to unfluence that proces.

In the example below we use the following setup:

\startexample
\starttyping
\setuplinenumbering[conversion=numbers,step=2,location=intext]
\stoptyping
\stopexample

and:

\startexample
\starttyping
\setuplinenumbering[conversion=characters,step=1,location=intext]
\stoptyping
\stopexample

\startsimplecolumns
\setuplines[before=,after=]
\setuplinenumbering[conversion=numbers,step=2,location=intext]
\startlines
\startlinenumbering
a macro is a piece of text
random at first sight
a bunch of stupid tokens that
looks less that awful right
\stoplinenumbering
\stoplines
\break
\setuplinenumbering[conversion=characters,step=1,location=intext]
\startlines
\startlinenumbering
but when fed to \TEX\ the program
you will be surprised
thanks to macros your text too
will look quite organized
\stoplinenumbering
\stoplines
\stopsimplecolumns

You can also mark lines in order to refer to specific line
numbers. This will be shown in in \in{chapter}[cross references].

\section[pages]{New page}
\index{new pages}
\index{new+page}
\macro{\tex{page}}

In some instances it is up to you to force, prevent or
encourage a new page.

\showsetup{page}

The possible set ups are explained in \in {table}
[tab:page]. If no setup is used \type {\page} will result in
a new page.

\placetable
  [here][tab:page]
  {Setups of \type {\page}.}
\starttable[|l|l|]
\HL
\VL \bf setup                                 \VL
    \bf result                                \VL\SR
\HL
\VL \type {yes}                                  \VL
     force a new page                            \VL\FR
\VL \type {makeup}                               \VL
     the same, without fill                      \VL\MR
\VL \type {no}                                   \VL
     when possible, avoid page break             \VL\MR
\VL \type {preference}                           \VL
     when possible, force page break             \VL\MR
\VL \type {bigpreference}                        \VL
     when possible, force page break, try harder \VL\MR
\VL \type {left}                                 \VL
     force a left page                           \VL\MR
\VL \type {right}                                \VL
     force a right page                          \VL\MR
\VL \type {disable}                              \VL
     ignore the next \type {\page} command       \VL\MR
\VL \type {last}                                 \VL
     add last page(s)                            \VL\MR
\VL \type {quadruple}                            \VL
     add pages until quadruple number of pages   \VL\MR
\VL \type {even}                                 \VL
     go to the next even page                    \VL\MR
\VL \type {odd}                                  \VL
     go to the next odd page                     \VL\MR
\VL \type {blank}                                \VL
     insert a completely blank page              \VL\MR
\VL \type {empty}                                \VL
     insert an empty page (with headers etc.)    \VL\MR
\VL \type {reset}                                \VL
     reset the disable command                   \VL\LR
\HL
\stoptable

The setups \type {last} and \type {quadruple} can be used in
double sided (reduced) typesetting. The first setup up will
add pages until an even number is obtained, the second set
up will add pages until the next quadruple is reached. When
you want to overrule the automatic page numbering you type
the pagenumber yourself:

\startexample
\starttyping
\page[25]
\stoptyping
\stopexample

You can also use a relative number like \type {[+4]}. You
can use this feature when you want to be on the safe side
and if you don't know at what page you are.

While generating empty pages you have to take
doublesidedness into account, for example:

\startexample
\starttyping
\page[right,empty,right]
\stoptyping
\stopexample

\section[pagenumbers]{Pagenumbers}
\index{pagenumbers}
\index{numbering+pages}
\index{mirroring}
\index{backspace}
\index{footers}
\index{headers}
\index{double||sided}
\index{single||sided}
\index{screen numbers}
\macro{\tex{setuppagenumbering}}
\macro{\tex{setuppagesubnumbering}}
\macro{\tex{setuppagenumber}}
\macro{\tex{pagenumber}}
\macro{\tex{subpagenumber}}
\macro{\tex{totalnumberofpages}}
\macro{\tex{numberofsubpages}}

At any location in the text the pagenumber can be set up with
the command:

\showsetup{setuppagenumber}

The pagenumber position on the page is defined by:

\showsetup{setuppagenumbering}

The position varies with the nature of the document. With
\type {conversion} we state the way we want to display the
number. With \type {location} we define pagenumber positions
like the bottom or top, left or right side or in the margin.
You can use combinations of these options. For example:

\startexample
\starttyping
\setuppagenumbering[location={header,inmargin}]
\stoptyping
\stopexample

\placetable
  [here][tab:setuppagenumbering]
  {setups to \type {\setuppagenumbering}.}
\starttable[|Tl|Tl|]
\HL
\VL alternative=singlesided \VL alternative=doublesided \VL\SR
\HL
\VL left, right             \VL marginedge              \VL\FR
\VL middle                  \VL middle                  \VL\MR
\VL margin                  \VL margin                  \VL\LR
\HL
\stoptable

Another alternative is \type {{singlesided,doublesided}}.
In this case headers and footers will be mirrored in a
double||sided document. The backspace is not mirrored (see
\in {figure} [fig:mirroring]).

\placefigure
  []
  [fig:mirroring]
  {Three ways to mirror.}
  {\getbuffer[sdsided]}

You can assign text to the parameters \type {left} and \type
{right}. These texts will encloses the pagenumber:

\startexample
\starttyping
\setuppagenumbering[conversion=romannumerals,left={--~},right={~--}]
\stoptyping
\stopexample

This will lead to: --~\romannumerals {8}~--. With \type
{style} you define the font and with \type {state}
pagenumbering is switched on and off.

Numbering can become very fancy when you use \type {command}
to execute an operation. This command has an argument and
will be executed every time a pagenumber is placed. A framed
pagenumber can be obtained by:

\startexample
\starttyping
\setuppagenumbering[command=\inframed]
\stoptyping
\stopexample

or partially framed by:

\startexample
\starttyping
\def\mypagenumber#1%
  {\inframed[frame=off,leftframe=on,rightframe=on]{#1}}

\setuppagenumbering[command=\mypagenumber]
\stoptyping
\stopexample

In this we use \type {\inframed} instead of \type {\framed},
because the pagenumber must align with the texts of the
headers and footers.

With \type {textseparator} you can define a separator
between the section and pagenumber. Default this is a --.
When the pagenumber is to appear at the margin the \type
{numberseparator} is placed between the number and the
footer text. Default this is a space with a width of 1em.

In interactive documents subpagenumbering is frequently used
for hyperlinking. When every new section is started on a new
page the footer text can be set up with:

\startexample
\starttyping
\setupsubpagenumber
  [way=byparagraph]
\setupfootertexts
  [screen {\subpagenumber} of {\numberofsubpages}] []
\stoptyping
\stopexample

The setup is done with:

\showsetup{setupsubpagenumber}

and the numbers themselves can be recalled by
\type {\subpagenumber} and \type {\numberofsubpages}. These
numbers are only reliable in headers and footers. In the
case of interactive documents a more abstract definition can
be used:

\startexample
\starttyping
\setupfootertexts[][{\interactionbar[alternative=d]}]
\stoptyping
\stopexample

In this case one can jump to the previous and following
subpages. The subnumbering can be reset with \type {[reset]}.

In a similar fashion one has access to the page number and
the total number of pages: \type {\pagenumber} and \type
{\totalnumberofpages}.

\section[headandfoot]{Headers and footers}
\index{header}
\index{footer}
\macro{\tex{definetext}}
\macro{\tex{setupheadertexts}}
\macro{\tex{setupfootertexts}}
\macro{\tex{setuptoptexts}}
\macro{\tex{setuptexttexts}}
\macro{\tex{setupbottomtexts}}
\macro{\tex{setupheader}}
\macro{\tex{setupfooter}}
\macro{\tex{setuptext}}
\macro{\tex{setuptop}}
\macro{\tex{setupbottom}}
\macro{\tex{noheadersandfooterlines}}
\macro{\tex{notopandbottomlines}}
\macro{\tex{donttest}}

Text in the header and footer are set up with the commands:

\showsetup{setupheadertexts}

\showsetup{setupfootertexts}

A great number of arguments can be added. When the first
argument is left out it is taken for granted that the footer
and header should be place under or over the pagebody (\type
{text}). The edge is located at the left side of the margin
and is only used in interactive documents where a extended
pagebody is needed.

The key \type {date} generates a date and \type {pagenumber}
generates the pagenumber. Part, chapter and section titles
can be summoned to appear in the header|| and footer text by
\type {part}, \type {chapter}, \type {paragraph} etc. By
default the mark mechanism is active. Sectionnumbers can
also be recalled: \type {chapternumber} etc.

Setting the \type {state} is done for the whole header, so
one should use the one||argument version:

\starttyping
\setupheader[state=high]
\stoptyping

Those who want more variations in headers and footers can
use four instead of two arguments. Four arguments have only
effect in double||sided documents.

\startexample
\starttyping
\setupfootertexts
  [even left][even right]
  [odd left][odd right]
\stoptyping
\stopexample

So there are different combinations of arguments possible:

\starttyping
\setupheadertexts
\setupheadertexts[mid text]
\setupheadertexts[left text][right text]
\setupheadertexts[left text][right text][left .][right .]
\setupheadertexts[location][left text][right text]
\setupheadertexts[location][left text][right text][left .][right .]
\stoptyping

Instead of text, one can specify keywords like \type
{chapter}, \type {date} or \type {pagenumber}. When the
pagenumber is positioned in this way, one should also say:

\starttyping
\setuppagenumbering[location=]
\stoptyping

The current setups of the headers and footers are cleared
when no values are stated in \type {\setupfootertexts}.
Problems can be expected when you use \setchars\ in your
setup. These have to be enclosed in curly brackets:

\startexample
\starttyping
\setupfootertexts[chapter][{\currentdate[month,year]}]
\stoptyping
\stopexample

The type setting of head|| and foot texts can be influenced
by:

\showsetup{setupheader}

and

\showsetup{setupfooter}

As with \type {\setup...texts} the first argument is
optional. The keys \type {state}, \type {before} and \type
{after} work on all parts of the pagebody, on the main text,
the margins and edges.

When \type {...width} is set up the text is clipped at the
given width. The key \type {strut} is important when footers
or headers contain other objects than text. When \type
{strut} is set to \type {no}, the object is not corrected
for linedepth. You could use the command \type {\showstruts}
to get some information on this phenomena.

The setups with \type {state} are explained in \in {table}
[tab:stelhoofdin]. You should bear in mind that page
numbering will always continue whether or not the pagenumbers
are placed.

\placetable
  [here][tab:stelhoofdin]
  {Setups with \type {\setupheader} and \type {\setupfooter}.}
\starttable[|l|l|]
\HL
\VL \bf setup         \VL \bf result                                 \VL\SR
\HL
\VL \type {normal}    \VL visible                                    \VL\FR
\VL \type {none}      \VL invisible, no whitespace                   \VL\MR
\VL \type {empty}     \VL one page invisble, whitespace              \VL\MR
\VL \type {high}      \VL one page visible, no whitespace            \VL\MR
\VL \type {start}     \VL visible                                    \VL\MR
\VL \type {nomarking} \VL leave out marks                            \VL\MR
\VL \type {stop}      \VL invisible, whitespace                      \VL\LR
\HL
\stoptable

When setups are done between \type {\start} and \type
{\stop} they will only work locally. This means that the
setups are reset after \type {stop}. Headers and footers may
appear even while you think new ones should appear. This is
due to the way \TEX\ determines valid breakpoints. One can
never be certain when such an automatic break will occur.
The solution is to force a new page by \type {\page} before
\type {\stop}.

Headers and footers can be switched off on a page by means
of:

\showsetup{noheaderandfooterlines}

Next to head|| and footertexts there are also over|| and
bottomtexts. These are setup in a similar way:

\showsetup{setuptoptexts}
\showsetup{setuptexttexts}
\showsetup{setupbottomtexts}

\showsetup{setuptop}
\showsetup{setuptext}
\showsetup{setupbottom}

\showsetup{notopandbottomlines}

When the height of an area equals zero, no text is placed.
By default the top and bottom area have zero height, so
setting their text areas without setting the height has no
effect.

At the instance of a new part or chapter we can deal in a
different way with the headers and footers. Suppose that a
default setup looks like this:

\starttyping
\setupheadertexts[pagenumber]
\setupfootertexts[chapter][paragraph]
\stoptyping

At the first page of new chapters this may look not too
good. Therefore we could state:

\starttyping
\setuphead[chapter][header=empty,footer=empty]
\stoptyping

However if we use it in this way we loose the pagenumber. A
more adequate solution is:

\starttyping
\definetext[chapter][footer][pagenumber]
\stoptyping

with:

\starttyping
\setuphead[chapter][header=high,footer=chapter,page=right]
\stoptyping

we obtain the desired effect. The pagenumber appears in the
foot and the header disappears completely. These kind of
commands are essential when you don't want to define all
kinds of setups locally in a text, for example before every
new chapter. This mechanism only works when going to a new
page enabled.

\showsetup{definetext}

\section[footnotes]{Footnotes}
\index{footnotes}
\index{symbols}
\macro{\tex{footnote}}
\macro{\tex{note}}
\macro{\tex{setupfootnotes}}
\macro{\tex{startlocalfootnotes}}
\macro{\tex{placelocalfootnotes}}
\macro{\tex{placefootnotes}}

In some texts you can't do without footnotes. The footnote
marker is placed in the text and the note itself is typeset
at another location in the text, usually at the bottom of
the page. Most often at the bottom of the page.

\showsetup{footnote}

A footnote number or ||symbol is recalled with:

\showsetup{note}

An example of footnotes is given below.

\startbuffer
The first compositions of the American composer Steve Reich will probably
only appreciated by the most \quote {purist} among those who like
minimal||music \footnote {A decent minimal is not so much characterized by
a minimal use of musical instruments, but more by subtle shifts in
polyphonic rhythms.}, his later works, like \quote {The Desert Music}, are
compositions for full orchestra, where the orchestra is extended with a for
Reich characteristic rhythm section \footnote {In most cases this section
consists of pianos, marimbas and xylophones.} and choir. Together
with John Adams, \footnote {His \quote {Fearful Symmetries} is a perfect mix
of classic, jazz, swing and pop music.} Reich can be considered one of
today's leading composers. It is, however, a pity that they can only be seen
\footnote {The nice thing about compositions like \quote {Drumming} and
\quote {Sextet} is de fact that \quotation {what the ear hears} differs
from what the \quotation {eye sees happening}.} and heard at the smaller
broad companies, like the \cap {VPRO}. \footnote{A non commercial Dutch
broadcast company.} \footnote {Sometimes also at other companies, because
somehow this kind of music is quite suited for impressive and|/|or
melodramatic documentaries.}
\stopbuffer

\startexample
\typebuffer
\stopexample

Undesired spaces are ignored. Spacing between two footnote
numbers or symbols is taken care of. The result looks like
this:

\startreality
\getbuffer
\stopreality

The type setting of the footnote can be setup with the
command below that is defined in the setup area of your
document.

\showsetup{setupfootnotes}

By default footnotes are placed at the bottom of a page.
When using columns you can set \type {location} to \type
{columns} so that the footnotes appear in the last column.

We can frame footnotes, place them in columns and decouple
them from a page. The meaning of this last option is
explained in an example.

\startbuffer
\startlocalfootnotes[n=0]
  \placetable
    {A (latin) table.}
    \placelegend
      {\starttable[|l|r|]
       \HL
       \VL Nota \footnote {Bene} \VL Bene \footnote {Nota} \VL\FR
       \VL Bene \footnote {Nota} \VL Nota \footnote {Bene} \VL\LR
       \HL
       \stoptable}
      {\placelocalfootnotes}
\stoplocalfootnotes
\stopbuffer

\typebuffer

The table enables the float placement mechanism, so we don't
know on which page the table nor the footnotes will appear.
So the footnotes are coupled to the table by using local
footnotes.

\getbuffer

\showsetup{startlocalfootnotes}

\showsetup{placelocalfootnotes}

Footnotes can be placed at the end of a chapter or a
document. The key \type {location} is set at \type {text}
and we use the following command to place the footnotes:

\showsetup{placefootnotes}

When \type {n} is set at~2, you can display the footnotes in
columns. This should be done at an early stage because \TEX\
is using the dimensions of the footnotes to determine the
page break. More information can be found in the source code
of the \CONTEXT\ module: \type {core-not.tex}.

The next example demonstrates that footnote numbers can be
replaced by footnote symbols. In this example \type
{conversion} is set at \type {set 3}.

\startitemize[3*broad,packed]

\setupfootnotes[conversion=set 3]

\sym{note:} use footnotes sparingly \footnote [first]
{During the development of \CONTEXT\ the footnote mechanism
was one of the first real challenges. And I'm challenged
still since I just encountered documents with footnotes
within footnotes.}

\sym{note:} be brief \footnote{Why? See note \note[first].}

\sym{note:} no notes are even better \footnote{QED.}

\stopitemize

Default the key \type {numbercommand} is set \type
{\high}, but other setups are allowed. You can also
work with:

\showsetup{setupfootnotedefinition}

to define the exact way of how to display the footnotes,
because the standard definition mechanism is used (see \in
{section} [definities]).

\section{Aligned boxes}
\index{alignment}
\macro{\tex{lbox}}
\macro{\tex{rbox}}
\macro{\tex{cbox}}
\macro{\tex{sbox}}
\macro{\tex{vbox}}
\macro{\tex{vtop}}
\macro{\tex{hbox}}
\macro{\tex{tbox}}
\macro{\tex{bbox}}

\TEX\ is basically aware of two kind of boxes: \type {\hbox}
and \type {\vbox}. A horizontal \type {\hbox} can be
considered a line, a \type {\vbox} a paragraph. There are
two types of vertical boxes: a \type {\vbox} aligns on the
baseline of the last line, while a \type {\vtop} aligns on
the first line.

\startbuffer
\hbox{\hbox{one} \vbox{two\par three} \vtop{four\par five}}
\stopbuffer

\startexample
\typebuffer
\stopexample

When we make the frames visible |<|in this case we said \type
{\showboxes} in advance|>| the example above becomes:

\startbaselinecorrection
\hsize3cm\showboxes\getbuffer
\stopbaselinecorrection

In addition \CONTEXT\ provides a lot of alternative boxes,
like: \type {\cbox}, \type {\lbox} and \type {\rbox}. These
commands can be used while defining your own macros, but
will seldom appear in the running text. Like in \type
{\hbox} and \type {\vbox} the dimension of the width can be
added.

\startexample
\starttyping
\cbox{... text ...}
\lbox to 4cm{... text ...}
\stoptyping
\stopexample

The reader is invited to experiment with these commands. A
new line is forced with \type {\\}.

For some very dedicated purposes there is \type {\sbox}.
This command is used to give a box the height of a strut.
You may forget this command.

To another category of boxes belong \type {\tbox} and \type
{\bbox}. Both are used within tables. Look at the example
below that illustrates their use.

\getbuffer[boxes-1]

The \type {\tbox} and \type {\bbox} are also used in
figures.

\getbuffer[boxes-2]

In \CONTEXT\ a complete repertoire of macros is available
that relies on boxes. For example we can add cutmarks to a
box:

\startexample
\typebuffer[boxes-3]
\stopexample

Be aware of the fact that such marks lie outside the boxes.

\startreality
\startbaselinecorrection
\getbuffer[boxes-3]
\stopbaselinecorrection
\stopreality

We can visualize boxes by using \type {\ruledhbox}, \type
{\ruledvbox} and \type {\ruledvtop} instead of \type
{\hbox}, \type {\vbox} and \type {\vtop}. With \type
{\showmakeup} we can visualise everything automatically and
we can get some insight on the features of \CONTEXT\ and
\TEX.

The next example shows that we can use \TEX\ for more than
only the straight forward typesetting. However, to be able
to do this, one should have some insight in the manipulation
of boxes. We use buffers to enhance comprehensibility.

\getbuffer[boxes-4]

\typebuffer[boxes-4]

This text is typeset in a framed box. We use two temporary
boxes. The first determines the height of the second one.
Instead of \type {\tfx\setupinterlinespace} you could use
\type {\switchtobodyfont} to switch to a narrower bodyfont.
(\type {[small]}). The \type {\par} is essential!

\startbuffer
\framed[offset=\bodyfontsize]
  {\setbox0=\vbox
     {\hsize 16em\switchtobodyfont[ss]\getbuffer[water]}
   \setbox2=\vbox to \ht0
     {\vfill\externalfigure[vew1091a][width=5cm]\vfill}
   \hskip1em\box2\hskip1em\box0\hskip1em}
\stopbuffer

\startexample
\typebuffer
\stopexample

The result |<|an example of a drinking water warning|>| is shown
below.

\startbaselinecorrection
\getbuffer
\stopbaselinecorrection

\section[typesetting]{Makeup}
\index{makeup}
\macro{\tex{definemakeup}}
\macro{\tex{setupmakeup}}
\macro[startnamemakeup]{\tex{start<<name>>makeup}}
\macro{\tex{startstandardmakeup}}

A document may have a titlepage, a colofon and some pages
that are not directly related to the main part of the
document. Mostly these pages are not numbered and can do
without headers and footers. Because their layout needs
extra attention we prefer the word makeup for defining
their specific layout.

The commands \type {\startstandardmakeup} and \type
{\stopstandardmakeup} exclude text from the standard
pagebody and its layout. Below a simple example is given.
You will notice commands like \type {\vfill}, \type
{\blank}, \type {\tf} and even \type {\crlf} and \type
{\vskip}.

\startexample
\starttyping
\startstandardmakeup
  \tfd Jobs around the house \blank[2*big]
  \tfb Part 1: Gas, water and  electricity \vfill
  \tfb J. Hagen \crlf A.F. Otten \blank
  \tfb Hasselt \crlf \currentdate[month,year]
\stopstandardmakeup
\stoptyping
\stopexample

In double||sided documents an empty page is generated that
functions as the backside of the title page. However
sometimes this backside should also be typeset.

\startexample
\starttyping
\startstandardmakeup[doublesided=no]
... the front
\stopstandardmakeup
\startstandardmakeup[page=no]
... the back
\stopstandardmakeup
\stoptyping
\stopexample

Because double||sided typesetting is turned off, a backside
page is not generated. And because the key \type {page} is
\type {no} the next page does not get the layout of a right
hand side page (this would be default).

With the command \type {\showframe} frames can be made
visible (temporarily) around the made up text. This is
very convenient during the typesetting of separate pages.

Next to the command \type {\startstandardmakeup} one can
define his own layout with different dimensions by means of:

\showsetup{definemakeup}

\showsetup{setupmakeup}

\showsetup{start<<name>>makeup}

The first command generates a \type {\start...stop}||pair
between which the new typesetting commands can be typed.
Bij default the result of this new layout is typeset on an empty
page. The new layout is marked with \type {<<name>>}, for
selection at a later stage (see \in {section} [selecteren]).

The commands that are provided after the key \type {commands}
are executed immediately when a new layout is called. In
this local layouts can be defined.

\stopcomponent

