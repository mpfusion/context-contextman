% interface=en

\startcomponent co-documents

\environment contextref-env
\product contextref


\startFLOWchart [opzet-1]

  \startFLOWcell
    \shape{procedure}
    \name{onderdeel}
    \location{1,1}
    \text{\translate[nl=onderdeel,en=component]}
  \stopFLOWcell
  \startFLOWcell
    \shape{procedure}
    \name{omgeving-3}
    \location{1,2}
    \text{\translate[nl=omgeving,en=environment]}
    \connect [tb] {onderdeel}
  \stopFLOWcell

\stopFLOWchart

\startFLOWchart [opzet-2]

  \startFLOWcell
    \shape{procedure}
    \name{produkt}
    \location{1,1}
    \text{\translate[nl=produkt,en=product]}
    \connect [rl] {onderdeel}
  \stopFLOWcell
  \startFLOWcell
    \shape{procedure}
    \name{omgeving-2}
    \location{1,2}
    \text{\translate[nl=omgeving,en=environment]}
    \connect [tb] {produkt}
  \stopFLOWcell

  \includeFLOWchart[opzet-1][x=2,y=1]

\stopFLOWchart

\startFLOWchart [opzet-3]

  \startFLOWcell
    \shape{procedure}
    \name{projekt}
    \location{1,1}
    \text{\translate[nl=projekt,en=project]}
    \connect [rl] {produkt}
  \stopFLOWcell
  \startFLOWcell
    \shape{procedure}
    \name{omgeving-1}
    \location{1,2}
    \text{\translate[nl=omgeving,en=environment]}
    \connect [tb] {projekt}
  \stopFLOWcell

  \includeFLOWchart[opzet-2][x=2,y=1]

\stopFLOWchart


\chapter[documents]{Documents}

\section{Introduction}

Why should one use \TEX\ in the first place? Many people
start using \TEX\ because they want to typeset math. Others
are charmed by the possibility of separating content and
make||up. Yet another kind of user longs for a programmable
system. And let us not forget those users that go for quality.

When using \TEX\ one does not easily run into capacity
problems. Typesetting large documents with hundreds of
pages is typically a job for \TEX. If possible, when coding
a document one should look beyond the current document.
These days we see documents that were originally typeset
for paper being published in electronic format. And how
about making a stripped version of a 700 page document? A
strict separation between content and layout (make||up) on
the one hand and an acceptable redundancy in structure on
the other is often enough to guarantee multiple use of one
document source.

A system like \CONTEXT\ is meant to make life easier.
When coding a document the feeling can surface that \quotation
{this or that should be easier}. This feeling often
reflects the truth and the answer to the question can often
be found in this manual, although sometimes obscured. It
takes some time to learn to think in structure and content,
certainly when one is accustomed to mouse driven
word processors. In this chapter we focus on the structure of
collections of documents.

\section[startstop]{Start and stop}
\index{start}
\index{stop}
\index{structure}
\macro{\tex{starttext}}

In a self contained text we use the following commands to
mark the begin and end of a text:

\startexample
\starttyping
\starttext
\stoptext
\stoptyping
\stopexample

The first command takes care of a number of initializations
and the last command tells \TEX\ that processing can stop.
When this command is left out \TEX\ will display a \type{*}
(a star) on the command line at the end of the job. \TEX\
will expect a command, for example \type{\end}.

It is advisable to type the document setups before the
\type{\start}||command, the so called setup area of the
document. In this way a clever word||processor can identify
where the text starts, and therefore can include those
setups when it partially processes the document, given of
course that it supports partial processing of files.

In the example below a very simple layout is being used.

\startexample
\starttyping
\starttext

\subject{Introduction}

\unknown\ America has always been a land set firmly not in the past, but
in the future. On a recent visit to England, I found dozens of wonderful
bookstores chock full of the past --- ancient history, rooms full of it,
and great literature in such monumental stacks as to be overwhelming. In
the usual American bookstore, history might occupy a few bookcases; great
literature has its honoured place, but this year's paperbacks dominate. The
past is not disregarded, but neither does it loom so large and run so deep
in our blood.

\blank

{\bf Greg Bear, introduction to Tangents (1989).}

\stoptext
\stoptyping
\stopexample

The commands \type{\starttext...\stoptext} may be nested.
Within a text a new text containing \type{\starttext} and
\type{\stoptext} may be loaded.

\section[structure]{Structure}
\index{structure}
\index{products}
\index{projects}
\index{environments}
\index{components}
\macro{\tex{startproject}}
\macro{\tex{startproduct}}
\macro{\tex{startenvironment}}
\macro{\tex{startlocalenvironment}}
\macro{\tex{startcomponent}}
\macro{\tex{project}}
\macro{\tex{product}}
\macro{\tex{environment}}
\macro{\tex{components}}

In this section a structured approach of managing your
documents is discussed. For very simple and self containing
documents you can use the following approach:

\starttyping
\environment this
\environment that

\starttext
... some interesting text ...
\stoptext
\stoptyping

When you have to typeset very bulky documents it is better
to divide your document in logical components. \CONTEXT\
allows you to setup a project structure to manage your
texts. You have to know that:

\startitemize

\item A group of texts that belong together have to be
      maintained as a whole. We call this a {\em project}.

\item Layout characteristics and macros have to be defined
      at the highest level. For this, the term {\em
      environment} has been reserved.

\item Texts that belong together in a project we call {\em
      products}.

\item A product can be divided into components, these
      components can be shared with other products.
      Components can be processed individually.

\stopitemize

Programmable word processors can be adapted to this
structure.

A {\em project}, {\em environment }, {\em product} or {\em
component} is started and stopped with one of the following
commands:

\showsetup{startproject}
\showsetup{startproduct}
\showsetup{startenvironment}
\showsetup{startcomponent}

Before a \type{\start}||\type{\stop}||pair commands can be
added. When a file is not found on the directory \CONTEXT\
looks for the files on higher level directories. This
enables the user to use one or more environments for
documents that are placed on several subdirectories.

\placetable
  [here][tab:structure commands]
  {The structure commands that can be used in the files that
   make up a project.}
\starttable[|l|c|c|c|c|]
\HL
\VL \bf command     \VL
    \bf project     \VL
    \bf environment \VL
    \bf product     \VL
    \bf component  \VL\SR
\HL
\VL \type{\project <<name>>}
    \VL           \VL        \VL (\yes) \VL (\yes) \VL\FR
\VL \type{\environment <<name>>}
    \VL    (\yes) \VL (\yes) \VL (\yes) \VL (\yes) \VL\MR
\VL \type{\product <<name>>}
    \VL     \yes  \VL        \VL        \VL (\yes) \VL\MR
\VL \type{\component <<name>>}
    \VL           \VL        \VL (\yes) \VL (\yes) \VL\LR
\HL
\stoptable

To treat products and components as individual documents,
the commands in \in{table}[tab:structure commands] are used.
The commands marked with \yes\ are obligatory and the
commands marked with (\yes) are optional. The content is
typed before the \type{\stop} command.

\startbuffer
\startproject documents

\environment layout

\product  teacher
\product  pupil
\product  curriculum

\stopproject
\stopbuffer

\startfiguretext[left,offset]{none}
  \startMiniFile
    \typebuffer
  \stopMiniFile
  An example of a project file.
\stopfiguretext

\startbuffer
\startproduct teacher

\project   documents

\component teacher1
\component teacher2

\stopproduct
\stopbuffer

\startfiguretext[left,offset]{none}
  \startMiniFile
    \typebuffer
  \stopMiniFile
  The product \type{teacher.tex} (a teacher manual)
  can be defined as shown on the opposite site.
\stopfiguretext

\startbuffer
\startcomponent teacher2

\project documents
\product teacher

... text ...

\stopcomponent
\stopbuffer

\startfiguretext[left,offset]{none}
  \startMiniFile
    \typebuffer
  \stopMiniFile
  Here we see the component.
\stopfiguretext

In most cases working with only \type {\starttext} and
\type {\stoptext} in combination with \type {\input} or
\type {\enviroment} is sufficient. A project structure has
advantages when you have to manage a great number of texts.
Although it is more obvious to process {\em products} as a
whole, it also enables you to process {\em components}
independently, given that the stucture is defined properly.

In principal a project file contains only a list of products
and environments. If you would process the project file all
products will be placed in one document. This is seldom
wanted.

If you have only one product, you don't really need a project file.
This manual for example has a product/component structure without a project
file. Every chapter is a component and the product file loads some
environments.

Schematically the coherence between files could be
displayed as illustrated in \in {figures} [fig:structure
1], \in [fig:structure 2] \in {and} [fig:structure 3].

\placefigure
  [here][fig:structure 1]
  {An example of project structure.}
  {\FLOWchart[opzet-3]}

\placefigure
  [here][fig:structure 2]
  {An example with only products.}
  {\FLOWchart[opzet-2]}

\placefigure
  [here][fig:structure 3]
  {An example with only one component.}
  {\FLOWchart[opzet-1]}

It is good practice to put all setups in one environment.
In case a component or product has a different layout you
could define {\em localenvironments}:

\startexample
\starttyping
\startlocalenvironment[<<names>>]
... setups ...
\stoplocalenvironment
\stoptyping
\stopexample

A local environment can be typed in an environment file or
is a separate file itself. When a separate file is used the
local environment is loaded with:

\startexample
\starttyping
\localenvironment <<name>>
\stoptyping
\stopexample

Below you will find an example of a project structure.

\startbuffer
\startproject demos

\environment environ
\product     example

\stopproject
\stopbuffer

\startfiguretext[left,offset]{none}
  \startMiniFile
    \typebuffer
  \stopMiniFile
  file: \type{demos.tex}
  \blank
  This file is used to define the products and environments.
\stopfiguretext

\startbuffer
\startenvironment environ

\setupwhitespace[big]

\setupfootertexts[part][chapter]

\stopenvironment
\stopbuffer

\startfiguretext[left,offset]{none}
  \startMiniFile
    \typebuffer
  \stopMiniFile
  file: \type{environ.tex}
  \blank
  In the environment we type the setups that relate to all
  the different products. More than one environment or local
  environments per product can be used.
\stopfiguretext

\startbuffer
\startproduct example

\project demos

\startfrontmatter
  \completecontent
\stopfrontmatter

\startbodymatter
  \component first
  \component second
\stopbodymatter

\startbackmatter
  \completeindex
\stopbackmatter

\stopproduct
\stopbuffer

\startfiguretext[left,offset]{none}
  \startMiniFile
    \typebuffer
  \stopMiniFile
  file: \type{example.tex}
  \blank
  The product file contains the structure of the product.
  Because indexes and registers can be evoked quite easily
  we do not use a separate file.
\stopfiguretext

\startbuffer
\startcomponent first

\environment environ
\product example

\part{One}

\completecontent

\chapter{First}

..... text .....

\chapter{Second}

..... text .....

\completeindex

\stopcomponent
\stopbuffer

\startfiguretext[left,offset]{none}
  \startMiniFile
    \typebuffer
  \stopMiniFile
  file: \type{first.tex}
  \blank
  In the components of a product we place the textual
  content, figures etc. It is also possible to request the
  tables of content and registers per product.
\stopfiguretext

\startbuffer
\startcomponent second

\environment environ
\product example

\part{Two}

\completecontent

\chapter{Alfa}

..... text .....

\chapter{Beta}

..... text .....

\completeindex

\stopcomponent
\stopbuffer

\startfiguretext[left,offset]{none}
  \startMiniFile
    \typebuffer
  \stopMiniFile
  file: \type{second.tex}
  \blank
  The product contains more than one component. We could have
  defined a product for each part and a component for each
  chapter.
\stopfiguretext

The files \type{first.tex}, \type{second.tex} and
\type{example.tex} can be processed separately. If you process an environment
there will be no pages of output.

\section[directories]{Directories}
\index{files+directories}
\index{directories}

Many \TEX\ implementations look for a file in all
directories and subdirectories when a requested file is not
in the current directory. This is not only time||consuming
but may lead to errors when the wrong file (a file with the
same name) is loaded.

For this reason \CONTEXT\ works somewhat differently. A file
that is not available on the working directory is searched
for on the parent directories. This means that environments
can be placed in directories that are parents to the
products that use them. For example:

\starttyping
/texfiles/course/layout.tex
/texfiles/course/teacher/manual.tex
/texfiles/course/student/learnmat.tex
/texfiles/course/otherdoc/sheets.tex
\stoptyping

The last three files (in different subdirectories) all use
the same environment \type {layout.tex}. So, instead of
putting all files into one directory, one can organize them
in subdirectories. When a project is properly set up, that
is, as long as the project file and specific environments
can be found, one can process components and products
independently.

\section[versie]{Versions}
\index{testing}
\index{references+checking}
\index{index+checking}
\macro{\tex{version}}

During the process of document production it is useful to
generate a provisional version. This version shows the
references and the typesetting failures. The provisional
version is produced when you type:

\showsetup{version}

By default the definitive version is produced. In case a
preliminary version is produced the word {\em concept} is
placed at the bottom of each page. The keyword \type
{temporary} shows some information on for instance overfull
lines, references, figure placement, and index entries. Most
messages are placed in the margin. In some cases these
messages refer to the next pages because \TEX\ is processing
in advance.

\section[modes]{Modes}
\index{selective typesetting}
\index{modes}
\index{styles}
\index{specials}
\index{output format}
\index[texexec+mode]{\TEXEXEC+mode}
\macro{\tex{setupoutput}}
\macro{\tex{startmode}}
\macro{\tex{startnotmode}}
\macro{\tex{enablemode}}
\macro{\tex{disablemode}}
\macro{\tex{doifmodeelse}}
\macro{\tex{doifmode}}
\macro{\tex{doifnotmode}}

\TEX\ can directly produce \DVI\ or \PDF. A document can be
designed for paper and screen, where the last category
often has additional functionality. From one document we
can generate different alternatives, both in size and in
design. So, from one source several alternatives can be
generated.

Processing a file in practice comes down to launching \TEX\
with the name of the file to be processed. Imagine that by
default we generate \DVI\ output. Switching to \PDF\ is
possible by enabling another output format in the file itself
or a configuration file, but both are far from comfortable.

\starttyping
\setupoutput[pdftex]
\stoptyping

for direct \PDF\ output, or for \PDF\ produced from
\POSTSCRIPT:

\starttyping
\setupoutput[dvips,acrobat]
\stoptyping

The key to the solution of this problem is \TEXEXEC. This
\PERL\ script provides \CONTEXT\ with a
command||line||interface. When we want \PDF\ instead of
\DVI, we can launch \TEXEXEC\ with:

\starttyping
texexec  --pdf  filename
\stoptyping

There are more options, like making A5||booklets; more on
these features can be found in the manual that comes with
\TEXEXEC. However, one option deserves more time: modes.

\starttyping
texexec  --pdf  --mode=screen  filename
\stoptyping

The idea behind modes is that within a style definition, at
each moment one can ask for in what mode the document is
processed. An example of a mode dependant definition is:

\starttyping
\startmode[screen]
  \setupinteraction[state=start]
  \setupcolors[state=start]
\stopmode
\stoptyping

if needed, accompanied by:

\starttyping
\startnotmode[screen]
  \setupcolors[state=start,conversion=always]
\stopnotmode
\stoptyping

One can also pass more than one mode, separated by comma's.
There are also some low level mode dependant commands.
Given that we are dealing with a screen mode, we can say:

\starttyping
\doifmodeelse {screen} {do this} {and not that}
\doifmode     {screen} {do something}
\doifnotmode  {screen} {do something else}
\stoptyping

A mode can be activated by saying:

\starttyping
\enablemode[screen]
\disablemode[screen]
\stoptyping

Again, we can pass more modes:

\starttyping
\enablemode[paper,A4]
\stoptyping

One strength of \TEXEXEC\ is that one is not forced to
enable modes in a file: one can simply pass a command line
switch. Just as with choosing the output format: the less
we spoil the document source with output and mode settings,
the more flexible we are.

To enable users to develop a style that adapts itself to
certain circumstances, \CONTEXT\ provides system modes. For
the moment there are:

\starttabulate[|lT|p|]
\NC *list         \NC the list one called for is placed indeed     \NC \NR
\NC *register     \NC the register one called for is placed indeed \NC \NR
\NC *interaction  \NC interaction (hyperlinks etc) are turned on   \NC \NR
\NC *sectionblock \NC the named sectionblock is entered            \NC \NR
\stoptabulate

System modes are prefixed by a \type {*}, so they will not
conflict with user modes. An example of a sectionblock mode
is \type {*frontmatter}. One can use these modes like:

\starttyping
\startmode[*interaction]
  \setuppapersize[S6][S6]
\stopmode
\stoptyping


\section{Modes Manual}

\todo{Merge with previous section}

Every user will at one moment run into modes. Modes are used for
conditional processing. You enable or disable modes:

\starttyping
\enablemode[screen]
\disablemode[proof]
\stoptyping

as well as prevent modes being set:

\starttyping
\preventmode[doublesided]
\stoptyping

Later on you can act upon this mode using:

\starttyping
\startmode[screen]
  \setupinteraction[state=start]
\stopmode
\stoptyping

The counterpart of this command is:

\starttyping
\startnotmode[screen]
  \setupinteraction[state=start]
\stopnotmode
\stoptyping

You can set modes in your document or in styles, but you can
also do that at runtime:

\starttyping
texexec --pdf --mode=screen --result=myfile-s myfile
texexec --pdf --mode=A4     --result=myfile-a myfile
texexec --pdf --mode=letter --result=myfile-l myfile
\stoptyping

You can test for more modes at the same time:

\starttyping
\startmode[color,colour]
  \setupcolors[state=start]
\stopmode
\stoptyping

If you want to satisfy a combination of modes, you use:

\starttyping
\startmode[final]
  \setuplayout[markings=on]
\stopmode
\startallmodes[final,color]
  \setuplayout[markings=color]
\stopallmodes
\stoptyping

The counterpart is

\starttyping
\startnotallmodes[print,proof]
  \setuplayout[markings=off]
\stopnotallmodes
\stoptyping

Instead of the \type {start}||\type {stop} variants, you can use
the \type {\doif} alternatives. These have the advantage that they
can be nested.

\starttyping
\doifmodeelse     {modes} {action} {alternative}
\doifmode         {modes} {action}
\doifnotmode      {modes} {action}
\doifallmodeselse {modes} {action} {alternative}
\doifallmodes     {modes} {action}
\doifnotallmodes  {modes} {action}
\stoptyping

Mode can be combined with variables:

\starttyping
\setupvariables[document][alternative=print]

\enablemode[document:\getvariable{document}{alternative}]

\startmode[document:print]
  ...
\stopmode

\startmode[document:screen]
  ...
\stopmode
\stoptyping

An alternative for such an selective approach is to use setups:

\starttyping
\setupvariables[document][alternative=print]

\startsetups[document:print]
  ...
\stopsetups

\startsetups[document:screen]
  ...
\stopsetups

\setups[document:\getvariable{document}{alternative}]
\stoptyping

The difference is that mode blocks are processed in the order that
the document (or style) is loaded, while setups are stored and
recalled later.

In addition to your own modes, \CONTEXT\ provides a couple
of system modes. These are preceded by a \type {*}, as in:

\starttyping
\startmode[*first]
  % this is the first run
\stopmode
\stoptyping
 The following system modes are available (more will implemented):

% \start \setuptype[color=]

\startswitch
{\type{color-c},\type{color-m},\type{color-y},\type{color-k}}

These are rather special modes related to color separation. They are
only set when channels are split off.

\stopswitch

\startswitch {\type{figure}}

This mode is set when a graphic is found. You can use this mode in
for instance figure postprocessing actions.

\stopswitch

\startswitch {\type{text}, \type{project}, \type{product},
\type{component}, \type{environment}}

These modes are set when one enters one of the associated
structuring environments. Nesting is supported.

\stopswitch

\startswitch {\type{list}}

After using \type {\determinelistcharacteristics} this mode reflects
if list entries were found.

\stopswitch

\startswitch {\type{pairedbox}} This mode is enabled when a paired
box (legenda and such) is constructed. \stopswitch

\startswitch {\type{combination}} This mode is enabled when a
combination (often used for graphics) is constructed. \stopswitch

\startswitch {\type{interaction}}

When interaction is enabled, this mode is true. You can for instance
use this mode to add different content to for instance screen and
paper versions.

\stopswitch

\startswitch {\type{register}}

After using \type {\determineregistercharacteristics} this mode
reflects if register entries were found.

\stopswitch

\startswitch {\type{sectionnumber}}

This mode is enabled when a section head is numbered. You can access
the mode while building the section head, which is true when you have
your own commands hooked into the head mechanism.

\stopswitch

\startswitch {\type{frontpart}, \type{bodypart}, \type{backpart},
\type{appendix}}

The state of main sections in a document as well as user defined
ones, are reflected in system modes.

\stopswitch

\startswitch {\type{suffix-\jobfilesuffix}}

You can use this mode to differentiate between input file types. We
use this for instance to distinguish between different XML content
variants when pretty|-|printing (given that they can be recognized
on their suffix).

\stopswitch

\startswitch {\type{first}}

Often multiple runs are needed to get a document right. Think of
cross references, object references, tables of contents, indices,
etc. You can use this mode to determine if the first run is taking
place. For instance, when you do real time graphic conversions, it
makes sense to do that only once.

\stopswitch

\startswitch {\type{last}}

This mode is set if the last run in a session is taking place.
Normally this is not known in advance, unless one has asked for an
additional imposition pass.

\stopswitch

\startswitch {\type{background}}

This mode is set when there is a (new) background defined.

\stopswitch

\startswitch {\type{postponing}}

While postponing some content using the postpone mechanism this mode
is enabled.

\stopswitch

\startswitch {\type{grid}}

When you are typesetting on a grid, special care has to be taken not
spoil grid snapping. You can use this mode to test if you are in
grid typesetting mode.

\stopswitch

\startswitch {\type{header}}

This mode is enabled when there is a page header, i.e.\ the header
has non|-|zero dimensions.

\stopswitch

\startswitch {\type{footer}}

This mode is enabled when there is a page footer, i.e.\ the header
has non|-|zero dimensions.

\stopswitch

\startswitch {\type{makeup}}

The makeup mechanisms are used to build single pages like title
pages. This mode is set during construction.

\stopswitch

\startswitch {\type{pdf}, \type{dvi}}

One of these modes is set, which one depends on the output driver
that is loaded.

\stopswitch

\startswitch {\type{*language-id}, \type{language-id}}

When a language is chosen, its id is set as mode. For example, when the main
language is English, and the current language Dutch, we can test for the
modes \type {**en} and \type {*nl} (watch the extra \type {*}).

\stopswitch

\startswitch {\type{marking}}

This flag is set when a marking (e.g.\ in a header or footer) is
being typeset (processed).

\stopswitch

\section {Regimes}

When you key in an english document, a normal \kap {QWERTY}
keyboard combined with the standard \ASCII\ character set
will do. However, in many countries dedicated keyboards and
corresponding input encodings are used. This means that
certain keystrokes correspond to non||standard \ASCII\
characters and these need to be mapped onto the characters
present in the font. Unless the input encoding matches the
output (font) encoding, intermediate steps are needed to
take care of the right mapping. For instance, input
code~145 can become command \type {\eacute} which can
result in character~123 of a certain font.

Although all kind of intermediate, direct or indirect,
mappings are possible, in \CONTEXT\ the preferred method is
to go by named glyphs. The advantage of this method is that
we can rather comfortably convert the input stream into
different output streams as needed for typesetting text (the
normal \TEX\ process) and embedding information in the file
(like annotations or font vectors needed for searching
documents).

The conversion from input characters into named glyphs is
handled by regimes. While further mapping is done
automatically and is triggered by internal processes,
regimes need to be chosen explicitly. This is because only
the user knows what he has input.

Most encodings (like \type {il2}) have an associated regime. You
can get some insight in what a regime involves by showing it:

\starttyping
\showregime[il2]
\stoptyping

In addition there are a couple of platform dependent ones:

\starttabulate[|lT|lp|]
\HL
\NC \bf regime \NC \bf platform \NC \NR
\HL
\NC ibm \NC the old standard \MSDOS\ code page       \NC \NR
\NC win \NC the western europe \MSWINDOWS\ code page \NC \NR
\HL
\stoptabulate

If you want to know what regimes are available, you can take a
look at the \type {regi-*.tex} files. A regime that becomes more
and more popular is the utf-8 regime. If you want some insight in
what vectors provide, you can use commands like:

\starttyping
\showunicodevector[001]
\stoptyping

and

\starttyping
\showunicodetable[001]
\stoptyping

where the last one produces a rather large table.


\stopcomponent

