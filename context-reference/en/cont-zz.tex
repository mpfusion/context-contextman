\toonkader

\stelkorpsin
  [lbr]

\switchnaarkorps
  [klein]

\stelwitruimtein
  [groot]

\stelkleurenin
  [status=start]

\startbuffer[voorb-1]
\stellayoutin 
  [kopwit=3cm,
   hoogte=24cm,
   hoofd=1.5cm,
   voet=1cm,
   bovenafstand=6pt,
   hoofdafstand=12pt,
   voetafstand=12pt,
   onderafstand=6pt,
   rugwit=4.5cm,
   breedte=10cm,
   linkermarge=0.75cm,
   rechtermarge=1.5cm,
   linkerrand=2cm,
   rechterrand=1.75cm,
   linkermargeafstand=18pt,
   rechterrandafstand=6pt,
   linkerrandafstand=12pt,
   rechterrandafstand=12pt]
\stopbuffer

\startbuffer[voorb-2]
\stelinteractiein
  [status=start,
   menu=aan]
\stopbuffer

\startbuffer[voorb-3]
\definieerinteractiemenu
  [onder]
  [{alfa[menu:a]},
   {beta[menu:b]},  
   {gamma[menu:c]},
   {delta[menu:d]}]

\definieerinteractiemenu
  [boven]
  [{alfa[menu:a]},
   {beta[menu:b]},
   {gamma[menu:c]},
   {delta[menu:d]}]

\definieerinteractiemenu
  [links]
  [{alfa[menu:a]},
   {beta[menu:b]},
   {gamma[menu:c]},
   {delta[menu:d]}]

\definieerinteractiemenu
  [rechts]
  [{alfa[menu:a]},
   {beta[menu:b]},
   {gamma[menu:c]},
   {delta[menu:d]}]
\stopbuffer

\startbuffer[voorb-4]
\stelinteractiemenuin 
  [onder,boven,links,rechts]
  [status=start,
   letter=klein,
   breedte=1.5cm]
\stopbuffer

\startbuffer[voorb-5]
\stelinteractiemenuin 
  [onder,boven,links,rechts]
  [achtergrond=raster,
   kader=uit]
\stopbuffer

\startbuffer[voorb-6]
\definieerinteractiemenu
  [extra]
  [rechts]
  [status=start]
  
\stelinteractiemenuin
  [extra]
  [{beta[menu:b]},
   {delta[menu:d]}]
\stopbuffer

\haalbuffer[voorb-1]
\haalbuffer[voorb-2]
\haalbuffer[voorb-3]
\haalbuffer[voorb-4]

\starttekst

\paginareferentie[menu:a]
\paginareferentie[menu:b]

De vier lokaties \type{links}, \type{rechts}, \type{onder}
en \type{boven} worden hier allemaal getoond. We zien dat de
menu's buiten de zetspiegel vallen. 

De hier getoonde zetspiegel is verre van fraai, maar toont
de positionering van de verschillende menu's in de randen en
boven en onderregels. 

\typebuffer[voorb-1]

We besparen de gebruiker de twee truukjes die nodig zijn om
dit binnen de bestaande layout te realiseren. 

De keus voor de plaats is subjectief. Links leent zich minder
voor rechtshandigen en andersom. Omdat we van links naar 
rechts lezen kan een menu links afleiden. Boven suggereert 
dat men over het 'papier' moet reiken. Bovendien staat daar 
als het tegenzit de interface van het programma. Blijven dus 
over rechts en onder.  

Standaard zijn teksten niet interactief, vandaar het
onderstaande comman\-do. We kunnen het plaatsen van menu's
uitzetten. Dit scheelt aanzienlijk in de verwerkingstijd. 

\typebuffer[voorb-2]

De menu's verschillen per bladzijde. Dat is een gevolg van 
het feit dat op deze drie bladzijden enkele verwijzingen 
worden gegenereerd. Op deze bladzijde is dat bijvoorbeeld:

\starttypen
\paginareferentie[menu:a]
\paginareferentie[menu:b]
\stoptypen

\pagina

\paginareferentie[menu:c]

\haalbuffer[voorb-5]

Omdat het vrij zinloos is naar een bladzijde te springen
als je hem toch al ziet, wordt het menu automatisch
aangepast. Menu||item delta wordt dan ook niet getoond. Omdat 
we een wat andere layout gebruiken, zien we wel een vlakje. 
Op de vorige bladzijde bleef het kader bij het ontbrekend 
item \type{[menu:c]} (gamma) echter achterwege. 

\typebuffer[voorb-5]

Menu's worden in de rand geplaatst. Het is technisch
mogelijk een menu in de marge, het hoofd of de voet te
plaatsen; het ligt echter meer voor de hand de breedte van
de marge en/of de hoogte van het hoofd en/of de voet gelijk
te maken aan \type{0cm}. De afstand tussen de rand en de
marge komen automatisch te vervallen. Hetzelfde geldt voor
de afstand tussen het hoofd en boven, en de voet en onder. 

De vier menu's zijn op gelijke wijze gedefinieerd. Let op het
gebruik van de verschillende haakjes. 

\typebuffer[voorb-3]

\pagina

\paginareferentie[menu:d]

De gebruiker wordt aangeraden eens wat te experimenteren met
de instellingen \type{voor}, \type{tussen} en \type{na}. Aan
deze variabelen kunnen \kap{\TeX}||commando's worden
toegekend, zoals \type{\hfill} (bij onder en boven) en
\type{\vfill} (bij links en rechts). 

Een menu wordt pas getoond als \type{status=start}.
Standaard is de breedte van de menu||items afgestemd op die
van de randen. Andere waarden zijn echter mogelijk. 

\typebuffer[voorb-4]

\haalbuffer[voorb-6]

Tot slot tonen we in dit voorbeeld dat we meerdere menu's
naast elkaar kunnen plaatsen. De parameter \type{afstand}
kan worden gebruikt om de afstand tussen de menu's in te
stellen. Standaard wordt de afstand tussen de marge en de
rand gebruikt. 

\typebuffer[voorb-6]

Op vergelijkbare wijze kunnen menu's boven elkaar worden
geplaatst. Wanneer we een menu willen onderdrukken,
gebruiken we \type{status=geen}. 

\stoptekst
