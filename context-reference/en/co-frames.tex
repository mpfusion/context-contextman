\startcomponent co-en-12

\environment contextref-env
\product contextref

\chapter[lines,frames]{Lines and frames}

\section{Introduction}

\TEX\ has an enormous capacity in handling text, but is very
weak at handling graphical information. Lines can be handled
adequately as long as you use vertical or horizontal lines.
However, you can do graphical work with \TEX\ by combining
\TEX\ and \METAPOST.

In this chapter we introduce a number of commands that
relate to drawing straight lines in your text. We will see a
very sophisticated command \type {\framed} that can be used
in many ways. The parameters of this command are also
available in other commands.

\section[single lines]{Single lines}
\index{lines}
\macro{\tex{hairline}}
\macro{\tex{thinrule}}
\macro{\tex{thinrules}}
\macro{\tex{setupthinrules}}
\macro{\tex{vl}}
\macro{\tex{hl}}

The simplest way to draw a line in \CONTEXT\ is:

\showsetup{hairline}

For example:

\startbuffer
\hairline
In what fairy tale is the wolf cut open and filled with stones? Was it in
{Little Red Riding-hood} or in \quote {The wolf and the seven goats}.
\hairline
\stopbuffer

\startexample
\typebuffer
\stopexample

This will become:

\startreality
\getbuffer
\stopreality

It does not look good at all. This is caused by the fact
that a drawn line gets its own vertical whitespace.
In \in{section}[textlines] we will show how to alter this.

The effects of the command \type{\hairline} is best
illustrated when we visualize \type{\strut}'s. We did so by
saying \type {\showstruts} first.

\page \message{harde paginaovergang ivm toonstrut}

\startreality
{\showstruts\hairline}
A strut is a character with a maximum height and depth, but
no width. The text in this example is surrounded by two
strutted lines.
{\showstruts\hairline}
\stopreality

It is  also possible to draw a line over the width of the
actual paragraph:

\showsetup{thinrule}

Or more than one lines by:

\showsetup{thinrules}

For example:

\startbuffer
\startitemize
\item question 1 \par \thinrules[n=2]
\item question 2 \par \thinrules[n=2]
\stopitemize
\stopbuffer

\startexample
\typebuffer
\stopexample

If you leave out a \type {\par} (or empty line), the thin
rules come after the text. Compare

\getbuffer

with

\startbuffer
\startitemize
\item question 1 \thinrules[n=2]
\item question 2 \thinrules[n=2]
\stopitemize
\stopbuffer

\getbuffer

The last example was keyed in as:

\typebuffer

The parameters are set with:

\showsetup{setupthinrules}

You can draw thin vertical or horizontal lines with the
commands:

\showsetup{vl}

\showsetup{hl}

The argument is optional. To \type{\vl} (\hbox spread .5em
{\hss \vl \hss}) you may pass a factor that relates to the
actual height of a line and to \type{\hl} (\hbox spread .5em
{\hss \hl \hss}) a width that relates to the width of an em.
So \type {\vl[2]} produces a rule with a height of two lines.

\section[fill in rules]{Fill in rules}
\index{lines}
\index{questionnaire}
\macro{\tex{fillinline}}
\macro{\tex{fillinrules}}
\macro{\tex{setupfillinline}}
\macro{\tex{setupfillinrules}}

On behalf of questionnaires there is the command:

\showsetup{fillinline}

With the accompanying setup command:

\showsetup{setupfillinlines}

The example:

\startbuffer
\fillinline[n=2,width=2cm]{name} \par
\fillinline[n=2,width=2cm]{address} \par
\stopbuffer

\startexample
\typebuffer
\stopexample

Leads to the next list:

\startreality
\getbuffer
\stopreality

An alternative is wanting the fill||in rule at the end of a
paragraph. Then you use the commands:

\showsetup{fillinrules}

\showsetup{setupfillinrules}

The next example will show the implications:

\startbuffer
\fillinline[width=3cm] Consumers in this shopping mall are frequently
confronted with questionnaires. Our hypothesis is that consumers rather
shop somewhere else than answer these kind of questionnaires. Do you
agree with this?
\stopbuffer

\startexample
\typebuffer
\stopexample

In this example we could of course have offered some
alternatives for answering this question. By setting the
width to \type {broad}, we get

\startreality
\getbuffer\par
\stopreality

The next set of examples demonstrate how we can influence the
layout.

\startbuffer
\fillinrules[n=2,width=fit]{first}
\fillinrules[n=2,width=broad]{first}
\fillinrules[n=2,width=3cm]{first}
\fillinrules[n=2,width=fit,distance=.5em,separator=:]{first}
\fillinrules[n=2,width=broad,distance=.5em]{first}{last}
\stopbuffer

\typebuffer

\getbuffer

\section[textlines]{Text lines}
\macro{\tex{textrule}}
\macro{\tex{setuptextruleen}}

A text line is drawn just before and/or after a paragraph.
The upper line may also contain text. The command is:

\showsetup{textrule}

An example:

\startbuffer
\textrule[top]{Instruments}
Some artists mention the instruments that they use during the production
of their \kap{CD}. In Peter Gabriel's \quote {Digging in the dust} he used
the {\em diembe}, {\em tama} and {\em surdu}. The information on another
song mentions the {\em doudouk}. Other \quote {unknown} instruments are
used on his \kap{cd} \quote {Passion}.
\textrule
\stopbuffer

\startexample
\typebuffer
\stopexample

This will result in:

\getbuffer

The behaviour of textlines is set up with the command below.
With the parameter \type{width} you set the length of the
line in front of the text.

\showsetup{setuptextrules}

These is also a \type {\start}||\type {\stop} alternative.
This one also honors the \type {bodyfont} parameter.

\showsetup{starttextrule}

\section[underline,overline,overstrike]{Underline}
\index{underline}
\index{overstrike}
\macro{\tex{underbar}}
\macro{\tex{underbars}}
\macro{\tex{overstrike}}
\macro{\tex{overstrikes}}

Underlining text is not such an ideal method to banner your
text. Nevertheless we introduced this feature in \CONTEXT.
Here is how it \underbar{works}. We use:

\showsetup{underbar}

\startbuffer
\underbar {A disadvantage of this command is that words can \underbar
{no} longer be hyphenated. This is a nasty side||effect. But we do
support \underbar {nested} underlining.}

\underbars {The spaces in the last paragraph were also underlined. If
we do not want that in this paragraph we use:}
\stopbuffer

\getbuffer

\showsetup{underbars}

From the input we can see that the hyphen results from the
compound word.

\typebuffer

The counterpart of these commands are:

\showsetup{overbar}

\showsetup{overbars}

You may wonder for what reasons we introduced these commands.
The reasons are mainly financial:

\startbuffer
\starttabulate[|l|r|]
\NC product 1 \NC          1.420  \NC \NR
\NC product 2 \NC          3.182  \NC \NR
\NC total     \NC \overbar{4.602} \NC \NR
\stoptabulate
\stopbuffer

\startreality
\getbuffer
\stopreality

This financial overview is made with:

\startexample
\typebuffer
\stopexample

The number of parameters in these commands is limited:

\showsetup{setupunderbar}

The alternatives are:
{\setupunderbar [alternative=a]\underbar {alternative~a},
 \setupunderbar [alternative=b]\underbar {alternative~b},
 \setupunderbar [alternative=c]\underbar {alternative~c}}
while another line thickness results in:
{\setupunderbar [rulethickness=1pt]\underbar {1pt~line},
 \setupunderbar [rulethickness=2pt]\underbar {2pt~line}}.

A part of the text can be \overstrike {striked} with the
command:

\showsetup{overstrike}

This command supports no nesting. Single \overstrikes {words
are striked} with:

\showsetup{overstrikes}

\section[framing,stp:framed]{Framing}
\index{framing}
\index{frames}
\macro{\tex{setupframedin}}
\macro{\tex{framed}}
\macro{\tex{inframed}}

Texts can be framed with the command: \type{\framed}. In its
most simple form the command looks like this:

\startbuffer
\framed{A button in an interactive document is a framed text
with specific characteristics.}
\stopbuffer

\startexample
\typebuffer
\stopexample

The becomes:

\startlinecorrection
\getbuffer
\stoplinecorrection

The complete definition of this command is:

\showsetup{framed}

You may notice that all arguments are optional.

\startbuffer
\framed
  [height=broad]
  {A framed text always needs special attention as far as the spacing
   is concerned.}
\stopbuffer

\startexample
\typebuffer
\stopexample

Here is the output of the previous source code:

\startlinecorrection
\getbuffer
\stoplinecorrection

For the height, the values \type {fit} and \type {broad} have
the same results. So:

\startbuffer
\hbox
  {\framed[height=broad]{Is this the spacing we want?}
   \hskip1em
   \framed[height=fit]  {Or isn't it?}}
\stopbuffer

\typebuffer

will give us:

\startlinecorrection
\getbuffer
\stoplinecorrection

To obtain a comparable layout between framed and non||framed
framing can be set on and off.

\startlinecorrection
\hbox{\framed[width=2cm,frame=on]{yes}
      \framed[width=2cm,frame=off]{no}
      \framed[width=2cm,frame=on]{yes}}
\hbox{\framed[width=2cm,frame=off]{no}
      \framed[width=2cm,frame=on]{yes}
      \framed[width=2cm,frame=off]{no}}
\stoplinecorrection

The rulethickness is set with the command \type
{\setuprulethickness} (see \in{section}[rulethickness]).

A framed text is typeset \quote {on top of} the baseline.
When you want real alignment you can use the command
\type{\inframed}.

\startbuffer
to \framed{frame} or to be \inframed{framed}
\stopbuffer

\startexample
\typebuffer
\stopexample

or:

\startreality
\getbuffer
\stopreality

It is possible to draw parts of the frame. In that case you
have to specify the separate sides of the frame with \type
{leftframe=on} and the alike.

We will now show some alternatives of the command \type
{\framed}. Please notice the influence of \type {offset}.
When no value is given, the offset is determined by the
height and depth of the \type {\strut}, that virtual
character with a maximum height and depth with no width.
When exact positioning is needed within a frame you set
\type {offset} at \type {none} (see also \in {tables}
[tab:strut 1], \in [tab:strut 2] \in {and} [tab:strut 3]).
Setting the \type {offset} to \type {none} or \type
{overlay}, will also disable the strut.

\def\toonframed[#1]%
  {\leavevmode\framed[#1]{\tttf#1}\par}

\startpacked
\toonframed[width=fit]
\toonframed[width=broad]
\toonframed[width=8cm,height=1.5em]
\toonframed[offset=5pt]
\toonframed[offset=0pt]
\toonframed[offset=none]
\toonframed[offset=overlay]
\toonframed[width=8cm,height=1.5em,offset=0pt]
\toonframed[width=8cm,height=1.5em,offset=none]
\stoppacked

The commands \type {\lbox} (ragged left), \type {\cbox}
(ragged center) and \type {\rbox} (ragged right) can be
combined with \type {\framed}:

\startbuffer[examp-1]
\framed
  [width=.2\hsize,height=3cm]
  {\lbox to 2.5cm{\hsize2.5cm left\\of the\\middle}}
\stopbuffer

\startbuffer[examp-2]
\framed
  [width=.2\hsize,height=3cm]
  {\cbox to 2.5cm{\hsize2.5cm just\\in the\\middle}}
\stopbuffer

\startbuffer[examp-3]
\framed
  [width=.2\hsize,height=3cm]
  {\rbox to 2.5cm{\hsize2.5cm right\\of the\\middle}}
\stopbuffer

\startlinecorrection
\startcombination[3]
  {\getbuffer[examp-1]} {\type{\lbox}}
  {\getbuffer[examp-2]} {\type{\cbox}}
  {\getbuffer[examp-3]} {\type{\rbox}}
\stopcombination
\stoplinecorrection

The second text is typed as follows:

\startexample
\typebuffer[examp-2]
\stopexample

There is a more convenient way to align a text, since we
have the parameters \type {align} and \type{top} and
\type{bottom}. In the next one shows the influence of \type
{top} and \type {bottom} (the second case is the default).

\startbuffer
\setupframed[width=.2\hsize,height=3cm,align=middle]
\startcombination[4]
  {\framed[bottom=\vss,top=\vss]{just\\in the\\middle}}
  {\type{top=\vss}\crlf\type{bottom=\vss}}
  {\framed[bottom=\vss,top=]    {just\\in the\\middle}}
  {\type{top=}    \crlf\type{bottom=\vss}}
  {\framed[bottom=,top=\vss]    {just\\in the\\middle}}
  {\type{top=\vss}\crlf\type{top=}}
  {\framed[bottom=,top=]        {just\\in the\\middle}}
  {\type{top=}    \crlf\type{bottom=}}
\stopcombination
\stopbuffer

\typebuffer

\startlinecorrection
\getbuffer
\stoplinecorrection

In the background of a framed text you can place a screen or
a coloured background by setting \type {background} at \type
{color} or \type {screen}. Don't forget to activate the the
colour mechanism by saying (\type
{\setupcolors[state=start]}).

\start
\setupframed[width=5cm,height=1cm]
\startlinecorrection
\startcombination[2*2]
  {\framed
    [background=screen]
    {\tfb In the}}
  {\tttf background=screen}
  {\framed
    [background=screen,backgroundscreen=0.7]
    {\tfb dark}}
  {\tttf background=screen \endgraf backgroundscreen=0.7}
  {\framed
    [background=color]
    {\tfb all cats}}
  {\tttf background=color}
  {\framed
    [background=color,backgroundcolor=red]
    {\tfb are grey.}}
  {\tttf background=color \endgraf backgroundcolor=red}
\stopcombination
\stoplinecorrection
\stop

There is also an option to enlarge a frame or the background
by setting the \type {frameoffset} and|/|or \type
{backgroundoffset}. These do not influence the dimensions.
Next to screens and colours you can also use your own kind
of backgrounds. This mechanism is described in \in {section}
[overlays].

The command \type{\framed} itself can be an argument
of \type{\framed}. We will obtain a framed frame.

\startbuffer[examp-1]
\framed
  [width=3cm,height=3cm]
  {\framed[width=2.5cm,height=2.5cm]{hello world}}
\stopbuffer

\startexample
\typebuffer[examp-1]
\stopexample

In that case the second frame is somewhat larger than
expected. This is caused by the fact that the first framed
has a strut. This strut is placed automatically to enable
typesetting one framed text next to another. We suppress
\type{\strut} with:

\startbuffer[examp-2]
\framed
  [width=3cm,height=3cm,strut=no]
  {\framed[width=2.5cm,height=2.5cm]{hello world}}
\stopbuffer

\startexample
\typebuffer[examp-2]
\stopexample

When both examples are placed close to one another we see the
difference:

\startlinecorrection
\startcombination
  {\showstruts\getbuffer[examp-1]} {\type{strut=yes}}
  {\showstruts\getbuffer[examp-2]} {\type{strut=no}}
\stopcombination
\stoplinecorrection

A \type {\hairline} is normally draw over the complete width
of a text (\type{\hsize}). Within a frame the line is drawn
from the left to the right of framed box.

Consequently the code:

\startbuffer
\framed[width=8cm,align=middle]
  {when you read between the lines \hairline
   you may see what effort it takes \hairline
   to write a macropackage}
\stopbuffer

\typebuffer

produces the following output:

\startlinecorrection
\getbuffer
\stoplinecorrection

When no width is specified only the vertical lines are
displayed.

\startbuffer
\framed
  {their opinions \hairline differ \hairline considerately}
\stopbuffer

\startlinecorrection
\getbuffer
\stoplinecorrection

Which was obtained with:

\startexample
\typebuffer
\stopexample

The default setup of \type{\framed} can be changed with
the command:

\showsetup{setupframed}

The command \type{\framed} is used within many other
commands. The combined use of \type{offset} and \type{strut}
may be very confusing. It really pays off to spend some time
playing with these macros and parameters, since you will
meet \type {\framed} in many other commands. Also, the
parameters \type{width} and \type{height} are very important
for the framing texts. For that reason we summarize the
consequences of their settings in \in {table} [tab:strut 1],
\in [tab:strut 2] \in {and} [tab:strut 3].

\startbuffer[table]
\starttable[|c|c|c|c|c|c|]
\HL
\VL
\VL                             \VL \FOUR{\tt offset} \VL\SR
\DC                 \DC         \DL[4]                   \DR
\VL                 \VL         \VL \tt .25ex
                                \VL \tt 0pt
                                \VL \tt none
                                \VL \tt overlay       \VL\SR
\HL
\VL \LOW{\tt strut} \VL \tt yes \VL \o[yes,.25ex]
                                \VL \o[yes,0pt]
                                \VL \o[yes,none]
                                \VL \o[yes,overlay]    \VL\SR
\DC                 \DL[1]                       \DC\DC\DC\DR
\VL                 \VL \tt no  \VL \o[no,.25ex]
                                \VL \o[no,0pt]
                                \VL \o[no,none]
                                \VL \o[no,overlay]     \VL\SR
\HL
\stoptable
\stopbuffer

\placetable
  [here][tab:strut 1]
  {The influence of \type{strut} and \type{offset} in
   \type{\framed} (1).}
  {\def\o[#1,#2]{\framed[strut=#1,offset=#2]{}}
   \getbuffer[table]}

\placetable
  [here][tab:strut 2]
  {The influence of \type{strut} and \type{offset} in
   \type{\framed} (2).}
  {\def\o[#1,#2]{\framed[strut=#1,offset=#2]{\TeX}}
   \getbuffer[table]}

\startbuffer[table]
\starttable[|c|c|c|c|]
\HL
\VL
\VL                                     \VL
    \TWO{\tt width}                     \VL\SR
\DC \DC \DL[2]                             \DR
\VL                                     \VL
                                        \VL
    \tt fit                             \VL
    \tt broad (\string\hsize=4cm)       \VL\SR
\HL
\VL \LOW{\tt height}                    \VL
    \tt fit                             \VL
    \o[fit,fit]                         \VL
    \hsize=4cm \o[fit,broad]            \VL\SR
\DC \DL[1]                              \DC\DR
\VL                                     \VL
    \tt broad                           \VL
    \o[broad,fit]                       \VL
    \hsize=4cm \o[broad,broad]          \VL\SR
\HL
\stoptable
\stopbuffer

\placetable
  [here][tab:strut 3]
  {The influence of \type{height} and \type{width} in
   \type{\framed}.}
  {\def\o[#1,#2]{\framed[height=#1,width=#2]{xxxx}}
   \getbuffer[table]}

\startbuffer
\placefigure
  [left]
  {none}
  {\framed[align=middle]{happy\\birthday\\to you}}
\stopbuffer

\getbuffer

At first sight it is not so obvious that \type {\framed} can
determine the width of a paragraph by itself. When we set
the parameter \type {align} the paragraph is first typeset
and then framed. This feature valuable when typesetting
titlepages. In the example left of this text, linebreaks are
forced by \type {\\}, but this is not mandatory. This
example was coded as follows:

\startexample
\typebuffer
\stopexample

The parameter \type {offset} needs some special attention.
By default it is set at \type {.25ex}, based on the
cureently selected font. The next examples will illustrate
this:

\startbuffer
\hbox{\bf \framed{test} \sl \framed{test} \tfa \framed{test}}
\hbox{\framed{\bf test} \framed{\sl test} \framed{\tfa test}}
\stopbuffer

\startexample
\typebuffer
\stopexample

The value of \type{1ex} outside \type{\framed} determines
the offset. This suits our purpose well.

\startlinecorrection
\getbuffer
\stoplinecorrection

The differences are very subtle. The distance between the
framed boxes depends on the actual font size,
the dimensions of the frame, the offset, and the strut.

\TEX\ can only draw straight lines. Curves are drawn with
small line pieces and effects the size of \DVI||files
considerately and will cause long processing times. Curves in
\CONTEXT\ are implemented by means of \POSTSCRIPT. There are
two parameters that affect curves: \type{corner} and
\type{radius}. When \type{corner} is set at \type{round},
round curves are drawn.

\startlinecorrection
\framed[corner=round]{Don't be to edgy.}
\stoplinecorrection

It is also possible to draw circles by setting \type{radius}
at half the width or height. But do not use this command for
drawing, it is meant for framing text. Use \METAPOST\
instead.

Technically speaking the background, the frame and the text
are separate components of a framed text. First the
background is set, then the text and at the last instance the
frame. The curved corner of a frame belongs to the frame and
is not influenced by the text. As long as the radius is
smaller than the offset no problems will occur.

\section[framed texts]{Framed texts}
\index{framing}
\index{frames}
\macro{\tex{defineframedtext}}
\macro{\tex{setupframedtexts}}
\macro{\tex{startframedtext}}

When you feel the urge to put a frame around or a backgroud
behind a paragraph there is the command:

\showsetup{startframedtext}

An application may look like this:

\startbuffer
\startframedtext[left]
From an experiment that was conducted by C. van Noort (1993) it was
shown that the use of intermezzos as an attention enhancer is not very
effective.
\stopframedtext
\stopbuffer

\startexample
\typebuffer
\stopexample

\getbuffer % geen zinnetje ervoor

This can be set up with:

\showsetup{setupframedtexts}

Framed texts can be combined with the place block mechanism,
as can be seen in \in {intermezzo} [int:demo 1].

\startbuffer
\placeintermezzo
  [here][int:demo 1]
  {An example of an intermezzo.}
  \startframedtext
    For millions of years mankind lived just like animals. Then
    something happened, which unleashed the power of our imagination.
    We learned to talk.
    \blank
    \rightaligned{--- The Division Bell / Pink Floyd}
  \stopframedtext
\stopbuffer

\startexample
\typebuffer
\stopexample

In this case the location of the framed text (between
\setchars) is left out.

\getbuffer

You can also draw a partial frame. The following setup
produces \in {intermezzo} [int:demo 2].

\startexample
\starttyping
\setupframedtexts[frame=off,topframe=on,leftframe=on]
\stoptyping
\stopexample

\start
\setupframedtexts[frame=off,topframe=on,leftframe=on]
\placeintermezzo
  [here][int:demo 2]
  {An example of an intermezzo.}
\startframedtext
Why are the world leaders not moved by songs like {\em Wozu sind Kriege
da?} by Udo Lindenberg. I was, and now I wonder why wars go on and on.
\stopframedtext
\stop

You can also use a background. When the background is active
it looks better to omit the frame.

\start
\setupframedtexts[frame=off,background=screen]
\placeintermezzo
  [here][]
  {An example of an intermezzo with background.}
\startframedtext
An intermezzo like this will draw more attention, but the readability
is far from optimal. However, you read can it. This inermezzo was set
up with :

\starttyping
\setupframedtexts[frame=off,background=screen]
\stoptyping
\stopframedtext
\stop

\in {Intermezzo} [int:color] demonstrate how to use some
color:

\startbuffer
\setupframedtexts
  [background=screen,
   frame=off,
   rightframe=on,
   framecolor=darkgreen,
   rulethickness=3pt]

\placeintermezzo
  [here][int:color]
  {An example of an intermezzo with a trick.}
  \startframedtext
    The trick is really very simple. But the fun is gone when Tom, Dick
    and Harry would use it too.
  \stopframedtext
\stopbuffer

\typebuffer

\getbuffer

So, in order to get a partial frame, we have to set the whole
\type {frame} to \type {off}. This is an example of a
situation where we can get a bit more readable source when
we say:

\startbuffer
\startbuffer
\startframedtext ... \stopframedtext
\stopbuffer

\placeintermezzo
  [here][int:color]
  {An example of an intermezzo with a trick.}{\getbuffer}
\stopbuffer

\typebuffer

You do not want to set up a framed text every time you need
it, so there is the following command:

\showsetup{defineframedtext}

\startbuffer
\defineframedtext
  [musicfragment]
  [frame=off, rightframe=on, leftframe=on]

\placeintermezzo
  [here][]
  {An example of a predefined framed text.}
\startmusicfragment
Imagine that there are fragments of music in your interactive document.
You will not be able to read undisturbed.
\stopmusicfragment
\stopbuffer

The definition:

\startexample
\typebuffer
\stopexample

results in:

\getbuffer

\section[margin rules]{Margin rules}
\index{margin+lines}
\macro{\tex{startmarginrule}}
\macro{\tex{marginrule}}
\macro{\tex{setupmarginrule}}

To add some sort of flags to paragraphs you can draw
vertical lines in the margin. This can be used to indicate
that the paragraph was altered since the last version. The
commands are:

\showsetup{startmarginrule}

\showsetup{marginrule}

The first command is used around paragraphs, the second
within a paragraph.

By specifying a level you can suppress a margin rule. This
is done by setting the \quote {global} level higher than the
\quote {local} level.

\showsetup{setupmarginrules}

In the example below we show an application of the use of
margin rules.

\startbuffer
\startmarginrule
The sound of a duck is a good demonstration of how different
people listen to a sound. Everywhere in Europe the sound is
equal. But in every country it is described differently:
kwaak||kwaak (Netherlands), couin||couin (French),
gick||gack (German), rap||rap (Danish) and mech||mech
(Spanish). If you speak these words aloud you will notice
that
\startmarginrule[4] in spite of the \stopmarginrule
consonants the sound is really very well described. And what
about a cow, does it say boe, mboe or mmmmmm?
\stopmarginrule
\stopbuffer

\startexample
\typebuffer
\stopexample

Or:\footnote{G.C. Molewijk, Spellingsverandering van zin
naar onzin (1992).}

\getbuffer

If we would have set \type {\setupmarginrules[level=2]} we
would have obtained a margin rule in the middle of the
paragraph. In this example we also see that the thickness of
the line is adapted to the level. You can undo this feature
with \type{\setupmarginrules[thickness=1]}.

\section[black rules]{Black rules}
\index{black rules}
\macro{\tex{blackrule}}
\macro{\tex{blackrules}}
\macro{\tex{setupblackrules}}

Little black boxes |<|we call them black rules|>| (\blackrule)
can be drawn by \type{\blackrule}:

\showsetup{blackrule}

When the setup is left out, the default setup is used.

\showsetup{setupblackrules}

The height, depth and width of a black rule are in
accordance with the usual height, depth and width of \TEX.
When we use the key \type {max} instead of a real value the
dimensions of \TEX's \type{\strutbox} are used. When we set
all three dimensions to \type {max} we get: \blackrule
[width=max, height=max, depth=max].

\inleft{\blackrule}Black rules may have different purposes.
You can use them as identifiers of sections or subsections.
This paragraph is tagged by a black rule with default
dimensions: \type{\inleft{\blackrule}}.

A series of black rules can be typeset by \type{\blackrules}:

\showsetup{blackrules}

\inleft{\blackrules}There are two versions. Version
\type{a} sets \type{n} black rules next to each other with an
equal specified width. Version~\type{b} divides the
specified width over the number of rules. This paragraph is
tagged with \type {\inleft{\blackrules}}. The setup after
\type {\blackrule} and \type {\blackrules} are optional.

\section[grids]{Grids}
\index{grids}
\index{squares}
\macro{\tex{grid}}

We can make squared paper (a sort of grid) with the
command:

\showsetup{grid}

The default setup produces:

\startlinecorrection
\grid
\stoplinecorrection

It is used in the background when defining interactive areas
in a figure. And for the sake of completeness it is
described in this chapter.

\stopcomponent

