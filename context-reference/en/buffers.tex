
% These buffers should move to their chapter includes

\MPinclusions
  {color gray;
   gray := (.95,.95,.95);}

\def\MPcircle#1#2#3#4#5%
  {\startuseMPgraphic{circle}
     path p;
     p := fullcircle xscaled #1\space yscaled #2;
     pickup pencircle scaled #3;
     fill p withcolor \MPcolor{#5};
     draw p withcolor \MPcolor{#4};
   \stopuseMPgraphic
   \useMPgraphic{circle}}

\defineoverlay
  [MPcircleA]
  [\MPcircle{\overlaywidth}{\overlayheight}{2pt}{FrameColor}{GrayColor}]

\startbuffer[fig-cont]

\unprotect

\setupframed[\c!background=MPcircleA,\c!frame=\v!off,\c!offset=\v!none]

\framed
  [\c!width=6cm,\c!height=5cm]
  {\vfil \ConTeXt \vfil
   \leavevmode\framed[\c!width=.7\hsize,\c!height=.5\hsize]
     {\vfil plain\ \TeX \vfil
      \leavevmode\framed[\c!width=.7\hsize,\c!height=.3\hsize]{\TeX}
      \vfil\vfil}
   \vfil\vfil}

\protect

\stopbuffer

\startbuffer[kerning]

\startlinecorrection
\vbox
  {\forgetall
   \vfill
   \definefont[test][ComputerModern at 48pt]\test
   \hbox{box}
   \vskip-36pt
   \def\\#1%
     {\toprulefalse
      \bottomrulefalse
      \ruledhbox{\vrule width 0pt height 72pt#1}}
   \hbox{\\b\\o\\x}}
\stoplinecorrection

\stopbuffer

\startbuffer[ok-1]

\startlinecorrection
\ruledhbox to \hsize
  {\hss
   \definefont[test][LucidaBright             at 48pt]\test ok\setstrut\strut\hss
   \definefont[test][LucidaSans               at 48pt]\test ok\hss
   \definefont[test][LucidaSans-Typewriter    at 48pt]\test ok\hss
   \definefont[test][LucidaHandwriting-Italic at 48pt]\test ok\hss
   \definefont[test][LucidaCalligraphy-Italic at 48pt]\test ok\hss}
\stoplinecorrection

\stopbuffer

\startbuffer[ok-2]

\startlinecorrection
\ruledhbox to \hsize
  {\hss
   \definefont[test][cmr5  at 48pt]\test ok\setstrut\strut\hss
   \definefont[test][cmr7  at 48pt]\test ok\hss
   \definefont[test][cmr9  at 48pt]\test ok\hss
   \definefont[test][cmr12 at 48pt]\test ok\hss
   \definefont[test][cmr17 at 48pt]\test ok\hss}
\stoplinecorrection

\stopbuffer

\startbuffer[styles]

\vskip1ex
\startcombination[3*3]
  {\rmtfc\setupinterlinespace   Serif\end}  {}
  {\sstfc\setupinterlinespace    Sans\end}   {}
  {\tttfc\setupinterlinespace    Mono\end}   {}
  {\rmtfc\setupinterlinespace Regular\end}{}
  {\sstfc\setupinterlinespace Support\end}{}
  {\tttfc\setupinterlinespace    Mono\end}   {}
  {\rmtfc\setupinterlinespace   Roman\end}  {}
  {\sstfc\setupinterlinespace    Sans\end}   {}
  {\tttfc\setupinterlinespace    Type\end}   {}
\stopcombination
\vskip1ex

\stopbuffer

\startbuffer[em-ex-1]

\def\jump{\vl\hskip1em\vl}%
\def\mmmm{\vl M\vl}%
\def\dash{\vl---\vl}%
\def\numb{\vl12\vl}%

\starttable[|c|c|c|c|c|c|]
\HL
\VL \type{\tf} \VL \type{\bf} \VL \type{\sl}  \VL
    \type{\tt} \VL \type{\ss} \VL \type{\tfx} \VL\SR
\HL
\VL \tf\numb   \VL \bf\numb   \VL \sl \numb   \VL
    \tt\numb   \VL \ss\numb   \VL \tfx\numb   \VL\FR
\VL \tf\mmmm   \VL \bf\mmmm   \VL \sl \mmmm   \VL
    \tt\mmmm   \VL \ss\mmmm   \VL \tfx\mmmm   \VL\MR
\VL \tf\jump   \VL \bf\jump   \VL \sl \jump   \VL
    \tt\jump   \VL \ss\jump   \VL \tfx\jump   \VL\MR
\VL \tf\dash   \VL \bf\dash   \VL \sl \dash   \VL
    \tt\dash   \VL \ss\dash   \VL \tfx\dash   \VL\LR
\HL
\stoptable

\stopbuffer

\startbuffer[em-ex-2]

\def\show
  {\hbox
    {\forgetall
     \offinterlineskip
     \vbox
       {\hsize1em\hl[1]\endgraf\vskip1ex\hl[1]}%
     \hskip.25em
     \vbox
       {\hsize.5em \vskip\linewidth x\vskip\linewidth}}}%

\starttable[|c|c|c|c|c|c|]
\HL
\VL \type{\tf} \VL \type{\bf} \VL \type{\sl}  \VL
    \type{\tt} \VL \type{\ss} \VL \type{\tfx} \VL\SR
\HL
\VL \tf\show   \VL \bf\show   \VL \sl\show    \VL
    \tt\show   \VL \ss\show   \VL \tfx\show   \VL\SR
\HL
\stoptable

\stopbuffer

\startbuffer[font-1]
\definefontsynonym [Sans]            [Helvetica]
\definefontsynonym [SansBold]        [Helvetica-Bold]
\definefontsynonym [SansItalic]      [Helvetica-Oblique]
\definefontsynonym [SansSlanted]     [Helvetica-Oblique]
\definefontsynonym [SansBoldItalic]  [Helvetica-BoldOblique]
\definefontsynonym [SansBoldSlanted] [Helvetica-BoldOblique]
\definefontsynonym [SansCaps]        [Helvetica]

\definebodyfont [14.4pt,12pt,11pt,10pt,9pt,8pt,7pt,6pt,5pt] [ss] [default]
\stopbuffer

\startbuffer[font-2]
\definefontsynonym [Helvetica-Bold] [hvb] [encoding=texnansi]
\stopbuffer

\startbuffer[font-3]
\definefontsynonym [Helvetica-Bold] [phvb] [encoding=ec]
\stopbuffer

\startbuffer[font-4]
\definefontsynonym [Regular] [Serif]
\definefontsynonym [Roman]   [Serif]
\stopbuffer

\startbuffer[font-5]
\definebodyfont [default] [rm]
  [ tf=Serif        sa 1,
   tfa=Serif        sa a,
      ...
    sl=SerifSlanted sa 1,
   sla=SerifSlanted sa a,
      ...]
\stopbuffer

\startbuffer[font-6]
\definebodyfont [12pt] [rm]
  [ tf=cmr12,
   tfa=cmr12 scaled \magstep1,
   tfb=cmr12 scaled \magstep2,
   tfc=cmr12 scaled \magstep3,
   tfd=cmr12 scaled \magstep4,
    bf=cmbx12,
    it=cmti12,
    sl=cmsl12,
    bi=cmbxti10 at 12pt,
    bs=cmbxsl10 at 12pt,
    sc=cmcsc10 at 12pt]
\stopbuffer

\startbuffer[font-7]
\definebodyfont [12pt,11pt,10pt,9pt,8pt] [ss]
  [tf=hv  sa 1.000,
   bf=hvb sa 1.000,
   it=hvo sa 1.000,
   sl=hvo sa 1.000,
  tfa=hv  sa 1.200,
  tfb=hv  sa 1.440,
  tfc=hv  sa 1.728,
  tfd=hv  sa 2.074,
   sc=hv  sa 1.000]
\stopbuffer

\startbuffer[font-8]
\definebodyfont [12pt,11pt,10pt,9pt,8pt] [ss]
  [tf=hv sa 1, tfa=hv sa a, tfb=hv sa b, tfc=hv sa c, tfd=hv sa d]
\stopbuffer

\startbuffer[enco-1]
\startmapping[texnansi]
  \definecasemap 228 228 196  \definecasemap 196 228 196
  \definecasemap 235 235 203  \definecasemap 203 235 203
  \definecasemap 239 239 207  \definecasemap 207 239 207
  \definecasemap 246 246 214  \definecasemap 214 246 214
  \definecasemap 252 252 220  \definecasemap 220 252 220
  \definecasemap 255 255 159  \definecasemap 159 255 159
\stopmapping
\stopbuffer

\startbuffer[enco-2]
\startencoding[texnansi]
  \defineaccent " a 228
  \defineaccent " e 235
  \defineaccent " i 239
  \defineaccent " o 246
  \defineaccent " u 252
  \defineaccent " y 255
\stopencoding
\stopbuffer

\startbuffer[enco-3]
\startencoding[texnansi]
  \definecharacter ae 230
  \definecharacter oe 156
  \definecharacter o  248
  \definecharacter AE 198
\stopencoding
\stopbuffer

\startbuffer[font-10]
\definefontsynonym [twelvepoint] [12pt]
\definefontsynonym [xii]         [12pt]
\stopbuffer

\startbuffer[font-11]
\definefontstyle [rm,roman,serif,regular]    [rm]
\definefontstyle [ss,sansserif,sans,support] [ss]
\definefontstyle [tt,teletype,type,mono]     [tt]
\definefontstyle [hw,handwritten]            [hw]
\definefontstyle [cg,calligraphic]           [cg]
\stopbuffer

\startbuffer[font-12]
\definestyle [normal]                  [\tf]  []
\definestyle [bold]                    [\bf]  []
\definestyle [type]                    [\tt]  []
\definestyle [italic]                  [\it]  []
\definestyle [slanted]                 [\sl]  []
\definestyle [bolditalic,italicbold]   [\bs]  []
\definestyle [boldslanted,slantedbold] [\bs]  []
\definestyle [small,smallnormal]       [\tfx] []
\stopbuffer

\startbuffer[font-13]
\definefontsynonym [OldStyle] [MathItalic]
\stopbuffer

\startbuffer[math-1]
$\tf x^2+\bf x^2+\sl x^2+\it x^2+\bs x^2+ \bi x^2 =\rm 6x^2$
$\tf x^2+\bf x^2+\sl x^2+\it x^2+\bs x^2+ \bi x^2 =\tf 6x^2$
$\tf x^2+\bf x^2+\sl x^2+\it x^2+\bs x^2+ \bi x^2 =\bf 6x^2$
$\tf x^2+\bf x^2+\sl x^2+\it x^2+\bs x^2+ \bi x^2 =\sl 6x^2$
\stopbuffer

\startbuffer[math-2]
$\tf\mf x^2 + x^2 + x^2 + x^2 + x^2 + x^2 = 6x^2$
$\bf\mf x^2 + x^2 + x^2 + x^2 + x^2 + x^2 = 6x^2$
$\sl\mf x^2 + x^2 + x^2 + x^2 + x^2 + x^2 = 6x^2$
$\bs\mf x^2 + x^2 + x^2 + x^2 + x^2 + x^2 = 6x^2$
$\it\mf x^2 + x^2 + x^2 + x^2 + x^2 + x^2 = 6x^2$
$\bi\mf x^2 + x^2 + x^2 + x^2 + x^2 + x^2 = 6x^2$
\stopbuffer

\startbuffer[math-3]
$\bf x^2 + x^2 + x^2 + x^2 + x^2 + x^2 = \mf 6x^2$
\stopbuffer

\startbuffer[math-4]
$\bf x^2 + \hbox{whatever} + \sin(2x)$
\stopbuffer

\startbuffer[math-5]
\definebodyfont [12pt] [mm]
  [ex=cmex10 at 12pt,
   mi=cmmi12,
   sy=cmsy10 at 12pt]
\stopbuffer

\startbuffer[math-6]
\definebodyfont [10pt,11pt,12pt] [mm]
  [tf=Sans          sa 1,
   bf=SansBold      sa 1,
   sl=SansItalic    sa 1,
   ex=MathExtension sa 1,
   mi=MathItalic    sa 1,
   sy=MathSymbol    sa 1]

\setupbodyfont
\stopbuffer

\startbuffer[fontfil]
\starttable[|Tl|l|]
\HL
\NC font-cmr \NC Computer Modern Roman        \NC\AR
\NC font-csr \NC Computer Slavik Roman (?)    \NC\AR
\NC font-con \NC Concrete Roman               \NC\AR
\NC font-eul \NC Euler                        \NC\AR
\NC font-ams \NC American Mathematics Society \NC\AR
\HL
\NC font-ant \NC Antykwa Torunska             \NC\AR
\HL
\NC font-lbr \NC Lucida Bright                \NC\AR
\HL
\NC font-pos \NC Base PostScript Fonts        \NC\AR
\NC font-ptm \NC Times Roman                  \NC\AR
\NC font-phv \NC Helvetica                    \NC\AR
\NC font-pcr \NC Courier                      \NC\AR
\HL
\NC font-fil \NC Standard Filenames           \NC\AR
\NC font-ber \NC Karl Berry FileNames         \NC\AR
\HL
\stoptable
\stopbuffer

\startbuffer[encofil]
\starttable[|Tl|l|]
\HL
\NC enco-ans \NC TeXnansi                     \NC\AR
\NC enco-ec  \NC European Computer            \NC\AR
\NC enco-il2 \NC ISO Latin 2                  \NC\AR
\NC enco-plr \NC Polish Roman                 \NC\AR
\HL
\NC enco-ibm \NC default IBM PC code page     \NC\AR
\NC enco-win \NC default MS Windows code page \NC\AR
\HL
\stoptable
\stopbuffer

\def\ShowComposed #1
  {\handletokens#1\with\type\VL#1\VL\hyphenatedword{#1}}

\startbuffer[hyph-1]
\starttable[|l|l|l|]
\HL
\VL \bf input \VL \bf normal \VL \bf hyphenated \VL\SR
\HL
\VL \ShowComposed intra||word   \VL\FR
\VL \ShowComposed intra|-|word  \VL\MR
\VL \ShowComposed intra|(|word) \VL\MR
\VL \ShowComposed (intra|)|word \VL\MR
\VL \ShowComposed intra|--|word \VL\MR
\VL \ShowComposed intra|~|word  \VL\LR
\HL
\stoptable
\stopbuffer

\startbuffer[messian]
The French composer {\fr Olivier Messiaen} wrote \quote {\fr Quatuor pour
la fin du temps} during the World War II in a concentration camp. This
may well be one of the most moving musical pieces of that period.
\stopbuffer
