\startcomponent ma-cb-fr-processingfiles

\product ma-cb-fr

\enableregime[il1]

%\chapter{Processing steps}
\chapter{Les �tapes de la compilation}

\index[texutil]{\TEXUTIL}
% \index[tuo]{{\tt tuo}--file}
\index[tuo]{{\tt tuo}--fichier}
% \index{input file+processing}
\index{inclusion de fichier+compilation}

% During processing \CONTEXT\ writes information in the file
% \type{myfile.tui}. This information is used in the next
% pass. Part of this information is processed by the program
% \TEXUTIL. Information on registers and lists are written in
% the file \type{myfile.tuo}. The information in this file is
% filtered and used (when necessary) by \CONTEXT.

Au cours de la compilation, \CONTEXT\ �crit des informations dans le
fichier \type{monfichier.tui}. Ces informations sont utilis�es lors de
la �~passe~� suivante. Une partie de cette information est utilis�e par le
programme \TEXUTIL. L'information concernant les index et les listes
sont �crits dans le fichier \type{monfichier.tuo}. L'information dans ce
fichier est filtr�e et utilis�e (si n�cessaire) par \CONTEXT.

\starttyping
texutil --references filename
\stoptyping

% When \CONTEXT\ cannot find a figure, you can generate a
% figure auxilliary file by saying:

Quand \CONTEXT\ ne peut pas trouver une figure, on peut g�n�rer un
fichier auxiliaire de figure en saisissant :

\starttyping
texutil --figures *.*
\stoptyping

% or whatever specification suits.

ou n'importe quelle autre sp�cification qui convient.

% When one wants to convert \EPS\ illustrations to \PDF\ ones,
% there is:

Si l'on veut convertir une illustration \EPS\ en \PDF\, on peut passer
la commande :

\starttyping
texutil --figures --epspage --epspdf
\stoptyping

% One can use \TEXEXEC\ to run \CONTEXT:

On peut utiliser \TEXEXEC\ pour lancer \CONTEXT\ :

\starttyping
texexec filename
\stoptyping

% \CONTEXT\ runs as many times as needed to get the references
% straight. One can also specify specific needs on the command
% line. For instance if pdf output is desired you type:

\CONTEXT\ effectuera toutes les passes n�cessaires pour �tablir
correctement les r�f�rences crois�es du document. Certaines options sont
disponibles. Par exemple, si le format de sortie d�sir� est pdf, on
saisira :

\starttyping
texexec --pdf filename
\stoptyping

% When in doubt, say \type{--help} and you get all the
% information needed to proceed. A \TEXEXEC\ manual is
% available.

En cas de doute, choisissez l'option \type{--help} et vous obtiendrez
toute l'information n�cessaire pour compiler votre document. Un manuel
pour \TEXEXEC\ est �galement disponible.

\stopcomponent
