\startcomponent ma-cb-fr-introduction

\product ma-cb-fr

\CONTEXT\ est un syst�me d'ing�nierie de document bas� sur \TEX.
\TEX\ est un syst�me de composition de document et un langage de
programmation pour la composition et la r�alisation de
document. \CONTEXT\ est facile � utiliser et vous permet de concevoir
des documents papier et �lectronique complexes.

Ce manuel decrit les capacit�s de \CONTEXT\ ainsi que les commandes
disponibles et leurs fonctionnalit�s.\footnote{Toutes les documents
  papiers et �lectroniques concernant \CONTEXT\ ont �t� r�alis�es avec
  \CONTEXT. Tous les fichiers sources de ces r�alisations sont ou
  seront rendus disponible �lectroniquement afin de fournir un aper�u
  de la mani�re dont ils ont �t� fait.}

\CONTEXT\ est d�velopp� pour des applications pratiques: la
composition et la r�alisation de documents allant de livres simples
aux manuels techniques et supports de cours dans une forme papier ou
�lectronique.
Ce manuel d'introduction decrit les fonctionnalit�s de \CONTEXT\
n�cessaire � l'utilisation d'�l�ments textuels standards au sein d'un
manuel ou d'un support de cours. \CONTEXT, n�anmoins, est capable
de beaucoup plus et pour les utilisateurs d�sireux d'aller plus loin
il y a d'autres manuels et fichiers source \CONTEXT\ disponibles.

\CONTEXT\ dispose d'une interface multilingue afin de permettre aux
utilisateurs de travailler avec \CONTEXT\ dans leur propre
langage. \CONTEXT\ et ce manuel sont disponibles en hollandais,
allemand, anglais et fran�ais.

\stopcomponent
