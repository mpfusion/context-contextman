\startcomponent ma-cb-fr-structure

\product ma-cb-fr

\chapter{D�finir un document}

Chaque document d�bute par \type{\starttext} et finit par
\type{\stoptext}. Toute la saisie de texte est plac�e entre ces deux
commandes et \CONTEXT\ traitera uniquement cet information.

Les informations relatives aux r�glages sont plac�es juste avant
\type{\starttext}.

\startbuffer
\setupbodyfont[12pt]
\starttext
Ceci est un document � ligne unique
\stoptext
\stopbuffer

\typebuffer

� l'int�rieur du bloc \type{\starttext} $\cdots$ \type{\stoptext}, un
document est susceptible d'�tre divis� en quatre parties principales:

\startitemize[n,packed]
\item la pr�face ({\em front matter}),
\item le corps du doment ({\em body matter}),
\item l'�pilogue ({\em back matter}),
\item les annexes.
\stopitemize

Les diff�rentes parties sont d�finies par l'interm�diaire de:

\starttyping
\startfrontmatter ... \stopfrontmatter
\startbodymatter  ... \stopbodymatter
\startbackmatter  ... \stopbackmatter
\startappendices  ... \stopappendices
\stoptyping

Au sein de la pr�face autant que de l'�pilogue, la commande
\type{\chapter} produit un titre non num�rot� apparaissant dans la
table des mati�res. Ce type de section est plus particuli�rement
utilis� pour les tables des mati�res , des figures ou des tableaux, la
pr�face, les remerciements {\em etc.} Ce type de section dispose
souvent d'une num�rotation romaine des pages.

La section d'annexe est �videmment utilis� pour les annexes. Les
titres peuvent �tre compos� de diff�rentes mani�res : par exemple,
les commandes de type \type{\chapter} peuvent utiliser une
num�rotation alphab�tique.

Le style des sections peut �tre r�gl� par :

\shortsetup{setupsectionblock}

\stopcomponent
