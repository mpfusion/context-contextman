\startcomponent ma-cb-fr-footnotes

\product ma-cb-fr

%\chapter{Footnotes}
\chapter{Notes de bas de page}

%\index{footnote}
\index{notes de bas de page}

\Command{\tex{footnote}}
\Command{\tex{setupfootnotes}}

% If you want to annotate your text you can use
% \type{\footnote}. The command looks like this:
Si vous souhaitez annoter votre texte, vous pouvez utiliser la commande
\type{\footnote}. La commande s'utilise de la mani�re suivante :

\shortsetup{footnote}

% The bracket pair is optional and contains a logical name.
% The curly braces contain the text you want to display at
% the foot of the page.

La paire de crochets est optionnelle ; elle contient un nom logique,
afin de pouvoir y faire r�f�rence. Les accolades contiennent le texte
que vous voulez afficher au bas de la page.

% The same footnote number can be called with its logical name.

Le num�ro d'une note de bas de page peut �tre cit� en utilisant son nom
logique.

\shortsetup{note}

% If you have typed this text:
Si vous saisissez le texte suivant :

\startbuffer
The Hanse was a late medieval commercial alliance of towns in the
regions of the North and the Baltic Sea. The association was formed
for the furtherance and protection of the commerce of its
members.\footnote[war]{This was the source of jealousy and fear among
other towns that caused a number of wars.} In the Hanse period there
was a lively trade in all sorts of articles such as wood, wool,
metal, cloth, salt, wine and beer.\note[war] The prosperous trade
caused an enormous growth of welfare in the Hanseatic
towns.\footnote{Hasselt is one of these towns.}
\stopbuffer

\typebuffer

% It would look like this:
vous obtiendrez ce r�sultat :

\getbuffer

% The footnote numbering is done automatically. The command
% \type{\setupfootnotes} enables you to influence the display
% of footnotes:

La num�rotation des notes de bas de page est automatique. La commande
\type{\setupfootnote} vous permet de configurer leur pr�sentation.

\shortsetup{setupfootnotes}

% Footnotes can be set at the bottom of a page but also at
% other locations, like the end of a chapter. This is done
% with the command:

Les notes de bas de page peuvent �tre plac�es au bas de la page mais,
�galement, � d'autres endroits comme, par exemple, � la fin d'un
chapitre. On peut r�aliser cela avec la commande :

\shortsetup{placefootnotes}

% You can also couple footnotes to a table. In that case we
% speak of local footnotes. The commands are:

On peut �galement placer des notes dans un tableau. Dans ce cas, on
parle de notes locales. Les commandes sont les suivantes :

\shortsetup{startlocalfootnotes}

\shortsetup{placelocalfootnotes}

% An example illustrates the use of local footnotes:

L'exemple suivant illustre l'utilisation des notes locales :

\startbuffer
\startlocalfootnotes[n=0]
  \placetable
    {Decline of Hasselt's productivity.}
     \starttable[|l|c|c|c|c|]
       \HL
       \NC
       \NC Ovens\footnote{Source: Uit de geschiedenis van Hasselt.}
       \NC Blacksmiths\NC Breweries \NC Potteries \NC\SR
       \HL
       \NC 1682 \NC 15 \NC 9 \NC 3 \NC 2 \NC\FR
       \NC 1752 \NC ~6 \NC 4 \NC 0 \NC 0 \NC\LR
       \HL
       \stoptable
  \placelocalfootnotes
\stoplocalfootnotes
\stopbuffer

\typebuffer

\getbuffer

\stopcomponent
