\startcomponent ma-cb-fr-heads

\product ma-cb-fr

% TODO: Pluriel d'en-t�te ?

% Attention, il y a confusion entre heads et headers (dans la version
% anglaise aussi, apparemment) : titres (heads) et en-t�tes (headers).

% \chapter[headers]{Headers}
\chapter[titres]{Titres}

% \index{headers}
\index{titres}

\Command{\tex{chapter}}
\Command{\tex{paragraph}}
\Command{\tex{subparagraph}}
\Command{\tex{title}}
\Command{\tex{subject}}
\Command{\tex{subsubject}}
\Command{\tex{setuphead}}
\Command{\tex{setupheads}}

% The structure of a document is determined by its headers.
% Headers (heads) are created with the commands shown in
% \in{table}[tab:headers]:

La structure d'un document est d�finie par ses titres ({\em heads}).
Ils sont cr��s au moyen des commandes de la table
\in{table}[tab:headers]:

% \placetable[here][tab:headers]{Headers.}
\placetable[here][tab:headers]{Titres}
\starttable[|l|l|]
\HL
\NC \bf Titres num�rot�s \NC \bf Titres non num�rot�s \NC\SR
\HL
\NC \type{\chapter}        \NC \type{\title}              \NC\FR
\NC \type{\section}        \NC \type{\subject}            \NC\MR
\NC \type{\subsection}     \NC \type{\subsubject}         \NC\MR
\NC \type{\subsubsection}  \NC \type{\subsubsubject}      \NC\MR
\NC $\cdots$               \NC $\cdots$                   \NC\LR
\HL
\stoptable

\shortsetup{chapter}
\shortsetup{section}
\shortsetup{subsection}
\shortsetup{title}
\shortsetup{subject}
\shortsetup{subsubject}

% These commands will produce a header in a
% predefined fontsize and fonttype with some vertical
% spacing before and after the header.
Ces commandes produisent un titre dans un corps et un style de police
pr�d�finis, avec un espacement vertical avant et apr�s le titre.

% The heading commands can take several arguments, like in:
% TODO: Plusieurs <- 1 ou 2 ?
Chacune de ces commandes peut prendre plusieurs arguments ; comparez :

% \title[hasselt-by-night]{Hasselt by night}
\starttyping
\title[hasselt-de-nuit]{Hasselt de nuit}
\stoptyping

% and
et

\starttyping
\title{Hasselt de nuit}
\stoptyping

% The bracket pair is optional and used for internal
% references. If you want to refer to this header you type for
% example \type{\at{page}[hasselt-by-night]}.
La paire de crochets est facultative, et utilis�e comme r�f�rence interne ; par
exemple, pour renvoyer automatiquement au num�ro de page du titre � Hasselt de
nuit �, on utilisera \type{\at{page}[hasselt-de-nuit]}.

% Of course these headers can be set to your own preferences
% and you can even define your own headers. This is done by
% the command \type{\setuphead} and \type{\definehead}.
Bien entendu, ces titres peuvent �tre personnalis�s. Vous pouvez m�me
d�finir vos propres commandes pour cr�er de nouveaux types de titres. On
utilise pour cela les commandes \type{\definehead} et \type{\setuphead}.

\shortsetup{definehead}
\shortsetup{setuphead}

Ainsi, les commandes ci-dessous d�finissent le titre \type{\montitre},
et donnent un exemple d'utilisation :

\startbuffer
\definehead
  [montitre]
  [section]

\setuphead
  [montitre]
  [numberstyle=bold,
   textstyle=bold,
   before=\hairline\blank,
   after=\nowhitespace\hairline]

\montitre[refmontitre]{Hasselt fait la une}
\stopbuffer
% \myheader[myhead]{Hasselt makes headlines}
% TODO: les grands titres?

\typebuffer

% A new header \type{\myheader} is defined and it inherits the
% properties of \type{\section}. It would look something
% like this:
Un nouveau titre \type{\montitre} h�rite des propri�t�s de
\type{\section}, et la commande \type{\setuphead} en modifie certaines.
% TODO: pr�ciser ?

\getbuffer

% There is one other command you should know now, and that is
% \type{\setupheads}. You can use this command to set up the
% numbering of the numbered headers. If you type:
Une autre commande permet de modifier l'apparence des titres, c'est
\type{\setupheads} (au pluriel). On peut l'utiliser pour r�gler la
num�rotation des titres num�rot�s. L'instruction

% TODO: inmargin or margin?
\startbuffer
\setupheads
  [alternative=inmargin,
   separator=--]
\stopbuffer

\typebuffer

% all numbers will appear in the margin. Section 1.1 would
% look like 1--1.
placera les num�ros dans la marge, s�par� par le symbole �~--~�, le cas
�ch�ant. Ainsi, la section 1.1 appara�tra comme 1--1.
% TODO: add a note explaining \type{--} types --?

% Commands like \type{\setupheads} are typed in the
% set up area of your input file.
Ces commandes sont � placer dans le pr�ambule de votre document.

\shortsetup{setupheads}

\stopcomponent
