% Translation:
\startcomponent ma-cb-cz-layout
\project ma-cb
\environment ma-cb-env-cz

\product ma-cb-cz

\chapter{Page layout}

\index{layout}
\index{page design}

\Command{\tex{setuplayout}}
\Command{\tex{adaptlayout}}
\Command{\tex{showlayout}}
\Command{\tex{showsetups}}
\Command{\tex{showmakeup}}
\Command{\tex{startlocal}}

The page layout is defined by:

\shortsetup{setuplayout}

This command is typed in the set up area of your input file.

You have to familiarize yourself with the parameters that
describe your page layout. A page is divided in a number of
areas such as the text, margin, head, foot etc. The size of
these areas can be set with \type{\setuplayout}.

The different areas in the page layout are shown in
\in{figure}[fig:pagelayout].

\placefigure
  [here]
  [fig:pagelayout]
  {The page layout.}
  {\hbox{\framed[width=1.5cm,frame=off]{}
   \framed[width=2cm,frame=off]{\tt left}
   \framed[width=2.5cm,frame=off]{\tt left}
   \framed[width=3cm,frame=off]{\tt text}
   \framed[width=2.5cm,frame=off]{\tt right}
   \framed[width=2cm,frame=off]{\tt right}}
   \hbox{\framed[width=1.5cm,frame=off]{}
   \framed[width=2cm,frame=off]{\tt edge}
   \framed[width=2.5cm,frame=off]{\tt margin}
   \framed[width=3cm,frame=off]{\tt text}
   \framed[width=2.5cm,frame=off]{\tt margin}
   \framed[width=2cm,frame=off]{\tt edge}}
   \hbox{\framed[width=1.5cm,frame=off]{\tt top}
   \framed[width=2cm]{}
   \framed[width=2.5cm]{}
   \framed[width=3cm]{}
   \framed[width=2.5cm]{}
   \framed[width=2cm]{}}
   \hbox{\framed[width=1.5cm,frame=off]{\tt header}
   \framed[width=2cm]{}
   \framed[width=2.5cm]{}
   \framed[width=3cm]{}
   \framed[width=2.5cm]{}
   \framed[width=2cm]{}}
   \hbox{\framed[width=1.5cm,frame=off,height=3cm]{\tt text}
   \framed[width=2cm,height=3cm]{}
   \framed[width=2.5cm,height=3cm]{}
   \framed[width=3cm,height=3cm]{}
   \framed[width=2.5cm,height=3cm]{}
   \framed[width=2cm,height=3cm]{}}
   \hbox{\framed[width=1.5cm,frame=off]{\tt footer}
   \framed[width=2cm]{}
   \framed[width=2.5cm]{}
   \framed[width=3cm]{}
   \framed[width=2.5cm]{}
   \framed[width=2cm]{}}
   \hbox{\framed[width=1.5cm,frame=off]{\tt bottom}
   \framed[width=2cm]{}
   \framed[width=2.5cm]{}
   \framed[width=3cm]{}
   \framed[width=2.5cm]{}
   \framed[width=2cm]{}}}

If you want to look at your page layout you can type the
command \type{\showframe} and process one page or the whole
file. The areas are shown in a number of frames.

The command \type{\showsetups} shows the values of the
parameters. A combination of both commands is
\type{\showlayout}.

The values of the layout parameters are available
as commands (see \in{table}[tab:parameters]).
This enables you to work more accurately when defining
measures of columns, figures and tables. A few of these
parameters are explained in \in{table}[tab:fewparameters].

\placetable
  [here,force]
  [tab:fewparameters]
  {A few parameters as commands.}
\starttable[|l|l|]
\HL
\NC \bf Commands         \NC \bf Meaning               \NC\SR
\HL
\NC \type{\makeupwidth}  \NC width of the typing area  \NC\FR
\NC \type{\makeupheight} \NC height of the typing area \NC\MR
\NC \type{\textwidth}    \NC width of the text area    \NC\MR
\NC \type{\textheight}   \NC height of the text area   \NC\LR
\HL
\stoptable

If you want to define the width of a column or the height of
a figure you can do it relative to the \type{\makeupwidth}
or \type{\makeupheight}. Changes in page width or page height
will alter columns and figures proportionally.

\startbuffer
\placefigure
  [here]
  [fig:stepgable]
  {A stepgable.}
  {\externalfigure[ma-cb-19][width=.6\textwidth]}
\stopbuffer

\typebuffer

After processing this would become:

\getbuffer

The other distances and measures are shown in
\in{table}[tab:parameters].

The command \type{\setuplayout} is defined in the set up
area of your input file, before the
\type{\starttext} command. This means that the values of
the parameters have a global effect and are default
throughout the document. Nevertheless you might want to make
slight changes in the page design for a number of pages.

\startbuffer
\adaptlayout[21,38][height=+.5cm]
\stopbuffer

\typebuffer

In this case page 21 and 38 have a height of .5~cm + default
height.

For local changes in the page layout you can use:

\shortsetup{startlocal}

For example:

\startbuffer

\start

\startlocal
  \setuplayout[height=+.5cm]
\stoplocal

Hasselt has a completely different layout than most other cities
because of its fortifications and moats.

\stop

\stopbuffer

\typebuffer

It is advisable not to use these local changes to often.
It is always better to alter the text than to change the
page layout.

\startbuffer

\starttablehead
\HL
\NC \bf Parameter       \NC \bf Available command       \NC\SR
\HL
\stoptablehead

\starttabletail
\HL
\stoptabletail

\starttables[|l|l|]
\NC topdistance         \NC \type{\topheight}           \NC\FR
\NC topheight           \NC \type{\topheight}           \NC\MR
\NC headerdistance      \NC \type{\headerdistance}      \NC\MR
\NC headerheight        \NC \type{\headerheight}        \NC\MR
\NC topspace            \NC \type{\topspace}            \NC\MR
\NC headlevel           \NC \type{\headerlevel}         \NC\MR
\NC backspace           \NC \type{\backspace}           \NC\MR
\NC margindistance      \NC \type{\margindistance}      \NC\MR
\NC marginwidth         \NC \type{\marginwidth}         \NC\MR
\NC leftmarginwidth     \NC \type{\leftmarginwidth}     \NC\MR
\NC rightmarginwidth    \NC \type{\rightmarginwidth}    \NC\MR
\NC edgedistance        \NC \type{\edgedistance}        \NC\MR
\NC edgewidth           \NC \type{\edgewidth}           \NC\MR
\NC leftedgewidth       \NC \type{\leftedgewidth}       \NC\MR
\NC rightedgewidth      \NC \type{\rightedgewidth}      \NC\MR
\NC paperheight         \NC \type{\paperheight}         \NC\MR
\NC paperwidth          \NC \type{\paperwidth}          \NC\MR
\NC makeupheight        \NC \type{\makeupheight}        \NC\MR
\NC makeupwidth         \NC \type{\makeupwidth}         \NC\MR
\NC textheight          \NC \type{\textheight}          \NC\MR
\NC textwidth           \NC \type{\textwidth}           \NC\MR
\NC footerdistance      \NC \type{\footerdistance}      \NC\MR
\NC footerheight        \NC \type{\footerheight}        \NC\MR
\NC bottomdistance      \NC \type{\bottomdistance}      \NC\MR
\NC bottomheight        \NC \type{\bottomheight}        \NC\LR
\stoptables

\stopbuffer

\splitfloat[lines=1]
  {\placetable
     [here][tab:parameters]
     {Parameters for page design.}}
  {\getbuffer}

\stopcomponent
