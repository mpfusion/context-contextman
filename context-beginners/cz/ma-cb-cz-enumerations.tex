% Translation:
\startcomponent ma-cb-cz-enumerations
\project ma-cb
\product ma-cb-cz

\environment ma-cb-env-cz

\chapter{Numbered definitions}

\index{numbered definition}

\Command{\tex{defineenumeration}}
\Command{\tex{setupenumerations}}

With \type{\defineenumeration} you can number text elements
like remarks or questions. If you want to make numbered
remarks in your document you use:

\shortsetup{defineenumeration}

For example:

\startbuffer[a]
\defineenumeration
  [remark]
  [location=top,
   text=Remark,
   inbetween=\blank,
   after=\blank]
\stopbuffer

\typebuffer[a]

Now the new commands \type{\remark}, \type{\subremark},
\type{\resetremark} and \type{\nextremark} are available and
you can type remarks like this:

\startbuffer[b]
\remark In the early medieval times Hasselt was a place of
pilgrimage. The {\em Heilige Stede} (Holy Place) was torn down during
the Reformation. In 1930, after 300 years the {\em Heilige Stede} was
reopened.

\subremark Nowadays the {\em Heilige Stede} is closed again but once
a year an open air service is held on the same spot. \par
\stopbuffer

\typebuffer[b]

\start
\getbuffer[a]\getbuffer[b]
\stop

You can reset numbering with \type{\resetremark} of
\type{\resetsubremark} or increment a number with
\type{\nextremark} of \type{\nextsubremark}. This is
normally done automatically per chapter, section or
whatever.

You can set up the layout of \type{\defineenumeration} with:

\shortsetup{setupenumerations}

You can also vary the layout of {\bf Remark} and {\bf
Subremark} in the example above by:

\starttyping
\setupenumeration[remark][headstyle=bold]
\setupenumeration[subremark][headstyle=slanted]
\stoptyping

If a number becomes obsolete you can type:

\starttyping
\remark[-]
\stoptyping

If the remark contains more than one paragraph you will
have to use the command pair
\type{\startremark} $\cdots$ \type{\stopremark} that becomes
available after defining {\bf Remark} with
\type{\defineenumeration[remark]}.

So the example above would look like this:

\startbuffer[c]
\startremark
In the early medieval times Hasselt was a place of pilgrimage. The
{\em Heilige Stede} (Holy Place) was torn down during the
Reformation.

After 300 years in 1930 the {\em Heilige Stede} was reopened.
Nowadays the {\em Heilige Stede} is closed again but once a year an
open air service is held on the same spot.
\stopremark
\stopbuffer

\typebuffer[c]

\start
\getbuffer[a]\getbuffer[c] \par
\stop

\stopcomponent
