% Translation: 2006-03-24 Vit Zyka
% Checking:    2006-11-16 zyka
\startcomponent ma-cb-cz-metapost
\project ma-cb
\product ma-cb-cz
\environment ma-cb-env-cz

%\chapter{Graphical extension / \METAPOST}
\chapter{Grafick� roz���en� / \METAPOST}

%\index[metapost]{\METAPOST}
\index[metapost]{\METAPOST}
%\index{graphical features}
\index{grafika}
\index{vektorov� grafika}

%The graphical possibilities of \TEX||related macro packages
%are rather limited. However by using the graphical package
%\METAPOST\ of John Hobby a complete range of graphical
%features has become available that may improve the look of
%your documents.

Grafick� mo�nosti makrojazyka \TEX u jsou dosti omezen�. Pou�ijeme-li
v�ak grafick� syst�m \METAPOST\ od Johna Hobbyho, z�sk�me �plnou
mno�inu grafick�ch prvk�, kter� mohou zlep�it vzhled na�eho dokumetu.

%In \CONTEXT\ there is a direct link to \METAPOST\ so users
%can apply the features of \METAPOST\ directly into their
%documents. The chapter headers and page numbers of this
%manual are extended by some graphical elements that are
%generated by \METAPOST.

\CONTEXT\ p��mo spolupracuje s~\METAPOST em, tak�e u�ivatel m��e
vyu��t v�echny vlastnosti \METAPOST u p��mo ve sv�m dokumentu. Nadpisy
kapitol a ��sla str�nek tohoto manu�lu jsou zv�razn�ny grafikou p��mo
generovanou \METAPOST em.

%The usage and features of \METAPOST\ within \CONTEXT\ are
%described in the extensive \METAFUN\ manual.

Pou�it� \METAPOST u v~\CONTEXT u je pops�no v~rozs�hl�m manu�lu
\from[manual:metafun].

\stopcomponent
