% Translation:
\startcomponent ma-cb-cz-whatever
\project ma-cb
\product ma-cb-cz

\environment ma-cb-env-cz

\chapter[]{Miscellaneous}

\section{A titlepage}

\index{titlepage}

\Command{\tex{startstandardmakeup}}
\Command{\tex{definemakeup}}
\Command{\tex{setupmakeup}}

In the first example of this manual on \at{page}[inputfile]
we used the command:

\shortsetup{start<<name>>makeup}

This command can be used to define titlepages. Such a
command is needed since title pages often have a different
layout than that of the bodytext. With the command pair
\type{\start ... \stopstandardmakeup} you can make up a page
within the default page dimensions.

A simple titlepage may look like this:

\startbuffer
\startstandardmakeup
\blank
\rightaligned{\tfd Hasselt in the 21st century}
\blank
\rightaligned{\tfb The future}
\vfill
\rightaligned{\tfa C. van Marle}
\rightaligned{Hasselt, 2001}
\stopstandardmakeup
\stopbuffer

\typebuffer

In a doublesided document you have to go through some
additional actions to typeset the back of the titlepage.

\startbuffer
\startstandardmakeup[doublesided=no]
\blank
\rightaligned{\tfd Hasselt in the 21st century}
\blank
\rightaligned{\tfb The future}
\vfill
\rightaligned{\tfa C. van Marle}
\rightaligned{Hasselt, 2001}
\stopstandardmakeup
\startstandardmakeup[page=no]
\vfill
\copyright 2001

This book is dedicated to the people living in Hasselt. We
want to thank photographer J. Jonker for manipulating the
photos in this book in such a way that readers can get a
clear picture of Hasselt's future look.
\stopstandardmakeup
\stopbuffer

\typebuffer

Your own make ups can be made and set up with:

\shortsetup{definemakeup}

and

\shortsetup{setupmakeup}

\section[floatingblocks]{Floating blocks}

\index{floating blocks}

\Command{\tex{definefloat}}
\Command{\tex{setupfloat}}
\Command{\tex{setupfloats}}
\Command{\tex{setupcaptions}}
\Command{\tex{placeintermezzo}}

A block in \CONTEXT\ is a text element, for example a table
or a figure that you can process in a special way. You have
already seen the use of \type{\placefigure} and
\type{\placetable}. These are both examples of floating
blocks. The floating mechanism is described in
\in{chapter}[figures] and \in[tables].

You can define these kind of blocks yourself with:

\shortsetup{definefloat}

The bracket pairs are used for the name in singular and
plural form. For example:

\starttyping
\definefloat[intermezzo][intermezzi]
\stoptyping

Now the following commands are available:

\starttyping
\placeintermezzo[][]{}{}
\startintermezzotext ... \stopintermezzotext
\placelistofintermezzi
\completelistofintermezzi
\stoptyping

The newly defined floating block can be set up with:

\shortsetup{setupfloat}

You can set up the layout of floating blocks with:

\shortsetup{setupfloats}

You can set up the numbering and the labels with:

\shortsetup{setupcaption}

These commands are typed in the set up area of your input
file and will have a global effect on all floating blocks.

\startbuffer
\setupfloats[location=middle]
\setupcaption[location=bottom,headstyle=boldslanted]

\placeintermezzo{An intermezzo.}
\startframedtext
At the beginning of this century there was a tramline from Zwolle to
Blokzijl via Hasselt. Other means of transport became more important
and just before the second world war the tramline was stopped.
Nowadays such a tramline would have been very profitable.
\stopframedtext
\stopbuffer

\typebuffer

\start
\getbuffer
\stop

\section[textblocks]{Text blocks}

\index{text blocks}

\Command{\tex{defineblock}}
\Command{\tex{useblocks}}
\Command{\tex{hideblocks}}
\Command{\tex{setupblock}}

Another type of block is a text block. A text block for
example is one or more paragraphs you want to use several times.

You have to define a text block with:

\shortsetup{defineblock}

You give the name of text blocks between brackets;
you can also type a list of names if you separate them by
commas. For example you can define:

\starttyping
\defineblock[dutch]
\stoptyping

After defining the text block the following command is
available:

\starttyping
\begindutch ... \enddutch
\stoptyping

Text blocks are manipulated with:

\shortsetup{hideblocks}
\shortsetup{useblocks}
\shortsetup{keepblocks}
\shortsetup{selectblocks}

An example shows the possibilities of text blocks.

\startbuffer
\defineblock[dutch,english]

\hideblocks[dutch,english]

\beginenglish[dedemsvaart-e]
After 1810 the Dedemsvaart brought some prosperity to Hasselt. All
ships went through the canals of Hasselt and the shops on both
sides of the canals prospered.
\endenglish

\begindutch[dedemsvaart-d]
Sinds 1810 veroorzaakte de Dedemsvaart enige welvaart in Hasselt.
Alle schepen voeren door de grachten en de winkels aan weerszijden
van de gracht floreerden.
\enddutch

\useblocks[english][dedemsvaart-e]
\stopbuffer

\typebuffer

This will result in:

\getbuffer

If you continue defining these blocks you could make
a bilingual manual. For that purpose it is also possible to
store the text blocks in an external file. This would look
something like this:

\startbuffer
\setupblock[dutch][file=store-d]
\stopbuffer

\typebuffer

The Dutch text blocks are stored in the file \type
{store-d.tex} and the text fragments can be called upon by
their logical names.

\section{Storing text for later use}

\index{storing text}

\Command{\tex{startbuffer}}
\Command{\tex{getbuffer}}
\Command{\tex{typebuffer}}
\Command{\tex{setupbuffer}}

You can store information temporarily for future use in your
document with:

\shortsetup{startbuffer}

For example:

\starttyping
\startbuffer[visit]
If you want to see what Hasselt has in store you should come and
visit it some time. If you take this manual with you, you will
recognise some locations.
\stopbuffer

\getbuffer[visit]
\stoptyping

With \type{\getbuffer[visit]} you recall the stored text.
The logical name is optional. With \type{\typebuffer[visit]}
you get back the typeset version of the content of the
buffer.

Buffers are set up with:

\shortsetup{setupbuffer}

\section{Hiding text}

\index{hiding text}

\Command{\tex{starthiding}}

Text can be hidden with:

\shortsetup{starthiding}

The text in between will not be processed.

\section{Lines}

\index{lines}

\Command{\tex{hairline}}
\Command{\tex{starttextrule}}
\Command{\tex{thinrule}}
\Command{\tex{thinrules}}
\Command{\tex{setupthinrules}}

There are many comands to draw lines. For a single line you
type:

\shortsetup{hairline}

or:

\shortsetup{thinrule}

For more lines you type:

\shortsetup{thinrules}

Text in combination with lines is also possible:

\startbuffer
\starttextrule{Hasselt -- Amsterdam}
If you draw a straight line from Hasselt to Amsterdam you would have
to cover a distance of almost 145 \Kilo \Meter.
\stoptextrule

If you draw two straight lines from Hasselt to Amsterdam you would
have to cover a distance of almost 290 \Kilo \Meter.

Amsterdam \thinrules[n=3] Hasselt
\stopbuffer

\getbuffer

\typebuffer

You always have to be careful in drawing lines. Empty lines
around \type{\thinrules} must not be forgotten and the
vertical spacing is always a point of concern.

You can set up line spacing with:

\shortsetup{setupthinrules}

There are a few complementary commands that might be very
useful.

\shortsetup{setupfillinrules}
\shortsetup{setupfillinlines}

These commands are introduced in the examples below:

\startbuffer
\setupfillinrules[width=2cm]
\setupfillinlines[width=3cm]

\fillinrules[n=1]{\bf name}
\fillinrules[n=3]{\bf adress}

\fillinline{Can you please state the \underbar{number} of houses
            in Hasselt.} \par

Strike out \overstrikes{Hasselt in this text}\periods[18]
\stopbuffer

\typebuffer

This will become:

\getbuffer

These commands are used in questionaires. Text that is
struck out or underlined will not be hyphenated.

\section{Super- and subscript in text}

\index{subscript}
\index{superscript}

\Command{\tex{low}}
\Command{\tex{high}}
\Command{\tex{lohi}}

\startbuffer
It is very easy to put \high{superscript} and \low{subscript} in your
text. What would you call this version \lohi{subscript}{superscript}?
It looks strange!
\stopbuffer

\getbuffer

This ugly text was made with \type{\low{}}, \type{\high{}}
and \type{\lohi{}{}}. The text was placed between the curly
braces.

\section{Date}

\index{date}

\Command{\tex{currentdate}}

You can introduce the system date in your text with:

\starttyping
\currentdate
\stoptyping

\section{Positioning}

\index{positioning}

\Command{\tex{position}}
\Command{\tex{setupositioning}}

Sometimes you feel the need to position text on a page or
within a text element. You can position text with:

\shortsetup{position}

The parenthesis enclose the $x,y$ coordinates, the curly
braces enclose the text you want to position.

You can set up the $x,y$ axes with:

\shortsetup{setuppositioning}

You can use units and scaling factors. An example will
illustrate \type{\position}.

\startbuffer
\def\dicefive%
  {\framed
     [width=42pt,height=42pt,offset=0pt]
     {\setuppositioning
        [unit=pt,factor=12,xoffset=-11pt,yoffset=-8pt]%
      \startpositioning
        \position(1,1){$\bullet$}%
        \position(1,3){$\bullet$}%
        \position(2,2){$\bullet$}%
        \position(3,1){$\bullet$}%
        \position(3,3){$\bullet$}%
      \stoppositioning}}

\placefigure{This is five.}{\dicefive}
\stopbuffer

\typebuffer

This is a rather complex example but it would look something
like this.

\getbuffer

\section{Rotating text, figures and tables}

\index{rotating}

\Command{\tex{rotate}}

In a number of cases you would like to rotate text or
figures. You can rotate text and objects with:

\shortsetup{rotate}

The first bracket pair is optional. Within that bracket pair
you specify the rotation: \type{rotation=90}. The curly
braces contain the text or object you want to rotate.

\startbuffer
Hasselt got its municipal rights in 1252. From that time on it had
the \rotate[rotation=90]{right} to use its own seal on official
documents. This seal showed Holy Stephanus known as one of the first
Christian martyrs, and was the \rotate[rotation=270]{patron} of
Hasselt. After the Reformation the seal was redesigned and Stephanus
lost his `holiness' and was from that time on depicted without his
aureole.
\stopbuffer

\typebuffer

This results in a very ugly paragraph:

\getbuffer

You can rotate a figure just as easily:

\startbuffer
\placefigure
  [][fig:rotation]
  {The 180 \Degrees\ rotated fishing port (de Vispoort).}
  \rotate[rotation=180]{\externalfigure[ma-cb-15][width=10cm]}
\stopbuffer

\typebuffer

You can see in \in{figure}[fig:rotation] that it is not
always clear what you get when you rotate.

\getbuffer

We can set up rotating with:

\shortsetup{setuprotate}

\section{Carriage return}

\index{carriage return}

\Command{\tex{crlf}}
\Command{\tex{startlines}}

A new line can be enforced with:

\shortsetup{crlf}

When a number of lines should be followed by {\em carriage
return and line feed} you can use:

\shortsetup{startlines}

\starttyping
\startlines
.
.
.
\stoplines
\stoptyping

\startbuffer
On a wooden panel in the town hall of Hasselt you can read:

\startlines
Heimelijcken haet
eigen baet
jongen raet
Door diese drie wilt verstaen
is het Roomsche Rijck vergaen.
\stoplines

This little rhyme contains a warning for the magistrates of
Hasselt: don't allow personal benefits or feelings to
influence your wisdom in decision making.
\stopbuffer

\typebuffer

\getbuffer

In a few commands new lines are generated by \type{\\}. For
example if you type \type{\inmargin{in the\\margin}} then
the text will be divided over two lines.

\section{Hyphenation}

\index{hyphenation}
\index{language}

\Command{\tex{mainlanguage}}
\Command{\tex{language}}
\Command{\tex{nl}}
\Command{\tex{en}}
\Command{\tex{fr}}
\Command{\tex{sp}}
\Command{\tex{de}}

When writing multi-lingual texts you have to be aware of
the fact that hyphenation may differ from one language to
another.

To activate a language you type:

\shortsetup{mainlanguage}

Between the brackets you fill in \type{nl}, \type{fr},
\type{en}, \type{de} and \type{sp}.

To change from one language to another you can use the
shorthand versions:

\starttyping
\language[nl]  \language[en]  \language[de]
\language[fr]  \language[sp]
\stoptyping

or

\starttyping
\nl  \en  \de  \fr  \sp
\stoptyping

An example:

\startbuffer
If you want to know more about Hasselt, the best book to read is
probably {\nl \em Uit de geschiedenis van Hasselt} by
F.~Peereboom.
\stopbuffer

\typebuffer

\getbuffer

If a word is wrongly hyphenated you can define points of
hyphenation yourself. This is done in the set up area of
your input file:

\startbuffer
\hyphenation{his-to-ry}
\stopbuffer

\typebuffer

\section{Comment in input file}

\index{comment}
\index[percent]{\% in input file}

All text between \type{\starttext} and \type{\stoptext}
will be processed while running \CONTEXT. Sometimes however
you may have text fragments you don't want to be processed or
you want to comment on your \CONTEXT\ commands.

If you preceed your text with the percentage sign \type{%}
it will not be processed.

\startbuffer
% In very big documents you can use the command input for
% different files.
%
% For example:
%
% \input hass01.tex  % chapter 1 on Hasselt
% \input hass02.tex  % chapter 2 on Hasselt
% \input hass03.tex  % chapter 3 on Hasselt
\stopbuffer

\typebuffer

When you delete the \type{%} before \type{\input}
the three files will be processed. The comment describing the
contents of the files will not be processed.

\section{Input of another {\tt tex} file}

\index{input other \TEX--files}

\Command{\tex{input}}

In a number of situations you may want to insert other
\TEX\ files in your input file. For example, sometimes it is more
efficient to specify \CONTEXT\ sources in more than one file
in order to be able to partially process your files.

Another file (with the name \type{another.tex}) can be
inserted by:

\starttyping
\input another.tex
\stoptyping

The extension is optional so this will work too:

\starttyping
\input another
\stoptyping

The command \type{\input} is a \TEX\ command.

\stopcomponent
