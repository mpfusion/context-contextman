% Translation:
\startcomponent ma-cb-cz-gettingstarted
\project ma-cb
\product ma-cb-cz

\environment ma-cb-env-cz

\chapter{How to process a file}

\index{input file+processing}
\index[dvifile]{\type{dvi}--file}
\index[pdffile]{\type{pdf}--file}

If you want to process a \CONTEXT\ input file, you might
type at the command line prompt:

\starttyping
context filename
\stoptyping

The availability of the batch command \type{context} depends
on the system you're using. Ask your system administrator
the command you use to start \CONTEXT. If your filename is
\type{myfile.tex} this can be:

\starttyping
context myfile
\stoptyping

or when \TEXEXEC\ is properly installed:

\starttyping
texexec --pdf myfile
\stoptyping

the extension \type{.tex} is not needed.

After pressing \Enter\ processing will be started. \CONTEXT\
will show processing information on your screen. If
processing is succesful the command line prompt will return
and \CONTEXT\ will produce a \type{dvi} or \type{pdf} file.

If processing is not succesful ---for example because you
typed \type{\stptext} instead of \type{\stoptext}---
\CONTEXT\ produces a \type{ ? } on your terminal and tells
you it has just processed an error. It will give you some
basic information on the type of error and the line number
where the error becomes effective.

At the instant of \type{ ? } you can type:

\starttabulate[|||]
\NC \type{H} \NC for help information on your error \NC\NR
\NC \type{I} \NC for inserting the correct \CONTEXT\ command \NC\NR
\NC \type{Q} \NC for quiting and entering batch mode \NC\NR
\NC \type{X} \NC for exiting the running mode \NC\NR
\NC \Enter   \NC for ignoring the error \NC\NR
\stoptabulate

Most of the time you will type \Enter\ and processing will
continue. Then you can edit the input file and fix the error.

Some errors will produce a~\type{ * } on your screen and
processing will stop. This error is due to a fatal error in
your input file. You can't ignore this error and the only
option you have is to type \type{\stop} or {\sc Ctrl}~Z. The
program will be halted and you can fix the error.

During the processing of your input file \CONTEXT\ will also
inform you of what it is doing with your document. For
example it will show page numbers and information about
process steps. Further more it gives warnings. These are of
a typographical order and tells you when line breaking is not
successful. All information on processing is stored in a
\type{log} file that can be used for reviewing warnings and
errors and the respective line numbers where they occur in
your file.

When processing is succesful \CONTEXT\ produces a new file,
with the extension \type{.dvi} or \type{.pdf} so the files
\type{myfile.dvi} or \type{myfile.pdf} are generated. The
abbreviation \type{dvi} stands for Device Indepent. This
means that the file can be processed by a suitable
PostScript (\PS) printer driver to make the file suitable
for printing or viewing. The abbreviation \PDF\ stands for
Portable Document Format. This is a platform independent
format for printing and viewing.

\stopcomponent
