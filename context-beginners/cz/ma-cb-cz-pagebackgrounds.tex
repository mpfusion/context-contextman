% Translation: 2006-10-02 V�t Z�ka
% Checking:    2006-11-14 zyka
\startcomponent ma-cb-cz-pagebackgrounds
\project ma-cb
\product ma-cb-cz
\environment ma-cb-env-cz

%\chapter{Page backgrounds}
\chapter{Pozad� str�nky}

%\index{page areas}
%\index{background+page areas}
\index{plocha str�nky}
\index{pozad�+str�nky}

\Command{\tex{setupbackgrounds}}

%The page background can also be set, with:
Pozad� str�nky m��e b�t nastaveno pomoc�:

\shortsetup{setupbackgrounds}

%% The first two bracket pairs are used to define the page
%% areas. The last bracket pair is used for set up.
Prvn� dv� hranat� z�vorky definuj� ��st plochy str�nky, kter� chceme
nastavit pozad�. Posledn� z�vorka obsahuje jeho parametry.

\startbuffer
\hbox
  {\framed[width=1.5cm,frame=off]                   {}
   \framed[width=2cm,frame=off]                     {left}
   \framed[width=2.5cm,frame=off]                   {left}
   \framed[width=3cm,frame=off]                     {text}
   \framed[width=2.5cm,frame=off]                   {right}
   \framed[width=2cm,frame=off]                     {right}}
\hbox
  {\framed[width=1.5cm,frame=off]                   {}
   \framed[width=2cm,frame=off]                     {edge}
   \framed[width=2.5cm,frame=off]                   {margin}
   \framed[width=3cm,frame=off]                     {}
   \framed[width=2.5cm,frame=off]                   {margin}
   \framed[width=2cm,frame=off]                     {edge}}
\hbox
  {\framed[width=1.5cm,frame=off]                   {top}
   \framed[width=2cm]                               {}
   \framed[width=2.5cm]                             {}
   \framed[width=3cm]                               {}
   \framed[width=2.5cm]                             {}
   \framed[width=2cm]                               {}}
\hbox
  {\framed[width=1.5cm,frame=off]                   {header}
   \framed[width=2cm]                               {}
   \framed[width=2.5cm,background=screen]           {}
   \framed[width=3cm,background=screen]             {}
   \framed[width=2.5cm,background=screen]           {}
   \framed[width=2cm]                               {}}
\hbox
  {\framed[width=1.5cm,frame=off,height=3cm]        {text}
   \framed[width=2cm,height=3cm]                    {}
   \framed[width=2.5cm,height=3cm,background=screen]{}
   \framed[width=3cm,height=3cm,background=screen]  {}
   \framed[width=2.5cm,height=3cm,background=screen]{}
   \framed[width=2cm,height=3cm]                    {}}
\hbox
  {\framed[width=1.5cm,frame=off]                   {footer}
   \framed[width=2cm]                               {}
   \framed[width=2.5cm,background=screen]           {}
   \framed[width=3cm,background=screen]             {}
   \framed[width=2.5cm,background=screen]           {}
   \framed[width=2cm]                               {}}
\hbox
  {\framed[width=1.5cm,frame=off]                   {bottom}
   \framed[width=2cm]                               {}
   \framed[width=2.5cm]                             {}
   \framed[width=3cm]                               {}
   \framed[width=2.5cm]                             {}
   \framed[width=2cm]                               {}}
\stopbuffer

\placefigure
  [here]
  [fig:pageareas]
%  {The page areas defined in \type{\setupbackgrounds}.}
  {Plochy str�nky pou�iteln� v~\type{\setupbackgrounds}.}
  {\tt\getbuffer}

%% If you want to have backgrounds in the gray areas of the
%% page layout of \in{figure}[fig:pageareas] you type:
Pokud nap�. chcete m�t �ed� pozad� jako na
\in{obr�zku}[fig:pageareas], nap��eme:

\startbuffer
\setupbackgrounds
  [header,text,footer]
  [leftmargin,text,rightmargin]
  [background=screen]
\stopbuffer

\typebuffer

\stopcomponent
