% Translation:
\startcomponent ma-cb-cz-units
\project ma-cb
\product ma-cb-cz

\environment ma-cb-env-cz

\chapter[units]{Units}

\index{units}
\index[siunit]{\cap{SI}--unit}

\Command{\tex{unit}}
\Command{\tex{permille}}
\Command{\tex{percent}}

To force yourself to use dimensions and units consistently
throughout the document you can make your own list with
units. These are specified in the set up area of your
input file.

In \CONTEXT\ there is an external module available that
contains almost all \SI||units. When this module is loaded
with \type{\usemodule[units]} you can call units with:

\startbuffer
\Meter \Per \Square \Meter
\Cubic \Meter \Per \Sec
\Square \Milli \Meter \Per \Inch
\Centi \Liter \Per \Sec
\Meter \Inverse \Sec
\Newton \Per \Square \Inch
\Newton \Times \Meter \Per \Square \Sec
\stopbuffer

\typebuffer

It looks like a lot of typing but it does guarantee a
consistent use of units. The command \type{\unit} also
prevents the separation of value and unit at line breaks,
because a number typeset at the end of a line and the unit
at the beginning of the next one, is far from perfect.
These examples come out as:

\startnarrower
\startlines
\getbuffer
\stoplines
\stopnarrower

You can define your own units with:

\starttyping
\unit[Ounce]{oz}{}
\stoptyping

\unit[Ounce]{oz}{}

Later on in the document you can type \type{15.6 \Ounce}
that will be displayed as 15.6 \Ounce.

In order to write \percent\  and \permille\ in a consistent
way there are two specific commands:

\type{\percent} \crlf
\type{\permille}

\stopcomponent
