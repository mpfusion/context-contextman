% Translation: busaj
% Checking:    2006-11-14 zyka
\startcomponent ma-cb-cz-color
\project ma-cb
\product ma-cb-cz
\environment ma-cb-env-cz

%\chapter{Color}
\chapter{Barva}

%\index{color}
\index{barva}

\Command{\tex{setupcolors}}
\Command{\tex{color}}
\Command{\tex{definecolor}}

% Text can be set in color.
Text m��e b�t s�zen barevn�. 

\shortsetup{setupcolors}

%The use of colors has to be
%activated by:
Pou��v�n� barev se mus� zapnout povelem:

\starttyping
\setupcolors[state=start]
\stoptyping

% Now the basic colors are available (red, green and blue).
Nyn� m�me k~dispozici z�kladn� barvy (�ervenou -- red, zelenou --
green a modrou -- blue).

%\startbuffer
%\startcolor[red]
%Hasselt is a very \color[green]{colorful} town.
%\stopcolor
%\stopbuffer

\startbuffer
\startcolor[red]
Hasselt je velice \color[green]{barevn�} m�sto.
\stopcolor
\stopbuffer

\typebuffer

\getbuffer

% On a black and white printer you will see only grey shades.
% In an electronic document these colors will be as expected.
P�i pou�it� �ernob�l�ho tisku uvid�me jenom �ed� odst�ny. 
V~elektronick�m dokumentu barvy dopadnou podle o�ek�v�n�.

% You can define your own colors with:
M��ete tak� definovat sv� vlastn� barvy p��kazem:

\shortsetup{definecolor}

% For example:
Nap��klad:

%\startbuffer
%\definecolor[darkred]   [r=.5,g=.0,b=.0]
%\definecolor[darkgreen] [r=.0,g=.5,b=.0]
%\stopbuffer

\startbuffer
\definecolor[temneruda]   [r=.5,g=.0,b=.0]
\definecolor[tmavezelena] [r=.0,g=.5,b=.0]
\stopbuffer


\typebuffer

%Now the colors \type{darkred} and \type{darkgreen} are
%available.

\start
\setupcolors[state=start]
\definecolor[temneruda]   [r=.5,g=.0,b=.0]
\definecolor[tmavezelena] [r=.0,g=.5,b=.0]

Nyn� jsou k~dispozici barvy \color[temneruda]{\type{temneruda}}
a~\color[tmavezelena]{\type{tmavezelena}}.
\stop

\stopcomponent
