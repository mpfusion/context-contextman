% Translation:
\startcomponent ma-cb-cz-color
\project ma-cb
\product ma-cb-cz

\environment ma-cb-env-cz

\chapter{Color}

\index{color}

\Command{\tex{setupcolors}}
\Command{\tex{color}}
\Command{\tex{definecolor}}

Text can be set in color.

\shortsetup{setupcolors}

The use of colors has to be
activated by:

\starttyping
\setupcolors[state=start]
\stoptyping

Now the basic colors are available (red, green and blue).

\startbuffer
\startcolor[red]
Hasselt is a very \color[green]{colorful} town.
\stopcolor
\stopbuffer

\typebuffer

\getbuffer

On a black and white printer you will see only grey shades.
In an electronic document these colors will be as expected.

You can define your own colors with:

\shortsetup{definecolor}

For example:

\startbuffer
\definecolor[darkred]   [r=.5,g=.0,b=.0]
\definecolor[darkgreen] [r=.0,g=.5,b=.0]
\stopbuffer

\typebuffer

Now the colors \type{darkred} and \type{darkgreen} are
available.

\stopcomponent
