% Translation:
\startcomponent ma-cb-cz-tables

\product ma-cb-cz

\environment ma-cb-env-cz

\chapter[tables]{Tables}

\index{tables}
\index{floating blocks}

\Command{\tex{placetable}}
\Command{\tex{setuptables}}
\Command{\tex{starttable}}
\Command{\tex{startcombination}}
\Command{\tex{setupfloats}}
\Command{\tex{setupcaptions}}
\Command{\tex{NR}}
\Command{\tex{FR}}
\Command{\tex{LR}}
\Command{\tex{MR}}
\Command{\tex{SR}}
\Command{\tex{VL}}
\Command{\tex{NC}}
\Command{\tex{HL}}
\Command{\tex{DL}}
\Command{\tex{DC}}
\Command{\tex{DR}}
\Command{\tex{LOW}}
\Command{\tex{TWO}}
\Command{\tex{THREE}}

{\em In general, a table consists of columns which may be
independently left adjusted, centered, right adjusted, or
aligned on decimal points. Headings may be placed over single
columns or groups of columns. Table entries may contain
equations or several rows of text. Horizontal and vertical
lines may be drawn wholly or partially across the table.}

This is what Michael J. Wichura wrote in the preface of the
manual of \TABLE\ (\TABLE\ manual, 1988.). Michael Wichura
is also the author of the \TABLE\ macros \CONTEXT\ is
relying on when processing tables. A few \CONTEXT\ macros
were added to take care of consistent line spacing and to
make the interface a little less cryptic.\footnote{\CONTEXT\
was developed for non||technical users in the \cap{WYSIWYG}
era. Therefore a user friendly interface and easy file and
command handling were needed, and cryptic commands,
programming and logical reasoning had to be avoided.}

For placing a table the command \type{\placetable} is used
which is a predefined example of:

\shortsetup{placeblock}

For defining the table you use:

\shortsetup{starttable}

The definition of a table could look something like this:

\startbuffer
\placetable[here][tab:ships]{Ships that moored at Hasselt.}
\starttable[|c|c|]
\HL
\NC \bf Year \NC \bf Number of ships \NC\SR
\HL
\NC 1645     \NC 450                 \NC\FR
\NC 1671     \NC 480                 \NC\MR
\NC 1676     \NC 500                 \NC\MR
\NC 1695     \NC 930                 \NC\LR
\HL
\stoptable
\stopbuffer

\typebuffer

This table is typeset as \in{table}[tab:ships].

\getbuffer

The first command \type{\placetable} has the same function
as \type{\placefigure}. It takes care of spacing before and
after the table and numbering. Furthermore the floating
mechanism is initialized so the table will be placed at the
most optimal location of the page.

The table entries are placed between the \type{\starttable}
$\cdots$ \type{\stoptable} pair. Between the bracket pair
your can specify the table format with the column separators
\type{|} and the format keys (see
\in{table}[tab:formatkeys]).

\placetable
  []
  [tab:formatkeys]
  {Table format keys.}
\starttable[|l|l|]
\HL
\NC \bf Key     \NC \bf Meaning                                 \NC\SR
\HL
\NC \type{|}    \NC column separator                            \NC\FR
\NC \type{c}    \NC center                                      \NC\MR
\NC \type{l}    \NC flush left                                  \NC\MR
\NC \type{r}    \NC flush right                                 \NC\MR
\NC \type{s<n>} \NC set intercolumn space at value $n = 0, 1,2$ \NC\MR
\NC \type{w<>}  \NC set minimum column width at specified value \NC\LR
\HL
\stoptable

In addition to the format keys there are format commands.
\in{Table}[tab:formatcommands] shows a few of the essential
commands.

\placetable
  [here]
  [tab:formatcommands]
  {Table format commands.}
\starttable[|l|l|]
\HL
\NC \bf Command               \NC \bf Meaning                            \NC\SR
\HL
\NC \type{\JustLeft}          \NC flush left and suppress column format  \NC\FR
\NC \type{\JustRight}         \NC flush right and suppress column format \NC\MR
\NC \type{\JustCenter}        \NC center and suppress column format      \NC\MR
\NC \type{\SetTableToWidth{}} \NC specify exact table width              \NC\MR
\NC \type{\use{n}}            \NC use the space of the next $n$ columns  \NC\LR
\HL
\stoptable

In the examples you have seen so far a number of
\CONTEXT\ formatting commands were used. These commands are
somewhat longer than the original and less cryptic but they
also handle a lot of table typography. In
\in{table}[tab:contextformatcommands] an overview of these
commands is given.

\placetable
  [here]
  [tab:contextformatcommands]
  {\CONTEXT\ table format commands.}
{\setuptables[bodyfont=small]
\starttable[s1|l|l|l|]
\HL
\NC \bf Command       \NC
    \NC \bf Meaning                                \NC\SR
\HL
\NC \type{\NR}        \NC next row
    \NC make row with no vertical space adjustment \NC\FR
\NC \type{\FR}        \NC first row
    \NC make row, adjust upper spacing             \NC\MR
\NC \type{\LR}        \NC last row
    \NC make row, adjust lower spacing             \NC\MR
\NC \type{\MR}        \NC mid row
    \NC make row, adjust upper and lower spacing   \NC\MR
\NC \type{\SR}        \NC separate row
    \NC make row, adjust upper and lower spacing   \NC\MR
\NC \type{\VL}        \NC vertical line
    \NC draw a vertical line, go to next column    \NC\MR
\NC \type{\NC}        \NC next column
    \NC go to next column                          \NC\MR
\NC \type{\HL}        \NC horizontal line
    \NC draw a horizontal line                     \NC\MR
\NC \type{\DL}        \NC division line$^\star$
    \NC draw a division line over the next column  \NC\MR
\NC \type{\DL[n]}     \NC division line$^\star$
    \NC draw a division line over $n$ columns      \NC\MR
\NC \type{\DC}        \NC division column$^\star$
    \NC draw a space over the next column          \NC\MR
\NC \type{\DR} \NC division row$^\star$
    \NC make row, adjust upper and lower spacing   \NC\MR
\NC \type{\LOW{text}} \NC ---
    \NC lower {\em text}                           \NC\MR
\NC \type{\TWO}, \type{\THREE} etc.  \NC ---
    \NC use the space of the next {\em two}, {\em three} columns \NC\LR
\HL
\NC \use3 \JustLeft{$^\star$ \type{\DL, \DC} and \type{\DR}
    are used in combination.}                      \NC\FR
\stoptable}

The tables below are shown with their sources. You can
always read the \TABLE\ manual by M.J. Wichura for more
sophisticated examples.

\startbuffer
\placetable
  [here,force]
  [tab:effects of commands]
  {Effect of formatting commands.}
  {\startcombination[2*1]
     {\starttable[|c|c|]
      \HL
      \VL \bf Year \VL \bf Citizens \VL\SR
      \HL
      \VL 1675     \VL  ~428        \VL\FR
      \VL 1795     \VL  1124        \VL\MR
      \VL 1880     \VL  2405        \VL\MR
      \VL 1995     \VL  7408        \VL\LR
      \HL
      \stoptable}{standard}
     {\starttable[|c|c|]
      \HL
      \VL \bf Year \VL \bf Citizens \VL\NR
      \HL
      \VL 1675     \VL  ~428        \VL\NR
      \VL 1795     \VL  1124        \VL\NR
      \VL 1880     \VL  2405        \VL\NR
      \VL 1995     \VL  7408        \VL\NR
      \HL
      \stoptable}{only \type{\NR}}
   \stopcombination}
\stopbuffer

\typebuffer

In the example above the first table \type{\SR}, \type{\FR},
\type{\MR} and \type{\LR} are used. These commands take care
of line spacing within a table. As you can see below
the command \type{\NR} only starts a new row.

\getbuffer

In the example below column interspacing with the \type{s0}
and \type{s1} keys is shown.

\startbuffer
\startbuffer[one]
\starttable[|c|c|]
\HL
\VL \bf Year \VL \bf Citizens \VL\SR
\HL
\VL 1675 \VL  ~428 \VL\FR
\VL 1795 \VL  1124 \VL\MR
\VL 1880 \VL  2405 \VL\MR
\VL 1995 \VL  7408 \VL\LR
\HL
\stoptable
\stopbuffer

\startbuffer[two]
\starttable[s0 | c | c |]
\HL
\VL \bf Year \VL \bf Citizens \VL\SR
\HL
\VL 1675 \VL  ~428 \VL\FR
\VL 1795 \VL  1124 \VL\MR
\VL 1880 \VL  2405 \VL\MR
\VL 1995 \VL  7408 \VL\LR
\HL
\stoptable
\stopbuffer

\startbuffer[three]
\starttable[| s0 c | c |]
\HL
\VL \bf Year \VL \bf Citizens \VL\SR
\HL
\VL 1675 \VL  ~428 \VL\FR
\VL 1795 \VL  1124 \VL\MR
\VL 1880 \VL  2405 \VL\MR
\VL 1995 \VL  7408 \VL\LR
\HL
\stoptable
\stopbuffer

\startbuffer[four]
\starttable[| c | s0 c |]
\HL
\VL \bf Year \VL \bf Citizens \VL\SR
\HL
\VL 1675 \VL  ~428 \VL\FR
\VL 1795 \VL  1124 \VL\MR
\VL 1880 \VL  2405 \VL\MR
\VL 1995 \VL  7408 \VL\LR
\HL
\stoptable
\stopbuffer

\startbuffer[five]
\starttable[s1 | c | c |]
\HL
\VL \bf Year \VL \bf Citizens \VL\SR
\HL
\VL 1675 \VL  ~428 \VL\FR
\VL 1795 \VL  1124 \VL\MR
\VL 1880 \VL  2405 \VL\MR
\VL 1995 \VL  7408 \VL\LR
\HL
\stoptable
\stopbuffer

\placetable
  [here,force]
  [tab:example formatcommands]
  {Effect of formatting commands.}
  {\startcombination[3*2]
    {\getbuffer[one]}   {standard}
    {\getbuffer[two]}   {\type{s0}}
    {\getbuffer[three]} {\type{s0} in column~1}
    {\getbuffer[four]}  {\type{s0} in column~2}
    {\getbuffer[five]}  {\type{s1}}
    {}                  {}
  \stopcombination}
\stopbuffer

\typebuffer

After processing these tables come out as
\in{table}[tab:example formatcommands]. The default table
has a column interspacing of\type{s2}.

\getbuffer

Columns are often separated with a vertical line $|$ and
rows by a horizontal line.

\startbuffer
\placetable
  [here,force]
  [tab:divisions]
  {Effect of options.}
\starttable[|c|c|c|]
\NC Steenwijk  \NC Zwartsluis \NC Hasselt    \NC\SR
\DC            \DL            \DC               \DR
\NC Zwartsluis \VL Hasselt    \VL Steenwijk  \NC\SR
\DC            \DL            \DC               \DR
\NC Hasselt    \NC Steenwijk  \NC Zwartsluis \NC\SR
\stoptable
\stopbuffer

\typebuffer

\getbuffer

A more sensible example is given in the
\in{table}[tab:example contextcommands].

\startbuffer
\placetable
  [here,force]
  [tab:example contextcommands]
  {Effect of \CONTEXT\ formatting commands.}
\starttable[|l|c|c|c|c|]
\HL
\VL \FIVE \JustCenter{City council elections in 1994}    \VL\SR
\HL
\VL \LOW{Party} \VL \THREE{Districts}   \VL \LOW{Total} \VL\SR
\DC             \DL[3]                  \DC                \DR
\VL             \VL 1   \VL 2   \VL 3   \VL             \VL\SR
\HL
\VL PvdA        \VL 351 \VL 433 \VL 459 \VL 1243        \VL\FR
\VL CDA         \VL 346 \VL 350 \VL 285 \VL ~981        \VL\MR
\VL VVD         \VL 140 \VL 113 \VL 132 \VL ~385        \VL\MR
\VL HKV/RPF/SGP \VL 348 \VL 261 \VL 158 \VL ~767        \VL\MR
\VL GPV         \VL 117 \VL 192 \VL 291 \VL ~600        \VL\LR
\HL
\stoptable
\stopbuffer

\typebuffer

In the last column a \type{~} is used to simulate a four
digit number. The \type{~} has the width of a digit.

\getbuffer

Sometimes your tables get too big and you want to adjust, for
example, the body font or the vertical and/or horizontal spacing
around vertical and horizontal lines. This is done by:

\shortsetup{setuptables}

\startbuffer
\placetable
  [here,force]
  [tab:setuptable]
  {Use of \type{\setuptables}.}
{\startcombination[1*3]
{\setuptables[bodyfont=10pt]
\starttable[|c|c|c|c|c|c|]
\HL
\VL \use6 \JustCenter{Decline of wealth in
                      Dutch florine (Dfl)} \VL\SR
\HL
\VL Year \VL 1.000--2.000
         \VL 2.000--3.000
         \VL 3.000--5.000
         \VL 5.000--10.000
         \VL   over 10.000 \VL\SR
\HL
\VL 1675 \VL 22 \VL 7 \VL 5  \VL 4  \VL 5  \VL\FR
\VL 1724 \VL ~4 \VL 4 \VL -- \VL 4  \VL 3  \VL\MR
\VL 1750 \VL 12 \VL 3 \VL 2  \VL 2  \VL -- \VL\MR
\VL 1808 \VL ~9 \VL 2 \VL -- \VL -- \VL -- \VL\LR
\HL
\stoptable}{\tt bodyfont=10pt}
{\setuptables[bodyfont=8pt]
\starttable[|c|c|c|c|c|c|]
\HL
\VL \use6 \JustCenter{Decline of wealth in
                      Dutch florine (Dfl)} \VL\SR
\HL
\VL Year \VL 1.000--2.000
         \VL 2.000--3.000
         \VL 3.000--5.000
         \VL 5.000--10.000
         \VL   over 10.000 \VL\SR
\HL
\VL 1675 \VL 22 \VL 7 \VL 5  \VL 4  \VL 5  \VL\FR
\VL 1724 \VL ~4 \VL 4 \VL -- \VL 4  \VL 3  \VL\MR
\VL 1750 \VL 12 \VL 3 \VL 2  \VL 2  \VL -- \VL\MR
\VL 1808 \VL ~9 \VL 2 \VL -- \VL -- \VL -- \VL\LR
\HL
\stoptable}{\tt bodyfont=8pt}
{\setuptables[bodyfont=6pt,distance=small]
\starttable[|c|c|c|c|c|c|]
\HL
\VL \use6 \JustCenter{Decline of wealth in
                      Dutch florine (Dfl)} \VL\SR
\HL
\VL Year \VL 1.000--2.000
         \VL 2.000--3.000
         \VL 3.000--5.000
         \VL 5.000--10.000
         \VL   over 10.000 \VL\SR
\HL
\VL 1675 \VL 22 \VL 7 \VL 5  \VL 4  \VL 5  \VL\FR
\VL 1724 \VL ~4 \VL 4 \VL -- \VL 4  \VL 3  \VL\MR
\VL 1750 \VL 12 \VL 3 \VL 2  \VL 2  \VL -- \VL\MR
\VL 1808 \VL ~9 \VL 2 \VL -- \VL -- \VL -- \VL\LR
\HL
\stoptable}{\tt bodyfont=6pt,distance=small}
\stopcombination}
\stopbuffer

\typebuffer

\getbuffer

You can also set up the layout of tables with:

\shortsetup{setupfloats}

You can set up the numbering and the labels with:

\shortsetup{setupcaptions}

These  commands are typed in the set up area of your
input file and have a global effect on all floating blocks.

\startbuffer
\setupfloats[location=left]
\setupcaption[style=boldslanted]

\placetable[here][tab:opening hours]{Library opening hours.}
\starttable[|l|c|c|]
\HL
\VL \bf Day   \VL \use2 \bf Opening hours           \VL\SR
\HL
\VL Monday    \VL 14.00 -- 17.30 \VL 18.30 -- 20.30 \VL\FR
\VL Tuesday   \VL                \VL                \VL\MR
\VL Wednesday \VL 10.00 -- 12.00 \VL 14.00 -- 17.30 \VL\MR
\VL Thursday  \VL 14.00 -- 17.30 \VL 18.30 -- 20.30 \VL\MR
\VL Friday    \VL 14.00 -- 17.30 \VL                \VL\MR
\VL Saturday  \VL 10.00 -- 12.30 \VL                \VL\LR
\HL
\stoptable
\stopbuffer

\typebuffer

The result is displayed in \in{table}[tab:opening hours].

\start
\getbuffer
\stop

\stopcomponent
