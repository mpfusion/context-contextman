% Translation:
\startcomponent ma-cb-cz-frames
\project ma-cb
\product ma-cb-cz

\environment ma-cb-env-cz

\chapter{Outlined text}

\index{outline+text}

\Command{\tex{framed}}
\Command{\tex{setupframed}}
\Command{\tex{inframed}}

You can \inframed{outline} a text with \type{\framed}. The
command looks like this:

\shortsetup{framed}

The bracket pair is optional and contains the set up
parameters. The curly braces enclose the text. To be honest,
the outlined text in the last paragraph was done with
\type{\inframed}. This command takes care of the interline
spacing.

\startbuffer
\framed[height=3em,width=fit]{Hasselt needs more space}
\stopbuffer

\typebuffer

This becomes:

\startbaselinecorrection
\getbuffer
\stopbaselinecorrection

Some other examples of \type{\framed} and its set up
parameters are shown below. This time we use the in||line
alternative \type{\inframed}.

\startbuffer
\leftaligned
  {\inframed[width=fit]{People in Hasselt}}
\midaligned
  {\inframed[height=1.5cm,frame=off]{have a}}
\rightaligned
  {\inframed[background=screen]{historic background}}
\stopbuffer

\typebuffer

This leads to:

\getbuffer

The \type{\framed} command is very sophisticated and it is
used in many macros. The command to set up frames is:

\shortsetup{setupframed}

\stopcomponent
