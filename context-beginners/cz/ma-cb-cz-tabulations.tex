% Translation:
\startcomponent ma-cb-cz-tabulations

\product ma-cb-cz

\environment ma-cb-env-cz

\chapter[tabulation]{Tabulation / Paragraph formatting}

\index{tabulation}
\index{tables+running text}
\index{columns}
\Command{\tex{starttabulate}}
\Command{\tex{definetabulate}}
\Command{\tex{setuptabulate}}
\Command{\tex{NR}}
\Command{\tex{NC}}
\Command{\tex{startchemical}}

Sometimes you want to typeset paragraphs in a specific
formatted way. This is done with:

\shortsetup{starttabulate}

The tabulation mechanism is closely related to the
table mechanism. You can use the tabulation mechanism
in cases you want to type set complete paragraphs within a
cell. The tabulation mechanism works fine at a page break.

\startbuffer
\starttabulate[|w(1.5cm)B|p(6.0cm)|p|]
\NC 1252
    \NC Hasselt obtains its city charter from bishop Hendrik
        van Vianden.
    \NC Hendrik van Vianden was pressed by other towns not
        to agree with the charter. It took Hasselt a long
        period of time to convince the Bishop. After
        supporting the Bishop in a small war against the
        Drents, the charter was released. \NC\NR
\NC 1350
    \NC Hasselt joins the Hanzepact to protect their
        international trade.
    \NC The Hanzepact was of great importance for merchants
        in Hasselt. In those days trading goods were taxed
        at every city, highway or rivercrossing. After
        joining the Hanzepact duty free routes all over
        Europe became available to Hasselt. However
        important the Hanzepact was, Hasselt always stayed a
        minor member of the pact. \NC\NR
\stoptabulate
\stopbuffer

A tabulate definition could look like this:

\typebuffer

In this case the first column is 1.5 \Centi \Meter\ wide and
is type set bold (\type{B}). The second column has a width
of 6 \Centi \Meter\ and is type set like a paragraph. The
remaining horizontal space is used up by the last paragraph.

The example is typeset like this:

\getbuffer

Like in the table mechanism a number of formatting commands
and keys are used. A list of these commands and keys are
shown in \in{table}[tab:tabularformattingcommands].

\placetable[][tab:tabularformattingcommands]
  {Tabular commands.}
\starttable[|lT|l|lT|l|]
\NC l                 \NC left align
\NC I                 \NC \it italic
\NC \FR
\NC c                 \NC center
\NC R                 \NC \sl roman
\NC \MR
\NC r                 \NC right align
\NC S                 \NC \sl slanted
\NC \MR
\NC i\sl n            \NC spacing left
\NC T                 \NC \tt teletype
\NC \MR
\NC j\sl n            \NC spacing right
\NC m                 \NC in||line math
\NC \MR
\NC k\sl n            \NC spacing around
\NC M                 \NC display  math
\NC \MR
\NC w({\sl d})        \NC 1 line,   fixed width
\NC f\tex{command}    \NC font specification
\NC \MR
\NC p({\sl d})        \NC paragraph, fixed width
\NC b\arg{..}         \NC place \type{..} before the entry
\NC \MR
\NC p                 \NC paragraph, maximum width
\NC a\arg{..}         \NC place \type{..} after the entry
\NC \MR
\NC B                 \NC \bf boldface
\NC h\tex{command}    \NC apply \tex{command} on the entry
\NC \LR
\stoptable

Another example of paragraph formatting could look like this.

\startbuffer
\definetabulate[ChemPar][|l|p|l|]

\startChemPar
\NC Limekilns
    \NC Hasselt has its own limekilns. These were build in 1504
        and produced quick lime up to 1956. Nowadays they are a
        tourist attraction.
    \NC \chemical{CaCO_3,~,GIVES,~,CaO,~,+,~,CO_2} \NC\NR
\stopChemPar
\stopbuffer

\typebuffer

And it would come out like this:

\getbuffer

The chemical module is explained in another manual, because
not everybody is interested in chemical stuff.

Here we also introduced the command to define the paragraph
layout.

\shortsetup{definetabulate}

and we also have:

\shortsetup{setuptabulate}

\stopcomponent
