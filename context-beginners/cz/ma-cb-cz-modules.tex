% Translation:
\startcomponent ma-cb-cz-modules
\project ma-cb
\product ma-cb-cz
\environment ma-cb-env-cz

\chapter{Using modules}

\index{module+chemic}
\index{module+unit}
\index{module+chart}
\index{module+pictex}
\index{chemical structures}
\index{flowcharts}
\index{units}

\Command{\tex{usemodule}}

For reasons of efficiency the author decided to implement
some functionality of \CONTEXT\ by means of external
modules. At this moment you can load the following modules:

\startitemize[packed]
\item \type{chemic} for typesetting chemical structures
\item \type{units} for using \cap{SI} units
\item \type{pictex} for drawing pictures (is used in
      conjunction with module \type{chemic})
\item \type{chart} for drawing flowcharts and organograms
\stopitemize

Loading is done in the set up area of your input file and done
by means of:

\shortsetup{usemodule}

We have shown a number of examples of the module
\type{units}. Below we give two examples of the
modules \type{chemic} and \type{chart} without any further
explanations. These modules are described in two separate
manuals.

Chemical structures may look very impressive.

\startbuffer
\placeformula[-]
\startformula
\startchemical[scale=small,width=fit,top=3000,bottom=3000]
  \chemical[SIX,SB2356,DB14,Z2346,SR3,RZ3,-SR6,+SR6,-RZ6,+RZ6]
           [C,N,C,C,H,H,H]
  \chemical[PB:Z1,ONE,Z0,DIR8,Z0,SB24,DB7,Z27,PE][C,C,CH_3,O]
  \chemical[PB:Z5,ONE,Z0,DIR6,Z0,SB24,DB7,Z47,PE][C,C,H_3C,O]
  \chemical[SR24,RZ24][CH_3,H_3C]
  \bottext{Compound A}
\stopchemical
\stopformula
\stopbuffer

\getbuffer

\CONTEXT\ relies on \METAPOST\ to draw these kind of
chemical structures. Although these chemical structures are
defined with only two or three commands, it takes some
practice to get the right results. This is how the input
looks:

\typebuffer

While using the module \type{chart} a definition of an
organogram may look like this:

\setupFLOWcharts
  [breedte=11\bodyfontsize,
   hoogte=3\bodyfontsize,
   dx=1\bodyfontsize,
   dy=2\bodyfontsize]

\setupFLOWlines
  [pijl=nee]

\startbuffer
\startFLOWchart[organogram]
  \startFLOWcell
    \shape    {action}
    \name     {01}
    \location {2,1}
    \text     {Hasselt}
    \connect  [bt]{02}
    \connect  [bt]{03}
    \connect  [bt]{04}
  \stopFLOWcell
  \startFLOWcell
    \shape    {action}
    \name     {02}
    \location {1,2}
    \text     {Mastenbroek}
  \stopFLOWcell
  \startFLOWcell
    \shape    {action}
    \name     {03}
    \location {2,2}
    \text     {Genne}
  \stopFLOWcell
  \startFLOWcell
    \shape    {action}
    \name     {04}
    \location {3,2}
    \text     {Zwartewaterklooster}
  \stopFLOWcell
\stopFLOWchart

\FLOWchart[organogram]
\stopbuffer

\typebuffer

The result will be:

\midaligned{\getbuffer}

\stopcomponent
