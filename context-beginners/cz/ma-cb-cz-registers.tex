% Translation:
\startcomponent ma-cb-cz-registers

\product ma-cb-cz

\environment ma-cb-env-cz

\chapter{Registers}

\index{register}

\Command{\tex{index}}
\Command{\tex{placeindex}}
\Command{\tex{completeindex}}
\Command{\tex{defineregister}}
\Command{\tex{placeregister}}
\Command{\tex{completeregister}}
\Command{\tex{setupregister}}

It is possible to generate one or more registers. By default
the command \type{\index} is available. If you want to add
a word to the index you type:

\starttyping
\index{town hall}
\stoptyping

The word {\em town hall} will appear as an index entry. An
index is sorted in alphabetical order by an auxilliary
program. Sometimes the index word does not appear in normal alphabetic
order. For example, entries such as symbols have to provide extra
sorting information in order to produce a correct alphabetical
list:

\starttyping
\index[minus]{$-$}
\stoptyping

Sometimes you have sub- or sub sub entries. These can be
defined as follows:

\starttyping
\index{town hall+location}
\index{town hall+architecture}
\stoptyping

You can generate your indexlist with:

\starttyping
\placeindex
\stoptyping

or

\starttyping
\completeindex
\stoptyping

The command \type{\index} is a predefined \CONTEXT\ command,
but of course you can also define your own registers.

\shortsetup{defineregister}

For example if you want to make a new register based on the
streets in Hasselt you could type:

\starttyping
\defineregister[street][streets]
\stoptyping

Now a new register command \type{\street} is available. An
new index entry could be \type{\street{Ridderstraat}}. To
produce a list of entries you could now use:

\starttyping
\placestreets
\completestreets
\stoptyping

You can alter the display of the registers with:

\shortsetup{setupregister}

\stopcomponent
