% Translation:
\startcomponent ma-cb-cz-setupcommands

\product ma-cb-cz

\environment ma-cb-env-cz

\chapter{Setup commands}

\index{set up}
\index{layout}

Global commands are placed in the setup area of your input
file, before \type{\starttext}. In \in{appendix}[comdefs]
there is a complete overview of the available commands and
their parameters.

The set up commands all have the same structure. They look
something like:

\startbuffer
\setup{setupparagraphs}
\stopbuffer

\getbuffer

A set up command consist of a more or less logical name and a
number of bracket pairs. Bracket pairs may be optional and
in that case the \type{[]} are typeset slanted {\tt \sl []}.
In the definition the bracket pairs may contain:

\starttyping
\setupacommand[.1.][.2.][..,..=..,..]
\stoptyping

The commas indicate that a list of parameters can be
enclosed. In the options list following the definition, the
\type{.1.} and \type{.2.} show the possible options that can
be set in the first and second bracket pair respectively.
The parameters and their possible values are placed in the
third bracket pair.

The default options and parameter values are underlined.
Furthermore you will notice that some values are typeset in
a slanted way: {\sl section}, {\sl name}, {\sl dimension},
{\sl number}, {\sl command} and {\sl text}. This indicates
that you can set the value yourself.

\starttabulate[|S||]
\NC section   \NC a section name like chapter, section, subsection etc. \NC\NR
\NC name      \NC an identifier (logical name) \NC\NR
\NC dimension \NC a dimension with a unit in \type{cm}, \type{pt},
                  \type{em}, \type{ex}, \type{sp} or \type{in} \NC\NR
\NC number    \NC an integer \NC\NR
\NC command   \NC a command \NC\NR
\NC text      \NC text \NC\NR
\stoptabulate

\stopcomponent
