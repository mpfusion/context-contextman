% Translation:
% busaj
\startcomponent ma-cb-cz-usersetups
\project ma-cb
\product ma-cb-cz
\environment ma-cb-env-cz

%\chapter{User specifications}
\chapter{Specifikace u�ivatele}

%\index{\tt cont-sys.tex}
\index{\tt cont-sys.tex}

%When \CONTEXT\ is run a number of predefined parameters is
%loaded. These parameters are set up in the file
%\type{cont-sys.tex}. Users can define their own preferences
%(housestyle) in this file. Be aware of the fact that
%\CONTEXT\ has to be able to find this file. The
%\type{readme} file that goes with the distribution tells
%some more about site specific setups. The most important
%addition to this file probbably concerns the output:
B�hem provozu \CONTEXT{}u se nastavuje mno�stv� implicitn�ch
parametr�. Tyto parametry jsou zadefinov�ny v~souboru
\type{cont-sys.tex}. U�ivatel� m��ou v~tomto souboru 
definovat sv� vlastn� (dom�c�) priority. Bu�te si v�domi toho,
�e \CONTEXT\ mus� b�t schopen naj�t tento soubor. Soubor 
\type{readme}, kter� je sou��st� distribuce, informuje v�ce o~konkr�tn�ch
nastaven�ch. Nejd�le�itej�� dopln�k tohoto souboru se pravd�podobn� t�k�
v�stupu:

\starttyping
\setupoutput[pdftex]
\stoptyping

%tells \CONTEXT\ to produce \PDF\ output instead of \DVI,
%while
sd�l� \CONTEXT{}u vytvo�it v�stup \PDF\ nam�sto \DVI,
zat�mco

\starttyping
\setupoutput[dvipsone,dviwindo]
\stoptyping

%sets things up for those programs. By default \DVIPS\ output
%is set up.
nastav� v�ci pro tyto programy. P�edvolen� je \DVIPS{} v�stup.

\stopcomponent
