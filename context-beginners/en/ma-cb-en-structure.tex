\startcomponent ma-cb-en-structure

\product ma-cb-en

\chapter{Defining a document}

Every document is started with \type{\starttext} and closed
with \type{\stoptext}. All textual input is placed between
these two commands and \CONTEXT\ will only process that
information.

Setup information is placed in the set up area just before
\type{\starttext}.

\startbuffer
\setupbodyfont[12pt]
\starttext
This is a one line document.
\stoptext
\stopbuffer

\typebuffer

Within the \type{\starttext} $\cdots$ \type{\stoptext} a
document can be divided into four main divisions:

\startitemize[n,packed]
\item front matter
\item body matter
\item back matter
\item appendices
\stopitemize

The divisions are defined with:

\starttyping
\startfrontmatter ... \stopfrontmatter
\startbodymatter  ... \stopbodymatter
\startbackmatter  ... \stopbackmatter
\startappendices  ... \stopappendices
\stoptyping

In the front matter as well as back matter section the
command \type{\chapter} produces an un-numbered header in
the table of contents. This section is mostly used for the
table of contents, the list of figures and tables, the
preface, the acknowledgements etc. This section often
has a roman page numbering.

The appendices section is used for (indeed) appendices.
Headers may be typeset in a different way; for example,
\type{\chapter} may be numbered alphabetically.

Section style can be set up with:

\shortsetup{setupsectionblock}

\stopcomponent
