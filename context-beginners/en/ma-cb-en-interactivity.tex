% 2006-10-10 Vit Zyka: new interactionmenu syntax
\startcomponent ma-cb-en-interactivity
\project ma-cb
\product ma-cb-en
\environment ma-cb-env-en

\chapter{Interactive mode in electronic documents}

\section{Introduction}

\index[reader]{\READER}
\index[exchange]{\EXCHANGE}
\index[distiller]{\DISTILLER}

Nowadays documents can be made electronically available for
consulting on a computer and displaying on a computer screen.

Interaction means that you can click on active areas and
jump to the indicated areas. For example if you consult a
register you can click on a (active) page number and you will
jump to the corresponding page.

Interaction relates to:

\startitemize[packed]
\item active chapter numbers in table of content
\item active page numbers in registers
\item active page numbers, chapter numbers and figure numbers in
      internal references to pages, chapters, figures etc.
\item active titles, page numbers, and chapter numbers in
      external references to other interactive documents
\item active menus as navigation tools
\stopitemize

Interactivity depends on the program you use to view the
interactive document. We assume that you will use \PDFTEX\
for producing a \PDF\ document directly or use \DISTILLER\
to convert a \POSTSCRIPT\ file into a \PDF\ one. It is
obvious that you will then use \READER, \EXCHANGE, or
\GHOSTVIEW\ for viewing.

\CONTEXT\ is a very powerful system for producing electronic
or interactive \PDF\ documents. However only a few standard
features are described in this chapter. As the authors of
this manual are planning to make all \CONTEXT\ related
manuals electronically (sources included) available, reverse
engineering is one of the options to become more acquainted
with the possibilities of \CONTEXT.

\section{Interactive mode}

\index{interactive mode}

\Command{\tex{setupinteraction}}

The interactive mode is activated by:

\shortsetup{setupinteraction}

For example:

\startbuffer
\setupinteraction
  [state=start,
   color=green,
   style=bold]
\stopbuffer

\typebuffer

The hyper links are now generated automatically and the
active words are displayed in bold green.

The interactive document is considerably bigger (in MB's)
than its paper cousin because hyperlinks consume space. You
will also notice that processing time becomes longer.
Therefore it is advisable to de-activate the interactive mode
as long as your document is under construction.

\section{Interaction within a document}

\index{interaction+internal}

\Command{\tex{in}}
\Command{\tex{at}}
\Command{\tex{goto}}

Earlier you have seen how to make a reference with
\type{\in} and \type{\at}. You may have wondered why you had
to type \type{\in{chapter}[chap:introduction]}. In the first
place {\em chapter} and its corresponding chapter number
will not be separated at line breaking. In the second place
the word {\em chapter} and its number are typeset
differently in the interactive mode. This gives the user a
larger clickable area.

In interactive mode there is one other command that has
little meaning in the paper variant.

\shortsetup{goto}

The curly braces contain text, the brackets contain a
reference (logical name, location).

\startbuffer
In \goto{Hasselt}[fig:cityplan] all streets are build in a
circular way.
\stopbuffer

\typebuffer

In the interactive document {\em Hasselt} will be green and
active. You will jump to a map of Hasselt.

\section{Interaction between documents}

\index{interaction+external}

\Command{\tex{from}}
\Command{\tex{useexternaldocument}}

It is possible to link one document to another. First you
have to state that you want to refer to another document.
This is done by:

\shortsetup{useexternaldocument}

The first bracket pair must contain a logical name of the
document, the second pair the file name of the other document
and the third pair is used for the title of the document.

For refering to these other documents you can use:

\shortsetup{from}

The curly braces contain text and the brackets contain the
reference.

Look at the example below.

\startbuffer
\useexternaldocument[hia][hasbook][Hasselt in August]

Most tourist attractions are described in \from[hia]. A description
of the Eui||feest is found in \from[hia::euifeest]. A description
of the \goto{Eui||feest}[hia::euifeest] is found in \from[hia]. The
eui||feest is described on \at{page}[hia::euifeest] in \from[hia].
See for more information \in{chapter}[hia::euifeest] in \from[hia].
\stopbuffer

\typebuffer

The \type{\useexternaldocument} is usually typed in the
set up area of your input file.

After processing your input file (at least two times to get
the references right), and the file \type{hasbook.tex},
you will have two \PDF\ documents. The references
above have the following meaning:

\startitemize[packed]
\item \type{\from[hia]} will produce the active title you gave
      in the third bracket pair of
      \type{\useexternaldocument} and is linked to the
      first page of \type{hasbook.pdf}
\item \type{\from[hia::euifeest]} will produce an active title
      and is linked to the page where chapter Eui||feest
      begins
\item \type{\goto{Eui||feest}[hia::euifeest]} will produce an
      active word {\em Eui||feest} and is linked to the page
      where chapter Eui||feest begins
\item \type{\at{page}[hia::euifeest]} will produce an active
      word {\em page} and page number and is linked to that
      page
\item \type{\in{chapter}[hia::euifeest]} will produce on
      active word {\em chapter} and chapter number and is
      linked to that chapter
\stopitemize

As you can see the \type{::} separates the (logical) file name
and the destination in that file.

\section{Menus}

You can define navigation tools with:

\shortsetup{defineinteractionmenu}

The first bracket pair is used for a logical name that can
be used to recall the menu. The second pair contains the
location on the screen. The third pair is used for setting
up the menu.

A typical menu definition might look like this:

%% \startbuffer
%% \setupcolors
%%   [state=start]

%% \setupinteraction
%%   [state=start,
%%    menu=on]

%% \defineinteractionmenu
%%   [mymenu]
%%   [right]
%%   [state=start,
%%    align=middle,
%%    background=screen,
%%    frame=on,
%%    width=\marginwidth,
%%    style=smallbold,
%%    color=]

%% \setupinteractionmenu
%%   [mymenu]
%%   [{Content[content]},
%%    {Index[index]},
%%    {\vfill},
%%    {Stop[ExitViewer]}]
%% \stopbuffer
\startbuffer
\setupcolors
  [state=start]

\setupinteraction
  [state=start,
   menu=on]

\defineinteractionmenu
  [mymenu]
  [right]
  [state=start,
   align=middle,
   background=screen,
   frame=on,
   width=\marginwidth,
   style=smallbold,
   color=]

\startinteractionmenu[mymenu]
  \but [content] Content \\
  \but [index] Index \\
  \vfill \\
  \but [ExitViewer] Exit viewer \\
\stopinteractionmenu

\setupheadertexts[{\interactionmenu[mymenu]}]
\stopbuffer

\typebuffer

This will produce a menu on the right hand side of every
screen. The menu buttons contain the text {\em Content}, {\em
Index} and {\em Stop} with respectively the following
functions: jump to the table of contents, jump to the index
and leave the viewer. The labels to obvious destinations
like \type{content} and \type{index} are predefined. Other
predefined destinations are \type{FirstPage},
\type{LastPage}, \type{NextPage} and
\type{PreviousPage}.

An action like \type{ExitViewer} is necessary to make an
electronic document self containing. Other predefined
actions you can use are \type{PrintDocument},
\type{SearchDocument} and \type{PreviousJump}. The meaning of
these actions is obvious.

Menus are set up with:

%\shortsetup{setupinteractionmenu}
\shortsetup{startinteractionmenu}

\stopcomponent
