\startcomponent ma-cb-en-fonts

\product ma-cb-en

\chapter{Fonts and font switches}

\section{Introduction}

\index{Computer Modern Roman}
\index{Lucida Bright}
\index{AMS}
\index{\cap{PS}--fonts}

The default font in \CONTEXT\ is the {\em Computer Modern
Roman} (\type{cmr}). You can also use Lucida Bright
(\type{lbr}) as a full alternative and symbols of the {\em
American Mathematical Society} (\type{ams}). Standard
PostScript fonts (\type{pos}) are also available.

\section{Fontstyle and size}

\index{font+style}
\index{font+size}

\Command{\tex{setupbodyfont}}
\Command{\tex{switchtobodyfont}}

You select the font family, style and size for a document
with:

\shortsetup{setupbodyfont}

If you typed \type{\setupbodyfont[sansserif,9pt]}
{\switchtobodyfont[ss,9pt] in the setup area of the input
file your text would look something like this.}

For changes in mid-document and on section level you
should use:

\shortsetup{switchtobodyfont}

\startbuffer
On November 10th (one day before Saint Martensday) the youth of
Hasselt go from door to door to sing a special song and they
accompany themselves with a {\em foekepot}. They won't leave
before you give them some money or sweets. The song goes like this:

\startnarrower
\switchtobodyfont[small]
\startlines
Foekepotterij, foekepotterij,
Geef mij een centje dan ga'k voorbij.
Geef mij een alfje dan blijf ik staan,
'k Zal nog liever naar m'n arrenmoeder gaan.
Hier woont zo'n rieke man, die zo vulle g�ven kan.
G�f wat, old wat, g�f die arme stumpers wat,
'k Eb zo lange met de foekepot elopen.
'k Eb gien geld om brood te kopen.
Foekepotterij, foekepotterij,
Geef mij een centje dan ga'k voorbij.
\stoplines
\stopnarrower
\stopbuffer

\typebuffer

Notice that \type{\startnarrower} $\cdots$
\type{\stopnarrower} is also used as a begin and end of the
fontswitch. The function of \type{\startlines} and
\type{\stoplines} in this example is obvious.

\start
\getbuffer
\stop

If you want an overview of the available font family you can
type:

\startbuffer
\showbodyfont[cmr]
\stopbuffer

\typebuffer

\getbuffer

\section{Style and size switch in commands}

In a number of commands one of the parameters is
\type{character} to indicate the desired typestyle. For
example:

\startbuffer
\setuphead[chapter][style=\tfd]
\stopbuffer

\typebuffer

In this case the character size for chapters is indicated with
a command \type{\tfd}. But instead of a command you could
use the predefined options that are related to the actual
typeface:

\startbuffer
normal  bold  slanted  boldslanted  type  mediaeval
small  smallbold  smallslanted  smallboldslanted smalltype
capital cap
\stopbuffer

\typebuffer

\section{Local font style and size}

\Command{\tex{rm}}
\Command{\tex{ss}}
\Command{\tex{tt}}
\Command{\tex{sl}}
\Command{\tex{bf}}
\Command{\tex{tfa}}
\Command{\tex{tfb}}
\Command{\tex{tfc}}
\Command{\tex{tfd}}

In the running text (local) you can change the {\em
typestyle} into roman, sans serif and teletype with
\type{\rm}, \type{\ss} and \type{\tt}.

You can change the {\em typeface} like italic and boldface
with \type{\sl} and \type{\bf}.

The {\em typesize} is available from 4pt to 12pt and is
changed with \type{\switchtobodyfont}.

The actual style is indicated with \type{\tf}. If you want
to change into a somewhat greater size you can type
\type{\tfa}, \type{\tfb}, \type{\tfc} and \type{\tfd}. An
addition of \type{a}, \type{b}, \type{c} and \type{d} to
\type{\sl}, \type{\it} and \type{\bf} is also allowed.

\startbuffer
{\tfc Mintage}

In the period from {\tt 1404} till {\tt 1585} Hasselt had its own
{\sl right of coinage}. This right was challenged by other cities,
but the {\switchtobodyfont[7pt] bishops of Utrecht} did not honour
these {\slb protests}.
\stopbuffer

\typebuffer

The curly braces indicate begin and end of style or size
switches.

\getbuffer

\section{Redefining fontsize}

\index{fontsize}

\Command{\tex{definebodyfont}}

For special purposes you can define your own fontsize.

\shortsetup{definebodyfont}

A definition could look like this:

\startbuffer
\definebodyfont[10pt][rm][tfe=Regular at 36pt]

{\tfe Hasselt!}
\stopbuffer

\typebuffer

Now \type{\tfe} will produce 36pt characters saying:
{\hbox{\getbuffer}}

\section{Small caps}

\index{small caps}

\Command{\tex{kap}}

Abbreviations like \PDF\ (\infull{PDF}) are printed in
pseudo small caps. A small capital is somewhat smaller than
the capital of the actual typeface. Pseudo small caps are
produced with:

\shortsetup{kap}

If you compare \type{PDF}, \type{\kap{PDF}} and \type{\sc pdf}:

\midaligned{PDF, \kap{PDF} and {\sc pdf}}

you can see the difference. The command \type{\sc} shows the
real small caps. The reason for using pseudo small caps
instead of real small caps is just a matter of taste.

\section{Emphasized}

\index{emphasized}

\Command{\tex{em}}

To emphasize words consistently throughout your document
you use:

\starttyping
\em
\stoptyping

Empasized words appear in a slanted style.

\startbuffer
If you walk through Hasselt you should {\bf \em watch out} for
{\em Amsterdammers}. An {\em Amsterdammer} is {\bf \em not} a
person from Amsterdam but a little stone pilar used to separate
sidewalk and road. A pedestrian should be protected by these
{\em Amsterdammers} against cars but more often people get hurt
from tripping over them.
\stopbuffer

\typebuffer

\getbuffer

{\em An emphasize within an emphasize is {\em normal} again
and a boldface emphasize looks like {\bf this or \em this}}.

\section{Teletype / verbatim}

\index{type}
\index{verbatim}

\Command{\tex{starttyping}}
\Command{\tex{type}}
\Command{\tex{setuptyping}}
\Command{\tex{setuptype}}

If you want to display typed text and want to keep your
line breaking exactly as it is you use:

\shortsetup{starttyping}

In the text you can use:

\shortsetup{type}

The curly braces enclose the text you want in teletype.
You have to be careful with \type{\type} because the
line breaking mechanism does not work anymore.

You can set up the `typing' with:

\shortsetup{setuptyping}
\shortsetup{setuptype}

\page

\section{Encodings}

This section is stil to be filled.

\stopcomponent
