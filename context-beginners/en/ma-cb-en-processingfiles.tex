\startcomponent ma-cb-en-processingfiles

\product ma-cb-en

\chapter{Processing steps}

\index[texutil]{\TEXUTIL}
\index[tuo]{{\tt tuo}--file}
\index{input file+processing}

During processing \CONTEXT\ writes information in the file
\type{myfile.tui}. This information is used in the next
pass. Part of this information is processed by the program
\TEXUTIL. Information on registers and lists are written in
the file \type{myfile.tuo}. The information in this file is
filtered and used (when necessary) by \CONTEXT.

\starttyping
texutil --references filename
\stoptyping

When \CONTEXT\ cannot find a figure, you can generate a
figure auxilliary file by saying:

\starttyping
texutil --figures *.*
\stoptyping

or whatever specification suits.

When one wants to convert \EPS\ illustrations to \PDF\ ones,
there is:

\starttyping
texutil --figures --epspage --epspdf
\stoptyping

One can use \TEXEXEC\ to run \CONTEXT:

\starttyping
texexec filename
\stoptyping

\CONTEXT\ runs as many times as needed to get the references
straight. One can also specify specific needs on the command
line. For instance if pdf output is desired you type:

\starttyping
texexec --pdf filename
\stoptyping

When in doubt, say \type{--help} and you get all the
information needed to proceed. A \TEXEXEC\ manual is
available.

\stopcomponent
