\startcomponent ma-cb-en-pages

\product ma-cb-en

\chapter{Page breaking and page numbering}

\index{page breaking}
\index{page numbering}

\Command{\tex{page}}
\Command{\tex{setuppagenumbering}}
\Command{\tex{startpostponing}}

A page can be enforced or blocked by:

\shortsetup{page}

The options can be stated within the brackets. The options
and their meaning are presented in \in{table}[tab:page
options]

\placetable
  []
  [tab:page options]
  {Page options.}
\starttable[|l|l|]
\HL
\NC \bf Option \NC \bf Meaning \NC\SR
\HL
\NC \type{yes}           \NC enforce a page \NC\FR
\NC \type{makeup}        \NC enforce a page without filling \NC\MR
\NC \type{no}            \NC no page \NC\MR
\NC \type{preference}    \NC prefer a new page here \NC\MR
\NC \type{bigpreference} \NC great preference for a new page here \NC\MR
\NC \type{left}          \NC next page is a left handside page \NC\MR
\NC \type{right}         \NC next page is a right handside page \NC\MR
\NC \type{disable}       \NC following commands have no effect \NC\MR
\NC \type{reset}         \NC following commands do have effect \NC\MR
\NC \type{empty}         \NC insert an empty page \NC\MR
\NC \type{last}          \NC add pages till even number is reached \NC\MR
\NC \type{quadruple}     \NC add pages till a multiple of four
  is reached  \NC\LR
\HL
\stoptable

Page numbering happens automatically but you can enforce
a page number with:

\starttyping
\page[25]
\stoptyping

Sometimes it is better to state a relative page number like
\type{[+2]} or \type{[-2]}.

The position of the page numbers on a page depend on your own
preferences and if it concerns a one sided or double sided
document. Page numbering can be set up with:

\shortsetup{setuppagenumbering}

The preferences are placed within the brackets.

Tables or figures may take up a lot of space. The placing of
these text elements can be postponed till the next page break.
This is done with:

\shortsetup{startpostponing}

\startbuffer
\startpostponing
\placefigure
  {A postponed figure.}
  {\externalfigure[ma-cb-16][width=\textwidth]}
\stoppostponing
\stopbuffer

\typebuffer

The figure will be placed at the top of next page and will
cause minimal disruption of the running text.

\getbuffer

\stopcomponent
