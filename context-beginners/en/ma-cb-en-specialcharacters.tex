\startcomponent ma-cb-en-specialcharacters

\product ma-cb-en

\chapter[special chars]{Special characters}

\index{special characters}

You have seen that \CONTEXT\ commands are preceded by a
\tex{} (backslash). This means that \tex{} has a
special meaning to \CONTEXT. Aside from \tex{} there are
other characters that need special attention when you want
them to appear in verbatim mode or in text mode.
\in{Table}[tab:specchars] gives an overview of these special
characters and what you have to type to produce them.

\let\normalunderscore=\_
\let\normaltilde     =\~

\placetable[here,force][tab:specchars]
  {Special characters (1).}
  \starttable[|c|c|c|c|c|]
  \HL
  \NC \bf \LOW{Special character} \NC \use2 \bf Verbatim  \NC \use2 \bf Text \NC\FR
  \NC                         \NC \bf Type \NC \bf To produce \NC \bf Type \NC \bf To produce \NC\LR
  \HL
  \NC \type{#} \NC \type{\type{#}} \NC \type{#} \VL \type{\#} \NC \# \NC\FR
  \NC \type{$} \NC \type{\type{$}} \NC \type{$} \VL \type{\$} \NC \$ \NC\MR
  \NC \type{&} \NC \type{\type{&}} \NC \type{&} \VL \type{\&} \NC \& \NC\MR
  \NC \type} \NC \type \NC \% \NC\LR
  \HL
  \stoptable

Other special characters have a meaning in typesetting
mathematical expressions and some can be used in
math mode only (see \in{chapter}[formulas]).

\let\normalbar=|
\placetable
  [here,force]
  [tab:special chars]
  {Special characters (2).}
  \starttable[|c|c|c|c|c|]
  \HL
  \NC \bf \LOW{Special character} \NC \use2 \bf Verbatim  \NC \use2 \bf Text \NC\FR
  \NC                         \NC \bf Type \NC \bf To produce \NC \bf Type \NC \bf To produce \NC\LR
  \HL
  \NC \type{+} \NC \type{\type{+}} \NC \type{+} \VL \type{$+$} \NC $+$ \NC\FR
  \NC \type{-} \NC \type{\type{-}} \NC \type{-} \VL \type{$-$} \NC $-$ \NC\MR
  \NC \type{=} \NC \type{\type{=}} \NC \type{=} \VL \type{$=$} \NC $=$ \NC\MR
  \NC \type{<} \NC \type{\type{<}} \NC \type{<} \VL \type{$<$} \NC $<$ \NC\MR
  \NC \type{>} \NC \type{\type{>}} \NC \type{>} \VL \type{$>$} \NC $>$ \NC\LR
  \HL
  \stoptable

\stopcomponent
