\startcomponent ma-cb-zh-figures

\product ma-cb-zh

\chapter[figures]{插圖}

\index[chatu]{插圖}
\seeindex[zhaopian]{照片}{插圖}
\index[fudongkuai]{浮動塊}

\Command{\tex{placefigure}}
\Command{\tex{startfiguretext}}
\Command{\tex{setupfigures}}
\Command{\tex{startcombination}}
\Command{\tex{setupfloats}}
\Command{\tex{setupcaptions}}
\Command{\tex{externalfigure}}

你可以通過下面命令來插入照片或者圖片:

\startbuffer
\placefigure
   [][fig:church]
   {Stephanus 教堂。}
   {\externalfigure[ma-cb-24][width=.4\textwidth]}
\stopbuffer

\typebuffer

編譯之後,\in{圖}[fig:church] 就會出現在首個可放置的地方。

\getbuffer

\type{\placefigure} 命令用來處理自動編號以及插圖上下的間距。
此外,這條命令預置了一個浮動環境,即\CONTEXT\ 會在當前頁面上
尋找是否有地方可以放得下這張圖片。如果找不到,這張圖片就會先
擱置起來等放到下一個地方,然後繼續處理後續文本,這張圖片就會
在你文檔中來回浮動,直至找到一個最佳位置安放。你可以在這條命令
之後的第一個方括號內改變浮動方式。

\type{\placefigure} 其實是下面命令的一個預先定義好的特例:

\shortsetup{placeblock}

\in{表}[tab:placefigure] 列出了它的選項說明。

\placetable
  [here]
  [tab:placefigure]
  {\type{\placefigure} 命令的選項說明。}
\starttable[|l|l|]
\HL
\NC \bf 選項 \NC \bf 表示                             \NC\SR
\HL
\NC here       \NC 可能的話,盡量將圖片放在當前位置     \NC\FR
\NC force      \NC 強制將圖片放置於當前位置            \NC\MR
\NC page       \NC 將圖片獨立放到一個頁面              \NC\MR
\NC top        \NC 將圖片放到頁面的頂部                \NC\MR
\NC bottom     \NC 將圖片放到頁面的底部                \NC\MR
\NC left       \NC 將圖片放到左側空白邊內              \NC\MR
\NC right      \NC 將圖片放到右側空白邊內              \NC\MR
\NC margin     \NC 將圖片放到頁面空白邊內              \NC\LR
\HL
\stoptable

第二對方括號用來放置交叉引用的標簽。你可以通過輸入下列命令來引用
這幅圖片:

\starttyping
\in{figure}[fig:church]
\stoptyping

第一對花括號是用來放置圖片標題用的,隨便輸什麼都可以。如果你不想要
標題也不想要自動編號,可以輸入\type{{none}}。圖片的標題可以用\type{\setupcaptions}
命令來設置,而自動編號可以用\type{\setupnumbering} 命令來設置或重設
({\processingverbatimtrue 見\in{段落}[floatingblocks]})。

第二對花括號用來定義圖片,指定外部圖片的路徑。

下面一個例子中,我們將展示如何在\type{\placefigure{}{}}
命令的最後那個花括號中定義\inframed[height=1em]{\processingverbatimtrue 中國}。

\startbuffer
\placefigure
  {帶框的中國。}
  {\framed{\tfd 中國}}
\stopbuffer

\typebuffer

結果顯示:

{\processingverbatimtrue
\getbuffer
}

然而,你要插入的圖片很多情況下都是由像Corel Draw 或者Illustrator
程序生成的,或者是由PhotoShop 的濾鏡加工的掃描照片。因此需要插入
的圖片都是文件。\CONTEXT\ 支持所有後端驅動支持的文件格式。如果你
用\PDFTEX\ 的話,它支持\type{JPG} 文件、\type{PNG} 文件、\type {PDF}
文件和\METAPOST\ 輸出文件(\type{MPS} 文件)。用戶一般可以讓\CONTEXT\
自動蒐尋最可能的文件類型。

你可以發現,{\processingverbatimtrue 在\in{圖}[fig:canals]} 中,
一張照片與一幅矢量圖合併為一張圖片了。

\startbuffer
\placefigure
  [here,force]
  [fig:canals]
  {哈塞爾特運河。}
  {\startcombination[2*1]
     {\externalfigure[ma-cb-03][width=.4\textwidth]}
         {一張點陣照片}
     {\externalfigure[ma-cb-00][width=.4\textwidth]}
         {一幅矢量圖片}
   \stopcombination}
\stopbuffer

\getbuffer

這張圖片可以由下列命令來實現:

\typebuffer

這張圖片中,兩副圖用以下命令合併起來了:

\shortsetup{startcombination}

\type{\startcombination} $\cdots$ \type{\stopcombination}
環境用來將兩幅圖片合成一幅。你也可以在方括號內輸入圖片的數量。
如果你想以上下的形式合併兩張圖片,可以輸入\type{[1*2]}。你可以想象
得出如果用\type{[3*2]}({\processingverbatimtrue\type{[行數*列數]}})
合併六幅圖片後的樣子。

上面的那幾個例子足以創建使用插圖的文檔了。有時候,你會想讓你的文檔中
的文本與圖片排列更緊湊,那麼,你還可以用:

\shortsetup{startframedtext}

圖片與表格的文本混排環境是預先定義好的:

\startbuffer
\startfiguretext
  [left]
  [fig:citizens]
  {none}
  {\externalfigure[ma-cb-18][width=.5\makeupwidth]}
   由於經濟事態,哈塞爾特的市民數量經常變動。例如岱迪斯瓦爾特運河大約在一
   八一零年挖掘,這條運河貫穿哈塞爾特市,因此貿易繁忙。這導致了在最近的十
   年中,整個城市的人口增長了將近百分之四十。現在,岱迪瓦爾特運河已經沒有
   任何商業價值了,整條運河成了旅遊勝地。不過還是可見曾經繁盛的跡象。
\stopfiguretext
\stopbuffer

\typebuffer

顯示結果在下面的圖旁邊。

\start
\setuptolerance[verytolerant]
\getbuffer
\stop

\shortsetup{externalfigure}

最後一對花括弧中放入的命令\type{\externalfigure} 可以讓你隨意
處理圖片。\type{\externalfigure} 有兩對方括號參數,第一對方括號
用來輸入不帶擴展名的文件名,第二對用來指定文件的格式以及尺寸大小。
如果你輸入下面命令,那結果如何應該不難想象:\footnote{{\processingverbatimtrue
見\at{第}{~頁}[marginpicture]}}

\startbuffer[marginpicture]
\inmargin
  {\externalfigure
     [ma-cb-23]
     [width=\marginwidth]}
\stopbuffer

\typebuffer[marginpicture]

你可以用下面命令來設置插圖的樣式:

\shortsetup{setupfloats}

用下面命令可以設置自動編號和標題格式:

\shortsetup{setupcaptions}

這些命令可以放在輸入文件的設置區,它們對所有浮動塊都有效應。

\startbuffer
\setupfloats
  [location=right]
\setupcaptions
  [location=top,
   style=boldslanted]

\placefigure
  {哈塞爾特市的特徵景觀。}
  {\externalfigure[ma-cb-12][width=8cm]}
\stopbuffer

\typebuffer

{\getbuffer}

\stopcomponent
