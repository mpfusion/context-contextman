\startcomponent ma-cb-zh-units

\product ma-cb-zh

\chapter[units]{單位}

\index[danwei]{單位}
\index[siguojibiaozhundanwei]{\cap{SI} 國際標準單位}

\Command{\tex{unit}}
\Command{\tex{permille}}
\Command{\tex{percent}}

爲了強制全文使用統一的單位,你可以列一份單位清單。
你可以在輸入文件的設置區定義好這些單位。

\CONTEXT\ 有個專門模塊包含了幾乎所有的\SI||國際單位,你可以
通過\type{\usemodule[units]} 來加載這個模塊,然後就可以調用
其中的單位,比如:

\startbuffer
\Meter \Per \Square \Meter
\Cubic \Meter \Per \Sec
\Square \Milli \Meter \Per \Inch
\Centi \Liter \Per \Sec
\Meter \Inverse \Sec
\Newton \Per \Square \Inch
\Newton \Times \Meter \Per \Square \Sec
\stopbuffer

\typebuffer

看上去要輸入的字符還真不少,不過這樣的確可以確保全文單位的統一。
\type{\unit} 命令還可以防止數值與單位在換行処分開,因爲一行結束
的地方放上數值,然後下一行開頭放上它的單位太不完美了。上面的例子
效果如下:

\startnarrower
\startlines
\getbuffer
\stoplines
\stopnarrower

你可以仿照下面例子來定義自己的單位:

\starttyping
\unit[Ounce]{oz}{}
\stoptyping

\unit[Ounce]{oz}{}

在其後的文檔中,你就可以通過輸入\type{15.6 \Ounce} 來得到15.6 \Ounce。

下面兩條特殊命令可以使\percent\ 與\permille\ 也以統一的形式排版出來:

\type{\percent} \crlf
\type{\permille}

\stopcomponent
