\startcomponent ma-cb-zh-color

\product ma-cb-zh

\chapter{顔色}

\index{顔色}

\Command{\tex{setupcolors}}
\Command{\tex{color}}
\Command{\tex{definecolor}}

我們可以設置文本的顔色。

\shortsetup{setupcolors}

要使用顔色,首先要通過以下語句激活:

\starttyping
\setupcolors[state=start]
\stoptyping

至此,我們就可以調用一些基本的顔色了(比如紅、綠和藍)。

\startbuffer
\startcolor[red]
哈塞爾特是個\color[green]{多姿多彩}的城市。
\stopcolor
\stopbuffer

\typebuffer

\getbuffer

如果用黑白打印機打印出來,你只能看到灰色的痕跡。在電子文檔中,就能看到預期的顔色。

你可以用下面命令來定義你自己的顔色:

\shortsetup{definecolor}

例如:

\startbuffer
\definecolor[darkred]   [r=.5,g=.0,b=.0]
\definecolor[darkgreen] [r=.0,g=.5,b=.0]
\stopbuffer

\typebuffer

這樣,我們就可以調用\type{darkred} 和\type{darkgreen} 了。

\stopcomponent
