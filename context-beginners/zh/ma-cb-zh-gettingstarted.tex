\startcomponent ma-cb-zh-gettingstarted

\product ma-cb-zh

\chapter{如何編譯輸入文件}

\index[shuruwenjian]{輸入文件+編譯}
\index[dvifile]{\type{dvi}--文件}
\index[pdffile]{\type{pdf}--文件}

如果你希望變異 \CONTEXT\ 輸入文件,就應該在命令行提示符
後面輸入:

\starttyping
context filename
\stoptyping

這條批處理命令\type{context} 是否存在取決於你使用的系統。
你可以咨詢系統管理員用哪條命令來啓動\CONTEXT。如果你的文
件名是\type{myfile.tex},你可以輸入:

\starttyping
context myfile
\stoptyping

如果正確安裝了\TEXEXEC\ 的話,你還可以用:

\starttyping
texexec --pdf myfile
\stoptyping

擴展名\type{.tex} 不是必須要輸入的。

敲擊\Enter\ 鍵後,編譯就開始了。\CONTEXT\ 會在你顯示器上
顯示整個編譯過程。如果成功編譯的話,我們就會返回到命令提
示符,\CONTEXT\ 將生成\type{dvi} 文件或者\type{pdf} 文件。

如果編譯出錯了---比如你在輸入文件中輸入的是\type{\stptext}
而不是\type{\stoptext}---\CONTEXT\ 就會在終端給出一個\type{ ? }
提示符,然後告訴你編譯過程發生了錯誤。它會提供一些關於出錯
類型和錯誤發生所在哪一行的信息。

在\type{ ? } 之下,你可以輸入:

\starttabulate[|||]
\NC \type{H} \NC 得到你錯誤的幫助信息 \NC\NR
\NC \type{I} \NC 插入正確的\CONTEXT\ 命令 \NC\NR
\NC \type{Q} \NC 放棄處理並進入批處理模式 \NC\NR
\NC \type{X} \NC 退出運行模式 \NC\NR
\NC \Enter   \NC 跳過本次錯誤 \NC\NR
\stoptabulate

大多數情況下,你可以輸入\Enter\ 讓編譯繼續下去。然後你再編輯
輸入文件以糾正錯誤。

有些錯誤會在你屏幕上生成一個\type{ * } 符號,然後停止編譯。那是
由於你的輸入文件中有個重大的錯誤。你無法跳過這個錯誤,你衹能
選擇輸入\type{\stop} 或者{\sc Ctrl}~Z。然後程序就終止了,你才能
修正這個錯誤。

在編譯你的輸入文件時,\CONTEXT\ 也會告訴你它正在對你的文檔進行
什麼操作。例如,它會顯示頁碼和編譯步驟的信息。此外,它還會給出
警告。這些警告依照排版的順序,在斷行失敗的時候給出提示。編譯過
程的全部信息都存儲在\type{log} 文件中,利用它可以察看警告和錯誤
以及它們在你文件的哪一行出現的。

編譯順利結束後,\CONTEXT\ 便生成一個新文件,擴展名為\type{.dvi}
或\type{.pdf},即生成了\type{myfile.dvi} 或\type{myfile.pdf}。
\type{dvi} 是Device Indepent 的縮寫,表示這個文件被相配的PostScript
(\PS) 打印機驅動處理後可以用於打印或預覽。\PDF\ 是Portable
Document Format 的縮寫,它是種打印與預覽跟操作平臺無關的格式。

\stopcomponent
