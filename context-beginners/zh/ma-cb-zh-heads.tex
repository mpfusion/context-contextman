\startcomponent ma-cb-zh-heads

\product ma-cb-zh

\chapter[headers]{標題}

\index[biaoti]{標題}

\Command{\tex{chapter}}
\Command{\tex{paragraph}}
\Command{\tex{subparagraph}}
\Command{\tex{title}}
\Command{\tex{subject}}
\Command{\tex{subsubject}}
\Command{\tex{setuphead}}
\Command{\tex{setupheads}}

文檔的結構由它的標題決定,標題{\processingverbatimtrue 由\in{表}[tab:headers]} 所示
的命令創建:

\placetable[here][tab:headers]{標題。}
\starttable[|l|l|]
\HL
\NC \bf 編號的標題   \NC \bf 不編號的標題   \NC\SR
\HL
\NC \type{\chapter}       \NC \type{\title}           \NC\FR
\NC \type{\section}       \NC \type{\subject}         \NC\MR
\NC \type{\subsection}    \NC \type{\subsubject}      \NC\MR
\NC \type{\subsubsection} \NC \type{\subsubsubject}   \NC\MR
\NC $\cdots$              \NC $\cdots$                \NC\LR
\HL
\stoptable

\shortsetup{chapter}
\shortsetup{section}
\shortsetup{subsection}
\shortsetup{title}
\shortsetup{subject}
\shortsetup{subsubject}

這些命令將會生成定義好字體大小、字體樣式和上下間距的標題。

生成標題的命令有好幾個參數,比如:

\starttyping
\title[hasselt-by-night]{哈塞爾特之夜}
\stoptyping

或者

\starttyping
\title{哈塞爾特之夜}
\stoptyping

上面例子中那對方括號中的參數是可選的,用來作為內部引用的標簽。
例如想要引用這個標題的話,可以輸入\type{\at{page}[hasselt-by-night]}
這樣的命令。

當然,這些標題命令可以自己選擇參數,甚至可以定義自己的標題
命令。\type{\setuphead} 和\type{\definehead} 這兩條命令就是
用來自定義標題用的。

\shortsetup{definehead}

\shortsetup{setuphead}


\startbuffer
\definehead
  [myheader]
  [section]

\setuphead
  [myheader]
  [numberstyle=bold,
   textstyle=bold,
   before=\hairline\blank,
   after=\nowhitespace\hairline]

\myheader[myhead]{Hasselt makes headlines}
\stopbuffer

\typebuffer

這樣就定義了一條\type{\myheader} 命令,它繼承了\type{\section}
命令的屬性。它的效果如下:

\getbuffer

還有條命令你也應該知道,即\type{\setupheads}。這條命令可以用來
設置編號標題的編號方式。如果你輸入:

\startbuffer
\setupheads
  [alternative=inmargin,
   separator=--]
\stopbuffer

%\typebuffer

所有的編號都會顯示在頁邊空白処。第1.1 節會顯示為1--1。

像\type{\setupheads} 這樣的命令一般都放在你輸入文件的
設置區。

\shortsetup{setupheads}

\stopcomponent
