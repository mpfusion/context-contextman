\startcomponent ma-cb-zh-math

\product ma-cb-zh

\chapter[formulas]{排版數學公式}

\section{前言}

\index[shuxuegongshi]{數學公式}

\TEX\ {\em 就是}用來排版數學公式的程序。不過這一章並非你所
期待那樣大量地講述排版數學公式。我建議你多看些內容深點的
關於\TEX\ 的公式排版方面的書籍。例如:\footnote{本前言涉及
僅限於Arthur Samuel 的小冊子:{\em \TEX niques}。}

\startitemize[packed]
\item D.E. Knuth 撰寫的{\em The \TeX Book}
\item S. Levy 與R. Seroul 撰寫的{\em The Beginners Book of \TeX}
\stopitemize

此外,\CONTEXT\ 的數學模塊以及其相伴的手冊將會及時發佈。

\section{數學排版}

\index[shuxuemoshi]{數學模式}
\index[duligongshimoshi]{獨立公式模式}
\index[xingwenshuxuemoshi]{行文數學模式}

通常,排版普通文本與排版數學公式的慣例是不同的。\TEX“理解”這些慣例,
並能夠在生成文檔時加以區分處理。我們完全可以依賴\TEX\
來為我們創造高質量的數學公式結果。

數學公式的排版有很多慣例:

\startitemize[n,packed]

\item 數學公式中的字母以$math\ italic$ 字體排版(注意,不要與
      普通文本中的{\it italic} 字體混淆了)。

\item 使用如希臘字母($\alpha$,$\chi$)和數學符號($\leq$,
      $\geq$,$\in$)這樣的符號。

\item 字母之間的間距跟普通正文中的間距不同。

\item 調整數學表達式的方式與調整連續不斷正文的方式不同。

\item 上標、下標會自動調整為小號字體,比如$a^{b}_{c}$。

\item 某些符號在行文數學模式與獨立公式模式中顯示效果不同。

\stopitemize

排版數學公式時,你要進入一個叫做數學模式的模式下,在這個
模式中,我們可以依靠plain \TEX||命令來定義數學表達式。

數學模式有兩种情況:行文數學模式和獨立公式模式。行文數學模式
由\type{$} 和\type{$} 激活,而獨立公式模式由\type{$$} 和\type{$$}
激活。

\startbuffer
哈塞爾特佔地42,05 \Square \Kilo \Meter,如果你把這塊面積看作是一個圓,
而哈塞爾特市場為其圓心$M$ 的話,你可以用公式${{1}\over{4}} \pi r^2$ 算
得這個圓的直徑。
\stopbuffer

\typebuffer

結果顯示為:

\getbuffer

在${{1}\over{4}} \pi r^2$ 中的很多個\type{{}}(分組)是分割該
表達式的運算符的要點。如果你省略了最外面的花括號:\type{${1}\over{4} \pi r^2$},
你得到的結果${1}\over{4} \pi r^2$ 將不是你預期的。

字母與數字是以三種不同大小排版的:正文字體大小$a+b$,標注字體
大小$\scriptstyle a+b$ 與小標註字體大小$\scriptscriptstyle a+b$。
可以由\type{\scriptstyle} 命令與\type{\scriptscriptstyle} 命令
來改變。

形如$\int$ 和$\sum$ 這樣的符號在行文與獨立數學模式中顯示的形式
是不同的。比如我們輸入\type {$\sum_{n=1}^{m}$}
或\type {$\int_{-\infty}^{+\infty}$},我們將得到{$\sum_{n=1}^{m}$}
與{$\int_{-\infty}^{+\infty}$}。如果我們輸入\type {$$\sum_{n=1}^{m}$$}
與\type {$$\int_{-\infty}^{+\infty}$$} 來激活獨立公式模式,
我們會得到:

\startformula
\sum_{n=1}^{m} \quad {\mbox{與}} \quad \int_{-\infty}^{+\infty}
\stopformula

你也可以用\type {\nolimits} 與\type{\limits} 這兩條命令來改變\type{\sum}
與\type{\int}的顯示形式:

\startformula
\sum_{n=1}^{m}\nolimits \quad {\mbox{與}} \quad \int_{-\infty}^{+\infty}\limits
\stopformula

排版分數可以用\type{\over} 命令。在\CONTEXT\ 中,還有一條
命令可供使用:\type {\frac}。比如${\frac{a}{1+b}}+c$ 可以
由\type {${\frac{a}{1+b}}+c$} 來得到。

其他將兩部分曡放起來的命令有:

\startbuffer[atop]
${a} \atop {b}$
\stopbuffer
\startbuffer[choose]
${n+1} \choose {k}$
\stopbuffer
\startbuffer[brack]
${m} \brack {n}$
\stopbuffer
\startbuffer[brace]
${m} \brace {n-1}$
\stopbuffer

\starttabulate[|l|l|l|l|]
\NC \type {\atop}
\NC \typebuffer[atop]
\NC \mathstrut\getbuffer[atop]
\NC
\NC\NR
\NC \type {\choose}
\NC \typebuffer[choose]
\NC
\NC \mathstrut\getbuffer[choose]
\NC\NR
\NC \type {\brack}
\NC \typebuffer[brack]
\NC \mathstrut\getbuffer[brack]
\NC
\NC\NR
\NC \type {\brace}
\NC \typebuffer[brace]
\NC
\NC \mathstrut\getbuffer[brace]
\NC\NR
\stoptabulate

如果我們在(~) 和$\{~\}$ 這種定界符前面分別插入\type{\left}
和\type{\right} 命令的話,\TEX\ 能夠將它們自動放大至合適尺寸。
比如我們輸入\type{$$1+\left(\frac{1}{1-x^{x-2}}\right)^3$$}
就會得到:

\startformula
1+\left(\frac{1}{1-x^{x-2}}\right)^3
\stopformula

下標跟上標是通過輸入\quote{\type{_}} 和\quote{\type{^}} 調用的。
它們衹對緊跟的下一個字符有效果,如果上標或者下標中包含多個字符的話,
就需要以$\{$~$\}$ 將它們組合起來。

你還可以在定界符前插入\type{\bigl},\type{\Bigl},\type{\biggl},\type{\Biggl}
以及右邊它們對應的命令,這樣可以使用確定大小的定界符。更大的定界符
可以通過構造一個\type{\vbox},然後插入\type{\left} 和\type{\right}。
當我們輸入這樣一個表達式:\type{$$\left(\vbox to
16pt{}x^{2^{2^{2^{2}}}}\right)$$},就可以得到:

\startformula
\left(\vbox to 16pt{}x^{2^{2^{2^{2}}}}\right)
\stopformula

下列定界符在獨立公式模式下會自動縮放:

\starttabulate[|l|l|l|l|l|l|l|l|]
\NC \type{\lfloor}      \NC $\lfloor$
\NC \type{\langle}      \NC $\langle$
\NC \type{\vert}        \NC $\vert$
\NC \type{\downarrow}   \NC $\downarrow$
\NC\NR
\NC \type{\rfloor}      \NC $\rfloor$
\NC \type{\rangle}      \NC $\rangle$
\NC \type{\Vert}        \NC $\Vert$
\NC \type{\Downarrow}   \NC $\Downarrow$
\NC\NR
\NC \type{\lceil}       \NC $\lceil$
\NC \type{/}            \NC $/$
\NC \type{\uparrow}     \NC $\uparrow$
\NC \type{\updownarrow} \NC $\updownarrow$
\NC\NR
\NC \type{\rceil}       \NC $\rceil$
\NC \type{\backslash}   \NC $\backslash$
\NC \type{\Uparrow}     \NC $\Uparrow$
\NC \type{\Updownarrow} \NC $\Updownarrow$
\NC\NR
\stoptabulate

獨立公式模式下,我們應該衹排版一個分數,其它的都要換成\type{a/b}
形式。下列公式:

\startformula
a_0 + {\frac{a}{a_1 + \frac{1}{a_2}}}
\stopformula

是由\type{$$a_0+{\frac{a}{a_1+\frac{1}{a_2}}}$$} 得到的,
不過有一種更佳的方式是輸入\type{$$a_0 + {\frac{a}{a_1 + 1/a_2}}$$}
來得到:

\startformula
a_0 + {\frac{a}{a_1 + 1/a_2}}
\stopformula

此外,我們還可以用\type{\displaystyle} 命令。如果我們輸入:\type {$$a_0 +
{\frac{a}{a_1 + \frac{1} {\strut \displaystyle a_2}}}$$},
就能得到:

\startformula
a_0 + {\frac{a}{a_1 + \frac{1}{\displaystyle a_2}}}
\stopformula

下面我們來演示一下\type{\matrix},\type{\pmatrix},\type{\ldots},\type{\cdots}
與\type{\cases} 這幾條命令,不過並不會在這兒作更多說明。

\startbuffer[b]
$$
\stopbuffer

\startbuffer[a]
A=\left(\matrix{x-\lambda & 1         & 0         \cr
                0         & x-\lambda & 1         \cr
                0         & 0         & x-\lambda \cr}\right)
\stopbuffer

\typebuffer[b,a,b] \startformula\getbuffer[a]\stopformula

\startbuffer[a]
A=\left|\matrix{x-\mu& 1     & 0     \cr
                0    & x-\mu & 1     \cr
                0    & 0     & x-\mu \cr}\right|
\stopbuffer

\typebuffer[b,a,b] \startformula\getbuffer[a]\stopformula

\startbuffer[a]
A=\pmatrix{a_{11} & a_{12} & \ldots & a_{1n} \cr
           a_{21} & a_{22} & \ldots & a_{2n} \cr
           \vdots & \vdots & \ddots & \vdots \cr
           a_{m1} & a_{m2} & \ldots & a_{mn} \cr}
\stopbuffer

\typebuffer[b,a,b] \startformula\getbuffer[a]\stopformula

\startbuffer[a]
A=\pmatrix{a_{11} & a_{12} & \ldots & a_{1n} \cr
           a_{21} & a_{22} & \ldots & a_{2n} \cr
           \vdots & \vdots & \ddots & \vdots \cr
           a_{m1} & a_{m2} & \ldots & a_{mn} \cr}
\stopbuffer

\typebuffer[b,a,b] \startformula\getbuffer[a]\stopformula

\startbuffer[a]
|x|=\cases{ x, & \mbox{當} $x\geq0$; \cr
           -x, & \mbox{其它情況}     \cr}
\stopbuffer

\typebuffer[b,a,b]  \startformula\getbuffer[a]\stopformula

如果在數學表達式中要插入普通文本的話,有以下幾點需要考慮到。
首先,數學模式中空格是不會排版的,所以我們必須在空格之前添
加\type{ \ }(反斜杠)來強行插入。其次,我們還要切換字體,
因爲不能用$math\ italic$ 字體來顯示文本內容,而要用實際
正文字體。因此,我們可以在\CONTEXT\ 中輸入\type{$$x^3+{\tf lower\
order\ terms}$$} 來得到:

\startformula
x^3+{\tf lower\ order\ terms}
\stopformula

像$\sin$ 或$\tan$ 這樣的函數,排版時所用的字體在\TEX\ 中
已經預定義過了:

\starttabulate[|l|l|l|l|l|l|l|l|]
\NC \type{\arccos} \NC \type{\cos} \NC \type{\csc} \NC \type{\exp} \NC \type{\ker} \NC \type{\limsup} \NC \type{\min} \NC \type{\sinh} \NC\NR
\NC \type{\arcsin} \NC \type{\cosh} \NC \type{\deg} \NC \type{\gcd} \NC \type{\lg} \NC \type{\ln} \NC \type{\Pr} \NC \type{\sup} \NC\NR
\NC \type{\arctan} \NC \type{\cot} \NC \type{\det} \NC \type{\hom} \NC \type{\lim} \NC \type{\log} \NC \type{\sec} \NC \type{\tan} \NC\NR
\NC \type{\arg} \NC \type{\coth} \NC \type{\dim} \NC \type{\inf} \NC \type{\liminf} \NC \type{\max} \NC \type{\sin} \NC \type{\tanh} \NC\NR
\stoptabulate

如我們這樣輸入三角函數\type {$$\sin2\theta=2\sin\theta\cos\theta$$}
或者極限函數\type {$$\lim_{x \to 0} {\frac{\sin x}{x}}=1$$},就能得到:

\startformula
\sin 2\theta=2\sin\theta\cos\theta
\quad {\tf or} \quad
\lim_{x\to0}{\frac{\sin x}{x}}=1
\stopformula

數學表達式中的對齊需要引起特別的注意。有時在多行表達式中,我們
想在“$=$”処對齊,就可以通過\type{\eqalign} 命令實現。比如我們輸入:

\startbuffer
$$\eqalign{
      ax^2+bx+c &= 0                                \cr
      x         &= \frac{-b \pm \sqrt{b^2-4ac}}{2a} \cr}$$
\stopbuffer

\typebuffer

就能得到:

\startformula
\eqalign{
      ax^2+bx+c &= 0                                \cr
      x         &= \frac{-b \pm \sqrt{b^2-4ac}}{2a} \cr}
\stopformula

有時我們希望在多処地方都對齊。請觀察下面例子中的第二行公式,
並注意它的定義:

\startbuffer
$$\eqalign{
      ax+bx+\cdots+yx+zx &         = x(a +b+ \cdots \cr
                         &\phantom{= x(a~}+y+z)     \cr
                         &         = y              \cr}$$
\stopbuffer

\typebuffer

結果為:

\startformula
\eqalign{
      ax+bx+\cdots+yx+zx &         = x(a +b+ \cdots \cr
                         &\phantom{= x(a~}+y+z)     \cr
                         &         = y              \cr}
\stopformula

類似於\type{\phantom} 命令,還有\type{\hphantom} 命令與\type{\vphantom}
命令,前者的高度和深度為零,後者的寬度為零。

你可以完全依靠\TEX\ 自己調整數學表達式中的間隔。不過,某些情況下
你可能想更改間隔。用下面的命令即可實現:

\starttabulate[|l|r|]
\NC \type{\!} \NC $-\frac{1}{6}$\type{\quad} \NC\NR
\NC \type{\,} \NC $\frac{1}{6}$\type{\quad}  \NC\NR
\NC \type{\>} \NC $\frac{2}{9}$\type{\quad}  \NC\NR
\NC \type{\;} \NC $\frac{5}{18}$\type{\quad} \NC\NR
\stoptabulate

這些“間隔”與\type {\quad} 有確定的比例關係,而一個\type {\quad}
就是當前字體下,一個大寫字母“M”的寬度。

\type{\prime} 命令用來排版區別同一變量不同的值符號。
比如$y_1^\prime+y_2^{\prime\prime}$ 對應的代碼
是\type{$y_1^\prime+y_2^{\prime\prime}$}。

表達式$\root 3 \of {x^2+y^2}$ 由\type{$\root 3 \of {x^2+y^2}$}
生成。

本節的最後,我們來講講\type{\mathstrut} 命令,我們可以
用這條命令保持形式統一,比如可以用在開根符號中。我們可以
輸入\type{$\sqrt{\mathstrut a}+\sqrt{\mathstrut d}+\sqrt{\mathstrut y}$}
來得到$\sqrt{\mathstrut a}+\sqrt{\mathstrut d}+\sqrt{\mathstrut y}$,
而不會像$\sqrt{a}+\sqrt{d}+\sqrt{y}$ 這樣參差不齊。

\in{附錄}[overviews] 列出了全部數學命令。

\section{放置公式}

\index[gongshi]{公式}

\Command{\tex{placeformula}}
\Command{\tex{startformula}}
\Command{\tex{setupformulae}}

我們可以用下面命令來排版自動編號的公式:

\shortsetup{placeformula}
\shortsetup{startformula}

擧兩個例子:

\startbuffer
\placeformula[formula:aformula]
\startformula
   y=x^2
\stopformula

\placeformula
\startformula
  \int_0^1 x^2 dx
\stopformula
\stopbuffer

\typebuffer

\getbuffer

\type {\startformula} $\cdots$ \type{\stopformula} 命令取代了
公式開頭和結束的\type{$$}。因此,如果你輸入:

\startbuffer
$$
\int_0^1 x^2 dx
$$
\stopbuffer

\typebuffer

排版出來結果是一個{\em 顯示}在頁面中間的表達式,不過那可不像上一個
例子那樣排列整齊。

\getbuffer

\type{\placeformula} 命令用來處理公式周圍的間隔與公式的編號。方括號
中的內容是可選的,用來作爲交叉引用的標簽,或者打開關閉自動編號。

\startbuffer
\placeformula[first one]
\startformula
  y=x^2
\stopformula

\placeformula[middle one]
\startformula
  y=x^3
\stopformula

\placeformula[last one]
\startformula
  y=x^4
\stopformula
\stopbuffer

\getbuffer

\in{公式}[middle one] 是由下面命令排版的:

\startbuffer
\placeformula[middle one]
  \startformula
     y=x^3
  \stopformula
\stopbuffer

\typebuffer

\type{[middle one]} 這個標簽可以讓我們引用這個公式,只需
輸入{\processingverbatimtrue\type{\in{公式}[middle one]}}
即可實現對該公式的引用。

如果你不打算對該公式自動編號,可以輸入:

\type{\placeformula[-]}

公式的編號是通過\type{\setupnumbering} 命令來設置的。本手冊中
的編號設置為\type{\setupnumbering[way=bychapter]},表示章編號
後跟公式編號,且如果開始新一章,公式編號即被清零。為了全文的統一,
表格、圖片等的編號也是如此設置的。因此,你可以在輸入文件的設置區
使用\type{\setupnumbering} 命令。

公式可以由下面命令來設置:

\shortsetup{setupformulae}

\stopcomponent
