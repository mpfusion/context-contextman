\startcomponent ma-cb-zh-itemizations

\product ma-cb-zh

\chapter[itemize]{羅列項目}

\index[luoliexiangmu]{羅列項目}
\index[xiangmuliebiao]{項目列表}

\Command{\tex{startitemize}}
\Command{\tex{setupitemize}}
\Command{\tex{definesymbol}}
\Command{\tex{item}}
\Command{\tex{head}}

安排信息的途徑之一就是挨個列舉或者概述各項信息。列表
命令形如:

\shortsetup{startitemize}

例如:

\startbuffer
\startitemize[R,packed,broad]
\item 無道人之短,無說己之長,施人慎勿念,受施慎勿忘;
\item 俗譽不足慕,唯仁為紀綱,隱身而後動,謗議庸何傷;
\item 無使名過實,守愚聖所臧,柔弱生之徒,老氏誡剛強;
\item 在涅貴不緇,曖曖內含光,硜硜鄙夫介,悠悠故難量;
\item 慎言節飲食,知足勝不祥,行之苟有恆,久久自芬芳。
\stopitemize
\stopbuffer

\typebuffer

在\type{\startitemize} $\cdots$ \type{\stopitemize} 環境
中,可以以\type{\item} 新增一條項目。在\type{\item} 後面
的空格不能省略。在上述的例子中,\type{R} 指定了項目列表
一大寫羅馬數字編號,\type{packed} 使行間距盡可能小。而\type{broad}
參數會調整項目符後的水平間距。上述代碼得到的結果為:

\getbuffer

羅列項目需要兩個步驟,因此你必須要將你的輸入文件編譯兩次
才能得到最佳效果。方括號中可以放置項目符號樣式和其他局部
設置。

\placetable
  [here]
  [tab:itemsetup]
  {項目列表中的符號。}
\starttable[|l|l|]
\HL
\NC \bf 參數 \NC \bf 符號   \NC\SR
\HL
\NC 1            \NC $\bullet$            \NC\FR
\NC 2            \NC $-$                  \NC\MR
\NC 3            \NC $\star$              \NC\MR
\NC $\vdots$     \NC $\vdots$             \NC\MR
\NC n            \NC 1 2 3 4 $\cdots$     \NC\MR
\NC a            \NC a b c d $\cdots$     \NC\MR
\NC A            \NC A B C D $\cdots$     \NC\MR
\NC r            \NC i ii iii iv $\cdots$ \NC\MR
\NC R            \NC I II III IV $\cdots$ \NC\LR
\HL
\stoptable

你也可以通過\type{\definesymbol} 命令自己定義鍾意的列表
符號。例如你輸入下列命令:

\startbuffer
\definesymbol[5][$\clubsuit$]

\startitemize[5,packed]
\item 四十年來家國,三千裡地山河。鳳閣龍樓連霄漢,玉樹瓊枝作煙蘿。幾曾識干戈。
\item 一旦歸為臣虜,沉腰潘鬢消磨。最是倉皇辭廟日,教坊猶奏離別歌。垂淚對宮娥。
\stopitemize
\stopbuffer

\typebuffer

結果為:

\getbuffer

有些情況下,你需要在列表項目中放置標題。那就應該將\type{\item}
命令換作\type{\head} 命令。

\startbuffer
《九歌》者,屈原之所作也。昔楚國南郢之邑,沅、湘之間,其俗信鬼而好祠。其祠,
必作歌樂鼓舞以樂諸神。屈原放逐,竄伏其域,懷憂苦毒,愁思沸鬱。出見俗人祭祀之
禮,歌舞之樂,其詞鄙陋。因為作《九歌》之曲,上陳事神之敬,下見己之冤結,託之
以風諫。故其文意不同,章句雜錯,而廣異義焉。

\startitemize

\head 東皇太一

      吉日兮辰良,穆將愉兮上皇。撫長劍兮玉珥,璆鏘鳴兮琳琅。瑤席兮玉瑱,盍將
      把兮瓊芳。蕙肴蒸兮蘭藉,奠桂酒兮椒漿。揚枹兮拊鼓,疏緩節兮安歌,陳竽瑟
      兮浩倡。靈偃蹇兮姣服,芳菲菲兮滿堂。五音紛兮繁會,君欣欣兮樂康。

\head 雲中君

      浴蘭湯兮沐芳,華采衣兮若英。靈連蜷兮既留,爛昭昭兮未央。蹇將憺兮壽宮,
      與日月兮齊光。龍駕兮帝服,聊翱遊兮周章。靈皇皇兮既降,猋遠舉兮雲中。覽
      冀州兮有餘,橫四海兮焉窮。思夫君兮太息,極勞心兮忡忡。

\head 湘君

      君不行兮夷猶,蹇誰留兮中洲?美要眇兮宜修,沛吾乘兮桂舟。令沅湘兮無波,
      使江水兮安流!望夫君兮未來,吹參差兮誰思!駕飛龍兮北征,邅吾道兮洞庭。
      薜荔柏兮蕙綢,蓀橈兮蘭旌。望涔陽兮極浦,橫大江兮揚靈。揚靈兮未極,女嬋
      媛兮為余太息。橫流涕兮潺湲,隱思君兮陫側。桂櫂兮蘭枻,斲冰兮積雪。采薜
      荔兮水中,搴芙蓉兮木末。心不同兮媒勞,恩不甚兮輕絕。石瀨兮淺淺,飛龍兮
      翩翩。交不忠兮怨長,期不信兮告余以不閒。朝騁騖兮江皋,夕弭節兮北渚。鳥
      次兮屋上,水周兮堂下。捐余玦兮江中,遺余佩兮醴浦。采芳洲兮杜若,將以遺
      兮下女。時不可兮再得,聊逍遙兮容與。

\head 湘夫人

      帝子降兮北渚,目眇眇兮愁予。嫋嫋兮秋風,洞庭波兮木葉下。登白薠兮騁望,
      與佳期兮夕張。鳥萃兮蘋中,罾何為兮木上。沅有茞兮醴有蘭,思公子兮未敢言。
      荒忽兮遠望,觀流水兮潺湲。麋何食兮庭中?蛟何為兮水裔?朝馳余馬兮江皋,
      夕濟兮西澨。聞佳人兮召予,將騰駕兮偕逝。築室兮水中,葺之兮荷蓋。蓀壁兮
      紫壇,播芳椒兮成堂。桂棟兮蘭橑,辛夷楣兮葯房。罔薜荔兮為帷,擗蕙櫋兮既
      張。白玉兮為鎮,疏石蘭兮為芳。芷葺兮荷屋,繚之兮杜衡。合百草兮實庭,建
      芳馨兮廡門。九嶷繽兮並迎,靈之來兮如雲。捐余袂兮江中,遺余褋兮醴浦。搴
      汀洲兮杜若,將以遺兮遠者。時不可兮驟得,聊逍遙兮容與。

\head 大司命

      廣開兮天門,紛吾乘兮玄雲。令飄風兮先驅,使涷雨兮灑塵。君迴翔兮以下,踰
      空桑兮從女。紛總總兮九州,何壽夭兮在予!高飛兮安翔,乘清氣兮御陰陽。吾
      與君兮齋速,導帝之兮九坑。靈衣兮被被,玉佩兮陸離。壹陰兮壹陽,眾莫知兮
      余所為。折疏麻兮瑤華,將以遺兮離居。老冉冉兮既極,不寖近兮愈疏。乘龍兮
      轔轔,高駝兮沖天。結桂枝兮延佇,羌愈思兮愁人。愁人兮柰何,願若今兮無虧。
      固人命兮有當,孰離合兮可為?

\head 少司命

      秋蘭兮麋蕪,羅生兮堂下。綠葉兮素枝,芳菲菲兮襲予。夫人自有兮美子,蓀何
      以兮愁苦!秋蘭兮青青,綠葉兮紫莖。滿堂兮美人,忽獨與余兮目成。入不言兮
      出不辭,乘回風兮載雲旗。悲莫悲兮生別離,樂莫樂兮新相知。荷衣兮蕙帶,儵
      而來兮忽而逝。夕宿兮帝郊,君誰須兮雲之際?與女遊兮九河,衝風至兮水揚波。
      與女沐兮咸池,晞女髮兮陽之阿。望美人兮未來,臨風怳兮浩歌。孔蓋兮翠旍,
      登九天兮撫彗星。竦長劍兮擁幼艾,蓀獨宜兮為民正。

\head 東君

      暾將出兮東方,照吾檻兮扶桑。撫余馬兮安驅,夜皎皎兮既明。駕龍輈兮乘雷,
      載雲旗兮委蛇。長太息兮將上,心低佪兮顧懷。羌聲色兮娛人,觀者憺兮忘歸。
      緪瑟兮交鼓,簫鍾兮瑤虡,鳴箎兮吹竽,思靈保兮賢姱。翾飛兮翠曾,展詩兮會
      舞。應律兮合節,靈之來兮蔽日。青雲衣兮白霓裳,舉長矢兮射天狼。操余弧兮
      反淪降,援北斗兮酌桂漿。撰余轡兮高駝翔,杳冥冥兮以東行。

\head 河伯

      與女遊兮九河,衝風起兮橫波。乘水車兮荷蓋,駕兩龍兮驂螭。登崑崙兮四望,
      心飛揚兮浩蕩。日將暮兮悵忘歸,惟極浦兮寤懷。魚鱗屋兮龍堂,紫貝闕兮朱宮。
      靈何為兮水中,乘白黿兮逐文魚。與女遊兮河之渚,流澌紛兮將來下。子交手兮
      東行,送美人兮南浦。波滔滔兮來迎,魚鱗鱗兮媵予。

\head 山鬼

      若有人兮山之阿,被薜荔兮帶女羅。既含睇兮又宜笑,子慕予兮善窈窕。乘赤豹
      兮從文狸,辛夷車兮結桂旗。被石蘭兮帶杜衡,折芳馨兮遺所思:「余處幽篁兮
      終不見天,路險難兮獨後來。」表獨立兮山之上,雲容容兮而在下。杳冥冥兮羌
      晝晦,東風飄兮神靈雨。留靈脩兮憺忘歸,歲既晏兮孰華予。采三秀兮於山間,
      石磊磊兮葛蔓蔓。怨公子兮悵忘歸,君思我兮不得閒。山中人兮芳杜若,飲石泉
      兮蔭松柏。君思我兮然疑作。雷填填兮雨冥冥,猿啾啾兮又夜鳴。風颯颯兮木蕭
      蕭,思公子兮徒離憂。

\head 國殤

      操吳戈兮被犀甲,車錯轂兮短兵接。旌蔽日兮敵若雲,矢交墜兮士爭先。
      凌余陣兮躐余行,左驂殪兮右刃傷。霾兩輪兮縶四馬,援玉枹兮擊鳴鼓。
      天時墜兮威靈怒,嚴殺盡兮棄原野。出不入兮往不反,平原忽兮路超遠。
      帶長劍兮挾秦弓,首身離兮心不懲。誠既勇兮又以武,終剛強兮不可凌。
      身既死兮神以靈,子魂魄兮為鬼雄。

\head 禮魂

      成禮兮會鼓,傳芭兮代舞,姱女倡兮容與。春蘭兮秋菊,長無絕兮終古。

\stopitemize
\stopbuffer

\typebuffer

\type{\head} 命令可以用\type{\setupitemize} 來設置。萬一
出現分頁的情況,\type{\head} 就會出現在新頁面上。

上述的例子顯示結果如下:

\getbuffer

\in{表}[tab:tablesetup] 詳細說明了設置項目列表的參數。

\placetable
  [here,force]
  [tab:tablesetup]
  {項目列表設置參數。}
\starttable[|l|l|]
\HL
\NC \bf 設置 \NC \bf 表示                           \NC\SR
\HL
\NC standard   \NC 標準(全局)設置                 \NC\FR
\NC packed     \NC 項目之間不插入豎直間隙           \NC\MR
\NC serried    \NC 項目符號與文本之間不插入水平間隙 \NC\MR
\NC joinedup   \NC 整個列表環境前後不插入豎直間隙   \NC\MR
\NC broad      \NC 項目符號與文本之間插入水平間隙   \NC\MR
\NC inmargin   \NC 項目符號放入頁邊空白內           \NC\MR
\NC atmargin   \NC 項目符號放入頁邊空白邊           \NC\MR
\NC stopper    \NC 項目符號後插入句點               \NC\MR
\NC columns    \NC 將項目分在多欄內                 \NC\MR
\NC intro      \NC 禁止在緒言後分頁                 \NC\MR
\NC continue   \NC 項目的編號從上一次計數処繼續     \NC\LR
\HL
\stoptable

你可以在\type{\startitemize} 中設置參數,不過,為了全文的一致性,
你可以在\type{\setupitemize} 中統一設置。

\type{columns} 參數一般和數字一起使用。如果你輸入:

\startbuffer
\startitemize[n,columns,four]
\item 趙錢孫李
.
.
.
\item 奚范彭郎
\stopitemize
\stopbuffer

\typebuffer

就會顯示:

\startbuffer
\startitemize[n,columns,four]
\item 趙錢孫李
\item 周吳鄭王
\item 馮陳褚衛
\item 蔣沈韓楊
\item 朱秦尤許
\item 何呂施張
\item 孔曹嚴華
\item 金魏陶姜
\item 戚謝鄒喻
\item 柏水竇章
\item 雲蘇潘葛
\item 奚范彭郎
\stopitemize
\stopbuffer

\getbuffer

某些情況下,你會想要在項目列表之後插入一小段文字,然後再繼續
上一次列表的編號。這時,你可以輸入%
\type{\startitemize[continue,columns,three,broad]},
這樣就會繼續上一次的編號,而將項目分放在三欄內。

\startbuffer
\startitemize[continue, columns, three, broad]
\item 魯韋昌馬
\item 苗鳳花方
\item 俞任袁柳
\item 酆鮑史唐
\item 費廉岑薛
\item 雷賀倪湯
\item 滕殷羅畢
\item 郝鄔安常
\item 樂于時傅
\item 皮卞齊康
\item 伍余元卜
\item 顧孟平黃
\item 和穆蕭尹
\item 姚邵湛汪
\item 祁毛禹狄
\item 米貝明臧
\stopitemize
\stopbuffer

\getbuffer

\type{broad} 參數會在項目符號與文本之間插入更大的水平間隙。

\shortsetup{setupitemize}

項目列表中的列表會自動以正確形式排版出來。比如:

\startbuffer
在荷蘭,城市可以決定一系列的稅收制度。因此,每個城鎮之間的日常開銷都是不一樣
的。超過百分之五十的稅收都是互不相同的,例如:

\setupitemize[2][width=5em]
\startitemize[n]

\item 房地產所有權稅

      房地產所有權稅分為兩部分:

      \startitemize[a,packed]
      \item 所有者稅
      \item 承租者稅
      \stopitemize

      如果房地產沒有出租,則這兩部分稅收均由房地產所有者支付。

\item 狗証費用

      飼養狗的主人需要支付費用。當狗死亡或者轉賣後,狗主人必須通知市政厛。

\stopitemize
\stopbuffer

\typebuffer

這樣,\type{\setupitemize[2][width=5em]} 命令設置好了第二級
項目列表中項目符號與文本之間的水平距離。

結果顯示:

\start
\getbuffer
\stop

\stopcomponent
