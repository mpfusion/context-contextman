\startcomponent ma-cb-zh-commands

\product ma-cb-zh

\chapter{定義命令與宏}

\CONTEXT\ 是基於\TEX\ 的一組宏。\TEX\
既是個印刷排版系統,也是種編程語言。這就意味著如果你需要這種靈活性的話,
你可以自己編寫程序。

你可以通過下面的命令來定義一條新命令:

\shortsetup{define}

下面的示範將會說明如何使用這條命令。

也許你有份佈滿圖片的文檔,而你實在懶得輸入:

\startbuffer
\placefigure
  [here,force]
  [fig:logical name]
  {標題。}
  {\externalfigure[filename][width=5cm]}
\stopbuffer

\typebuffer

你可以用幾個參數來定義一條你自己的命令,比如:

\startitemize[packed]
\item 標簽名
\item 標題
\item 文件名
\stopitemize

你可以這樣定義與調用你的命令:

\startbuffer
\define[3]\myputfigure
  {\placefigure
     [here,force][fig:#1]
     {#2}{\externalfigure[#3][width=5cm]}}

\myputfigure{lion}{這頭荷蘭獅是名守衛。}{ma-cb-13}
\stopbuffer

\typebuffer

方括號中間的數 \type{[3]} 表示你將使用三個參數:\type{#1}, \type{#2} 和 \type{#3}。
在調用\type{\myputfigure} 時你必須將這三個參數輸入到花括弧中。結果如下:

\getbuffer

這樣,你可以利用自己的創造力編寫出非常複雜的命令。

除了定義命令,你還可以定義\type{\start} $\cdots$ \type{\stop} 命令組。

\shortsetup{definestartstop}

例如:

\startbuffer
\definestartstop
   [stars]
   [commands={\inleft{\hbox to \leftmarginwidth
                    {\leaders\hbox{$\star$}\hfill}}},
    before=\blank,
    after=\blank]

\startstars
{\em 洛陽}處天下之中,挾{\em 殽}、{\em 黽}之阻,當{\em 秦}、
{\em 隴}之襟喉,而{\em 趙}、{\em 魏}之走集,蓋四方必爭之地
也。天下常無事則已,有事則{\em 洛陽}必先受兵。予故嘗曰:
「{\em 洛陽}之盛衰,天下治亂之候也。」

方{\em 唐貞觀}、{\em 開元}之間,公卿貴戚開館列第於東都者,
號千有餘邸。及其亂離,繼以五季之酷,其池塘竹樹,兵車蹂蹴,
廢而為丘墟;高亭大榭,煙火焚燎,化而為灰燼,與{\em 唐}共滅
而俱亡者,無餘處矣。予故嘗曰:「園囿之興廢,{\em 洛陽}盛衰
之候也。」

且天下之治亂,候於{\em 洛陽}之盛衰而知;{\em 洛陽}之盛衰,
候於園囿之興廢而得。則《名園記》之作,予豈徒然哉?

嗚呼!公卿大夫方進於朝,放乎一己之私以自為,而忘天下之治忽,
欲退享此,得乎?{\em 唐}之末路是矣!
\stopstars
\stopbuffer

\typebuffer

結果如下:

\getbuffer

\stopcomponent
