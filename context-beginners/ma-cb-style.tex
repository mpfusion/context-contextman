\startenvironment ma-cb-style

\def\GenericDir{../generic}
\setupexternalfigures[directory={\GenericDir}]

% The strange mix of english and dutch is due to the fact
% that much of the core of \CONTEXT\ is still dutch.
%
% Beware, this module looks awful. This is due to the fact
% that we use one setup for many interface languages and many
% alternative layouts. A single language setup looks much
% better and readible.

\usemodule[res-trace] % so that we know what graphics are used

\usemodule[units]
\usemodule[chemic]
\usemodule[chart]

\unprotect

% The layout dimensions are based on the A4 paper dimensions.
% That way users can print this manual themselves.

\setuplayout
  [\c!backspace=3cm,
   \c!margin=2cm,
   \c!margindistance=.5cm,
   \c!width=15cm,
   \c!topspace=2cm,
   \c!header=2cm,
   \c!footer=2cm,
   \c!height=25.7cm]

% There will be three versions:
%
% \startopsomming
% \som  bound: the real paper one
% \som  print: the user printable one
% \som  screen: the interactive version
% \stopopsomming
%
% This mode mechanism is rather new but painfully simple.

\startmode[bound]

\setuplayout
  [%\c!marking=\v!on,
   %\c!scale=.8,
   \c!location=\v!doublesided]

\stopmode

% The lucida fonts look a bit more informal.

\startmode[atpragma]

    \usetypescriptfile[type-buy]
    \usetypescript[lucida][texnansi]
    \setupbodyfont[lucida,10pt]

\stopmode

\startnotmode[atpragma]

    \usetypescriptfile[type-buy]
    \usetypescript[palatino][\defaultencoding]
    \setupbodyfont[palatino,10pt]

\stopnotmode

% We enable colors and define an aditional color, used in the
% chapter headings.

\setupcolors
  [\c!state=\v!start]

% \doifmode{bound}
%   {\setupcolors[\c!conversie=\v!altijd]}

\definecolor
  [headrule][r=0,g=1,b=0]

% Let's keep the text compact.

\setupwhitespace
  [\v!medium]

\setupblank
  [\v!medium]

% We indent verbatim with the default indenting value.

\setuptyping
  [\c!margin=\v!standard,
   \c!blank=\v!medium]

% Manuals as usual need a bit more tolerance, because a lot
% of in||line verbatim is used.

\setuptolerance
  [\v!verytolerant]

% This manual makes heavy use of backgrounds. During a run,
% about 800 metaclips are generated.

\defineoverlay
  [KopAchtergrond]
  [\MPclipOne{\overlaywidth}{\overlayheight}{10pt}{gray}{red}]

\defineoverlay
  [MargeAchtergrond]
  [\MPclipOne{\overlaywidth}{\overlayheight}{7.5pt}{gray}{red}]

\defineoverlay
  [IndexAchtergrond]
  [\MPclipOne{\overlaywidth}{\overlayheight}{5pt}{gray}{red}]

\defineoverlay
  [StatusAchtergrondL]
  [\MPclipFourL{\overlaywidth}{\overlayheight}{5pt}]

\defineoverlay
  [StatusAchtergrondR]
  [\MPclipFourR{\overlaywidth}{\overlayheight}{5pt}]

\defineoverlay
  [TitelAchtergrond]
  [\MPclipOne{\overlaywidth}{\overlayheight}{20pt}{gray}{red}]

\defineoverlay
  [AuteurAchtergrond]
  [\MPclipOne{\overlaywidth}{\overlayheight}{15pt}{gray}{red}]

\defineoverlay
  [IntroAchtergrond]
  [\MPclipOne{\overlaywidth}{\overlayheight}{5pt}{lightgray}{lightred}]

\defineoverlay
  [MenuAchtergrond]
  [\MPclipTwoA{\overlaywidth}{\overlayheight}{3pt}{3pt}{red}{white}]

\defineoverlay
  [SetupAchtergrond]
  [\MPclipTwoB{\overlaywidth}{\overlayheight}{5pt}{8pt}{blue}{lightgray}]

\defineoverlay
  [NummerAchtergrond]
  [\MPclipFive{\overlaywidth}{\overlayheight}{30pt}{5pt}]

\defineoverlay
  [FiguurAchtergrond]
  [\MPclipSeven{\overlaywidth}{\overlayheight}{2pt}]

% We precede footnotes by a nice line. Due to preprocessing
% this is a to be a 'use' graphic, not a 'resuse' one! When
% I got a more clearer vision of the concept, 'reused'
% graphics will be possible too. (The solution is rather
% simple, but sort of fuzzy.)

\unexpanded\def\VoetnootLijn
  {\dimen0=.4\makeupwidth
   \expanded{\MPclipThree{\the\dimen0}{2pt}}}

\setupfootnotes
  [\c!rule=\VoetnootLijn]

% This looks ok, but in a manual like this, it is confusing.
%
% \starttypen
% \stelplaatsblokin
%   [\v!figuur]
%   [\c!background=FiguurAchtergrond,
%    \c!frame=\v!off,\c!backgroundoffset=3pt]
% \stoptypen
%
% So, I decided to comment this one out.

% Chapter titles have a fancy shape around them. Because we
% have a lot of small chapters, we don't go to a new page.

\setuphead
  [\v!chapter]
  [\c!command=\HeadCommand,
   \v!appendix\c!label=none,
   \c!page=,
   \c!before=\vskip36pt plus 6pt minus 6pt,
   \c!after=\vskip24pt]

% Titles look the same, but here we go to a new page.

\setuphead
  [\v!title]
  [\c!command=\HeadCommand,
   \c!page=\v!right,
   \c!before=\vskip36pt plus 6pt minus 6pt,
   \c!after=\vskip24pt]

% The next definition is not that complicated, once one knows
% a bit of \TEX. Watch the \type{\getpagestatus} and
% \type{\ifrightpage}.

% \def\HeadCommand#1#2%
%   {\hbox to \hsize
%      {\getpagestate
%       \doifmodeelse{screen}
%         {\hss}
%         {\ifrightpage\hss\fi}%
%       \framed
%         [\c!background=KopAchtergrond,
%          \c!width=\v!fit,
%          \c!height=\v!fit,
%          \c!frame=\v!off,
%          \c!strut=\v!no,
%          \c!offset=24pt,
%          \c!align=\v!middle]
%         {\doifmode{*sectionnumber}
%            {#1\kern.5em% strut niet geset, zou wel moeten
%             \hbox{\color[headrule]{\vrule\!!width1pt\!!height1.5\ht\strutbox\!!depth1.25\dp\strutbox}}%
%             \kern.5em}%
%          #2}%
%       \doifmodeelse{screen}
%         {\hss}
%         {\ifrightpage\else\hss\fi}}}

\def\HeadCommand#1#2%
  {\alignedline{\v!outer}{\v!left}
     {\framed
        [\c!background=KopAchtergrond,
         \c!width=\v!fit,
         \c!height=\v!fit,
         \c!frame=\v!off,
         \c!strut=\v!no,
         \c!offset=24pt,
         \c!align=\v!middle]
        {\doifmode{*sectionnumber}
           {#1\kern.5em% strut niet geset, zou wel moeten
            \hbox{\color[headrule]{\vrule\!!width1pt\!!height1.5\ht\strutbox\!!depth1.25\dp\strutbox}}%
            \kern.5em}%
         #2}}}

% The current chapter number is typeset in the (outer) margin
% and slowly moves down. We could have directly put it in the
% margin but using the footermargin as starting point works
% better.

\setupfootertexts
  [\v!margin]
  [][\PlaatschapterStatus]

\unexpanded\def\PlaatschapterStatus
  {\determineheadnumber[\v!chapter]%
   \ifnum\currentheadnumber>0
     \vbox to \makeupheight
       {\scratchcounter=\lastpage
        \advance\scratchcounter by -\realpageno
        \vskip2cm
        \vskip0pt plus \realpageno cm
        \edef\StatusAchtergrond
          {StatusAchtergrond%
%           \ifodd\realpageno R\else\ifdubbelzijdig L\else R\fi\fi}%
           \ifodd\realpageno R\else\ifdoublesided L\else R\fi\fi}%
        \framed
          [\c!background=\StatusAchtergrond,
           \c!width=36pt,
           \c!height=72pt,
           \c!frame=\v!off]
          {\lower.5\dp\strutbox\hbox
             {\bfb\getmarking[\v!chapter\v!number]}}
        \vskip0pt plus \scratchcounter cm
        \vskip2cm}
   \fi}

% The index is put on a double collumned grid. The numbers is
% surrounded by a shape.

\setupregister
  [\v!index]
  [\c!command=\IndexCommando,
   \c!before={\blank[\v!line]},
   \c!after=]

\unexpanded\def\IndexCommando#1%
  {\fittoptobaselinegrid
   \framed
     [\c!background=IndexAchtergrond,
      \c!width=36pt,
      \c!height=\v!fit,
      \c!align=\v!middle,
      \c!frame=\v!off,
      \c!offset=4pt]
     {#1}}

% The very first version used keywords in the margin. The
% next settings applied to the level one margin word (we can
% stack margin words, you see).
%
% \starttyping
% \setupinmargin
%   [\c!location=\v!left]
%
% \setupinmargin
%   [1]
%   [\c!background=MargeAchtergrond,
%    \c!width=\v!fit,
%    \c!height=\v!fit,
%    \c!align=\v!middle,
%    \c!style=\tfx\setupinterlinespace,
%    \c!offset=8pt]
% \stoptyping
%
% When bound, we use a double sided layout and put the
% pagenumber in the margin, enhanced by a fancy background.

\startmode[bound]

\setuppagenumbering[\c!alternative=\v!doublesided]

\stopmode

\setuppagenumbering
  [\c!location={\v!footer,\v!middle},
   \c!command=\NummerCommando]

\unexpanded\def\NummerCommando#1%
  {\framed
     [\c!background=NummerAchtergrond,
      \c!frame=\v!off,
      \c!offset=6pt]
     {\lower.5\dp\strutbox\hbox spread 60pt{\hss#1\hss}}}

% We put the chapter title in the head. If we wouldn't have
% to center, the more simple setting would be:
%
% \starttypen
% \setupheadertexts[\v!chapter][]
% \stoptypen

\setupheadertexts
  []

\setupheadertexts
  [{\hfill\getmarking[\v!chapter]\hfill}]
  []

% Guess what the next one does.

\setupitemgroup
  [\v!itemize]
  [1][\v!autointro]

% Indeed, it prevents breaks between one||liners and
% following itemizations.

% The cover page is a real piece of art. Just watch.

\newbox\CoverBackgroundBox

\def\CoverBackground
  {\ifvoid\CoverBackgroundBox
     \global\setbox\CoverBackgroundBox=\vbox to \paperheight
       {\hsize\paperwidth
        \emergencystretch3em
        \iflocation
          \switchtobodyfont[5pt]
        \else
          \switchtobodyfont[7pt]
        \fi
        \parfillskip0pt
        \setupsetup
          [\c!command=\test]
        \def\test##1%
          {\dontleavehmode
           \framed
             [\c!background=IntroAchtergrond,
              \c!frame=\v!off]
             {\type{##1}}%\writetexcommand{##1}}%
           \hskip1.25em plus 1em minus 1em}
% beter :
%
% \oninterlineskip
% \baselineskip \vsize plus 1fill minus 1fill
% \parfillskip0pt
% \leftskip3pt
% \rightskip3pt
% \leavevmode\plaatselkesetup\unskip
%
        \beginofshapebox
          \leftskip3pt
          \rightskip3pt
          \plaatselkesetup
        \endofshapebox
        \doreshapebox
          {\box\shapebox}
          {\penalty\shapepenalty}
          {\kern\shapekern}
          {\vfil}
        \kern3pt
        \vfilneg
        \flushshapebox
        \vfilneg
        \kern3pt}%
   \fi
   \copy\CoverBackgroundBox}

\defineoverlay[geintje][\CoverBackground]

% In interactive mode we make the text area active. Click and
% go to the table of contents.

\unexpanded\def\TitleButton% zou iets standaards moeten zijn
  {\button                 % nog eens iets achter page
     [\c!height=\vsize,    % moeten kunnen zetten
      \c!width=\hsize,
      \c!frame=\v!off]
     {}%
     [contents]}

\def\TitlePage#1#2#3%
  {\setupbackgrounds
     [\v!rightpage]
     [\c!background=geintje]
   \setuptexttexts
     [\TitleButton]
     []
   \startmakeup
     [\v!standard]
     [\c!doublesided=\v!empty,
      \c!headerstate=\v!none,
      \c!footerstate=\v!none]
   \hbox to \hsize
     {\hss
      \definedfont[SansBold at 40pt]% \setfont lsb at 40pt    % change this into something available
      \framed
        [\c!background=TitelAchtergrond,
         \c!width=\v!fit,
         \c!height=\v!fit,
         \c!align=\v!middle,
         \c!frame=\v!off,
         \c!offset=40pt]
        {#1}}
   \vfill
   \hbox to \hsize
     {\definedfont[SansBold at 14pt]% \setfont lsb at 14pt    % change this into something available
      \framed
        [\c!background=AuteurAchtergrond,
         \c!frame=\v!off,
         \c!width=\v!fit,
         \c!height=\v!fit,
         \c!align=\v!middle,
         \c!offset=20pt]
        {#2}
      \hss}
   \hbox to \hsize
     {\hss
      \definedfont[SansBold at 20pt]% \setfont lsb at 20pt    % change this into something available
      \framed
        [\c!background=AuteurAchtergrond,
         \c!frame=\v!off,
         \c!width=\v!fit,
         \c!height=\v!fit,
         \c!align=\v!middle,
         \c!offset=35pt]
        {#3\\PRAGMA ADE}}
   \stopmakeup
   %\setuplayout[\c!marking=\v!off]
   \setuptexttexts
   \setupbackgrounds
     [\v!rightpage]
     [\c!background=]
   \doifmode{screen}
     {\setupbackgrounds
        [\v!page]
        [\c!background=\v!screen,
         \c!backgroundscreen=.95]
      \setupbackgrounds
        [\v!text][\v!text]
        [\c!backgroundoffset=.25cm,
         \c!depth=.125cm,
         \c!background=\v!color,
         \c!backgroundcolor=white]}}

% The backpage uses the same background and overlays a piece
% of text.

\def\StartBackPage
  {\page[\v!yes,\v!blank,\v!left]
   \setupbackgrounds
     [\v!leftpage]
     [\c!background=geintje]
   \setuptexttexts
     [\TitleButton][]
   \startmakeup
     [\v!standard]
     [\c!page=,
      \c!doublesided=\v!no,
      \c!headerstate=\v!none,
      \c!footerstate=\v!none]
   \setuptolerance
     [\v!verytolerant]
   \vfill
   \framed
     [\c!background=\v!color,
      \c!backgroundcolor=white,
      \c!frame=\v!off,
      \c!offset=10pt,
      \c!corner=\v!round,
      \c!width=.4\makeupwidth,
      \c!height=\textheight,
      \c!align=\v!middle,
      \c!strut=\v!no]
     \bgroup
     \vfil}

\def\StopBackPage
  {\vfil
   \egroup
   \hfill
   \vfill
   \stopmakeup}

% We draw a nice line between columns. The next command does
% the job. Of course a normal line can be set more easily,
% but here we hook in a command.

\unexpanded\def\KolommenLijn
  {\vsmash
     {\hbox to 36pt
        {\hss
         \vbox
           {\vskip-\maxcolumnheight
            \dimen0=\maxcolumnheight
            \advance\dimen0 by \maxcolumndepth
            \expanded{\MPclipSix{\the\dimen0}{2pt}}}%
         \hss}}}

% To save space we don't start chapters on a new page, except
% in appendices and the introduction. These settings happen
% in dedicated setups sections (see later). We also add some
% white space between table of content entries.

\setupsectionblock
  [\v!frontmatter]
  [\c!page=\v!right,
   \c!before={\setups{introduction}\writebetweenlist[\v!chapter]{\blank[\v!line]}}]

\setupsectionblock
  [\v!bodymatter]
  [\c!page=\v!right,
   \c!before={\setups{bodymatter}\writebetweenlist[\v!chapter]{\blank[\v!line]}}]

\setupsectionblock
  [\v!appendix]
  [\c!page=\v!right,
   \c!before={\setups{appendix}\writebetweenlist[\v!chapter]{\blank[\v!line]}}]

\setupsectionblock
  [\v!backmatter]
  [\c!page=\v!right,
   \c!before={\setups{backmatter}\writebetweenlist[\v!chapter]{\blank[\v!line]}}]

\startsetups introduction
  \setuphead[\v!chapter][\c!page=\v!right]
\stopsetups

\startsetups bodymatter
  \setuphead[\v!chapter][\c!page=]
\stopsetups

\startsetups appendix
  \setuphead[\v!chapter][\c!page=\v!right]
\stopsetups

%\setuplabeltext [\csname s!\currentmainlanguage\endcsname] [\v!appendix=]

\startsetups backmatter
  \setuphead[\v!chapter][\c!page=\v!right]
\stopsetups

% In normal documents one will never find awful things like
% below. Because we want an international setup, we just call
% the chapters in an indirect way.

\def\Introduction#1% nothing funny yet
  {\getvalue{\v!title}[intro]{#1}} % == \titel[intro]{#1}

% In a previous version I aligned the columns to the middle.
% Watch the column postprocessor routine!

\def\TableOfContents#1%
  {\getvalue{\v!title}[contents]{#1}
   \bgroup
   %\def\postprocesscolumnline##1%
   %  {\ifodd\currentcolumn
   %     \hss##1\relax
   %   \else
   %     \relax##1\hss
   %   \fi}%
   \startcolumns[\c!n=2,\c!distance=36pt,\c!rule=\KolommenLijn]
   \placelist
     [\v!chapter]
     [\c!criterium=\v!all,
      \c!before=,
      \c!after=\vskip0pt] % makes it breakable
   \stopcolumns
   \egroup}

\def\NormalIndex#1%
  {\getvalue{\v!chapter}[subind]{#1}
   \bgroup
   \startcolumns[\c!n=2,\c!distance=36pt,\c!rule=\KolommenLijn]
   \placeregister[\v!index]
   \stopcolumns
   \egroup}

\def\CommandIndex#1%
  {\getvalue{\v!chapter}[comind]{#1}
   \bgroup
   \startcolumns[\c!n=2,\c!distance=36pt,\c!rule=\KolommenLijn]
   \placeregister[Command][\c!before={\blank[\v!line]}]
   \stopcolumns
   \egroup}

\def\CommandList#1%
  {\getvalue{\v!chapter}[comdefs]{#1}
   \bgroup
   \component ma-cb-\currentmainlanguage-commandlist
   \blank[2*\v!big]
   \startcolumns[\c!n=2,\c!distance=36pt,\c!rule=\KolommenLijn]
     \switchtobodyfont[7pt]
     \setupsetup[\c!reference=2]
     \placesetup
   \stopcolumns
   \egroup}

\defineregister
  [Command]
  [Commands]

\setupregister
  [Command]
  [bagger=oeps,\c!indicator=\v!off]

\def\Copyright{%
  \doiffileelse{ma-cb-\currentmainlanguage-copyright.tex }
    {\component ma-cb-\currentmainlanguage-copyright.tex }
%    {\component \GenericDir/ma-cb-copyright.tex }
    {\component ma-cb-copyright.tex }
}
\def\Colofon  {%
  \doiffileelse{ma-cb-\currentmainlanguage-colophon.tex }
    {\component ma-cb-\currentmainlanguage-colophon.tex }
%    {\component \GenericDir/ma-cb-colophon.tex }
    {\component ma-cb-colophon.tex }
}

% Oh, how far more easy it would have been if we could put
% this in the text directly. But consistency is needed.

% Info sources
\definedescription[infosource][location=top,headstyle=bold,width=fit]

\protect

\stopenvironment
