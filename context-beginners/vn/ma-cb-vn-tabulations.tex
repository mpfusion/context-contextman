\startcomponent ma-cb-vn-tabulations
\project ma-cb
\product ma-cb-vn
\environment ma-cb-env-vn

%\chapter[tabulation]{Tabulation / Paragraph formatting}
\chapter[tabulation]{Định dạng đoạn văn}

%\index{tabulation}
%\index{tables+running text}
%\index{columns}
\index{xếp cột}
\index{bảng+chữ liên tục}
\index{cột}

\Command{\tex{starttabulate}}
\Command{\tex{definetabulate}}
\Command{\tex{setuptabulate}}
\Command{\tex{NR}}
\Command{\tex{NC}}
\Command{\tex{startchemical}}

%Sometimes you want to typeset paragraphs in a specific
%formatted way. This is done with:
Thỉnh thoảng bạn cần sắp chữ cho đoạn văn theo cách định dạng riêng biệt. Việc
này được thực hiện với:

\shortsetup{starttabulate}

%The tabulation mechanism is closely related to the
%table mechanism. You can use the tabulation mechanism
%in cases you want to type set complete paragraphs within a
%cell. The tabulation mechanism works fine at a page break.
Cơ chế xếp cột gần tương tự như cơ chế tạo bảng. Bạn có thể dùng cơ chế xếp
cột trong trường hợp bạn cần nhập tập hợp các đoạn văn hoàn chỉnh vào trong
một ô. Cơ chế xếp cột làm việc tốt lúc sang trang.

%% \startbuffer
%% \starttabulate[|w(1.5cm)B|p(6.0cm)|p|]
%% \NC 1252
%%     \NC Hasselt obtains its city charter from bishop Hendrik
%%         van Vianden.
%%     \NC Hendrik van Vianden was pressed by other towns not
%%         to agree with the charter. It took Hasselt a long
%%         period of time to convince the Bishop. After
%%         supporting the Bishop in a small war against the
%%         Drents, the charter was released. \NC\NR
%% \NC 1350
%%     \NC Hasselt joins the Hanzepact to protect their
%%         international trade.
%%     \NC The Hanzepact was of great importance for merchants
%%         in Hasselt. In those days trading goods were taxed
%%         at every city, highway or rivercrossing. After
%%         joining the Hanzepact duty free routes all over
%%         Europe became available to Hasselt. However
%%         important the Hanzepact was, Hasselt always stayed a
%%         minor member of the pact. \NC\NR
%% \stoptabulate
%% \stopbuffer
\startbuffer
\starttabulate[|w(1.5cm)B|p(6.0cm)|p|]
\NC 1252
    \NC Hasselt đạt được hiến chương thành phố nhờ giám mục Hendrik van Vianden.
    \NC Hendrik van Vianden bị ép buộc bởi các thị trấn khác không đồng ý với
        hiến chương. Mất cả môt thời kì dài để nghe theo Giám mục. Sau đó,
	được sự hỗ trợ của Hội đồng giám mục trong cuộc chiến nhỏ với người
	Drents, hiến chương được thông qua. \NC\NR
\NC 1350
    \NC Hasselt gia nhập Hanzepact để bảo vệ mậu dịch quốc tế của họ.
    \NC Hanzepact là thế lực lớn nhất trong việc buôn bán ở Hasselt. Trong
        những ngày này, hàng hóa giao thương bị đánh thuế khắp mọi thành phố,
	đường cao tốc hay điểm qua sông. Sau khi gia nhập Hanzepact, các tuyến
	đường miễn thuế toàn Châu Âu đều sẵn sàng với Hasselt. Tuy nhiên, điểm
	quan trọng là trong Hanzepact, Hasselt luôn là thành viên không quan
	trọng. \NC\NR
\stoptabulate
\stopbuffer

%A tabulate definition could look like this:
Một định nghĩa cách xếp cột có thể trông như thế này:
\typebuffer

%In this case the first column is 1.5 \Centi \Meter\ wide and
%is type set bold (\type{B}). The second column has a width
%of 6 \Centi \Meter\ and is type set like a paragraph. The
%remaining horizontal space is used up by the last paragraph.
Trong trường hợp này, cột đầu tiên rộng 1.5 \Centi \Meter\ và được đặt chữ đậm
(\type{B}). Cột thứ hai có độ rộng là 6 \Centi \Meter và nhập như một đoạn
văn. Khoảng trống chiều ngang còn lại được dùng bởi đoạn văn còn lại.

%The example is typeset like this:
Ví dụ được sắp như sau:

\getbuffer

%Like in the table mechanism a number of formatting commands
%and keys are used. A list of these commands and keys are
%shown in \in{table}[tab:tabularformattingcommands].
Giống như cơ chế tạo bảng, một số các lệnh và khóa định dạng được dùng. Một
danh sách các lệnh và khóa này được thể hiện trong \in{table}[tab:tabularformattingcommands]

%% \placetable[][tab:tabularformattingcommands]
%%   {Tabular commands.}
%% \starttable[|lT|l|lT|l|]
%% \NC l                 \NC left align
%% \NC I                 \NC \it italic
%% \NC \FR
%% \NC c                 \NC center
%% \NC R                 \NC \sl roman
%% \NC \MR
%% \NC r                 \NC right align
%% \NC S                 \NC \sl slanted
%% \NC \MR
%% \NC i\sl n            \NC spacing left
%% \NC T                 \NC \tt teletype
%% \NC \MR
%% \NC j\sl n            \NC spacing right
%% \NC m                 \NC in||line math
%% \NC \MR
%% \NC k\sl n            \NC spacing around
%% \NC M                 \NC display  math
%% \NC \MR
%% \NC w({\sl d})        \NC 1 line,   fixed width
%% \NC f\tex{command}    \NC font specification
%% \NC \MR
%% \NC p({\sl d})        \NC paragraph, fixed width
%% \NC b\arg{..}         \NC place \type{..} before the entry
%% \NC \MR
%% \NC p                 \NC paragraph, maximum width
%% \NC a\arg{..}         \NC place \type{..} after the entry
%% \NC \MR
%% \NC B                 \NC \bf boldface
%% \NC h\tex{command}    \NC apply \tex{command} on the entry
%% \NC \LR
%% \stoptable
\placetable[][tab:tabularformattingcommands]
  {Lệnh xếp cột.}
\starttable[|lT|l|lT|l|]
\NC l                 \NC gióng hàng bên trái
\NC I                 \NC \it chữ nghiêng
\NC \FR
\NC c                 \NC canh giữa
\NC R                 \NC \sl chữ La mã
\NC \MR
\NC r                 \NC gióng hàng bên phải
\NC S                 \NC \sl chữ xiên
\NC \MR
\NC i\sl n            \NC khoảng trắng bên trái
\NC T                 \NC \tt kiểu đánh máy
\NC \MR
\NC j\sl n            \NC khoảng trắng bên phải
\NC m                 \NC in||line math
\NC \MR
\NC k\sl n            \NC khoảng trắng xung quanh
\NC M                 \NC hiển thị kí tự toán
\NC \MR
\NC w({\sl d})        \NC 1 dòng, độ rộng cố định
\NC f\tex{command}    \NC font chữ xác định
\NC \MR
\NC p({\sl d})        \NC đoạn văn, độ rộng cố định
\NC b\arg{..}         \NC đặt \type{..} trước mục
\NC \MR
\NC p                 \NC đoạn văn, độ rộng tối đa
\NC a\arg{..}         \NC đặt \type{..} sau mục
\NC \MR
\NC B                 \NC \bf chữ đậm
\NC h\tex{command}    \NC chấp nhận \tex{command} trong mục
\NC \LR
\stoptable

%Another example of paragraph formatting could look like this.
Một ví dụ khác về định dạng trang:

%% \startbuffer
%% \definetabulate[ChemPar][|l|p|l|]

%% \startChemPar
%% \NC Limekilns
%%     \NC Hasselt has its own limekilns. These were build in 1504
%%         and produced quick lime up to 1956. Nowadays they are a
%%         tourist attraction.
%%     \NC \chemical{CaCO_3,~,GIVES,~,CaO,~,+,~,CO_2} \NC\NR
%% \stopChemPar
%% \stopbuffer
\startbuffer
\definetabulate[ChemPar][|l|p|l|]

\startChemPar
\NC Lò vôi
    \NC Hasselt có các lò vôi riêng. Chúng được xây dựng vào năm 1504 và sản
        xuất vôi đến năm 1956. Ngày nay, chúng là các điểm du lịch hấp dẫn.
    \NC \chemical{CaCO_3,~,GIVES,~,CaO,~,+,~,CO_2} \NC\NR
\stopChemPar
\stopbuffer

\typebuffer

%And it would come out like this:
Và nó được xuất ra như thế này:

\getbuffer

%The chemical module is explained in another manual, because
%not everybody is interested in chemical stuff.
Mođun hóa học được diễn giải trong một sổ tay khác bởi vì không phải ai cũng
thích hóa học.

%Here we also introduced the command to define the paragraph
%layout.
Ở đây, chúng chúng ta cũng giới thiệu lệnh để định nghĩa khung nền của đoạn
văn.

\shortsetup{definetabulate}

%and we also have:
và chúng ta cũng có:

\shortsetup{setuptabulate}

\stopcomponent
