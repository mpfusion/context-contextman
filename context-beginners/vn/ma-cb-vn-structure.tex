\startcomponent ma-cb-vn-structure
\project ma-cb
\product ma-cb-vn
\environment ma-cb-env-vn

%% \chapter{Defining a document}
\chapter{Định nghĩa một tài liệu}

%% Every document is started with \type{\starttext} and closed
%% with \type{\stoptext}. All textual input is placed between
%% these two commands and \CONTEXT\ will only process that
%% information.
Mỗi tài liệu được bắt đầu bằng \type{\starttext} và được đóng bằng
\type{\stoptext}. Tất cả nhập liệu nguyên văn được được đặt giữa hai lệnh này
và \CONTEXT\ sẽ chỉ thực thi thông tin đó.

%% Setup information is placed in the set up area just before
%% \type{\starttext}.
Thông tin thiết lập được đặt trong phần thiết lập nằm trước \type{\starttext}.

%% \startbuffer
%% \setupbodyfont[12pt]
%% \starttext
%% This is a one line document.
%% \stoptext
%% \stopbuffer

%% \typebuffer
\startbuffer
\setupbodyfont[12pt]
\starttext
Đây là tài liệu một dòng.
\stoptext
\stopbuffer

\typebuffer

%% Within the \type{\starttext} $\cdots$ \type{\stoptext} a
%% document can be divided into four main divisions:
Bên trong \type{\starttext} $\cdots$ \type{\stoptext}, một tài liệu có thể
được chia thành bốn phân đoạn chính:

%% \startitemize[n,packed]
%% \item front matter
%% \item body matter
%% \item back matter
%% \item appendices
%% \stopitemize
\startitemize[n,packed]
\item phần đầu (front)
\item phần thân (body)
\item phần sau (back)
\item phụ lục (appendices)
\stopitemize

%% The divisions are defined with:
Các phân đoạn được định nghĩa bằng:

\starttyping
\startfrontmatter ... \stopfrontmatter
\startbodymatter  ... \stopbodymatter
\startbackmatter  ... \stopbackmatter
\startappendices  ... \stopappendices
\stoptyping

%% In the front matter as well as back matter section the
%% command \type{\chapter} produces an un-numbered header in
%% the table of contents. This section is mostly used for the
%% table of contents, the list of figures and tables, the
%% preface, the acknowledgements etc. This section often
%% has a roman page numbering.
Ở phần đầu và phần sau khi dùng lệnh \type{\chapter} sẽ không được đánh số trong
bảng nội dung. Các phần này thường được dùng cho bảng nội dung, danh sách hình
ảnh và bảng, lời tựa, lời cảm ơn, ... Các mục này thường được đánh số La Mã.

%% The appendices section is used for (indeed) appendices.
%% Headers may be typeset in a different way; for example,
%% \type{\chapter} may be numbered alphabetically.
Phần phụ lục được dùng cho phụ lục (nghĩa đen). Cách đánh số có thể được sắp xếp
theo cách khác, ví dụ \type{\chapter} có thể được đánh số theo kí tự alphabe.

%% Section style can be set up with:
Kiểu mỗi phần có thể được thiết lập bằng:

\shortsetup{setupsectionblock}

\stopcomponent
