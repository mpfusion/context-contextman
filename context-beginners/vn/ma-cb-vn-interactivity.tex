\startcomponent ma-cb-vn-interactivity
\project ma-cb
\product ma-cb-vn
\environment ma-cb-env-vn

%\chapter{Interactive mode in electronic documents}
\chapter{Kiểu tương tác trong tài liệu số}

%\section{Introduction}
\section{Lời nói đầu}

\index[reader]{\READER}
\index[exchange]{\EXCHANGE}
\index[distiller]{\DISTILLER}

%Nowadays documents can be made electronically available for
%consulting on a computer and displaying on a computer screen.
Ngày nay, tài liệu có thể được số hóa để hiển thị trên màn hình máy vi tính để
xem và nghiền ngẫm trên đó.

%Interaction means that you can click on active areas and
%jump to the indicated areas. For example if you consult a
%register you can click on a (active) page number and you will
%jump to the corresponding page.
Sự tương tác có nghĩa là bạn có thể nhấp chuột vào vùng đang xem và được di
chuyển đến vùng được đánh dấu. Ví dụ nếu bạn tra cứu một chỉ mục, bạn có thể
nhấp chuột vào số trang và sẽ được đưa đến trang đó.

%Interaction relates to:
Sự tương tác liên quan đến

%%\startitemize[packed]
%% \item active chapter numbers in table of content
%% \item active page numbers in registers
%% \item active page numbers, chapter numbers and figure numbers in
%%       internal references to pages, chapters, figures etc.
%% \item active titles, page numbers, and chapter numbers in
%%       external references to other interactive documents
%% \item active menus as navigation tools
%%\stopitemize
\startitemize[packed]
\item số chương trong bảng mục lục.
\item số trang trong bảng chỉ mục
\item số trang, số chương và số của hình trong các tham khảo bên trong đến
	trang, chương, hình,...
\item các tiêu đề, số trang, số chương trong các tham chiếu đến các tài liệu
	tương tác bên ngoài khác
\item các menu như là các công cụ di chuyển (trong tài liệu)
\stopitemize

%Interactivity depends on the program you use to view the
%interactive document. We assume that you will use \PDFTEX\
%for producing a \PDF\ document directly or use \DISTILLER\
%to convert a \POSTSCRIPT\ file into a \PDF\ one. It is
%obvious that you will then use \READER, \EXCHANGE, or
%\GHOSTVIEW\ for viewing.
Các tác động qua lại dựa vào chương trình bạn dùng để xem tài liệu. Chúng ta
giả định rằng bạn dùng \PDFTEX\ để tạo tài liệu \PDF\ trực tiếp hoặc dùng
\DISTILLER\ để chuyển đổi một tập tin \POSTSCRIPT\ sang \PDF. Và hiển nhiên là
bạn sẽ dùng \READER, \EXCHANGE\ hoặc \GHOSTVIEW\ để xem.

%\CONTEXT\ is a very powerful system for producing electronic
%or interactive \PDF\ documents. However only a few standard
%features are described in this chapter. As the authors of
%this manual are planning to make all \CONTEXT\ related
%manuals electronically (sources included) available, reverse
%engineering is one of the options to become more acquainted
%with the possibilities of \CONTEXT.
\CONTEXT\ là một hệ thống rất mạnh để tạo ra các tài liệu số hoặc các tài liệu
\PDF\ tương tác. Tuy nhiên chỉ có một vài chức năng chuẩn được diễn giải trong
chương này. Các tác giả của sổ tay này đang có kế hoạch số hóa tất cả các tài
liệu liên quan đến \CONTEXT\ (bao gồm cả mã nguồn), dịch ngược là một tùy chọn
trở nên quen thuộc với \CONTEXT.

%\section{Interactive mode}
\section{Kiểu tương tác}

%\index{interactive mode}
\index{kiểu tương tác}

\Command{\tex{setupinteraction}}

%The interactive mode is activated by:
Kiểu tương tác được kích hoạt với lệnh:

\shortsetup{setupinteraction}

%For example:
Ví dụ:

\startbuffer
\setupinteraction
  [state=start,
   color=green,
   style=bold]
\stopbuffer

\typebuffer

%The hyper links are now generated automatically and the
%active words are displayed in bold green.
Các liên kết bây giờ được tự động tạo và các từ liên kết sẽ được hiển thị màu
xanh lá và chữ đậm.

%The interactive document is considerably bigger (in MB's)
%than its paper cousin because hyperlinks consume space. You
%will also notice that processing time becomes longer.
%Therefore it is advisable to de-activate the interactive mode
%as long as your document is under construction.
Tài liệu liên kết phình ra đáng kể (theo đơn vị MB) hơn người anh em giấy của
nó vì các liên kết chiếm dụng nhiều khoảng trống. Bạn sẽ để ý thấy rằng thời
gian biên dịch trở nên lâu hơn. Cho nên, nếu tài liệu của bạn vẫn đang còn
biên soạn thì nên tắt chức năng tương tác trong tài liệu.

%\section{Interaction within a document}
\section{Tương tác trong tài liệu}

\index{tương tác+bên trong}

\Command{\tex{in}}
\Command{\tex{at}}
\Command{\tex{goto}}

%Earlier you have seen how to make a reference with
%\type{\in} and \type{\at}. You may have wondered why you had
%to type \type{\in{chapter}[chap:introduction]}. In the first
%place {\em chapter} and its corresponding chapter number
%will not be separated at line breaking. In the second place
%the word {\em chapter} and its number are typeset
%differently in the interactive mode. This gives the user a
%larger clickable area.
Trong phần đầu, bạn đã học cách tạo một tham chiếu với lệnh \type{\in} và
\type{\at}. Có thể bạn đã tự hỏi tại sao bạn phải nhập vào
\type{\in{chương}[chap:introduction]}. Trong phần đầu tiên, {\em chương} và số
chương của nó sẽ không bị tách ra lúc xuống dòng. Trong phần thứ hai, {\em
chương} và số chương của nó sẽ được sắp xếp khác biệt trong kiểu tương tác.
Điều này sẽ tạo một vùng nhấp chuột rộng hơn cho người dùng.

%In interactive mode there is one other command that has
%little meaning in the paper variant.
Trong kiểu tương tác, có một lệnh khác có ít ý nghĩa hơn trong dạng định dạng
giấy.

\shortsetup{goto}

%The curly braces contain text, the brackets contain a
%reference (logical name, location).
Cặp ngoặc móc chứa chữ, cặp ngoặc vuông chứa tham chiếu (tên logic, vị trí).

%\startbuffer
%In \goto{Hasselt}[fig:cityplan] all streets are build in a
%circular way.
%\stopbuffer
\startbuffer
Ở \goto{Hasselt}[fig:cityplan] các con phố được xây dựng theo một đường vòng.
\stopbuffer

\typebuffer

%In the interactive document {\em Hasselt} will be green and
%active. You will jump to a map of Hasselt.
Trong tài liệu tương tác, {\em Hasselt} sẽ được tô màu xanh lá và nhấp chuột
được. Bạn sẽ được đưa đến bản đồ của Hasselt.

%\section{Interaction between documents}
\section{Tương tác giữa các tài liệu}

%\index{interaction+external}
\index{tương tác+bên ngoài}

\Command{\tex{from}}
\Command{\tex{useexternaldocument}}

%It is possible to link one document to another. First you
%have to state that you want to refer to another document.
%This is done by:
Có thể liên kết một tài liệu đến một tài liệu khác. Đầu tiên, bạn phải khai
báo rằng bạn muốn tham chiếu đến tài liệu khác với lệnh:

\shortsetup{useexternaldocument}

%The first bracket pair must contain a logical name of the
%document, the second pair the file name of the other document
%and the third pair is used for the title of the document.
Cặp ngoặc vuông đầu tiên phải chứa một tên logic của tài liệu, cặp thứ hai
chứa tên tập tin của tài liệu cần tham chiếu và cặp thứ ba chứa tiêu đề của
tài liệu.

%For refering to these other documents you can use:
Để tham chiếu đến các tài liệu này, bạn dùng:

\shortsetup{from}

%The curly braces contain text and the brackets contain the
%reference.
Cặp ngoặc móc chứa chữ và cặp ngoặc vuông chứa tham chiếu.

%Look at the example below.
Xem ví dụ bên dưới.

%% \startbuffer
%% \useexternaldocument[hia][hasbook][Hasselt in August]

%% Most tourist attractions are described in \from[hia]. A description
%% of the Eui||feest is found in \from[hia::euifeest]. A description
%% of the \goto{Eui||feest}[hia::euifeest] is found in \from[hia]. The
%% eui||feest is described on \at{page}[hia::euifeest] in \from[hia].
%% See for more information \in{chapter}[hia::euifeest] in \from[hia].
%% \stopbuffer
\startbuffer
\useexternaldocument[hia][hasbook][Hasselt vào tháng Tám]

Hầu hết các điểm du lịch được mô tả trong \from[hia]. Một mô tả về Eui||feest
có trong \from[hia::euifeest]. Một mô tả về \goto{Eui||feest}[hia::euifeest]
có trong \from[hia]. Mục eui||feest được mô tả tại \at{trang}[hia::euifeest]
trong \from[hia]. Xem thêm thông tin tại \in{chương}[hia::euifeest] trong
\from[hia].
\stopbuffer

\typebuffer

%The \type{\useexternaldocument} is usually typed in the
%set up area of your input file.
Lệnh \type{\useexternaldocument} luôn được đặt trong phần thiết đặt của tập
tin nhập liệu.

%After processing your input file (at least two times to get
%the references right), and the file \type{hasbook.tex},
%you will have two \PDF\ documents. The references
%above have the following meaning:
Sau khi biên dịch tập tin nhập liệu (tối thiểu hai lần để có tham chiếu chính
xác) và tập tin \type{hasbook.tex}, bạn sẽ có hai tài liệu \PDF. Các tham
chiếu trên có ý nghiĩ sau:

%%\startitemize[packed]
%% \item \type{\from[hia]} will produce the active title you gave
%%       in the third bracket pair of
%%       \type{\useexternaldocument} and is linked to the
%%       first page of \type{hasbook.pdf}
%% \item \type{\from[hia::euifeest]} will produce an active title
%%       and is linked to the page where chapter Eui||feest
%%       begins
%% \item \type{\goto{Eui||feest}[hia::euifeest]} will produce an
%%       active word {\em Eui||feest} and is linked to the page
%%       where chapter Eui||feest begins
%% \item \type{\at{page}[hia::euifeest]} will produce an active
%%       word {\em page} and page number and is linked to that
%%       page
%% \item \type{\in{chapter}[hia::euifeest]} will produce on
%%       active word {\em chapter} and chapter number and is
%%       linked to that chapter
%%\stopitemize
\startitemize[packed]
\item \type{\from[hia]} sẽ tạo tiêu đề liên kết bạn đã đặt trong cặp ngoặc
vuông thứ ba của lệnh \type{useexternaldocument} và được liên kết đến trang
đầu tiên của \type{hasbook.pdf}.
\item \type{\from[hia::euifeest]} sẽ tạo một tiêu đề liên kết và được liên kết
đến trang bắt đầu của chương Eui||feest.
\item \type{\goto{Eui||feest}[hia::euifeest]} sẽ tạo một từ liên kết {\em
Eui||feest} và được liên kết đến trang bắt đầu của chương Eui||feest.
\item \type{\at{page}[hia::euifeest]} sẽ tạo một từ {\em trang} và số trang
liên kết đến trang đó.
\item \type{\in{chapter}[hia::euifeest]} sẽ tạo từ liên kết {\em chương} và số
chương liên kết đến chương đó.
\stopitemize

%As you can see the \type{::} separates the (logical) file name
%and the destination in that file.
Như bạn đã thấy, kí tự \type{::} chia cách tên tập tin (logic) và đích đến
trong tập tin đó.

%\section{Menus}
\section{Menu}

%You can define navigation tools with:
Bạn có thể định nghĩa các công cụ di chuyển (trong tài liệu) với lệnh:

\shortsetup{defineinteractionmenu}

%The first bracket pair is used for a logical name that can
%be used to recall the menu. The second pair contains the
%location on the screen. The third pair is used for setting
%up the menu.
Cặp ngoặc vuông đầu tiên được dùng cho tên logic để có thể gọi lại menu. Cặp
ngoặc vuông thứ hai chứa vị trí trên màn hình. Cặp ngoặc vuông thứ ba được
dùng để thiết đặt menu.

%A typical menu definition might look like this:
Một định nghĩa menu đặc trưng có thể trông như thế này:

%% \startbuffer
%% \setupcolors
%%   [state=start]

%% \setupinteraction
%%   [state=start,
%%    menu=on]

%% \defineinteractionmenu
%%   [mymenu]
%%   [right]
%%   [state=start,
%%    align=middle,
%%    background=screen,
%%    frame=on,
%%    width=\marginwidth,
%%    style=smallbold,
%%    color=]

%% \setupinteractionmenu
%%   [mymenu]
%%   [{Content[content]},
%%    {Index[index]},
%%    {\vfill},
%%    {Stop[ExitViewer]}]
%% \stopbuffer
\startbuffer
\setupcolors
  [state=start]

\setupinteraction
  [state=start,
   menu=on]

\defineinteractionmenu
  [mymenu]
  [right]
  [state=start,
   align=middle,
   background=screen,
   frame=on,
   width=\marginwidth,
   style=smallbold,
   color=]

\startinteractionmenu[mymenu]
  \but [content] Nội dung \\
  \but [index] Bảng tra \\
  \vfill \\
  \but [ExitViewer] Thoát \\
\stopinteractionmenu

\setupheadertexts[{\interactionmenu[mymenu]}]
\stopbuffer

\typebuffer

%This will produce a menu on the right hand side of every
%screen. The menu buttons contain the text {\em Content}, {\em
%Index} and {\em Stop} with respectively the following
%functions: jump to the table of contents, jump to the index
%and leave the viewer. The labels to obvious destinations
%like \type{content} and \type{index} are predefined. Other
%predefined destinations are \type{FirstPage},
%\type{LastPage}, \type{NextPage} and
%\type{PreviousPage}.
Ví dụ trên sẽ tạo một menu phía tay phải mỗi khung hình. Các nút trên menu
chứa các chữ {\em Nội dung}, {\em Bảng tra} và {\em Thoát} thực hiện các chức
năng: di chuyển đến bảng mục lục, di chuyển đến bảng tra và thoát khỏi chương
trình xem tài liệu. Các nhãn đến các đích như \type{content} và \type{index}
được định nghĩa trước. Các đích đến được đinh nghĩa trước khác là
\type{FirstPage}, \type{LastPage}, \type{NextPage} và \type{PreviousPage}.

%An action like \type{ExitViewer} is necessary to make an
%electronic document self containing. Other predefined
%actions you can use are \type{PrintDocument},
%\type{SearchDocument} and \type{PreviousJump}. The meaning of
%these actions is obvious.
Một hành động như \type{Thoát} là cần thiết mà một tài liệu số nên có. Các
hành động khác bạn có thể dùng là \type{PrintDocument}, \type{SearchDocument}
và \type{PreviousJump}. Ý nghĩa của các hành động này rất rõ ràng.

%Menus are set up with:
Menu được thiết lập với lệnh:

%\shortsetup{setupinteractionmenu}
\shortsetup{startinteractionmenu}

\stopcomponent
