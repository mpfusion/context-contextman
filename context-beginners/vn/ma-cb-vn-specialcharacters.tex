\startcomponent ma-cb-vn-specialcharacters
\project ma-cb
\product ma-cb-vn
\environment ma-cb-env-vn

%% \chapter[special chars]{Special characters}
\chapter[special chars]{Kí tự đặc biệt}

%\index{special characters}
\index{kí tự đặc biệt}

%% You have seen that \CONTEXT\ commands are preceded by a
%% \tex{} (backslash). This means that \tex{} has a
%% special meaning to \CONTEXT. Aside from \tex{} there are
%% other characters that need special attention when you want
%% them to appear in verbatim mode or in text mode.
%% \in{Table}[tab:specchars] gives an overview of these special
%% characters and what you have to type to produce them.
Bạn đã thấy rằng các lệnh \CONTEXT\ theo sau một \tex{} (dấu chéo ngược). Điều
này nghĩa là \tex{} có một ý nghĩa đặc biệt đối với \CONTEXT. Ngoài \tex{} nói
riêng, còn có những kí tự đặc biệt khác cần quan tâm đặc biệt khi bạn muốn
chúng xuất hiện trong nguyên mẫu hoặc trong phương thức văn bản.
\in{Table}[tab:specchars] cho bạn cái nhìn tổng quan về những kí tự đặc biệt
này cách bạn nhập để tạo chúng.

\let\normalunderscore=\_
\let\normaltilde     =\~

%% \placetable[here,force][tab:specchars]
%%   {Special characters (1).}
%%   \starttable[|c|c|c|c|c|]
%%   \HL
%%   \NC \bf \LOW{Special character} \NC \use2 \bf Verbatim  \NC \use2 \bf Text \NC\FR
%%   \NC                         \NC \bf Type \NC \bf To produce \NC \bf Type \NC \bf To produce \NC\LR
%%   \HL
%%   \NC \type{#} \NC \type{\type{#}} \NC \type{#} \VL \type{\#} \NC \# \NC\FR
%%   \NC \type{$} \NC \type{\type{$}} \NC \type{$} \VL \type{\$} \NC \$ \NC\MR
%%   \NC \type{&} \NC \type{\type{&}} \NC \type{&} \VL \type{\&} \NC \& \NC\MR
%%   \NC \type} \NC \type \NC \% \NC\LR
%%   \HL
%%   \stoptable
\placetable[here,force][tab:specchars]
  {Kí tự đặc biệt (1).}
  \starttable[|c|c|c|c|c|]
  \HL
  \NC \bf \LOW{Kí tự đặc biệt} \NC \use2 \bf Nguyên mẫu  \NC \use2 \bf Văn bản \NC\FR
  \NC                         \NC \bf Nhập \NC \bf Xuất \NC \bf Nhập \NC \bf Xuất \NC\LR
  \HL
  \NC \type{#} \NC \type{\type{#}} \NC \type{#} \VL \type{\#} \NC \# \NC\FR
  \NC \type{$} \NC \type{\type{$}} \NC \type{$} \VL \type{\$} \NC \$ \NC\MR
  \NC \type{&} \NC \type{\type{&}} \NC \type{&} \VL \type{\&} \NC \& \NC\MR
  \NC \type} \NC \type \NC \% \NC\LR
  \HL
  \stoptable

%% Other special characters have a meaning in typesetting
%% mathematical expressions and some can be used in
%% math mode only (see \in{chapter}[formulas]).
Các kí tự đặc biệt khác được dùng để diễn đạt xếp chữ toán học và một số khác
chỉ được dùng trong phương thức toán (xem \in{chương}[formulas]).

%% \let\normalbar=|
%% \placetable
%%   [here,force]
%%   [tab:special chars]
%%   {Special characters (2).}
%%   \starttable[|c|c|c|c|c|]
%%   \HL
%%   \NC \bf \LOW{Special character} \NC \use2 \bf Verbatim  \NC \use2 \bf Text \NC\FR
%%   \NC                         \NC \bf Type \NC \bf To produce \NC \bf Type \NC \bf To produce \NC\LR
%%   \HL
%%   \NC \type{+} \NC \type{\type{+}} \NC \type{+} \VL \type{$+$} \NC $+$ \NC\FR
%%   \NC \type{-} \NC \type{\type{-}} \NC \type{-} \VL \type{$-$} \NC $-$ \NC\MR
%%   \NC \type{=} \NC \type{\type{=}} \NC \type{=} \VL \type{$=$} \NC $=$ \NC\MR
%%   \NC \type{<} \NC \type{\type{<}} \NC \type{<} \VL \type{$<$} \NC $<$ \NC\MR
%%   \NC \type{>} \NC \type{\type{>}} \NC \type{>} \VL \type{$>$} \NC $>$ \NC\LR
%%   \HL
%%   \stoptable
\let\normalbar=|
\placetable
  [here,force]
  [tab:special chars]
  {Kí tự đặc biệt (2).}
  \starttable[|c|c|c|c|c|]
  \HL
  \NC \bf \LOW{Kí tự đặc biệt} \NC \use2 \bf Nguyên mẫu  \NC \use2 \bf Văn bản \NC\FR
  \NC                         \NC \bf Nhập \NC \bf Xuất \NC \bf Nhập \NC \bf Xuất \NC\LR
  \HL
  \NC \type{+} \NC \type{\type{+}} \NC \type{+} \VL \type{$+$} \NC $+$ \NC\FR
  \NC \type{-} \NC \type{\type{-}} \NC \type{-} \VL \type{$-$} \NC $-$ \NC\MR
  \NC \type{=} \NC \type{\type{=}} \NC \type{=} \VL \type{$=$} \NC $=$ \NC\MR
  \NC \type{<} \NC \type{\type{<}} \NC \type{<} \VL \type{$<$} \NC $<$ \NC\MR
  \NC \type{>} \NC \type{\type{>}} \NC \type{>} \VL \type{$>$} \NC $>$ \NC\LR
  \HL
  \stoptable

\stopcomponent
