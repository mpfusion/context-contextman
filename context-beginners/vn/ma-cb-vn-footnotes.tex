\startcomponent ma-cb-vn-footnotes
\project ma-cb
\product ma-cb-vn
\environment ma-cb-env-vn

%\chapter{Footnotes}
\chapter{Ghi chú}

\index{footnote}
\index{ghi chú}

\Command{\tex{footnote}}
\Command{\tex{setupfootnotes}}

%If you want to annotate your text you can use
%\type{\footnote}. The command looks like this:
Nếu bạn muốn chú giải văn bản của bạn, bạn có thể dùng \type{\footnote}. Câu
lệnh thực hiện như thế này:

\shortsetup{footnote}

%The bracket pair is optional and contains a logical name.
%The curly braces contain the text you want to display at
%the foot of the page.
Cặp ngoặc vuông là tùy chọn và chứa một tên logic. Cặp ngoặc móc chứa đoạn
văn bạn muốn hiển thị tại phần chân của trang.

%The same footnote number can be called with its logical name.
Số ghi chú giống nhau có thể được gọi với tên logic của nó.

\shortsetup{note}

%If you have typed this text:
Nếu bạn đã nhập đoạn văn sau:

% \startbuffer
% The Hanse was a late medieval commercial alliance of towns in the
% regions of the North and the Baltic Sea. The association was formed
% for the furtherance and protection of the commerce of its
% members.\footnote[war]{This was the source of jealousy and fear among
% other towns that caused a number of wars.} In the Hanse period there
% was a lively trade in all sorts of articles such as wood, wool,
% metal, cloth, salt, wine and beer.\note[war] The prosperous trade
% caused an enormous growth of welfare in the Hanseatic
% towns.\footnote{Hasselt is one of these towns.}
% \stopbuffer
\startbuffer
Hội Buôn Bán là liên minh các thành phố thương mại của các vùng phía Bắc và
biển Ban\-tíc vào cuối thời trung cổ. Sự liên kết được hình thành để đẩy mạnh
và bảo vệ nền thương mại cho các thành viên của nó.\footnote[war]{Đây là nguồn
gốc của sự ghen tị và e ngại các thành phố khác xuất phát từ một vài cuộc
chiến tranh.} Trong thời kì Hội Buôn Bán, có một sự giao thương sinh động
trong tất cả các loại vật phẩm như gỗ, len, kim loại, vải, muối, rượu vang và
bia.\note[war] Nền giao thương thịnh vượng tạo nên một sự phát triển to lớn về
phúc lợi công cộng trong các thành phố thuộc Hội Buôn Bán.\footnote{Hasselt
nằm trong các thành phố này.}
\stopbuffer

\typebuffer

%It would look like this:
Nó được xuất ra như thế này:

\getbuffer

%The footnote numbering is done automatically. The command
%\type{\setupfootnotes} enables you to influence the display
%of footnotes:
Số ghi chú được đánh tự động. Lệnh \type{\setupfootnotes} cho phép bạn tác
động đến cách hiển thị của ghi chú:

\shortsetup{setupfootnotes}

%Footnotes can be set at the bottom of a page but also at
%other locations, like the end of a chapter. This is done
%with the command:
Ghi chú có thể được đặt tại phần dưới của trang nhưng cũng có thể nằm tại vị
trí khác như là cuối một chương. Lệnh sau thực hiện việc đó:

\shortsetup{placefootnotes}

%You can also couple footnotes to a table. In that case we
%speak of local footnotes. The commands are:
Bạn cũng có thể ghép các ghi chú vào trong một bảng. Trong trường hợp đó,
chúng ta nó về các ghi chú cục bộ. Câu lệnh là:

\shortsetup{startlocalfootnotes}

\shortsetup{placelocalfootnotes}

%An example illustrates the use of local footnotes:
Một ví dụ làm rõ cách dùng ghi chú cục bộ:

% \startbuffer
% \startlocalfootnotes[n=0]
%   \placetable
%     {Decline of Hasselt's productivity.}
%      \starttable[|l|c|c|c|c|]
%        \HL
%        \NC
%        \NC Ovens\footnote{Source: Uit de geschiedenis van Hasselt.}
%        \NC Blacksmiths\NC Breweries \NC Potteries \NC\SR
%        \HL
%        \NC 1682 \NC 15 \NC 9 \NC 3 \NC 2 \NC\FR
%        \NC 1752 \NC ~6 \NC 4 \NC 0 \NC 0 \NC\LR
%        \HL
%        \stoptable
%   \placelocalfootnotes
% \stoplocalfootnotes
% \stopbuffer
\startbuffer
\startlocalfootnotes[n=0]
  \placetable
    {Sự suy sụp năng suất lao động của Hasselt.}
     \starttable[|l|c|c|c|c|]
       \HL
       \NC
       \NC Lò bánh\footnote{Nguồn: Uit de geschiedenis van Hasselt.}
       \NC Lò rèn\NC Nhà máy bia \NC Xưởng gốm \NC\SR
       \HL
       \NC 1682 \NC 15 \NC 9 \NC 3 \NC 2 \NC\FR
       \NC 1752 \NC ~6 \NC 4 \NC 0 \NC 0 \NC\LR
       \HL
       \stoptable
  \placelocalfootnotes
\stoplocalfootnotes
\stopbuffer

\typebuffer

\getbuffer

\stopcomponent
