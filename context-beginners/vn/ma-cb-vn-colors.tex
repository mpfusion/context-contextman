\startcomponent ma-cb-vn-color
\project ma-cb
\product ma-cb-vn
\environment ma-cb-env-vn

%\chapter{Color}
\chapter{Màu sắc}

\index{màu sắc}

\Command{\tex{setupcolors}}
\Command{\tex{color}}
\Command{\tex{definecolor}}

%Text can be set in color.
Chúng ta có thể tô màu cho chữ bởi lệnh:

\shortsetup{setupcolors}

%The use of colors has to be
%activated by:
Cách dùng màu sắc được bắt đầu với lệnh:

\starttyping
\setupcolors[state=start]
\stoptyping

%Now the basic colors are available (red, green and blue).
Bây giờ có thể bắt đầu với các màu cơ bản (đỏ, xanh lá và xanh dương).

%\startbuffer
%\startcolor[red]
%Hasselt is a very \color[green]{colorful} town.
%\stopcolor
%\stopbuffer
\startbuffer
\startcolor[red]
Hasselt là một thành phố \color[green]{đầy màu sắc}.
\stopcolor
\stopbuffer
\typebuffer

\getbuffer

%On a black and white printer you will see only grey shades.
%In an electronic document these colors will be as expected.
Trên các bản in trắng đen, bạn sẽ chỉ thấy các bóng mờ màu xám nhưng trên các
tài liệu số, những màu này sẽ được hiển thị đầy đủ.

%You can define your own colors with:
Bạn có thể tự xác định màu với lệnh:

\shortsetup{definecolor}

%For example:
Ví dụ:

%\startbuffer
%\definecolor[darkred]   [r=.5,g=.0,b=.0]
%\definecolor[darkgreen] [r=.0,g=.5,b=.0]
%\stopbuffer
\startbuffer
\definecolor[đỏ_đậm]   [r=.5,g=.0,b=.0]
\definecolor[xanh_lá_đậm] [r=.0,g=.5,b=.0]
\stopbuffer

\typebuffer

%Now the colors \type{darkred} and \type{darkgreen} are
%available.
Bây giờ, bạn có thể dùng các màu \type{đỏ_đậm} và \type{xanh_lá_đậm} đã định
trước ở trên.

\stopcomponent
