\startcomponent ma-cb-vn-descriptions
\project ma-cb
\product ma-cb-vn
\environment ma-cb-env-vn

%\chapter{Definitions}
\chapter{Lời định nghĩa}

\index{định nghĩa}

\Command{\tex{definedescription}}
\Command{\tex{setupdescriptions}}

%If you want to display notions, concepts and ideas in a
%consistent manner you can use:
Nếu bạn muốn hiển thị các quan điểm, khái niệm và ý kiến theo một kiểu phù
hợp, bạn có thể dùng:

\shortsetup{definedescription}

%For example:
Ví dụ:

% \startbuffer
% \definedescription
%   [concept]
%   [location=serried,headstyle=bold,width=broad]

% \concept{Hasselter juffer} A sort of biscuit made of puff pastry and
% covered with sugar. It tastes very sweet. \par
% \stopbuffer
\startbuffer
\definedescription
  [concept]
  [location=serried,headstyle=bold,width=broad]

\concept{Hasselter juffer} Là một loại bánh bích quy được làm từ bột bánh xốp
và được phủ đường. Mùi vị của nó rất ngọt. \par
\stopbuffer

\typebuffer

%It would look like this:
Nó trông như thế này:

\getbuffer

%But you can also choose other layouts:
Nhưng bạn có thể chọn một khung nền khác:

% \startbuffer
% \definedescription
%   [concept]
%   [location=top,
%    headstyle=bold,
%    width=broad,
%    style=slanted]

% \concept{Hasselter bitter} A very strong alcoholic drink (up to 40\%)
% mixed with herbs to give it a special taste. It is sold in a stone
% flask and it should be served {\em ijskoud} (as cold as ice). \par

% \definedescription
%   [concept]
%   [location=inmargin,headstyle=bold,width=broad]

% \concept{Euifeest} A harvest home to celebrate the end of a period of
% hard work. The festivities take place in the last week of August.
% \par

% \stopbuffer
\startbuffer
\definedescription
  [concept]
  [location=top,
   headstyle=bold,
   width=broad,
   style=slanted]

\concept{Hasselter bitter} Một thứ nước uống có độ cồn rất mạnh (lên đến 40\%)
được pha trộn từ các loại dược thảo cho mùi vị rất đặc biệt. Nó được để trong
chai bẹt bằng đá (lạnh như nước đá). \par

\definedescription
  [concept]
  [location=inmargin,headstyle=bold,width=broad]

\concept{Euifeest} Một chỗ ở thu hoạch để vui chơi vào cuối vụ mùa. Ngày hội
được tổ chức vào tuần cuối cùng của tháng Tám.
\par

\stopbuffer

\start
\getbuffer
\stop

%If you have more than one paragraph in such a definition you can use
%a \type{\start...}||\type{\stop...} pair.
Nếu bạn có nhiều hơn một đoạn văn đặt trong diễn giải, bạn có thể dùng cặp
\type{\start...}||\type{\stop...}.

% \startbuffer
% \definedescription
%   [concept]
%   [location=right,
%    headstyle=bold,
%    width=broad]

% \startconcept{Euifeest} A harvest home to celebrate the end of a
% period of hard work.
% This event takes place at the end of August and lasts one week. The
% city is completely illuminated and the streets are decorated. This
% feast week ends with a {\em Braderie}.
% \stopconcept
% \stopbuffer
\startbuffer
\definedescription
  [concept]
  [location=right,
   headstyle=bold,
   width=broad]

\startconcept{Euifeest} Một chỗ ở thu hoạch để vui chơi vào cuối vụ mùa.
Ngày hội được tổ chức vào tuần cuối cùng của tháng Tám. Thành phố được treo đèn
và đường phố được trang trí. Tuần lễ hội này kết thúc với một {\em Baraderie}.
\stopconcept
\stopbuffer

\typebuffer

%This would become:
Sẽ cho ra:

\getbuffer

%Layout is set up within the second bracket pair of
%\type{\definedescription[][]}. But you can also use:
Khung nền được thiết lập với cặp ngoặc vuông thứ hai của lệnh
\type{\definedescription[][]}. Nhưng bạn cũng có thể dùng:

\shortsetup{setupdescriptions}

\stopcomponent
