\startcomponent ma-cb-vn-gettingstarted
\project ma-cb
\product ma-cb-vn
\environment ma-cb-env-vn

%\chapter{How to process a file}
\chapter{Phương thức thực thi một tập tin}

%\index{input file+processing}
%\index[dvifile]{\type{dvi}--file}
%\index[pdffile]{\type{pdf}--file}
\index{tập tin nhập liệu+quá trình thực thi}
\index[taptin_dvi]{\type{dvi}--taptin}
\index[taptin_pdf]{\type{pdf}--taptin}

%If you want to process a \CONTEXT\ input file, you might
%type at the command line prompt:
Nếu bạn muốn thực thi một tập tin nhập liệu \CONTEXT,bạn có thể gõ vào cửa sổ
dòng lệnh:

%\starttyping
%context filename
%\stoptyping
\starttyping
context ten_tap_tin
\stoptyping

%The availability of the batch command \type{context} depends
%on the system you're using. Ask your system administrator
%the command you use to start \CONTEXT. If your filename is
%\type{myfile.tex} this can be:
Lệnh \type{context} có thể phụ thuộc vào hệ thống của bạn đang dùng. Hỏi quản
trị hệ thống của bạn để biết lệnh bắt đầu \CONTEXT. Nếu tên tập tin của bạn là
\type{tap_tin.tex} thì bạn có thể dùng:

%\starttyping
%context myfile
%\stoptyping
\starttyping
context tap_tin
\stoptyping

%or when \TEXEXEC\ is properly installed:
hoặc nếu \TEXEXEC\ đã được cài đặt:

%\starttyping
%texexec --pdf myfile
%\stoptyping
\starttyping
texexec --pdf tap_tin
\stoptyping

%the extension \type{.tex} is not needed.
phần mở rộng \type{.tex} không cần thiết.

%After pressing \Enter\ processing will be started. \CONTEXT\
%will show processing information on your screen. If
%processing is succesful the command line prompt will return
%and \CONTEXT\ will produce a \type{dvi} or \type{pdf} file.
Sau khi nhấn \Enter\ quá trình thực thi sẽ được bắt đầu. \CONTEXT\ sẽ hiện
hiện thông tin về quá trình thực thi trên màn hình. Nếu quá trình thực thi
thành công, dấu nhắc dòng lệnh sẽ được trả về và \CONTEXT\ sẽ tạo một tập tin
\type{dvi} hoặc \type{pdf}.

%If processing is not succesful ---for example because you
%typed \type{\stptext} instead of \type{\stoptext}---
%\CONTEXT\ produces a \type{ ? } on your terminal and tells
%you it has just processed an error. It will give you some
%basic information on the type of error and the line number
%where the error becomes effective.
Nếu quá trình thực thi không thành công ---ví dụ bởi vì bạn đã gõ
\type{\stptext} thay vì \type{\stoptext}--- \CONTEXT\ hiện một \type{ ? } trên
trên thiết bị đầu cuối và báo với bạn nó vừa thực thi một lỗi. Nó sẽ đưa ra
vài thông tin cơ bản về loại lỗi và số dòng nơi lỗi có tác động.

%At the instant of \type{ ? } you can type:
Tại dấu \type{ ? } bạn có thể:

%\starttabulate[|||]
%\NC \type{H} \NC for help information on your error \NC\NR
%\NC \type{I} \NC for inserting the correct \CONTEXT\ command \NC\NR
%\NC \type{Q} \NC for quiting and entering batch mode \NC\NR
%\NC \type{X} \NC for exiting the running mode \NC\NR
%\NC \Enter   \NC for ignoring the error \NC\NR
%\stoptabulate
\starttabulate[|||]
\NC \type{H} \NC thông tin trợ giúp về lỗi \NC\NR
\NC \type{I} \NC nhập lại lệnh \CONTEXT\ chính xác \NC\NR
\NC \type{Q} \NC for quiting and entering batch mode \NC\NR
\NC \type{X} \NC thoát phương thức đang chạy \NC\NR
\NC \Enter   \NC bỏ qua lỗi \NC\NR
\stoptabulate

%Most of the time you will type \Enter\ and processing will
%continue. Then you can edit the input file and fix the error.
Thông thường bạn sẽ gõ \Enter\ quá trình thực thi sẽ lại tiếp tục. Sau đó, bạn
có thể chỉnh sửa tập tin nhập liệu và sửa lỗi.

%Some errors will produce a~\type{ * } on your screen and
%processing will stop. This error is due to a fatal error in
%your input file. You can't ignore this error and the only
%option you have is to type \type{\stop} or {\sc Ctrl}~Z. The
%program will be halted and you can fix the error.
Một vài lỗi sẽ tạo một dấu~\type{ * } trên màn hình và quá trình thực thi sẽ
chấm dứt. Lỗi này là do một lỗi nguy hiểm trong tập tin nhập liệu của bạn. Bạn
không thể bỏ qua mà chỉ có thể gõ \type{\stop} hoặc nhấn {\sc Ctrl}~Z. Chương
trình sẽ thoát và bạn có thể sửa lỗi.

%During the processing of your input file \CONTEXT\ will also
%inform you of what it is doing with your document. For
%example it will show page numbers and information about
%process steps. Further more it gives warnings. These are of
%a typographical order and tells you when line breaking is not
%successful. All information on processing is stored in a
%\type{log} file that can be used for reviewing warnings and
%errors and the respective line numbers where they occur in
%your file.
Trong suốt quá trình thực thi tập tin nhập liệu của bạn, \CONTEXT\ cũng sẽ
thông tin cho bạn những gì nó đang làm với tài liệu của bạn. Ví dụ, nó sẽ hiện
số trang và thông tin về các bước thực thi. Xa hơn nữa nó sẽ đưa ra các
cảnh báo. Những cảnh báo này thuộc về loại lỗi in và báo cho bạn biết khi
xuống dòng không thành công. Tất cả về quá trình thực thi được lưu trữ trong
tập tin \type{log} và bạn có thể xem lại các cảnh báo, các lỗi và các số dòng
tương ứng nơi chúng xuất hiện trong tập tin nhập liệu của bạn.

%When processing is succesful \CONTEXT\ produces a new file,
%with the extension \type{.dvi} or \type{.pdf} so the files
%\type{myfile.dvi} or \type{myfile.pdf} are generated. The
%abbreviation \type{dvi} stands for Device Indepent. This
%means that the file can be processed by a suitable
%PostScript (\PS) printer driver to make the file suitable
%for printing or viewing. The abbreviation \PDF\ stands for
%Portable Document Format. This is a platform independent
%format for printing and viewing.
Khi quá trình thực thi thành công, \CONTEXT\ tạo một tập tin mới với phần mở
rộng là \type{.dvi} hoặc \type{.pdf}. Đó là các tập tin \type{tap_tin.dvi}
hoặc \type{tap_tin.pdf} đã được sinh ra. Từ viết tắt \type{dvi} nghĩa là
Device Indepent. Tập tin này có thể được in hay xem bởi các máy in PostScript
(\PS). Từ viết tắt \PDF\ nghĩa là Portable Document Format, là một định dạng
độc lập nền tảng để in hoặc xem.

\stopcomponent
