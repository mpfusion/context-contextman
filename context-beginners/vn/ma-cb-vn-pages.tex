\startcomponent ma-cb-vn-pages
\project ma-cb
\product ma-cb-vn
\environment ma-cb-env-vn

%\chapter{Page breaking and page numbering}
\chapter{Sang trang và đánh số trang}

%\index{page breaking}
%\index{page numbering}
\index{sang trang}
\index{đánh số trang}

\Command{\tex{page}}
\Command{\tex{setuppagenumbering}}
\Command{\tex{startpostponing}}

%A page can be enforced or blocked by:
Một trang có thể được sắp xếp như sau:

\shortsetup{page}

%The options can be stated within the brackets. The options
%and their meaning are presented in \in{table}[tab:page
%options]
Các tùy chọn được bắt đầu với các dấu ngoặc vuông. Tùy chọn và ý nghĩa được
đặt trong \in{bảng}[tab:page options]

%\placetable
%  []
%  [tab:page options]
%  {Page options.}
%\starttable[|l|l|]
%\HL
%\NC \bf Option \NC \bf Meaning \NC\SR
%\HL
%\NC \type{yes}           \NC enforce a page \NC\FR
%\NC \type{makeup}        \NC enforce a page without filling \NC\MR
%\NC \type{no}            \NC no page \NC\MR
%\NC \type{preference}    \NC prefer a new page here \NC\MR
%\NC \type{bigpreference} \NC great preference for a new page here \NC\MR
%\NC \type{left}          \NC next page is a left handside page \NC\MR
%\NC \type{right}         \NC next page is a right handside page \NC\MR
%\NC \type{disable}       \NC following commands have no effect \NC\MR
%\NC \type{reset}         \NC following commands do have effect \NC\MR
%\NC \type{empty}         \NC insert an empty page \NC\MR
%\NC \type{last}          \NC add pages till even number is reached \NC\MR
%\NC \type{quadruple}     \NC add pages till a multiple of four
%  is reached  \NC\LR
%\HL
%\stoptable
\placetable
  []
  [tab:page options]
  {Các tùy chọn trang.}
\starttable[|l|l|]
\HL
\NC \bf Tùy chọn \NC \bf Ý nghĩa \NC\SR
\HL
\NC \type{yes}           \NC tạo một trang \NC\FR
\NC \type{makeup}        \NC tạo một trang không chiếm hết khoảng trống \NC\MR
\NC \type{no}            \NC không tạo trang \NC\MR
\NC \type{preference}    \NC ưu tiên một trang mới ở đây \NC\MR
\NC \type{bigpreference} \NC ưu tiên cao nhất cho một trang mới ở đây \NC\MR
\NC \type{left}          \NC trang kế là một trang mặt trái \NC\MR
\NC \type{right}         \NC trang kế là một trang mặt phải \NC\MR
\NC \type{disable}       \NC các lệnh theo sau ko có tác dụng \NC\MR
\NC \type{reset}         \NC các lệnh theo sau có tác dụng \NC\MR
\NC \type{empty}         \NC chèn một trang trắng \NC\MR
\NC \type{last}          \NC thêm trang cho đến số chẵn \NC\MR
\NC \type{quadruple}     \NC thêm trang cho đến bội số của bốn \NC\LR
\HL
\stoptable

%Page numbering happens automatically but you can enforce
%a page number with:
Việc đánh số trang được thực hiện tự động nhưng bạn có thể đánh số một trang
với lệnh:

\starttyping
\page[25]
\stoptyping

%Sometimes it is better to state a relative page number like
%\type{[+2]} or \type{[-2]}.
Thỉnh thoảng, tốt hơn là xác định cách đánh số trang liên quan như \type{[+2]}
hay \type{[-2]}.

%The position of the page numbers on a page depend on your own
%preferences and if it concerns a one sided or double sided
%document. Page numbering can be set up with:
Vị trí của số trang trong trang tùy thuộc vào sự ưu thích của chính bạn và nó
là tài liệu một mặt hay hai mặt. Cách đánh số trang có thể được thiết lập với
lệnh:

\shortsetup{setuppagenumbering}

%The preferences are placed within the brackets.
Các ưu tiên được đặt trong dấu ngoặc vuông.

%Tables or figures may take up a lot of space. The placing of
%these text elements can be postponed till the next page break.
%This is done with:
Bảng hay hình có thể lấy đi nhiều khoảng trống. Đặt các yếu tố này vào trang 
có thể bị hoãn lại cho đến khi sang trang mới. Điều này có thể được làm với
lệnh:

\shortsetup{startpostponing}

%\startbuffer
%\startpostponing
%\placefigure
%  {A postponed figure.}
%  {\externalfigure[ma-cb-16][width=\textwidth]}
%\stoppostponing
%\stopbuffer
\startbuffer
\startpostponing
\placefigure
  {Một hình ảnh bị hoãn.}
  {\externalfigure[ma-cb-16][width=\textwidth]}
\stoppostponing
\stopbuffer

\typebuffer

%The figure will be placed at the top of next page and will
%cause minimal disruption of the running text.
Hình sẽ được đặt tại đầu trang kế và sẽ gây ra sự phá vỡ nhỏ cho các chữ đang
chạy.

\getbuffer

\stopcomponent
