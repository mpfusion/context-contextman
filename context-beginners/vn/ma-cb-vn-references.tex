\startcomponent ma-cb-vn-references
\project ma-cb
\product ma-cb-vn
\environment ma-cb-env-vn

%\chapter{Refering to text elements}
\chapter{Tham chiếu đến phần văn bản}

%\index{refering}
%\index{label}
\index{tham chiếu}
\index{nhãn}

\Command{\tex{in}}
\Command{\tex{at}}
\Command{\tex{pagereference}}

%For referring to one location in a document from another you
%can use the command:
Để tham chiếu đến một vị trí trong tài liệu từ một tài liệu khác, bạn dùng
lệnh:

\shortsetup{in}

%The curly braces contain text, the brackets contain a
%logical label. If you have written a chapter header like
%this:
Cặp ngoặc móc chứa phần văn bản, cặp ngoặc vuông chứa nhãn. nếu bạn viết một
phần đầu chương thế này:

%\starttyping
%\chapter[hotel]{Hotels in Hasselt}
%\stoptyping
\starttyping
\chapter[hotel]{Khách sạn ở Hasselt}
\stoptyping

%then you can refer to this chapter with:
sau đó bạn có thể tham chiếu đến chương này với lệnh:

%\starttyping
%\in{chapter}[hotel]
%\stoptyping
\starttyping
\in{chương}[hotel]
\stoptyping

%After processing the chapter number is available and the
%reference could look something like: {\em chapter 23}.
%You can use \type{\in} for any references to text elements
%like chapters, sections, figures, tables, formulae
%etc.
Sau khi thực thi, số chương được tạo và phần tham chiếu trông như thế này:
{\em chương 23}. Bạn có thể dùng \type{\in} cho bất kì tham chiếu đến phần văn
bản nào như chương, tiết đoạn, hình, bảng, công thức, ...

%Another example:
Ví dụ khác:

%\startbuffer
% There are a number of things you can do in Hasselt:

%\startitemize[n,packed]
%\item swimming
%\item sailing
%\item[hiking] hiking
%\item biking
%\stopitemize

% Activities like \in{activity}[hiking] described on \at{page}[hiking]
% are very tiring.
%\stopbuffer
\startbuffer
Có một vài thứ bạn có thể làm ở Hasselt:

\startitemize[n,packed]
\item bơi lội
\item chèo thuyền
\item[hiking] đi bộ
\item đạp xe đạp
\stopitemize

Các hoạt động như \in{hoạt động}[hiking] mô tả trong \at{trang}[hiking] thì
rất mệt.
\stopbuffer

\typebuffer

%This would look like this:
Sẽ xuất ra thê này:

\getbuffer

%As you can see, it is also possible to  refer to pages. This is done with:
Như bạn thấy, nó có thể được tham chiếu đến trang. Việc này làm với lệnh:

\shortsetup{at}

%For example with:
Ví dụ:

%\starttyping
%\at{page}[hiking]
%\stoptyping
\starttyping
\at{trang}[hiking]
\stoptyping

%This command can be used in combination with:
Lệnh này có thể dùng kết hợp với:

\shortsetup{pagereference}

%and
và

\shortsetup{textreference}

%If you want to refer to the chapter {\em Hotels in Hasselt}
%you could type:
Nếu bạn muốn tham chiếu đến chương {\em Khách sạn ở Hasselt}, bạn có thể nhập:

%\startbuffer
%Look in \in{chapter}[hotel] on \at{page}[hotel] for a complete
%overview of accomodations in \pagereference[accomodation]Hasselt.
%\stopbuffer
\startbuffer
Hãy xem \in{chương}[hotel] tại \at{trang}[hotel] để có cái nhìn hoàn chỉnh về
các chỗ nghỉ trọ ở \pagereference[accomodation]Hasselt.
\stopbuffer

\typebuffer

%A chapter number and a page number will be generated when
%processing the input file. On another spot in the document
%you can refer to \type{accomodation} with
%\type{\at{page}[accomodation]}.
Số chương và số trang sẽ được tạo khi thực thi tập tin nhập liệu. Tại nơi khác
trong tài liệu, bạn có thể tham chiếu đến \type{accomodation} với
\type{\at{trang}[accomodation]}.

%You can also define a set of labels separated by commas.
Bạn cũng có thể một nhóm nhãn được phân cách bằng dấu phẩy.

%\startbuffer
%\placefigure
%  [here]
%  [fig:canals,fig:boats]
%  {A characteristic picture of Hasselt.}
%  {\externalfigure[ma-cb-08][width=10cm]}
%
%There are many canals in Hasselt (see \in{figure}[fig:canals]).
%.
%.
%.
%Boats can be moored in the canals of Hasselt (see
%\in{figure}[fig:boats]).
%\stopbuffer
\startbuffer
\placefigure
  [here]
  [fig:canals,fig:boats]
  {Một hình ảnh đặc trưng của Hasselt.}
  {\externalfigure[ma-cb-08][width=10cm]}

Có nhiều kênh ở Hasselt (xem \in{figure}[fig:canals]).
.
.
.
Thuyền có thể được neo đậu trên các kênh ở Hasselt (xem \in{figure}[fig:boats])
\stopbuffer

\typebuffer

%This might look like this:
Sẽ xuất ra thế này:

\getbuffer

\stopcomponent
