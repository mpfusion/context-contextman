\startcomponent ma-cb-vn-fonts
\project ma-cb
\product ma-cb-vn
\environment ma-cb-env-vn

%\chapter{Fonts and font switches}
\chapter{Font chữ và chuyển đổi font chữ}

%\section{Introduction}
\section{Lời nói đầu}

\index{Computer Modern Roman}
\index{Lucida Bright}
\index{AMS}
\index{\cap{PS}--fonts}

%The default font in \CONTEXT\ is the {\em Computer Modern
%Roman} (\type{cmr}). You can also use Lucida Bright
%(\type{lbr}) as a full alternative and symbols of the {\em
%American Mathematical Society} (\type{ams}). Standard
%PostScript fonts (\type{pos}) are also available.
Font chữ mặc định trong \CONTEXT\ là {\em Computer Modern Roman} (\type{cmr}).
Bạn cũng có thể dùng Lucida Bright (\type{lbr} như một thay thế hoàn chỉnh và
các kí tự của {\em Hội Toán Học Mĩ} ({\type{ams}). Ngoài ra bạn cũng có thể
dùng font chữ PostScript (\type{pos}).

%\section{Fontstyle and size}
\section{Kiểu font và kích cỡ}

\index{font+kiểu}
\index{font+cỡ}

\Command{\tex{setupbodyfont}}
\Command{\tex{switchtobodyfont}}

%You select the font family, style and size for a document
%with:
Bạn chọn loại font, kiểu và cỡ cho tài liệu với lệnh:

\shortsetup{setupbodyfont}

%If you typed \type{\setupbodyfont[sansserif,9pt]}
%{\switchtobodyfont[ss,9pt] in the setup area of the input
%file your text would look something like this.}
Nếu bạn dùng \type{\setupbodyfont[sansserif,9pt]} {switchbodyfont[ss,9pt]
trong phần thiết lập của tập tin nhập liệu, tài liệu của bạn sẽ trông như thế
này.}

%For changes in mid-document and on section level you
%should use:
Để thay đổi giữa tài liệu và tại các tiết đoạn, bạn nên dùng:

\shortsetup{switchtobodyfont}

%% \startbuffer
%% On November 10th (one day before Saint Martensday) the youth of
%% Hasselt go from door to door to sing a special song and they
%% accompany themselves with a {\em foekepot}. They won't leave
%% before you give them some money or sweets. The song goes like this:

%% \startnarrower
%% \switchtobodyfont[small]
%% \startlines
%% Foekepotterij, foekepotterij,
%% Geef mij een centje dan ga'k voorbij.
%% Geef mij een alfje dan blijf ik staan,
%% 'k Zal nog liever naar m'n arrenmoeder gaan.
%% Hier woont zo'n rieke man, die zo vulle g\egrave ven kan.
%% G\egrave f wat, old wat, g\egrave f die arme stumpers wat,
%% 'k Eb zo lange met de foekepot elopen.
%% 'k Eb gien geld om brood te kopen.
%% Foekepotterij, foekepotterij,
%% Geef mij een centje dan ga'k voorbij.
%% \stoplines
%% \stopnarrower
%% \stopbuffer
\startbuffer
Vào ngày 10 tháng Mười Một (một ngày trước Saint Martensday), bọn trẻ ở Hasselt
đi từ nhà này sang nhà khác, hát một bài hát đặc biệt và tự đệm theo bằng một
{\em foekepot}. Chúng sẽ không bỏ đi nếu bạn không cho chúng kẹo hoặc tiền.
Chúng hát thế này:

\startnarrower
\switchtobodyfont[small]
\startlines
Foekepotterij, foekepotterij,
Geef mij een centje dan ga'k voorbij.
Geef mij een alfje dan blijf ik staan,
'k Zal nog liever naar m'n arrenmoeder gaan.
Hier woont zo'n rieke man, die zo vulle g\egrave ven kan.
G\egrave f wat, old wat, g\egrave f die arme stumpers wat,
'k Eb zo lange met de foekepot elopen.
'k Eb gien geld om brood te kopen.
Foekepotterij, foekepotterij,
Geef mij een centje dan ga'k voorbij.
\stoplines
\stopnarrower
\stopbuffer

\typebuffer

%Notice that \type{\startnarrower} $\cdots$
%\type{\stopnarrower} is also used as a begin and end of the
%fontswitch. The function of \type{\startlines} and
%\type{\stoplines} in this example is obvious.
Hãy ghi nhớ rằng \type{\startnarrower} $\cdots$ \type{\stopnarrower} cũng được
dùng để bắt đầu và kết thúc chuyển đổi font. Chức năng của \type{\startlines}
và \type{stoplines} trong ví dụ này cũng giống như vậy.

\start
\getbuffer
\stop

%If you want an overview of the available font family you can
%type:
Nếu bạn muốn xem qua loại font đang có, bạn có thể dùng lệnh:

\startbuffer
\showbodyfont[cmr]
\stopbuffer

\typebuffer

\getbuffer

%\section{Style and size switch in commands}
\section{Chuyển kiểu và cỡ font với lệnh}

%In a number of commands one of the parameters is
%\type{character} to indicate the desired typestyle. For
%example:
Trong một số lệnh, một trong những tham số là \type{character} cho biết kiểu
font muốn có. Ví dụ:

\startbuffer
\setuphead[chapter][style=\tfd]
\stopbuffer

\typebuffer

%In this case the character size for chapters is indicated with
%a command \type{\tfd}. But instead of a command you could
%use the predefined options that are related to the actual
%typeface:
Trong trường hợp này, cỡ kí tự của chương được cho biết với lệnh \type{\tfd}.
Nhưng thay vì dùng lệnh, bạn có thể dùng các tùy chọn đã định nghĩa trước:

\startbuffer
normal  bold  slanted  boldslanted  type  mediaeval
small  smallbold  smallslanted  smallboldslanted smalltype
capital cap
\stopbuffer

\typebuffer

%\section{Local font style and size}
\section{Kiểu và cỡ font chữ cục bộ}

\Command{\tex{rm}}
\Command{\tex{ss}}
\Command{\tex{tt}}
\Command{\tex{sl}}
\Command{\tex{bf}}
\Command{\tex{tfa}}
\Command{\tex{tfb}}
\Command{\tex{tfc}}
\Command{\tex{tfd}}

%In the running text (local) you can change the {\em
%typestyle} into roman, sans serif and teletype with
%\type{\rm}, \type{\ss} and \type{\tt}.
Trong đoạn văn bản đang chạy (cục bộ), bạn có thể chuyển {\em kiểu font} thành
roman, sans serif và teletype với \type{\rm}, \type{\ss} và \type{\tt}.

%You can change the {\em typeface} like italic and boldface
%with \type{\sl} and \type{\bf}.
Bạn có thể chuyển {\em dạng font} thành nghiêng và đậm với \type{\sl} và
\type{\bf}.

%The {\em typesize} is available from 4pt to 12pt and is
%changed with \type{\switchtobodyfont}.
{\em Cỡ font} là từ 4pt đến 12pt và được chuyển với \type{\switchtobodyfont}.

%The actual style is indicated with \type{\tf}. If you want
%to change into a somewhat greater size you can type
%\type{\tfa}, \type{\tfb}, \type{\tfc} and \type{\tfd}. An
%addition of \type{a}, \type{b}, \type{c} and \type{d} to
%\type{\sl}, \type{\it} and \type{\bf} is also allowed.
Kiểu thật sự được cho biết với \type{\tf}. Nếu bạn muốn chuyển sang cỡ lớn
hơn, bạn dùng \type{\tfa}, \type{tfb}, \type{\tfc} và \type{tfd}. Cũng có thể
thêm một chữ trong các chữ \type{a}, \type{b}, \type{c} và \type{d} đến
\type{sl}, \type{it} và \type{bf}.

%\startbuffer
%{\tfc Mintage}

%In the period from {\tt 1404} till {\tt 1585} Hasselt had its own
%{\sl right of coinage}. This right was challenged by other cities,
%but the {\switchtobodyfont[7pt] bishops of Utrecht} did not honour
%these {\slb protests}.
%\stopbuffer
\startbuffer
{\tfc Tiền đúc}

Vào thời kì từ năm {\tt 1404} đến {\tt 1585} Hasselt có {\sl đồng tiền} riêng.
Đồng tiền này không được các thành phố khác chấp nhận nhưng
{\switchtobodyfont[7pt] hội đồng giám mục Utrecht} đã không chấp nhận những sự
{\slb phản đối} này.
\stopbuffer

\typebuffer

%The curly braces indicate begin and end of style or size
%switches.
Dấu ngoặc móc cho biết lúc bắt đầu và kết thúc chuyển kiểu hay cỡ font.

\getbuffer

%\section{Redefining fontsize}
\section{Định nghĩa lại cỡ font}

%\index{fontsize}
\index{cỡ font}

\Command{\tex{definebodyfont}}

%For special purposes you can define your own fontsize.
Bạn có thể định nghĩa lại cỡ font cho các mục đích đặc biệt.

\shortsetup{definebodyfont}

%A definition could look like this:
Một định nghĩa như thê này

\startbuffer
\definebodyfont[10pt][rm][tfe=Regular at 36pt]

{\tfe Hasselt!}
\stopbuffer

\typebuffer

%Now \type{\tfe} will produce 36pt characters saying:
%{\hbox{\getbuffer}}
Bây giờ \type{\tfe} sẽ tạo các kí tự 36pt: {\hbox{\getbuffer}}

%\section{Small caps}
\section{Chữ cái hoa nhỏ}

%\index{small caps}
\index{chữ cái hoa nhỏ}

%\Command{\tex{kap}}
\Command{\tex{cap}}

%Abbreviations like \PDF\ (\infull{PDF}) are printed in
%pseudo small caps. A small capital is somewhat smaller than
%the capital of the actual typeface. Pseudo small caps are
%produced with:
Các từ viết tắt như \PDF\ (\infull{PDF}) được in bằng các chữ cái hoa nhỏ giả.
Một chữ viết hoa nhỏ thì nhỏ hơn một chữ viết hoa dạng thật. Chữ cái hoa nhỏ
giả được tạo với lệnh:

%\shortsetup{kap}
\shortsetup{cap}

%If you compare \type{PDF}, \type{\kap{PDF}} and \type{\sc pdf}:
Nếu bạn so sánh \type{PDF}, \type{cap{PDF}} và \type{\sc pdf}:

%\midaligned{PDF, \kap{PDF} and {\sc pdf}}
\midaligned{PDF, \cap{PDF} và {\sc pdf}}

%you can see the difference. The command \type{\sc} shows the
%real small caps. The reason for using pseudo small caps
%instead of real small caps is just a matter of taste.
bạn sẽ thấy sự khác biệt. Lệnh \type{\sc} hiển thị chữ cái hoa thật. Lí do
dùng chữ cái hoa nhỏ giả chỉ là vấn đề sở thích.

%\section{Emphasized}
\section{Nhấn mạnh}

%\index{emphasized}
\index{nhấn mạnh}

\Command{\tex{em}}

%To emphasize words consistently throughout your document
%you use:
Để nhấn mạnh một từ, bạn dùng:

\starttyping
\em
\stoptyping

%Empasized words appear in a slanted style.
Từ được nhấn mạnh sẽ xuất hiện theo kiểu xiên.

%\startbuffer
%If you walk through Hasselt you should {\bf \em watch out} for
%{\em Amsterdammers}. An {\em Amsterdammer} is {\bf \em not} a
%person from Amsterdam but a little stone pilar used to separate
%sidewalk and road. A pedestrian should be protected by these
%{\em Amsterdammers} against cars but more often people get hurt
%from tripping over them.
%\stopbuffer
\startbuffer
Nếu bạn đi bộ dọc Hasselt bạn nên {\bf \em xem chừng} {\em Amsterdammer}.
Một {\em Amsterdammer} {\bf \em không phải} là người từ Amsterdam mà là hàng
đá nhỏ ngăn cách vỉa hè với đường. Một khách bộ hành sẽ được bảo vệ bởi các
{\em Amsterdammer} này khỏi xe hơi nhưng thường người ta dễ đau chân khi dạo
trên chúng.
\stopbuffer

\typebuffer

\getbuffer

%{\em An emphasize within an emphasize is {\em normal} again
%and a boldface emphasize looks like {\bf this or \em this}}.
{\em Một nhấn mạnh nằm trong một nhấn mạnh khác là {\em bình thường} và một
nhấn mạnh chữ đậm trông như {\bf thế này hoặc \em thế này}}.

%\section{Teletype / verbatim}
\section{Kiểu đánh máy / Nguyên văn}

%\index{type}
%\index{verbatim}
\index{loại}
\index{nguyên văn}

\Command{\tex{starttyping}}
\Command{\tex{type}}
\Command{\tex{setuptyping}}
\Command{\tex{setuptype}}

%If you want to display typed text and want to keep your
%line breaking exactly as it is you use:
Nếu bạn muốn hiển thị chữ đánh máy và muốn xuống dòng chính xác, bạn dùng:

\shortsetup{starttyping}

%In the text you can use:
Đối với chữ, bạn dùng:

\shortsetup{type}

%The curly braces enclose the text you want in teletype.
%You have to be careful with \type{\type} because the
%line breaking mechanism does not work anymore.
Cặp ngoặc móc chứa chữ bạn cần hiện kiểu đánh máy. Bạn nên cẩn thận với lệnh
\type{\type} vì xuống dòng sẽ không đúng.

%You can set up the `typing' with:
Bạn có thể thiết đặt kiểu đánh với:

\shortsetup{setuptyping}
\shortsetup{setuptype}

\page

%\section{Encodings}
\section{Bảng mã}

This section is stil to be filled.

\stopcomponent
