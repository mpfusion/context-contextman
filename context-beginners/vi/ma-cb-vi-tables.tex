\startcomponent ma-cb-vi-tables
\project ma-cb
\product ma-cb-vi
\environment ma-cb-env-vi

%\chapter[tables]{Tables}
\chapter[tables]{Bảng}

%\index{tables}
%\index{floating blocks}
\index{bảng}
\index{khối thay đổi}

\Command{\tex{placetable}}
\Command{\tex{setuptables}}
\Command{\tex{starttable}}
\Command{\tex{startcombination}}
\Command{\tex{setupfloats}}
\Command{\tex{setupcaptions}}
\Command{\tex{NR}}
\Command{\tex{FR}}
\Command{\tex{LR}}
\Command{\tex{MR}}
\Command{\tex{SR}}
\Command{\tex{VL}}
\Command{\tex{NC}}
\Command{\tex{HL}}
\Command{\tex{DL}}
\Command{\tex{DC}}
\Command{\tex{DR}}
\Command{\tex{LOW}}
\Command{\tex{TWO}}
\Command{\tex{THREE}}

%{\em In general, a table consists of columns which may be
%independently left adjusted, centered, right adjusted, or
%aligned on decimal points. Headings may be placed over single
%columns or groups of columns. Table entries may contain
%equations or several rows of text. Horizontal and vertical
%lines may be drawn wholly or partially across the table.}
{\em Tổng quát, một bảng chứa cột có thể độc lập canh trái, canh giữa, canh
phải hoặc gióng hàng theo các số thập phân. Đề mục có thể được đặt trên một
cột hoặc một nhóm cột. Các mục của bảng có thể chứa các phương trình hay vài
hàng văn bản. Các đường kẻ ngang và dọc có thể được vẽ toàn bộ hoặc không
hoàn chỉnh.}

%% This is what Michael J. Wichura wrote in the preface of the
%% manual of \TABLE\ (\TABLE\ manual, 1988.). Michael Wichura
%% is also the author of the \TABLE\ macros \CONTEXT\ is
%% relying on when processing tables. A few \CONTEXT\ macros
%% were added to take care of consistent line spacing and to
%% make the interface a little less cryptic.\footnote{\CONTEXT\
%% was developed for non||technical users in the \cap{WYSIWYG}
%% era. Therefore a user friendly interface and easy file and
%% command handling were needed, and cryptic commands,
%% programming and logical reasoning had to be avoided.}
Trên đây là những gì Michael J. Wichura đã vvieesttrong lời tựa của sổ tay
\TABLE\ (\TABLE\ manual, 1988.). Michael Wichura cũng là tác giả của macro
\TABLE\ mà \CONTEXT\ dựa vào khi thực thi tạo bảng. Một vài macro \CONTEXT\
được thêm vào để xem xét độ rộng đường kẻ phù hợp và cũng để tạo giao diện bớt
khó hiểu hơn.\footnote{\CONTEXT\ được phát triển cho những người dùng không
thành thạo kĩ thuật trong thời đại \cap{WYSIWYG}. Do vậy, một giao diện thân
thiện người dùng, sự dễ dàng trong cách tạo tập tin và trình bày câu lệnh là
điều mong muốn. Còn lại, các câu lệnh khó hiểu, các lý luận hợp logic và lập
trình đều được loại bỏ.}

%For placing a table the command \type{\placetable} is used
%which is a predefined example of:
Để đặt một bảng, bạn dùng lệnh \type{\placetable} như ví dụ cho trước:

\shortsetup{placetable}

%For defining the table you use:
Để định nghĩa bảng, bạn dùng:

\shortsetup{starttable}

%The definition of a table could look something like this:
Cách định nghĩa bảng có thể trông như thế này:

%% \startbuffer
%% \placetable[here][tab:ships]{Ships that moored at Hasselt.}
%% \starttable[|c|c|]
%% \HL
%% \NC \bf Year \NC \bf Number of ships \NC\SR
%% \HL
%% \NC 1645     \NC 450                 \NC\FR
%% \NC 1671     \NC 480                 \NC\MR
%% \NC 1676     \NC 500                 \NC\MR
%% \NC 1695     \NC 930                 \NC\LR
%% \HL
%% \stoptable
%% \stopbuffer
\startbuffer
\placetable[here][tab:tàu]{Số tàu neo tại Hasselt.}
\starttable[|c|c|]
\HL
\NC \bf Năm \NC \bf Số lượng tàu \NC\SR
\HL
\NC 1645     \NC 450                 \NC\FR
\NC 1671     \NC 480                 \NC\MR
\NC 1676     \NC 500                 \NC\MR
\NC 1695     \NC 930                 \NC\LR
\HL
\stoptable
\stopbuffer

\typebuffer

%This table is typeset as \in{table}[tab:ships].
Bảng này được sắp chữ như trong \in{bảng}[tab:tàu].

\getbuffer

%The first command \type{\placetable} has the same function
%as \type{\placefigure}. It takes care of spacing before and
%after the table and numbering. Furthermore the floating
%mechanism is initialized so the table will be placed at the
%most optimal location of the page.
Lệnh đầu tiên \type{\placetable} có cùng chức năng như lệnh
\type{\placefigure}. Nó giữ khoảng trống trước và sau bảng và cách đánh số. Xa
hơn, cơ chế thay đổi được khởi chạy để mà bảng sẽ được đặt ở vị trí tối ưu
nhất trong trang.

%The table entries are placed between the \type{\starttable}
%$\cdots$ \type{\stoptable} pair. Between the bracket pair
%your can specify the table format with the column separators
%\type{|} and the format keys (see
%\in{table}[tab:formatkeys]).
Các mục trong bảng được đặt giữa hai cặp khóa \type{\starttable} $\cdots$
\type{stoptable}. Giữa cặp ngoặc vuông, bạn có thể xác định kiểu bảng với dấu
tách cột \type{|} và các khóa định dạng (xem \in{bảng}[tab:khóa định dạng]).

%% \placetable
%%   []
%%   [tab:formatkeys]
%%   {Table format keys.}
%% \starttable[|l|l|]
%% \HL
%% \NC \bf Key     \NC \bf Meaning                                 \NC\SR
%% \HL
%% \NC \type{|}    \NC column separator                            \NC\FR
%% \NC \type{c}    \NC center                                      \NC\MR
%% \NC \type{l}    \NC flush left                                  \NC\MR
%% \NC \type{r}    \NC flush right                                 \NC\MR
%% \NC \type{s<n>} \NC set intercolumn space at value $n = 0, 1,2$ \NC\MR
%% \NC \type{w<>}  \NC set minimum column width at specified value \NC\LR
%% \HL
%% \stoptable
\placetable
  []
  [tab:khóa định dạng]
  {Bảng các khóa định dạng.}
\starttable[|l|l|]
\HL
\NC \bf Khóa    \NC \bf Ý nghĩa                                        \NC\SR
\HL
\NC \type{|}    \NC dấu tách cột                                       \NC\FR
\NC \type{c}    \NC canh giữa                                          \NC\MR
\NC \type{l}    \NC thẳng theo bên trái                                \NC\MR
\NC \type{r}    \NC thẳng theo bên phải                                \NC\MR
\NC \type{s<n>} \NC đặt giá trị khoảng trống giữa các cột $n = 0, 1,2$ \NC\MR
\NC \type{w<>}  \NC đặt giá trị nhỏ nhất cho độ rộng của cột           \NC\LR
\HL
\stoptable

%In addition to the format keys there are format commands.
%\in{Table}[tab:formatcommands] shows a few of the essential
%commands.
Thêm vào các khoá định dạng có các lệnh định dạng. \in{Bảng}[tab:lệnh định
dạng] thể hiện vài lệnh thông dụng.

%\placetable
%  [here]
%  [tab:formatcommands]
%  {Table format commands.}
%\starttable[|l|l|]
%\HL
%\NC \bf Command               \NC \bf Meaning                            \NC\SR
%\HL
%\NC \type{\JustLeft}          \NC flush left and suppress column format  \NC\FR
%\NC \type{\JustRight}         \NC flush right and suppress column format \NC\MR
%\NC \type{\JustCenter}        \NC center and suppress column format      \NC\MR
%\NC \type{\SetTableToWidth{}} \NC specify exact table width              \NC\MR
%\NC \type{\use{n}}            \NC use the space of the next $n$ columns  \NC\LR
%\HL
%\stoptable
\placetable
  [here]
  [tab:lệnh định dạng]
  {Bảng các lệnh định dạng.}
\starttable[|l|l|]
\HL
\NC \bf Lệnh               \NC \bf Ý nghĩa                            \NC\SR
\HL
\NC \type{\JustLeft}          \NC thẳng trái và bỏ định dạng cột         \NC\FR
\NC \type{\JustRight}         \NC thẳng phải và bỏ định dạng cột         \NC\MR
\NC \type{\JustCenter}        \NC canh giữa và bỏ định dạng cột          \NC\MR
\NC \type{\SetTableToWidth{}} \NC xác định chính xác độ rộng của bảng    \NC\MR
\NC \type{\use{n}}            \NC dùng khoảng trắng của cột thứ $n$ kế   \NC\LR
\HL
\stoptable

%In the examples you have seen so far a number of
%\CONTEXT\ formatting commands were used. These commands are
%somewhat longer than the original and less cryptic but they
%also handle a lot of table typography. In
%\in{table}[tab:contextformatcommands] an overview of these
%commands is given.
Trong các ví dụ, bảng đã thấy nhiều lệnh định dạng \CONTEXT\ được dùng. Những
lệnh này dài hơn nguyên mẫu và ít khó hiểu nhưng chúng cũng tạo được nhiều
kiểu trình bày bảng. Trong \in{bảng}[tab:lệnh định dạng context], bạn sẽ có
cái nhìn tổng quan hơn về các lệnh này.
%% \placetable
%%   [here]
%%   [tab:formatcommands]
%%   {\CONTEXT\ table format commands.}
%% {\setuptables[bodyfont=small]
%% \starttable[s1|l|l|l|]
%% \HL
%% \NC \bf Command       \NC
%%     \NC \bf Meaning                                \NC\SR
%% \HL
%% \NC \type{\NR}        \NC next row
%%     \NC make row with no vertical space adjustment \NC\FR
%% \NC \type{\FR}        \NC first row
%%     \NC make row, adjust upper spacing             \NC\MR
%% \NC \type{\LR}        \NC last row
%%     \NC make row, adjust lower spacing             \NC\MR
%% \NC \type{\MR}        \NC mid row
%%     \NC make row, adjust upper and lower spacing   \NC\MR
%% \NC \type{\SR}        \NC separate row
%%     \NC make row, adjust upper and lower spacing   \NC\MR
%% \NC \type{\VL}        \NC vertical line
%%     \NC draw a vertical line, go to next column    \NC\MR
%% \NC \type{\NC}        \NC next column
%%     \NC go to next column                          \NC\MR
%% \NC \type{\HL}        \NC horizontal line
%%     \NC draw a horizontal line                     \NC\MR
%% \NC \type{\DL}        \NC division line$^\star$
%%     \NC draw a division line over the next column  \NC\MR
%% \NC \type{\DL[n]}     \NC division line$^\star$
%%     \NC draw a division line over $n$ columns      \NC\MR
%% \NC \type{\DC}        \NC division column$^\star$
%%     \NC draw a space over the next column          \NC\MR
%% \NC \type{\DR} \NC division row$^\star$
%%     \NC make row, adjust upper and lower spacing   \NC\MR
%% \NC \type{\LOW{text}} \NC ---
%%     \NC lower {\em text}                           \NC\MR
%% \NC \type{\TWO}, \type{\THREE} etc.  \NC ---
%%     \NC use the space of the next {\em two}, {\em three} columns \NC\LR
%% \HL
%% \NC \use3 \JustLeft{$^\star$ \type{\DL, \DC} and \type{\DR}
%%     are used in combination.}                      \NC\FR
%% \stoptable}
\placetable
  [here]
  [tab:lệnh định dạng context]
  {Các lệnh định dạng bảng của \CONTEXT.}
{\setuptables[bodyfont=small]
\starttable[s1|l|l|l|]
\HL
\NC \bf Lệnh       \NC
    \NC \bf Ý nghĩa                                  \NC\SR
\HL
\NC \type{\NR}        \NC hàng kế
    \NC tạo hàng nhưng không điều chỉnh khoảng trắng \NC\FR
\NC \type{\FR}        \NC hàng đầu tiên
    \NC tạo hàng, điều chỉnh khoảng trắng phía trên       \NC\MR
\NC \type{\LR}        \NC hàng cuối
    \NC tạo hàng, điều chỉnh khoảng trắng phía dưới       \NC\MR
\NC \type{\MR}        \NC hàng giữa
    \NC tạo hàng, điều chỉnh khoảng trắng trên, dưới      \NC\MR
\NC \type{\SR}        \NC hàng riêng rẽ
    \NC tạo hàng, diều chỉnh khoảng trắng trên, dưới   \NC\MR
\NC \type{\VL}        \NC đường kẻ thẳng
    \NC vẽ đường kẻ thẳng, chuyển đến cột kế    \NC\MR
\NC \type{\NC}        \NC cột kế
    \NC chuyển đến cột kế                          \NC\MR
\NC \type{\HL}        \NC đường kẻ ngang
    \NC vẽ đường kẻ ngang                     \NC\MR
\NC \type{\DL}        \NC đường kẻ phân cách$^\star$
    \NC vẽ một đường kẻ ngăn cách đến cột kế       \NC\MR
\NC \type{\DL[n]}     \NC đường kẻ phân cách$^\star$
    \NC vẽ một đường kẻ ngăn cách đến $n$ cột      \NC\MR
\NC \type{\DC}        \NC đường kẻ ngăn cách$^\star$
    \NC vẽ một khoảng trắng đến cột kế         \NC\MR
\NC \type{\DR} \NC đường kẻ phân cách$^\star$
    \NC tạo hàng, điều chỉnh khoảng trắng trên, dưới   \NC\MR
\NC \type{\LOW{văn bản}} \NC ---
    \NC {\em văn bản} thấp hơn                          \NC\MR
\NC \type{\TWO}, \type{\THREE} etc.  \NC ---
    \NC dùng khoảng trắng của {\em hai}, {\em ba} cột kế \NC\LR
\HL
\NC \use3 \JustLeft{$^\star$ \type{\DL, \DC} và \type{\DR}
    được dùng trong dạng kết hợp.}                      \NC\FR
\stoptable}

%The tables below are shown with their sources. You can
%always read the \TABLE\ manual by M.J. Wichura for more
%sophisticated examples.
Các bảng bên dưới được thể hiện cùng với mã của chúng. Bạn có thể đọc sổ tay
\TABLE\ của M.J. Wichura để biết thêm các ví dụ phức tạp

%% \startbuffer
%% \placetable
%%   [here,force]
%%   [tab:effects of commands]
%%   {Effect of formatting commands.}
%%   {\startcombination[2*1]
%%      {\starttable[|c|c|]
%%       \HL
%%       \VL \bf Year \VL \bf Citizens \VL\SR
%%       \HL
%%       \VL 1675     \VL  ~428        \VL\FR
%%       \VL 1795     \VL  1124        \VL\MR
%%       \VL 1880     \VL  2405        \VL\MR
%%       \VL 1995     \VL  7408        \VL\LR
%%       \HL
%%       \stoptable}{standard}
%%      {\starttable[|c|c|]
%%       \HL
%%       \VL \bf Year \VL \bf Citizens \VL\NR
%%       \HL
%%       \VL 1675     \VL  ~428        \VL\NR
%%       \VL 1795     \VL  1124        \VL\NR
%%       \VL 1880     \VL  2405        \VL\NR
%%       \VL 1995     \VL  7408        \VL\NR
%%       \HL
%%       \stoptable}{only \type{\NR}}
%%    \stopcombination}
%% \stopbuffer
\startbuffer
\placetable
  [here,force]
  [tab:tác dụng của lệnh]
  {Tác dụng của các lệnh định dạng.}
  {\startcombination[2*1]
     {\starttable[|c|c|]
      \HL
      \VL \bf Năm \VL \bf Số dân \VL\SR
      \HL
      \VL 1675     \VL  ~428        \VL\FR
      \VL 1795     \VL  1124        \VL\MR
      \VL 1880     \VL  2405        \VL\MR
      \VL 1995     \VL  7408        \VL\LR
      \HL
      \stoptable}{chuẩn}
     {\starttable[|c|c|]
      \HL
      \VL \bf Năm \VL \bf Số dân \VL\NR
      \HL
      \VL 1675     \VL  ~428        \VL\NR
      \VL 1795     \VL  1124        \VL\NR
      \VL 1880     \VL  2405        \VL\NR
      \VL 1995     \VL  7408        \VL\NR
      \HL
      \stoptable}{chỉ \type{\NR}}
   \stopcombination}
\stopbuffer

\typebuffer

%In the example above the first table \type{\SR}, \type{\FR},
%\type{\MR} and \type{\LR} are used. These commands take care
%of line spacing within a table. As you can see below
%the command \type{\NR} only starts a new row.
Trong ví dụ trên, bảng đầu tiên có \type{\SR}, \type{FR}, \type{MR} và
\type{LR} được dùng. Những lệnh này giữ khoảng cách trong bảng. Như bạn thấy,
bảng bên dưới, lệnh \type{NR} chỉ bắt đầu một hàng mới.

\getbuffer

%In the example below column interspacing with the \type{s0}
%and \type{s1} keys is shown.
Trong ví dụ bên dưới, khoảng cách các cột được thể hiện với các khóa \type{s0}
và \type{s1}.

%% \startbuffer[one]
%% \starttable[|c|c|]
%% \HL
%% \VL \bf Year \VL \bf Citizens \VL\SR
%% \HL
%% \VL 1675 \VL  ~428 \VL\FR
%% \VL 1795 \VL  1124 \VL\MR
%% \VL 1880 \VL  2405 \VL\MR
%% \VL 1995 \VL  7408 \VL\LR
%% \HL
%% \stoptable
%% \stopbuffer

%% \startbuffer[two]
%% \starttable[s0 | c | c |]
%% \HL
%% \VL \bf Year \VL \bf Citizens \VL\SR
%% \HL
%% \VL 1675 \VL  ~428 \VL\FR
%% \VL 1795 \VL  1124 \VL\MR
%% \VL 1880 \VL  2405 \VL\MR
%% \VL 1995 \VL  7408 \VL\LR
%% \HL
%% \stoptable
%% \stopbuffer

%% \startbuffer[three]
%% \starttable[| s0 c | c |]
%% \HL
%% \VL \bf Year \VL \bf Citizens \VL\SR
%% \HL
%% \VL 1675 \VL  ~428 \VL\FR
%% \VL 1795 \VL  1124 \VL\MR
%% \VL 1880 \VL  2405 \VL\MR
%% \VL 1995 \VL  7408 \VL\LR
%% \HL
%% \stoptable
%% \stopbuffer

%% \startbuffer[four]
%% \starttable[| c | s0 c |]
%% \HL
%% \VL \bf Year \VL \bf Citizens \VL\SR
%% \HL
%% \VL 1675 \VL  ~428 \VL\FR
%% \VL 1795 \VL  1124 \VL\MR
%% \VL 1880 \VL  2405 \VL\MR
%% \VL 1995 \VL  7408 \VL\LR
%% \HL
%% \stoptable
%% \stopbuffer

%% \startbuffer[five]
%% \starttable[s1 | c | c |]
%% \HL
%% \VL \bf Year \VL \bf Citizens \VL\SR
%% \HL
%% \VL 1675 \VL  ~428 \VL\FR
%% \VL 1795 \VL  1124 \VL\MR
%% \VL 1880 \VL  2405 \VL\MR
%% \VL 1995 \VL  7408 \VL\LR
%% \HL
%% \stoptable
%% \stopbuffer

%% \placetable
%%   [here,force]
%%   [tab:example formatcommands]
%%   {Effect of formatting commands.}
%%   {\startcombination[3*2]
%%     {\getbuffer[one]}   {standard}
%%     {\getbuffer[two]}   {\type{s0}}
%%     {\getbuffer[three]} {\type{s0} in column~1}
%%     {\getbuffer[four]}  {\type{s0} in column~2}
%%     {\getbuffer[five]}  {\type{s1}}
%%     {}                  {}
%%   \stopcombination}
%% \stopbuffer
\startbuffer
\startbuffer[one]
\starttable[|c|c|]
\HL
\VL \bf Năm \VL \bf Số dân \VL\SR
\HL
\VL 1675 \VL  ~428 \VL\FR
\VL 1795 \VL  1124 \VL\MR
\VL 1880 \VL  2405 \VL\MR
\VL 1995 \VL  7408 \VL\LR
\HL
\stoptable
\stopbuffer

\startbuffer[two]
\starttable[s0 | c | c |]
\HL
\VL \bf Năm \VL \bf Số dân \VL\SR
\HL
\VL 1675 \VL  ~428 \VL\FR
\VL 1795 \VL  1124 \VL\MR
\VL 1880 \VL  2405 \VL\MR
\VL 1995 \VL  7408 \VL\LR
\HL
\stoptable
\stopbuffer

\startbuffer[three]
\starttable[| s0 c | c |]
\HL
\VL \bf Năm \VL \bf Số dân \VL\SR
\HL
\VL 1675 \VL  ~428 \VL\FR
\VL 1795 \VL  1124 \VL\MR
\VL 1880 \VL  2405 \VL\MR
\VL 1995 \VL  7408 \VL\LR
\HL
\stoptable
\stopbuffer

\startbuffer[four]
\starttable[| c | s0 c |]
\HL
\VL \bf Năm \VL \bf Số dân \VL\SR
\HL
\VL 1675 \VL  ~428 \VL\FR
\VL 1795 \VL  1124 \VL\MR
\VL 1880 \VL  2405 \VL\MR
\VL 1995 \VL  7408 \VL\LR
\HL
\stoptable
\stopbuffer

\startbuffer[five]
\starttable[s1 | c | c |]
\HL
\VL \bf Năm \VL \bf Số dân \VL\SR
\HL
\VL 1675 \VL  ~428 \VL\FR
\VL 1795 \VL  1124 \VL\MR
\VL 1880 \VL  2405 \VL\MR
\VL 1995 \VL  7408 \VL\LR
\HL
\stoptable
\stopbuffer

\placetable
  [here,force]
  [tab:ví dụ về lệnh định dạng]
  {Tác dụng của các lệnh định dạng.}
  {\startcombination[3*2]
    {\getbuffer[one]}   {standard}
    {\getbuffer[two]}   {\type{s0}}
    {\getbuffer[three]} {\type{s0} in column~1}
    {\getbuffer[four]}  {\type{s0} in column~2}
    {\getbuffer[five]}  {\type{s1}}
    {}                  {}
  \stopcombination}
\stopbuffer

\typebuffer

%After processing these tables come out as
%\in{table}[tab:example formatcommands]. The default table
%has a column interspacing of\type{s2}.
Sau khi thực thi, các bảng này xuất ra như \in{bảng}[tab:ví dụ về lệnh định
dạng]. Bảng mặc định có khoảng cách một cột là \type{s2}

\getbuffer

%Columns are often separated with a vertical line $|$ and
%rows by a horizontal line.
Cột thường được ngăn cách bởi các đường kẻ thẳng $|$ và hàng được ngăn cách
bởi các đường kẻ ngang.

%% \startbuffer
%% \placetable
%%   [here,force]
%%   [tab:divisions]
%%   {Effect of options.}
%% \starttable[|c|c|c|]
%% \NC Steenwijk  \NC Zwartsluis \NC Hasselt    \NC\SR
%% \DC            \DL            \DC               \DR
%% \NC Zwartsluis \VL Hasselt    \VL Steenwijk  \NC\SR
%% \DC            \DL            \DC               \DR
%% \NC Hasselt    \NC Steenwijk  \NC Zwartsluis \NC\SR
%% \stoptable
%% \stopbuffer
\startbuffer
\placetable
  [here,force]
  [tab:đường phân cách]
  {Tác dụng của các tùy chọn.}
\starttable[|c|c|c|]
\NC Steenwijk  \NC Zwartsluis \NC Hasselt    \NC\SR
\DC            \DL            \DC               \DR
\NC Zwartsluis \VL Hasselt    \VL Steenwijk  \NC\SR
\DC            \DL            \DC               \DR
\NC Hasselt    \NC Steenwijk  \NC Zwartsluis \NC\SR
\stoptable
\stopbuffer

\typebuffer

\getbuffer

%A more sensible example is given in the
%\in{table}[tab:example contextcommands].
Một ví dụ dễ nhận thấy hơn được thể hiện trong \in{bảng}[tab:ví dụ lệnh
context].

%% \startbuffer
%% \placetable
%%   [here,force]
%%   [tab:example contextcommands]
%%   {Effect of \CONTEXT\ formatting commands.}
%% \starttable[|l|c|c|c|c|]
%% \HL
%% \VL \FIVE \JustCenter{City council elections in 1994}    \VL\SR
%% \HL
%% \VL \LOW{Party} \VL \THREE{Districts}   \VL \LOW{Total} \VL\SR
%% \DC             \DL[3]                  \DC                \DR
%% \VL             \VL 1   \VL 2   \VL 3   \VL             \VL\SR
%% \HL
%% \VL PvdA        \VL 351 \VL 433 \VL 459 \VL 1243        \VL\FR
%% \VL CDA         \VL 346 \VL 350 \VL 285 \VL ~981        \VL\MR
%% \VL VVD         \VL 140 \VL 113 \VL 132 \VL ~385        \VL\MR
%% \VL HKV/RPF/SGP \VL 348 \VL 261 \VL 158 \VL ~767        \VL\MR
%% \VL GPV         \VL 117 \VL 192 \VL 291 \VL ~600        \VL\LR
%% \HL
%% \stoptable
%% \stopbuffer
\startbuffer
\placetable
  [here,force]
  [tab:ví dụ lệnh context]
  {Tác dụng của các lệnh định dạng trong \CONTEXT.}
\starttable[|l|c|c|c|c|]
\HL
\VL \FIVE \JustCenter{Bầu cử hội đồng thành phố trong năm 1994}    \VL\SR
\HL
\VL \LOW{Đảng} \VL \THREE{Quận}   \VL \LOW{Tổng} \VL\SR
\DC             \DL[3]                  \DC                \DR
\VL             \VL 1   \VL 2   \VL 3   \VL             \VL\SR
\HL
\VL PvdA        \VL 351 \VL 433 \VL 459 \VL 1243        \VL\FR
\VL CDA         \VL 346 \VL 350 \VL 285 \VL ~981        \VL\MR
\VL VVD         \VL 140 \VL 113 \VL 132 \VL ~385        \VL\MR
\VL HKV/RPF/SGP \VL 348 \VL 261 \VL 158 \VL ~767        \VL\MR
\VL GPV         \VL 117 \VL 192 \VL 291 \VL ~600        \VL\LR
\HL
\stoptable
\stopbuffer

\typebuffer

%In the last column a \type{~} is used to simulate a four
%digit number. The \type{~} has the width of a digit.
Trong cột cuối, một \type{~} được dùng để giả lập một số bốn chữ số. Kí tự
\type{~} có độ rộng của một chữ số.

\getbuffer

%Sometimes your tables get too big and you want to adjust, for
%example, the body font or the vertical and/or horizontal spacing
%around vertical and horizontal lines. This is done by:
Thỉnh thoảng bảng của bạn nhận được quá lớn và bạn muốn điều chỉnh, ví dụ:
font chữ phần thân hay khoảng trắng dọc, ngang quanh các đường kẻ. Việc này
được làm như sau:

\shortsetup{setuptables}

%% \startbuffer
%% \placetable
%%   [here,force]
%%   [tab:setuptable]
%%   {Use of \type{\setuptables}.}
%% {\startcombination[1*3]
%% {\setuptables[bodyfont=10pt]
%% \starttable[|c|c|c|c|c|c|]
%% \HL
%% \VL \use6 \JustCenter{Decline of wealth in
%%                       Dutch florine (Dfl)} \VL\SR
%% \HL
%% \VL Year \VL 1.000--2.000
%%          \VL 2.000--3.000
%%          \VL 3.000--5.000
%%          \VL 5.000--10.000
%%          \VL   over 10.000 \VL\SR
%% \HL
%% \VL 1675 \VL 22 \VL 7 \VL 5  \VL 4  \VL 5  \VL\FR
%% \VL 1724 \VL ~4 \VL 4 \VL -- \VL 4  \VL 3  \VL\MR
%% \VL 1750 \VL 12 \VL 3 \VL 2  \VL 2  \VL -- \VL\MR
%% \VL 1808 \VL ~9 \VL 2 \VL -- \VL -- \VL -- \VL\LR
%% \HL
%% \stoptable}{\tt bodyfont=10pt}
%% {\setuptables[bodyfont=8pt]
%% \starttable[|c|c|c|c|c|c|]
%% \HL
%% \VL \use6 \JustCenter{Decline of wealth in
%%                       Dutch florine (Dfl)} \VL\SR
%% \HL
%% \VL Year \VL 1.000--2.000
%%          \VL 2.000--3.000
%%          \VL 3.000--5.000
%%          \VL 5.000--10.000
%%          \VL   over 10.000 \VL\SR
%% \HL
%% \VL 1675 \VL 22 \VL 7 \VL 5  \VL 4  \VL 5  \VL\FR
%% \VL 1724 \VL ~4 \VL 4 \VL -- \VL 4  \VL 3  \VL\MR
%% \VL 1750 \VL 12 \VL 3 \VL 2  \VL 2  \VL -- \VL\MR
%% \VL 1808 \VL ~9 \VL 2 \VL -- \VL -- \VL -- \VL\LR
%% \HL
%% \stoptable}{\tt bodyfont=8pt}
%% {\setuptables[bodyfont=6pt,distance=small]
%% \starttable[|c|c|c|c|c|c|]
%% \HL
%% \VL \use6 \JustCenter{Decline of wealth in
%%                       Dutch florine (Dfl)} \VL\SR
%% \HL
%% \VL Year \VL 1.000--2.000
%%          \VL 2.000--3.000
%%          \VL 3.000--5.000
%%          \VL 5.000--10.000
%%          \VL   over 10.000 \VL\SR
%% \HL
%% \VL 1675 \VL 22 \VL 7 \VL 5  \VL 4  \VL 5  \VL\FR
%% \VL 1724 \VL ~4 \VL 4 \VL -- \VL 4  \VL 3  \VL\MR
%% \VL 1750 \VL 12 \VL 3 \VL 2  \VL 2  \VL -- \VL\MR
%% \VL 1808 \VL ~9 \VL 2 \VL -- \VL -- \VL -- \VL\LR
%% \HL
%% \stoptable}{\tt bodyfont=6pt,distance=small}
%% \stopcombination}
%% \stopbuffer
\startbuffer
\placetable
  [here,force]
  [tab:thiết lập bảng]
  {Cách dùng \type{\setuptables}.}
{\startcombination[1*3]
{\setuptables[bodyfont=10pt]
\starttable[|c|c|c|c|c|c|]
\HL
\VL \use6 \JustCenter{Sự mất giá của đồng tiền Hà Lan (Dfl)} \VL\SR
\HL
\VL Năm \VL 1.000--2.000
         \VL 2.000--3.000
         \VL 3.000--5.000
         \VL 5.000--10.000
         \VL   over 10.000 \VL\SR
\HL
\VL 1675 \VL 22 \VL 7 \VL 5  \VL 4  \VL 5  \VL\FR
\VL 1724 \VL ~4 \VL 4 \VL -- \VL 4  \VL 3  \VL\MR
\VL 1750 \VL 12 \VL 3 \VL 2  \VL 2  \VL -- \VL\MR
\VL 1808 \VL ~9 \VL 2 \VL -- \VL -- \VL -- \VL\LR
\HL
\stoptable}{\tt bodyfont=10pt}
{\setuptables[bodyfont=8pt]
\starttable[|c|c|c|c|c|c|]
\HL
\VL \use6 \JustCenter{Sự mất giá của đồng tiền Hà Lan (Dfl)} \VL\SR
\HL
\VL Năm \VL 1.000--2.000
         \VL 2.000--3.000
         \VL 3.000--5.000
         \VL 5.000--10.000
         \VL   over 10.000 \VL\SR
\HL
\VL 1675 \VL 22 \VL 7 \VL 5  \VL 4  \VL 5  \VL\FR
\VL 1724 \VL ~4 \VL 4 \VL -- \VL 4  \VL 3  \VL\MR
\VL 1750 \VL 12 \VL 3 \VL 2  \VL 2  \VL -- \VL\MR
\VL 1808 \VL ~9 \VL 2 \VL -- \VL -- \VL -- \VL\LR
\HL
\stoptable}{\tt bodyfont=8pt}
{\setuptables[bodyfont=6pt,distance=small]
\starttable[|c|c|c|c|c|c|]
\HL
\VL \use6 \JustCenter{Sự mất giá của đồng tiền Hà Lan (Dfl)} \VL\SR
\HL
\VL Năm \VL 1.000--2.000
         \VL 2.000--3.000
         \VL 3.000--5.000
         \VL 5.000--10.000
         \VL   over 10.000 \VL\SR
\HL
\VL 1675 \VL 22 \VL 7 \VL 5  \VL 4  \VL 5  \VL\FR
\VL 1724 \VL ~4 \VL 4 \VL -- \VL 4  \VL 3  \VL\MR
\VL 1750 \VL 12 \VL 3 \VL 2  \VL 2  \VL -- \VL\MR
\VL 1808 \VL ~9 \VL 2 \VL -- \VL -- \VL -- \VL\LR
\HL
\stoptable}{\tt bodyfont=6pt,distance=small}
\stopcombination}
\stopbuffer

\typebuffer

\getbuffer

%You can also set up the layout of tables with:
Bạn cũng có thể thiết lập khung nền của bảng với:

\shortsetup{setupfloats}

%You can set up the numbering and the labels with:
Bạn có thể thiết lập cách đánh số và đặt nhãn bảng với:

\shortsetup{setupcaptions}

%These  commands are typed in the set up area of your
%input file and have a global effect on all floating blocks.
Những lệnh này được nhập vào trong phần thiết lập của tập tin nhập liệu và có
tác dụng toàn bộ lên tất cả các khối thay đổi.

%% \startbuffer
%% \setupfloats[location=left]
%% \setupcaption[style=boldslanted]

%% \placetable[here][tab:opening hours]{Library opening hours.}
%% \starttable[|l|c|c|]
%% \HL
%% \VL \bf Day   \VL \use2 \bf Opening hours           \VL\SR
%% \HL
%% \VL Monday    \VL 14.00 -- 17.30 \VL 18.30 -- 20.30 \VL\FR
%% \VL Tuesday   \VL                \VL                \VL\MR
%% \VL Wednesday \VL 10.00 -- 12.00 \VL 14.00 -- 17.30 \VL\MR
%% \VL Thursday  \VL 14.00 -- 17.30 \VL 18.30 -- 20.30 \VL\MR
%% \VL Friday    \VL 14.00 -- 17.30 \VL                \VL\MR
%% \VL Saturday  \VL 10.00 -- 12.30 \VL                \VL\LR
%% \HL
%% \stoptable
%% \stopbuffer
\startbuffer
\setupfloats[location=left]
\setupcaption[style=boldslanted]

\placetable[here][tab:giờ mở cửa]{Giờ mở cửa thư viện.}
\starttable[|l|c|c|]
\HL
\VL \bf Ngày   \VL \use2 \bf Giờ mở cửa           \VL\SR
\HL
\VL Thứ Hai    \VL 14.00 -- 17.30 \VL 18.30 -- 20.30 \VL\FR
\VL Thứ Ba     \VL                \VL                \VL\MR
\VL Thứ Tư     \VL 10.00 -- 12.00 \VL 14.00 -- 17.30 \VL\MR
\VL Thứ Năm    \VL 14.00 -- 17.30 \VL 18.30 -- 20.30 \VL\MR
\VL Thứ Sáu    \VL 14.00 -- 17.30 \VL                \VL\MR
\VL Thứ Bảy    \VL 10.00 -- 12.30 \VL                \VL\LR
\HL
\stoptable
\stopbuffer

\typebuffer

%The result is displayed in \in{table}[tab:opening hours].
Kết quả được hiển thị trong \in{bảng}[tab:giờ mở cửa]

\start
\getbuffer
\stop

\stopcomponent
