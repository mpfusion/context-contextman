\startcomponent ma-cb-en-figures
\project ma-cb
\product ma-cb-en
\environment ma-cb-env-vi

%\chapter[figures]{Figures}
\chapter[figures]{Hình Ảnh}

%\index{figure}
\index{hình ảnh}
%\seeindex{picture}{figure}
%\index{floating blocks}
\index{khối thay đổi}

\Command{\tex{placefigure}}
\Command{\tex{startfiguretext}}
\Command{\tex{setupfigures}}
\Command{\tex{startcombination}}
\Command{\tex{setupfloats}}
\Command{\tex{setupcaptions}}
\Command{\tex{externalfigure}}

%Photographs and pictures can be inserted in your document
%with the following command:
Tranh và ảnh có thể được chèn vào tài liệu của bạn với các lệnh sau:

%\startbuffer
%\placefigure
%   [][fig:church]
%   {Stephanus Church.}
%   {\externalfigure[ma-cb-24][width=.4\textwidth]}
%\stopbuffer
\startbuffer
\placefigure
   [][fig:nhà thờ]
   {Nhà thờ Stephanus.}
   {\externalfigure[ma-cb-24][width=.4\textwidth]}
\stopbuffer

\typebuffer

%After processing this will come out as
%\in{figure}[fig:church] at the first available place.
Sau khi thực thi, lệnh trên sẽ xuất ra như \in{hình}[fig:nhà thờ]

\getbuffer

%The command \type{\placefigure} handles numbering and
%vertical spacing before and after your figure. Furthermore
%this command initializes a float mechanism, which means that
%\CONTEXT\ looks whether there is enough space for your
%figure on the page. If not the figure will be placed at
%another location and the text carries on, while the figure
%floats in your document until the optimal location is found.
%You can influence this mechanism within the first bracket
%pair.
Lệnh \type{\placefigure} đánh số và tạo khoảng trắng trước và sau hình ảnh của
bạn. Xa hơn, lệnh này khởi chạy một cơ chế thay đổi nghĩa là \CONTEXT\ tìm trong
trang một khoảng trống đủ cho hình ảnh của bạn. Nếu không có, hình ảnh sẽ được
đặt ở một vị trí khác trong khi đoạn văn bản vẫn được tiếp tục sắp xếp cho đến
khi vị trí tốt nhất được tìm thấy. Bạn có thể tác động cơ chế này trong cặp
ngoặc vuông.

%The command \type{\placefigure} is a predefined example of:
Lệnh \type{\placefigure} là một ví dụ được xác định trước của:

% VZ (2006-11-13)
%\shortsetup{placeblock}
\shortsetup{placefigure}

%The options are described in \in{table}[tab:placefigure].
Các tùy chọn được diễn giải trong \in{bảng}[tab:placefigure].

%\placetable
  %[here]
  %[tab:placefigure]
  %{Options in \type{\placefigure}.}
%\starttable[|l|l|]
%\HL
%\NC \bf Option \NC \bf Meaning                             \NC\SR
%\HL
%\NC here       \NC put figure at this location if possible \NC\FR
%\NC force      \NC force figure placement here             \NC\MR
%\NC page       \NC put figure on its own page              \NC\MR
%\NC top        \NC put the figure at the top of the page   \NC\MR
%\NC bottom     \NC put the figure at the botom of the page \NC\MR
%\NC left       \NC place figure at the left margin         \NC\MR
%\NC right      \NC place figure at the right margin        \NC\MR
%\NC margin     \NC place figure in the margin              \NC\LR
%\HL
%\stoptable
\placetable
  [here]
  [tab:placefigure]
  {Tùy chọn của lệnh \type{\placefigure}.}
\starttable[|l|l|]
\HL
\NC \bf Tùy chọn \NC \bf Ý nghĩa                             \NC\SR
\HL
\NC here       \NC đặt hình ảnh tại vị trí này nếu có thể  \NC\FR
\NC force      \NC vị trí hình ảnh bắt buộc ở đây          \NC\MR
\NC page       \NC đặt hình ảnh trong trang riêng của nó   \NC\MR
\NC top        \NC đặt hình ảnh tại phần trên của trang    \NC\MR
\NC bottom     \NC đặt hình ảnh tại phần dưới của trang    \NC\MR
\NC left       \NC đặt hình ảnh tại lề bên trái            \NC\MR
\NC right      \NC đặt hình ảnh tại lề bên phải            \NC\MR
\NC margin     \NC đặt hình ảnh tại lề                     \NC\LR
\HL
\stoptable

%The second bracket pair is used for cross-referencing. You can
%refer to this particular figure by typing:
Cặp ngoặc vuông thứ hai được dùng để tham khảo chéo. Bạn có thể tham khảo đến
riêng hình ảnh này bằng cách:

%\starttyping
%\in{figure}[fig:church]
%\stoptyping
\starttyping
\in{hình}[fig:nhà thờ]
\stoptyping

%The first brace pair is used for the caption. You can type
%any text you want. If you want no caption and no number, you
%can type \type{{none}}. The figure labels are set up with
%\type{\setupcaptions} and the numbering is (re)set by
%\type{\setupnumbering} (see \in{paragraph}[floatingblocks]).
Cặp ngoặc móc đầu tiên được dùng để chú thích. Bạn có thể nhập bất cứ chữ nào
bạn muốn. Nếu bạn muốn không chú thích và đánh số, bạn có thể đặt
\type{{none}}. Nhãn ảnh được thiết lập với \type{\setupcaptions} và đánh số
được (tái) thiết lập với \type{\setupnumbering} (xem
		\in{đoạn}[floatingblocks]).

%The second brace pair is used for defining the figure and
%addressing the file names of external figures.
Cặp ngoặc móc thứ hai được dùng xác định đặc điểm để hình ảnh hiển thị và
thông báo tên tập tin hình ảnh được dùng.

%In the next example you see how
%\inframed[height=1em]{Hasselt} is defined within the last
%brace pair to show you the function of \type{\placefigure{}{}}.
Trong ví dụ kế, chúng ta xem \inframed[height=1em]{Hasselt} được xác định như
thế nào trong cặp ngoặc móc cuối để bạn thấy được chức năng của
\type{\placefigure{}{}}.

%\startbuffer
%\placefigure
%  {A framed Hasselt.}
%  {\framed{\tfd Hasselt}}
%\stopbuffer
\startbuffer
\placefigure
  {Hasselt trong khung.}
  {\framed{\tfd Hasselt}}
\stopbuffer

\typebuffer

%This will produce:
Sẽ cho ra:

\getbuffer

%However, your pictures are often created using programs like
%Corel Draw or Illustrator and photos are --- after scanning ---
%improved in packages like PhotoShop. Then the figures are
%available as files. \CONTEXT\ supports all types that are handled
%by the backend driver used. If you use \PDFTEX\ you can insert
%\type {JPG}, \type {PNG} and (pages from) \type {PDF} files as
%well as \METAPOST\ output (\type {MPS} files). Users normally can
%trust \CONTEXT\ to find the best possible file type.
Tuy nhiên, tranh của bạn thường được tạo bởi các chương trình như Corel Draw
hoặc Illustrator và ảnh --- sau khi scan --- được biên tập bở Photoshop. Khi
đó, các hình ảnh là các tập tin. \CONTEXT\ hỗ trợ tất cả loại tập tin được sử
dụng bởi các chương trình điều khiển nền. Nếu bạn dùng \PDFTEX\ bạn có thể
chèn \type {JPG}, \type {PNG} và (các trang trong) tập tin \type {PDF} cũng
như kết quả \METAPOST\ (tập tin \type {MPS}). Người dùng bình thường có thể
tin \CONTEXT\ nhận ra loại tập tin tốt nhất có thể.

%In \in{figure}[fig:canals] you see a photo and a graphic
%combined into one figure.
Trong \in{hình}[fig:kênh], bạn thấy một bức ảnh và một bức tranh được lồng vào
trong một khung hình.

%\startbuffer
%\placefigure
%  [here,force]
%  [fig:canals]
%  {The Hasselt Canals.}
%  {\startcombination[2*1]
%     {\externalfigure[ma-cb-03][width=.4\textwidth]}
%         {a bitmap picture}
     %{\externalfigure[gracht][width=.4\textwidth]}
%     {\externalfigure[ma-cb-00][width=.4\textwidth]}
%         {a vector graphic}
%   \stopcombination}
%\stopbuffer
\startbuffer
\placefigure
  [here,force]
  [fig:kênh]
  {Kênh Hasselt.}
  {\startcombination[2*1]
     {\externalfigure[ma-cb-03][width=.4\textwidth]}
         {ảnh bitmap}
     {\externalfigure[ma-cb-00][width=.4\textwidth]}
         {tranh vector}
   \stopcombination}
\stopbuffer

\getbuffer

%You can produce this figure by typing something like:
Bạn có thể tạo hình này bằng cách:
\typebuffer

%In this figure two pictures are combined with:
Trong hình này, hai bức ảnh được kết hợp lại với:

\shortsetup{startcombination}

%The \type{\startcombination} $\cdots$
%\type{\stopcombination} pair is used for combining two
%pictures in one figure. You can type the number of pictures
%within the bracket pair. If you want to display one picture
%below the other you would have typed \type{[1*2]}. You can
%imagine what happens when you combine 6~pictures as
%\type{[3*2]} (\type{[rows*columns]}).
Cặp lệnh \type{\startcombination} $\cdots$ \type{\stopcombination} được dùng
để kết hợp các bức ảnh vào trong một khung hình. Bạn có thể nhập số bức ảnh
vào cặp ngoặc vuông. Nếu bạn muốn hiển thị ảnh này dưới ảnh kia, bạn nhập vào
\type{[1*2]}. Bạn có hình dung được chuyện gì khi bạn lồng ghép 6~bức ảnh
như thế này \type{[3*2]} (\type{hàng*cột]}).

%The examples shown above are enough for creating illustrated
%documents. Sometimes however you want a more integrated
%layout of the picture and the text. For that purpose
%you can use:
Các ví dụ ở trên đủ để tạo các tài liệu tranh ảnh. Tuy nhiên thỉnh thoảng bạn
cần một khung nền tích hợp ảnh và chữ. Để thực hiện ý định đó, bạn có thể
dùng:

\shortsetup{startframedtext}

%Figure and table texts are already predefined:
Hình và bảng chữ được xác định như sau:

%\startbuffer
%\startfiguretext
%  [left]
%  [fig:citizens]
%  {none}
%  {\externalfigure[ma-cb-18][width=.5\makeupwidth]}
%   Hasselt has always had a varying number of citizens due to
%   economic events. For example the Dedemsvaart was dug around
%   1810. This canal runs through Hasselt and therefore trade
%   flourished. This led to a population growth of almost 40\%
%   within 10~years. Nowadays the Dedemsvaart has no commercial
%   value anymore and the canals have become a tourist
%   attraction. But reminders of these prosperous times can be
%   found everywhere.
%\stopfiguretext
%\stopbuffer
\startbuffer
\startfiguretext
  [left]
  [fig:kênh]
  {none}
  {\externalfigure[ma-cb-18][width=.5\makeupwidth]}
   Hasselt có số dân luôn biến đổi do các sự kiện kinh tế. Ví dụ kênh
   Dedemsvaart được đào vào năm 1810. Con kênh này chạy xuyên qua Hasselt và 
   thúc đẩy giao thương. Việc này đã khiến dân số tăng lên khoảng 40\% trong
   10~năm. Ngày nay, kênh Dedemsvaart không còn giá trị thương mại nhiều và
   các con kênh trở thành điểm du lịch hấp dẫn. Nhưng sự gợi nhớ đến thời kì
   thịnh vượng này có thể tìm thấy mọi nơi.
\stopfiguretext
\stopbuffer

\typebuffer

%is shown in the figure below.
được hiển thị như hình bên dưới.

\start
\setuptolerance[verytolerant]
\getbuffer
\stop

\shortsetup{externalfigure}

%% %The last curly brace pair encloses the command
%% %\type{\externalfigure}. This command gives you the freedom
%% %to do anything you want with a figure.
%% %\type{\externalfigure} has two bracket pairs. The first is
%% %used for the exact file name without extension, the second
%% %for file formats and dimensions. It is not difficult to
%% %guess what happens if you type:\footnote{See
%% %\at{page}[marginpicture]}.
Cặp ngoặc móc cuối bao quanh lệnh \type{\externalfigure}. Lệnh này cho bạn sự
tự do làm bất cứ điều gì bạn muốn với hình. Lệnh \type{\externalfigure} có hai
cặp ngoặc vuông. Cặp đầu được dùng để chỉ chính xác tên tập tin không cần phần
mở rộng, cặp thứ hai dùng để định dạng và xác định kích thước tập tin. Không
khó để đoán chuyện gì nếu bạn nhập vào:\footnote{Xem
	\at{trang}[marginpicture].}

\startbuffer[marginpicture]
\inmargin
  {\externalfigure
     [ma-cb-23]
     [width=\marginwidth]}
\stopbuffer

\typebuffer[marginpicture]

%You can set up the layout of figures with:
Bạn có thể thiết lập khung nền của hình với:

\shortsetup{setupfloats}

%You can set up the numbering and the labels with:
Bạn có thể thiết lập cách đánh số và đặt nhãn với:

\shortsetup{setupcaptions}

%These commands are typed in the set up area of your input
%file and have a global effect on all floating blocks.
Những lệnh này được nhập trong phần thiết lập của tập tin nhập liệu và có tác
dụng toàn bộ trên tất cả các khối thay đổi.

%% \startbuffer
%% \setupfloats
%%   [location=right]
%% \setupcaptions
%%   [location=top,
%%    style=boldslanted]

%% \placefigure
%%   {A characteristic view on Hasselt.}
%%   {\externalfigure[ma-cb-12][width=8cm]}
%% \stopbuffer
\startbuffer
\setupfloats
  [location=right]
\setupcaptions
  [location=top,
   style=boldslanted]

\placefigure
  {Một góc nhìn dặc trưng ở Hasselt.}
  {\externalfigure[ma-cb-12][width=8cm]}
\stopbuffer

\typebuffer

{\getbuffer}

\stopcomponent
