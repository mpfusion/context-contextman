\startcomponent ma-cb-vi-heads
\project ma-cb
\product ma-cb-vi
\environment ma-cb-env-vi

%% \chapter[headers]{Headers}
\chapter[headers]{Headers}

%% \index{headers}
\index{headers}

\Command{\tex{chapter}}
\Command{\tex{paragraph}}
\Command{\tex{subparagraph}}
\Command{\tex{title}}
\Command{\tex{subject}}
\Command{\tex{subsubject}}
\Command{\tex{setuphead}}
\Command{\tex{setupheads}}

%% The structure of a document is determined by its headers.
%% Headers (heads) are created with the commands shown in
%% \in{table}[tab:headers]:
Cấu trúc của tài liệu được định rõ bởi phần đầu của nó. Phần đầu (head) được
tạo với các lệnh được cho trong
\in{table}[tab:headers]:

%% \placetable[here][tab:headers]{Headers.}
%% \starttable[|l|l|]
%% \HL
%% \NC \bf Numbered header   \NC \bf Un-numbered header   \NC\SR
%% \HL
%% \NC \type{\chapter}       \NC \type{\title}           \NC\FR
%% \NC \type{\section}       \NC \type{\subject}         \NC\MR
%% \NC \type{\subsection}    \NC \type{\subsubject}      \NC\MR
%% \NC \type{\subsubsection} \NC \type{\subsubsubject}   \NC\MR
%% \NC $\cdots$              \NC $\cdots$                \NC\LR
%% \HL
%% \stoptable
\placetable[here][tab:headers]{Header.}
\starttable[|l|l|]
\HL
\NC \bf Được đánh số   \NC \bf Không đánh số   \NC\SR
\HL
\NC \type{\chapter}       \NC \type{\title}           \NC\FR
\NC \type{\section}       \NC \type{\subject}         \NC\MR
\NC \type{\subsection}    \NC \type{\subsubject}      \NC\MR
\NC \type{\subsubsection} \NC \type{\subsubsubject}   \NC\MR
\NC $\cdots$              \NC $\cdots$                \NC\LR
\HL
\stoptable

\shortsetup{chapter}
\shortsetup{section}
\shortsetup{subsection}
\shortsetup{title}
\shortsetup{subject}
\shortsetup{subsubject}

%% These commands will produce a header in a
%% predefined fontsize and fonttype with some vertical
%% spacing before and after the header.
Những lệnh này sẽ tạo một header bằng kiểu và kích cỡ font được định trước với
một ít khỏoảg trắng trước và sau.

%% The heading commands can take several arguments, like in:
Các lệnh có thể có một vài đối số như thế này:

%% \starttyping
%% \title[hasselt-by-night]{Hasselt by night}
%% \stoptyping
%%
%% and
%%
%% \starttyping
%% \title{Hasselt by night}
%% \stoptyping
\starttyping
\title[hasselt-by-night]{Hasselt by night}
\stoptyping

và

\starttyping
\title{Hasselt by night}
\stoptyping

%% The bracket pair is optional and used for internal
%% references. If you want to refer to this header you type for
%% example \type{\at{page}[hasselt-by-night]}.
Cặp ngoặc vuông là tùy chọn và được dùng cho các tham khảo bên trong. Ví dụ, nếu
bạn muốn tham khảo đến header này bạn gõ \type{\at{trang}[hasselt-by-night]}.

%% Of course these headers can be set to your own preferences
%% and you can even define your own headers. This is done by
%% the command \type{\setuphead} and \type{\definehead}.
Dĩ nhiên các header này có thể được thiết đặt theo ý thích của bạn và bạn có
thể định nghĩa cho các header của bạn. Điều này được thực hiện bởi các lệnh
\type{\setuphead} và \type{\definehead}.

\shortsetup{definehead}

\shortsetup{setuphead}

\startbuffer
\definehead
  [myheader]
  [section]

%% \startbuffer
%% \definehead
%%   [myheader]
%%   [section]

%% \setuphead
%%   [myheader]
%%   [numberstyle=bold,
%%    textstyle=bold,
%%    before=\hairline\blank,
%%    after=\nowhitespace\hairline]

%% \myheader[myhead]{Hasselt makes headlines}
%% \stopbuffer

%% \typebuffer

%% A new header \type{\myheader} is defined and it inherits the
%% properties of \type{\section}. It would look something
%% like this:

%% \getbuffer
\setuphead
  [myheader]
  [numberstyle=bold,
   textstyle=bold,
   before=\hairline\blank,
   after=\nowhitespace\hairline]

\myheader[myhead]{Hasselt makes headlines}
\stopbuffer

\typebuffer

%A new header \type{\myheader} is defined and it inherits the
%properties of \type{\section}. It would look something
%like this:
Một header mới \type{\myheader} được định nghĩa và nó thừa hưởng các thuộc tính
của \type{\section}. Nó trông như thế này:

\getbuffer

%% There is one other command you should know now, and that is
%% \type{\setupheads}. You can use this command to set up the
%% numbering of the numbered headers. If you type:
Có một lệnh khác bạn nên biết, đó là \type{\setupheads}. Bạn có thể dùng lệnh
này để thiết lập số cho các header được đánh số. Nếu bạn gõ:

\startbuffer
\setupheads
  [alternative=inmargin,
   separator=--]
\stopbuffer

\typebuffer

%% all numbers will appear in the margin. Section 1.1 would
%% look like 1--1.
tất cả đánh số sẽ hiện ở mép trang. Mục 1.1 trông như 1--1.

%% Commands like \type{\setupheads} are typed in the
%% set up area of your input file.
Các lệnh như \type{setupheads} được gõ trong phần thiết lập của tập tin nhập
liệu.

\shortsetup{setupheads}

\stopcomponent
