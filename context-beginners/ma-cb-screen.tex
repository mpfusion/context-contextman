\startenvironment ma-cb-screen

% This setups adds some functionality as well as redefines
% some of the previously defined layout.

\unprotect

\definepapersize
  [LocalPaperFormat]
  [\c!width=28cm,
   \c!height=21cm]

\setuppapersize
  [LocalPaperFormat]
  [LocalPaperFormat]

\setuplayout
  [\c!location=\v!middle,
   \c!topspace=.5cm,
   \c!header=1.5cm,
   \c!height=20cm,
   \c!rightedge=5cm,
   \c!rightedgedistance=1cm]

\setupinteractionscreen
  [\c!width=28cm,
   \c!height=21cm,
   \c!option=\v!max]

% We place the pagenumber (not that useful in an interactive
% document) somewhere else and the chapter number in the
% footer.

\setuppagenumbering
  [\c!alternative=\v!singlesided,
   \c!location=,
   \c!command=\NummerCommando]

\unexpanded\def\NummerCommando#1% uitlijnen op onderkant voet
  {\hbox to \rightedgewidth
     {\scratchcounter=\lastpage
      \advance\scratchcounter by -\realpageno
      \hskip0pt plus \realpageno cm
      \framed
        [\c!background=NummerAchtergrond,
         \c!frame=\v!off,
         \c!offset=4pt]%
        {\lower.5\dp\strutbox\hbox spread 60pt{\hss\tx#1\hss}}% {\hss#1\hss}}%
      \hskip0pt plus \scratchcounter cm}}

\setupfootertexts
  [\v!edge]
  [][\v!pagenumber]

\unexpanded\def\PlaatsHoofdstukStatus
  {\determineheadnumber[\v!chapter]%
   \ifnum\currentheadnumber>0
     \hbox to \hsize
       {\hss
        \framed
          [\c!background=NummerAchtergrond,
           \c!frame=\v!off,
           \c!offset=6pt]
          {\lower.5\dp\strutbox\hbox spread 60pt
             {\hss\getmarking[\v!chapter\v!number]\hss}}%
        \hss}
   \fi}

\setupfootertexts
  [\v!margin]
  [][]

\setupfootertexts
  [\v!text]
  [][\PlaatsHoofdstukStatus]

\setupinteraction
  [\c!state=\v!start,
   \c!color=,
   \c!menu=\v!on]

% We let users click on the whole table of contents line and
% provide some menus.

\setuplist
  [\v!chapter]
  [\c!interaction=\v!all]

\setupinteractionmenu
  [\v!right]
  [\c!state=\v!start,
   \c!color=,
   \c!offset=4pt,
   \c!background=MenuAchtergrond,
   \c!frame=\v!off]

\startmode[**nl]

  \setupinteractionmenu
    [\v!right]
    [    {inhoud[contents]},
          {index[subind]},
     {commando's[comind]},
     {definities[comdefs]},
        {colofon[colofon]},
            {\vfill},
        {stoppen[\v!CloseDocument]},
     {\ZoekEnZoek{zoeken}},
          {terug[\v!PreviousJump]},
        {\HeenEnWeer}]

\stopmode

\startmode[**en,**uk]

  \setupinteractionmenu
    [\v!right]
    [  {contents[contents]},
          {index[subind]},
       {commands[comind]},
    {definitions[comdefs]},
        {colofon[colofon]},
            {\vfill},
           {exit[\v!CloseDocument]},
     {\ZoekEnZoek{search}},
        {go back[\v!PreviousJump]},
        {\HeenEnWeer}]

\stopmode

\startmode[**cz]

  \setupinteractionmenu
    [\v!right]
    [                             {obsah[contents]},
                  {rejst\rcaron\iacute k[subind]},
       {seznam p\rcaron\iacute kaz\uring[comind]},
     {definice p\rcaron\iacute kaz\uring[comdefs]},
                      {tir\aacute\zcaron[colofon]},
                                    {\vfill},
                                  {konec[\v!CloseDocument]},
                            {\ZoekEnZoek{vyhledej}},
                       {krok zp\ecaron t[\v!PreviousJump]},
                                  {\HeenEnWeer}]

\stopmode

% This not that \TEX nical definition deals with the two sets
% of buttons. Someday I'll make a general macro for this.

\def\TwoMenuButtons#1[#2]#3[#4]%
  {\hbox to \hsize
     {\dimen0=\hsize
      \advance\hsize by -12pt
      \menubutton[\v!right][\c!width=.5\hsize]{#1}[#2]%
      \hss
      \menubutton[\v!right][\c!width=.5\hsize]{#3}[#4]}}

\def\HeenEnWeer
  {\TwoMenuButtons{--}[\v!previouspage]{+}[\v!nextpage]}

\def\ZoekEnZoek#1%
  {\TwoMenuButtons{#1}[\v!SearchDocument]{+}[\v!SearchAgain]}

\protect

\stopenvironment
