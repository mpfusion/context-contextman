\startonderdeel ma-cb-nl-layout

\produkt ma-cb-nl

\hoofdstuk{Paginalayout}

\index{layout}
\index{paginalayout}
\index{pagina ontwerp}

\Command{\tex{stellayoutin}}
\Command{\tex{paslayoutaan}}
\Command{\tex{toonkader}}
\Command{\tex{tooninstellingen}}
\Command{\tex{toonlayout}}
\Command{\tex{startlokaal}}

De paginalayout van dit document is gedefinieerd met:

\shortsetup{stellayoutin}

U dient bekend te zijn met de parameters waarmee de
paginalayout kan worden ingesteld. Een pagina is ingedeeld
in vlakken die worden aangeduid met tekst, marge, hoofd,
voet enz.

De verschillende vlakken worden in
\in{figuur}[fig:paginalayout] schematisch weergegeven.

\startbuffer
\hbox
  {\omlijnd[breedte=1.5cm,kader=uit]           {}
   \omlijnd[breedte=2cm,kader=uit]             {linker}
   \omlijnd[breedte=2.5cm,kader=uit]           {linker}
   \omlijnd[breedte=3cm,kader=uit]             {tekst}
   \omlijnd[breedte=2.5cm,kader=uit]           {rechter}
   \omlijnd[breedte=2cm,kader=uit]             {rechter}}
\hbox
  {\omlijnd[breedte=1.5cm,kader=uit]           {}
   \omlijnd[breedte=2cm,kader=uit]             {rand}
   \omlijnd[breedte=2.5cm,kader=uit]           {marge}
   \omlijnd[breedte=3cm,kader=uit]             {tekst}
   \omlijnd[breedte=2.5cm,kader=uit]           {marge}
   \omlijnd[breedte=2cm,kader=uit]             {rand}}
\hbox
  {\omlijnd[breedte=1.5cm,kader=uit]           {boven}
   \omlijnd[breedte=2cm]                       {}
   \omlijnd[breedte=2.5cm]                     {}
   \omlijnd[breedte=3cm]                       {}
   \omlijnd[breedte=2.5cm]                     {}
   \omlijnd[breedte=2cm]                       {}}
\hbox
  {\omlijnd[breedte=1.5cm,kader=uit]           {hoofd}
   \omlijnd[breedte=2cm]                       {}
   \omlijnd[breedte=2.5cm]                     {}
   \omlijnd[breedte=3cm]                       {}
   \omlijnd[breedte=2.5cm]                     {}
   \omlijnd[breedte=2cm]                       {}}
\hbox
  {\omlijnd[breedte=1.5cm,kader=uit,hoogte=3cm]{tekst}
   \omlijnd[breedte=2cm,hoogte=3cm]            {}
   \omlijnd[breedte=2.5cm,hoogte=3cm]          {}
   \omlijnd[breedte=3cm,hoogte=3cm]            {}
   \omlijnd[breedte=2.5cm,hoogte=3cm]          {}
   \omlijnd[breedte=2cm,hoogte=3cm]            {}}
\hbox
  {\omlijnd[breedte=1.5cm,kader=uit]           {voet}
   \omlijnd[breedte=2cm]                       {}
   \omlijnd[breedte=2.5cm]                     {}
   \omlijnd[breedte=3cm]                       {}
   \omlijnd[breedte=2.5cm]                     {}
   \omlijnd[breedte=2cm]                       {}}
\hbox
  {\omlijnd[breedte=1.5cm,kader=uit]           {onder}
   \omlijnd[breedte=2cm]                       {}
   \omlijnd[breedte=2.5cm]                     {}
   \omlijnd[breedte=3cm]                       {}
   \omlijnd[breedte=2.5cm]                     {}
   \omlijnd[breedte=2cm]                       {}}
\stopbuffer

\plaatsfiguur
  [hier]
  [fig:paginalayout]
  {De vlakverdeling van een pagina.}
  {\tt\haalbuffer}

De paginalayout kan worden opgeroepen met \type{\toonkader}.
Na verwerking wordt de layout met kaders weergegeven. Het
commando \type{\tooninstellingen} geeft de instelwaarden
weer. Een combinatie van beide commando is
\type{\toonlayout}.

De waarde van de layout parameters zijn beschikbaar als
commando's (zie \in{tabel}[tab:parameters]). Dit maakt het
mogelijk nauwkeurig te werken bij het defini\"eren van
afmetingen van bijvoorbeeld kolommen, figuren en tabellen.
Een aantal van deze waarden wordt in
\in{tabel}[tab:aantalparameters] toegelicht.

\plaatstabel
  [hier,forceer]
  [tab:aantalparameters]
  {Een aantal parameters die als commando beschikbaar zijn.}
\starttabel[|l|l|]
\HL
\NC \bf Commando         \NC \bf Betekenis            \NC\SR
\HL
\NC \type{\zetbreedte}   \NC breedte van zetgebied    \NC\FR
\NC \type{\zethoogte}    \NC hoogte van het zetgebied \NC\MR
\NC \type{\tekstbreedte} \NC breedte van tekst vlak   \NC\MR
\NC \type{\teksthoogte}  \NC hoogte van tekst vlak    \NC\LR
\HL
\stoptabel

Indien u een breedte van een kolom of een figuur wilt
defini\"eren is het verstandig om deze te relateren aan de
\type{\zetbreedte} of \type{\zethoogte}. Bij verandering van
deze waarden worden de breedte of hoogte van de kolom of
figuur proportioneel meeveranderd.

\startbuffer
\plaatsfiguur
  [hier]
  [fig:trapgevel]
  {Een trapgevel.}
  {\externfiguur[ma-cb-19][breedte=.6\zetbreedte]}
\stopbuffer

\typebuffer

Na verwerking wordt \in{figuur}[fig:trapgevel] geplaatst.

\haalbuffer

De overige afstanden en maten worden in
\in{tabel}[tab:parameters] getoond.

Het commando \type{\stellayoutin} wordt gedefinieerd in het
instelgebied van de invoerfile, dus voor het
\type{\starttekst}||commando. Dit betekent dat de waarden een
globaal karakter hebben en betrekking hebben op het volledige
document. Kleine wijzigingen in die layout op lokaal niveau
worden gedaan met:

\startbuffer
\paslayoutaan[21,38][hoogte=+.5cm]
\stopbuffer

\typebuffer

In dit geval wordt op pagina 21 en 38 de standaardhoogte met
0,5 \Centi \Meter\ verhoogd.


Voor lokale aanpassingen in de layout kunt u gebruik maken
van:

\shortsetup{startlokaal}

\startbuffer
\start

\startlokaal
  \stellayoutin[hoogte=+.5cm]
\stoplokaal

Hasselt heeft een compleet andere vormgeving dan de meeste
andere steden als gevolg van de versterkingen en
verdedigingswerken.

\stop
\stopbuffer

\typebuffer

Het wordt afgeraden dergelijke tijdelijke aanpassingen te
vaak uit te voeren.

\startbuffer
\starttabelkop
\HL
\NC \bf Parameter        \NC \bf Beschikbaar als commando \NC\SR
\HL
\stoptabelkop

\starttabelstaart
\HL
\stoptabelstaart

\starttabellen[|l|l|]
\NC bovenafstand         \NC \type{\bovenafstand}        \NC\FR
\NC bovenhoogte          \NC \type{\bovenhoogte}         \NC\MR
\NC hoofdafstand         \NC \type{\hoofdafstand}        \NC\MR
\NC hoofdhoogte          \NC \type{\hoofdhoogte}         \NC\MR
\NC kopniveau            \NC \type{\kopniveau}           \NC\MR
\NC kopwit               \NC \type{\kopwit}              \NC\MR
\NC rugwit               \NC \type{\rugwit}              \NC\MR
\NC margeafstand         \NC \type{\margeafstand}        \NC\MR
\NC margebreedte         \NC \type{\margebreedte}        \NC\MR
\NC linkermargebreedte   \NC \type{\linkermargebreedte}  \NC\MR
\NC rechtermargebreedte  \NC \type{\rechtermargebreedte} \NC\MR
\NC randafstand          \NC \type{\randafstand}         \NC\MR
\NC randbreedte          \NC \type{\randbreedte}         \NC\MR
\NC linkerrandbreedte    \NC \type{\linkerrandbreedte}   \NC\MR
\NC rechterrandbreedte   \NC \type{\rechterrandbreedte}  \NC\MR
\NC papierbreedte        \NC \type{\papierbreedte}       \NC\MR
\NC papierhoogte         \NC \type{\papierhoogte}        \NC\MR
\NC zetbreedte           \NC \type{\zetbreedte}          \NC\MR
\NC zethoogte            \NC \type{\zethoogte}           \NC\MR
\NC tekstbreedte         \NC \type{\tekstbreedte}        \NC\MR
\NC teksthoogte          \NC \type{\teksthoogte}         \NC\MR
\NC voetafstand          \NC \type{\voetafstand}         \NC\MR
\NC voethoogte           \NC \type{\voethoogte}          \NC\MR
\NC onderhoogte          \NC \type{\onderhoogte}         \NC\MR
\NC onderafstand         \NC \type{\onderafstand}        \NC\LR
\stoptabellen
\stopbuffer

\splitsplaatsblok[regels=1]
  {\plaatstabel
     [hier][tab:parameters]
     {Parameters voor pagina layout.}}
  {\haalbuffer}

\stoponderdeel
