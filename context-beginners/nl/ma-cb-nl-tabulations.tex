\startonderdeel ma-cb-nl-tabulations

\produkt ma-cb-nl

\hoofdstuk{Tabulatie / Opmaak van alinea's}

\index{kolommen}
\index{alinea opmaak}
\index{tabulatie}
\index{tabel+lopende tekst}
\Command{\tex{starttabulatie}}
\Command{\tex{definieertabulatie}}
\Command{\tex{steltabulatiein}}
\Command{\tex{NR}}
\Command{\tex{NC}}
\Command{\tex{startchemie}}

Soms is het wenselijk om alineas zodanig vorm te geven dat
de informatie overzichtelijker wordt weergegeven. Dit wordt
gedaan met:

\shortsetup{starttabulatie}

Het tabulatiemechanisme is vergelijkbaar met het
tabelmechanisme. Tabulatie wordt vooral gebruikt als complete
alinea's in een cel van een tabel moet worden geplaatst. Het
tabulatiemechanisme werkt bovendien goed bij
pagina||overgangen.

Een tabulatie zou er als volgt uit kunnen zien:

\startbuffer
\starttabulatie[|w(1.5cm)B|p(6.0cm)|p|]
\NC 1252
    \NC Hasselt verkrijgt stadsrechten van bisschop Hendrik van
        Vianden.
    \NC Hendrik van Vianden werd door andere steden onder druk
        gezet om de stadsrechten niet te verlenen. Het kostte
        Hasselt veel tijd om de bisschop te overtuigen. Nadat
        Hasselt de bisschop had gesteund bij een kleine oorlog
        tegen de Drenten werden de stadsrechten verleend. \NC\NR
\NC 1350
    \NC Hasselt treedt toe tot het Hanzepact om haar internationale
        handel te beschermen.
    \NC Het Hanzepact was van groot belang voor de handelaren
        van Hasselt. Goederen werden in die dagen door iedere
        stad, op iedere hoofdstraat en bij iedere
        rivieroversteekplaats apart belast. Belastingvrije
        routes door heel Europa was een van de voordelen van
        het lidmaatschap. Hasselt is altijd een van de kleinere
        leden van het pact geweest. \NC\NR
\stoptabulatie
\stopbuffer

\typebuffer

Hierbij wordt de eerste kolom 1,5 \Centi \Meter\ breed en
vet (\type{B}) gezet. De tweede kolom is 6 \Centi \Meter\
breed en wordt als een alinea gezet. De resterende ruimte
wordt door de laatste alinea gebruikt.

\haalbuffer

Net als in het tabelmechanisme is er een aantal
formaataanduidingen en -commando's. In
\in{tabel}[tab:tabulatieformaataanduidingen] ziet u een
overzicht van deze aanduidingen en commando's.

\plaatstabel[][tab:tabulatieformaataanduidingen]
  {Tabulatie commando;'s.}
\starttabel[|lT|l|lT|l|]
\NC l                 \NC left align
\NC I                 \NC \it italic
\NC \FR
\NC c                 \NC center
\NC R                 \NC \sl roman
\NC \MR
\NC r                 \NC right align
\NC S                 \NC \sl slanted
\NC \MR
\NC i\sl n            \NC spacing left
\NC T                 \NC \tt teletype
\NC \MR
\NC j\sl n            \NC spacing right
\NC m                 \NC in||line math
\NC \MR
\NC k\sl n            \NC spacing around
\NC M                 \NC display  math
\NC \MR
\NC w({\sl d})        \NC 1 line,   fixed width
\NC f\tex{command}    \NC font specification
\NC \MR
\NC p({\sl d})        \NC paragraph, fixed width
\NC b\arg{..}         \NC place \type{..} before the entry
\NC \MR
\NC p                 \NC paragraph, maximum width
\NC a\arg{..}         \NC place \type{..} after the entry
\NC \MR
\NC B                 \NC \bf boldface
\NC h\tex{command}    \NC apply \tex{command} on the entry
\NC \LR
\stoptabel

Een tweede voorbeeld voor het herinrichten van
alineas vindt u hieronder.

\startbuffer
\definieertabulatie[ChemPar][|l|p|l|]

\startChemPar
\NC Kalkoven
    \NC Hasselt heeft zijn eigen kalkovens. Deze werden in 1504
        gebouwd en produceerden kalk tot 1956. Tegenwoordig
        vormen de ovens een toeristische attractie.
    \NC \chemie{CaCO_3,~,GIVES,~,CaO,~,+,~,CO_2} \NC\NR
\stopChemPar
\stopbuffer

\typebuffer

Het voorbeeld komt er ongeveer als volgt uit te zien:

\haalbuffer

De chemische module wordt in een andere handleiding
toegelicht.

In het voorbeeld wordt ook het commando ge\"{\i}ntroduceerd
om vooraf de alinealayout te defini\"eren.

\shortsetup{definieertabulatie}

en dan is er ook nog:

\shortsetup{steltabulatiein}

\stoponderdeel
