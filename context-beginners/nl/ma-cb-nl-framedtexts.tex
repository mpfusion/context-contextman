\startonderdeel ma-cb-nl-framedtexts

\produkt ma-cb-nl

\hoofdstuk{Omlijnde paragrafen}

\index{omlijnd+paragraaf}

\Command{\tex{startkadertekst}}
\Command{\tex{stelkadertekstenin}}

Voor het omkaderen van complete alineas wordt het volgende
commando||paar gebruikt:

\shortsetup{startkadertekst}

\startbuffer
\definieerplaatsblok[intermezzo]

\plaatsintermezzo[hier][blok:brug]{Een intermezzo.}
\startkadertekst[breedte=.8\zetbreedte]
Een brug over het Zwartewater was essentieel voor Hasselt.
De bisschop van Utrecht gaf zijn toestemming voor de bouw
in 1486.
\blanko
Andere steden in de omgeving van Hasselt waren bang voor de
hoge tolgelden die Hasselt bij passage over zo'n brug zou
kunnen vragen. Deze steden hebben de bouw lange tijd
tegengehouden.
\stopkadertekst
\stopbuffer

\typebuffer

Dit voorbeeld illustreert het commando
\type{\definieerplaatsblok}. Meer informatie vindt u in
\in{paragraaf}[tekstblokken]. De \type{\blanko} is
noodzakelijk om een lege regel af te dwingen.

\haalbuffer

Het omkaderen wordt ingesteld met:

\shortsetup{stelkadertekstenin}

\stoponderdeel
