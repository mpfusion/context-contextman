\startonderdeel ma-cb-nl-descriptions

\produkt ma-cb-nl

\hoofdstuk{Definities}

\index{definities}

\Command{\tex{doordefinieren}}
\Command{\tex{steldoordefinierenin}}

Om definities, begrippen en concepten enigszins consistent
weer te geven, wordt gebruik gemaakt van:

\shortsetup{doordefinieren}

Bijvoorbeeld:

\startbuffer
\doordefinieren
  [concept]
  [plaats=aanelkaar,kopletter=vet,breedte=ruim]

\concept{Hasselter juffer} Een traditioneel koekje van
besuikerd bladerdeeg. Erg zoet en erg lekker. Vergelijkbaar
met het Arnhemse meisje (krakeling). \par
\stopbuffer

\typebuffer

Een dergelijke definitie wordt als volgt weergegeven:

\haalbuffer

Maar het is ook mogelijk om een andere vormgeving te kiezen:

\startbuffer
\doordefinieren
  [concept]
  [plaats=boven,
   kopletter=vet,
   breedte=ruim,
   letter=schuin]

\concept{Hasselter bitter} Een sterk alcoholische drank (tot
40\%) gemengd met kruiden om een specifieke smaak te
verkrijgen. Het wordt verkocht in een stenen kruik en dient
koud te worden geserveerd. \par

\doordefinieren
  [concept]
  [plaats=inmarge,kopletter=vet,breedte=ruim]

\concept{Euifeest} Een hooifeest om het einde van een periode
van hard werken af te sluiten. De festiviteiten vinden in de
laatste week van augustus plaats. \par
\stopbuffer

\start
\haalbuffer
\stop

Indien de definitie uit meer dan een alinea bestaat kunt u
\type{\start...} $\cdots$ \type{\stop...} gebruiken.

\startbuffer
\doordefinieren
  [concept]
  [plaats=rechts,
   kopletter=vet,
   breedte=ruim]

\startconcept{Euifeest}
Een hooifeest om een periode van hard werken af te sluiten.
Het feest vindt plaats aan het eind van augustus en duurt
een week.
De stad wordt volledig versierd en verlicht en dit heeft
Hasselt de bijnaam {\em Lichtstad van het Noorden} gegeven.
Het feest eindigt in een traditionele braderie.
\stopconcept
\stopbuffer

\typebuffer

Dit wordt:

\haalbuffer

De vormgeving kan worden vastgelegd binnen het tweede paar
van vierkante haken in \type{\doordefinieren[][]}. Meestal
wordt dit echter gedaan met het commando:

\shortsetup{steldoordefinierenin}

\stoponderdeel
