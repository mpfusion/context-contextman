\startonderdeel ma-cb-nl-modules

\produkt ma-cb-nl

\hoofdstuk{Laden van modules}

\index{module+chemie}
\index{module+eenheid}
\index{module+chart}
\index{module+pictex}
\index{chemische structuren}
\index{stroomschema's}
\index{eenheden}

\Command{\tex{gebruikmodule}}

Uit oogpunt van efficiency is besloten om bepaalde
functionaliteit van \CONTEXT\ onder te brengen in modules.
Op dit moment zijn de volgende modules beschikbaar:

\startopsomming[opelkaar]
\som \type{eenheid} voor het gebruik van \kap{SI}||eenheden
\som \type{chemie} voor het zetten van chemische structuren
\som \type{pictex} voor het tekenen van plaatjes (wordt
      gebruikt in combinatie met module \type{chemie})
\som \type{chart} voor het tekenen van stroomschema's en
     organogrammen
\stopopsomming

Een module wordt in het instelgebied van de invoerfile
geladen door middel van:

\shortsetup{gebruikmodule}

Van de module \type{units} hebben we al meerdere voorbeelden
gegeven. Van de modules \type{chemie} en \type{chart} geven
we hier twee voorbeelden zonder nadere uitleg. Voor een
toelichting op beide modules verwijzen we naar de
afzonderlijke handleidingen.

Chemische structuren kunnen er indrukwekkend uitzien.

\startbuffer
\plaatsformule[-]
\startformule
\startchemie[schaal=klein,breedte=passend,boven=3000,onder=3000]
  \chemie[SIX,SB2356,DB14,Z2346,SR3,RZ3,-SR6,+SR6,-RZ6,+RZ6]
         [C,N,C,C,H,H,H]
  \chemie[PB:Z1,ONE,Z0,DIR8,Z0,SB24,DB7,Z27,PE][C,C,CH_3,O]
  \chemie[PB:Z5,ONE,Z0,DIR6,Z0,SB24,DB7,Z47,PE][C,C,H_3C,O]
  \chemie[SR24,RZ24][CH_3,H_3C]
  \bottext{Verbinding A}
\stopchemie
\stopformule
\stopbuffer

\haalbuffer

Hoewel chemische structuren met slechts twee commando's
worden gedefinieerd, is wel enige oefening nodig om de
juiste resultaten te verkrijgen. De invoer voor de formule
ziet er als volgt uit:

\typebuffer

Als we gebruik maken van de module \type{chart} voor het
defini\"eren van een organogram kan de definitie er als
volgt uitzien.

\setupFLOWcharts
  [breedte=11\bodyfontsize,
   hoogte=3\bodyfontsize,
   dx=1\bodyfontsize,
   dy=2\bodyfontsize]

\setupFLOWlines
  [pijl=nee]

\startbuffer
\startFLOWchart[organogram]
  \startFLOWcell
    \shape    {action}
    \name     {01}
    \location {2,1}
    \text     {Hasselt}
    \connect  [bt]{02}
    \connect  [bt]{03}
    \connect  [bt]{04}
  \stopFLOWcell
  \startFLOWcell
    \shape    {action}
    \name     {02}
    \location {1,2}
    \text     {Mastenbroek}
  \stopFLOWcell
  \startFLOWcell
    \shape    {action}
    \name     {03}
    \location {2,2}
    \text     {Genne}
  \stopFLOWcell
  \startFLOWcell
    \shape    {action}
    \name     {04}
    \location {3,2}
    \text     {Zwartewaterklooster}
  \stopFLOWcell
\stopFLOWchart

\FLOWchart[organogram]
\stopbuffer

\typebuffer

Het resultaat wordt dan:

\regelmidden{\haalbuffer}

\stoponderdeel
