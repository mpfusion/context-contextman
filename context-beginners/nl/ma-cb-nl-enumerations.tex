\startonderdeel ma-cb-nl-enumerations

\produkt ma-cb-nl

\hoofdstuk{Genummerde definities}

\index{genummerde definities}

\Command{\tex{doornummeren}}
\Command{\tex{steldoornummerenin}}

Met \type{\doornummeren} worden tekstelementen als
opmerkingen en vragen (antwoorden) gedefinieerd.
Genummerde opmerkingen worden verkregen door:

\shortsetup{doornummeren}

Bijvoorbeeld:

\startbuffer[a]
\doornummeren
  [opmerking]
  [plaats=boven,
   tekst=Opmerking,
   tussen=\blanko,
   na=\blanko]
\stopbuffer

\typebuffer[a]

Na definitie zijn de commando's \type{\opmerking},
\type{\subopmerking}, \type{\resetopmerking} en
\type{\volgendeopmerking} beschikbaar. Opmerkingen worden nu
als volgt ingevoerd:

\startbuffer[b]
\opmerking Aan het begin van de middeleeuwen was Hasselt een
bedevaartsoord. De zogenaamde {\em Heilige Stede} werd
afgebroken gedurende de reformatie. Na 300 jaar werd in 1930
de {\em Heilige Stede} weer heropend.

\subopmerking Tegenwoordig is de {\em Heilige Stede} gesloten
en is slechts eens per jaar geopend tijdens een kerkdienst in
de open lucht. \par
\stopbuffer

\typebuffer[b]

\start
\haalbuffer[a]\haalbuffer[b]
\stop

U kunt het nummeren herstellen \type{\resetopmerking} of
\type{\resetsubopmerking} of ophogen met
\type{\volgendeopmerking} of \type{\volgendesubopmerking}.
Normaal gebeurt dit automatisch per hoofdstuk.

De layout van genummerde definities wordt ingesteld met
\type{\doornummeren} of met:

\shortsetup{steldoornummerenin}

De layout van \type{\opmerking} en \type{\subopmerking} uit
het voorbeeld kan worden ingesteld met:

\starttypen
\steldoornummerenin[opmerking][kopletter=vet]
\steldoornummerenin[subopmerking][kopletter=schuin]
\stoptypen

Indien een nummer niet nodig is, typt u:

\starttypen
\opmerking[-]
\stoptypen

In het geval dat een tekstelement uit meerdere alineas
bestaat, moet het commando||paar \type{\startopmerking}
$\cdots$ \type{\stopopmerking} worden gebruikt. Dit
commando||paar is beschikbaar na definitie van {\bf
Opmerking} met \type{\doornummeren[opmerking]}.

Een langere definitie zou er als volgt uit kunnen zien:

\startbuffer[c]
\startopmerking
Aan het begin van de Middeleeuwen was Hasselt een
bedevaartsoord. De zogenaamde {\em Heilige Stede} werd
afgebroken gedurende de reformatie. Na 300 jaar werd in 1930
de {\em Heilige Stede} weer heropend.

Tegenwoordig is de {\em Heilige Stede} gesloten en slechts
eens per jaar geopend tijdens een kerkdienst in de open lucht.
\stopopmerking
\stopbuffer

\typebuffer[c]

\start
\haalbuffer[a]\haalbuffer[c] \par
\stop

\stoponderdeel
