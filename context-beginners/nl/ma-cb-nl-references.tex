\startonderdeel ma-cb-nl-references

\produkt ma-cb-nl

\hoofdstuk[verwijzen]{Verwijzen naar tekstelementen}

\index{verwijzen}
\index{label}

\Command{\tex{in}}
\Command{\tex{op}}
\Command{\tex{paginareferentie}}

Om te verwijzen van de ene naar de andere plaats in het
document kan het volgende commando worden gebruikt:

\shortsetup{in}

De accolades bevatten tekst, de haken bevatten een logische
naam (een label). Als u in een document een volgend
hoofdstuk hebt gedefinieerd:

\starttypen
\hoofdstuk[hotel]{Hotels in Hasselt}
\stoptypen

dan kunt u naar dat hoofdstuk verwijzen met:

\starttypen
\in{hoofdstuk}[hotel]
\stoptypen

Na verwerking is het hoofdstuknummer beschikbaar en de
verwijzing zou er als volgt uit kunnen zien: {\em hoofdstuk
23}. Het commando \type{\in} wordt gebruikt voor allerlei
verwijzingen naar hoofdstukken, paragrafen, figuren,
tabellen, formules enz.

Een ander voorbeeld:

\startbuffer
Er is een aantal dingen dat u in Hasselt kunt doen:

\startopsomming[n,opelkaar]
\som zwemmen
\som zeilen
\som[wandelen] wandelen
\som fietsen
\stopopsomming

Activiteiten als \in{activiteit}[wandelen] die zijn
beschreven op \op{pagina}[wandelen] zijn erg vermoeiend.
\stopbuffer

\typebuffer

Dit komt er als volgt uit te zien:

\haalbuffer

Het label \type{wandelen} geeft dus een itemnummer terug,
maar ook een paginanummer. Het is dus mogelijk om naar
pagina's te verwijzen met:

\shortsetup{op}

Bijvoorbeeld met:

\starttypen
\op{pagina}[wandelen]
\stoptypen

Dit commando wordt veel gebruikt in combinatie met:

\shortsetup{paginareferentie}

en

\shortsetup{tekstreferentie}

Als u naar het hoofdstuk {\em Hotels in Hasselt} en de
bijbehorende bladzijde wilt verwijzen, kunt u bijvoorbeeld
de volgende tekst invoeren:

\startbuffer
Kijk in \in{hoofdstuk}[hotel] op \op{pagina}[hotel] voor een
compleet overzicht van  accomodaties in
\paginareferentie[accomodaties]Hasselt.
\stopbuffer

\typebuffer

Er worden een hoofdstuknummer en een paginanummer
gegenereerd bij het verwerken van de invoerfile. Op andere
plaatsen in het document kan naar de locatie
\type{accomodaties} worden verwezen met
\type{\op{pagina}[accomodaties]}.

Het is ook toegestaan om een serie labels te defini\"eren:

\startbuffer
\plaatsfiguur
  [hier]
  [fig:eengracht,fig:eenboot]
  {Een gracht in Hasselt.}
  {\externfiguur[ma-cb-08][breedte=10cm]}

Er zijn veel grachten in Hasselt (zie \in{figuur}[fig:eengracht]).
.
.
.
Boten mogen gewoon in de gracht aanleggen (zie
\in{figuur}[fig:eenboot]).
\stopbuffer

\typebuffer

Dit wordt:

\haalbuffer

\stoponderdeel
