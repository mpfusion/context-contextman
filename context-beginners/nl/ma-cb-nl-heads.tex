\startonderdeel ma-cb-nl-heads

\produkt ma-cb-nl

\hoofdstuk[koppen]{Koppen}

\index{koppen}

\Command{\tex{hoofdstuk}}
\Command{\tex{paragraaf}}
\Command{\tex{subparagraaf}}
\Command{\tex{titel}}
\Command{\tex{onderwerp}}
\Command{\tex{subonderwerp}}
\Command{\tex{stelkopin}}
\Command{\tex{stelkoppenin}}

De structuur van een document wordt bepaald door zijn
koppen, zoals hoofdstukken, paragrafen en subpragrafen.
Koppen worden gecre\"eerd met de commando's die in
\in{tabel}[tab:koppen] staan:

\plaatstabel[hier][tab:koppen]{Koppen.}
\starttabel[|l|l|]
\HL
\NC \bf Genummerde kop    \NC \bf Ongenummerde kop \NC\SR
\HL
\NC \tex{hoofdstuk}       \NC \tex{titel}           \NC\FR
\NC \tex{paragraaf}       \NC \tex{onderwerp}       \NC\MR
\NC \tex{subparagraaf}    \NC \tex{subonderwerp}    \NC\MR
\NC \tex{subsubparagraaf} \NC \tex{subsubonderwerp} \NC\MR
\NC \onbekend             \NC \onbekend             \NC\LR
\HL
\stoptabel

\shortsetup{hoofdstuk}
\shortsetup{paragraaf}
\shortsetup{subparagraaf}
\shortsetup{titel}
\shortsetup{onderwerp}
\shortsetup{subonderwerp}

Deze commando's produceren een kop en een kopnummer in een
bepaalde grootte en met een vooraf ingestelde verticale
witruimte voor en na de kop.

De koppen hebben twee verschijningsvormen. Bijvoorbeeld:

\starttypen
\titel[hasselt bij nacht]{Hasselt bij nacht}
\stoptypen

en

\starttypen
\titel{Hasselt bij nacht}
\stoptypen

De vierkante haken zijn optioneel en worden gebruikt voor
interne verwijzingen. Verwijzen doet u bijvoorbeeld met:
\type{\op{pagina}[hasselt bij nacht]}.

Natuurlijk kunnen koppen in een door u zelf gedefinieerde
vormgeving worden weergegeven. Dit gebeurt met de commando's
\type{\stelkopin} en \type{\definieerkop}.

\shortsetup{definieerkop}
\shortsetup{stelkopin}

\startbuffer
\definieerkop
  [mijnkop]
  [paragraaf]

\stelkopin
  [mijnkop]
  [nummerletter=vet,
   tekstletter=vet,
   voor={\haarlijn\pagina[nee]},
   na={\geenwitruimte\pagina[nee]\haarlijn}]

\mijnkop[mijnkop]{In Hasselt wonen kopstukken}
\stopbuffer

\typebuffer

Een nieuwe kop \type{\mijnkop} wordt gedefinieerd en erft
daarbij de eigenschappen van \type{\paragraaf}. Een
dergelijke kop ziet er als volgt uit:

\haalbuffer

Een ander commando met betrekking tot koppen is
\type{\stelkoppenin}. U kunt dit commando gebruiken
voor het instellen van de nummering van genummerde koppen.
Als u typt:

\startbuffer
\stelkoppenin
  [variant=inmarge,
   scheider=--]
\stopbuffer

\typebuffer

zullen alle nummers in de marge verschijnen en worden
subnummers als volgt weergegeven: 1--1.

Commando's als \type{\stelkoppenin} worden bij voorkeur in
het instelgebied van uw invoerfile geplaatst.

\shortsetup{stelkoppenin}

\stoponderdeel
