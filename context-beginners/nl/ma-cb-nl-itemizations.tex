\startonderdeel ma-cb-nl-itemizations

\produkt ma-cb-nl

\hoofdstuk[opsommingen]{Opsommingen}

\index{opsommingen}
\index{kolommen in opsommingen}

\Command{\tex{startopsomming}}
\Command{\tex{stelopsommingin}}
\Command{\tex{definieersymbool}}
\Command{\tex{som}}
\Command{\tex{kop}}

Informatie kan worden gestructureerd met behulp van
opsommingen. Er zijn genummerde en ongenummerde opsommingen.
Het commando om opsommingen te maken ziet er als volgt uit:

\shortsetup{startopsomming}

Bijvoorbeeld:

\startbuffer
\startopsomming[R,opelkaar,ruim]
\som Hasselt ontstond in 14e eeuw.
\som Hasselt staat bekend als een Hanzestad.
\som Hasselt's naam is ontleend aan een boom.
\stopopsomming
\stopbuffer

\typebuffer

Binnen het \type{\startopsomming} $\cdots$
\type{\stopopsomming} paar wordt ieder nieuw gegeven gestart
met het commando \type{\som}. De spatie achter \type{\som}
is vereist. In het bovenstaande voorbeeld specificeert
\type{R} dat een romeinse nummering is gewenst en
\type{opelkaar} zorgt ervoor dat de items zonder witruimte
opelkaar worden geplaatst. De instelling \type{ruim} zorgt
voor extra ruimte na de scheider. In gezette vorm ziet het
voorbeeld er als volgt uit:

\haalbuffer

Voor het zetten van opsommingen zijn twee verwerkingsslagen
nodig. Dit betekent dat u uw invoerfile twee keer door
\CONTEXT\ moet laten verwerken om een correcte layout te
verkrijgen. Tussen de vierkante haken staat informatie over
de itemscheiders en locale instellingen van opsommingen.

\plaatstabel
  [hier]
  [tab:iteminstellingen]
  {Itemscheiders in opsommingen.}
\starttabel[|l|l|]
\HL
\NC \bf Argument \NC \bf Itemscheider \NC\SR
\HL
\NC 1        \NC $\bullet$             \NC\FR
\NC 2        \NC $-$                   \NC\MR
\NC 3        \NC $\star$               \NC\MR
\NC $\vdots$ \NC $\vdots$              \NC\MR
\NC n        \NC 1 2 3 4 \onbekend     \NC\MR
\NC a        \NC a b c d \onbekend     \NC\MR
\NC A        \NC A B C D \onbekend     \NC\MR
\NC r        \NC i ii iii iv \onbekend \NC\MR
\NC R        \NC I II III IV \onbekend \NC\LR
\HL
\stoptabel

U kunt natuurlijk ook een eigen itemscheider defini\"eren door
middel van het commando \type{\definieersymbool}. Als u
bijvoorbeeld het volgende invoert:

\startbuffer
\definieersymbool[5][$\clubsuit$]

\startopsomming[5,opelkaar]
\som Hasselt werd gebouwd op een rivierduin.
\som Hasselt ligt op een kruising van twee rivieren.
\stopopsomming
\stopbuffer

\typebuffer

Krijgt u:

\haalbuffer

Soms zijn binnen een opsomming koppen gewenst. In die gevallen
wordt in plaats van het commando \type{\som} het commando
\type{\kop} ingevoerd.

\startbuffer
Hasselt ligt in Overijssel en er is een aantal gebruiken dat
typerend is voor deze provincie.

\startopsomming

\kop kraamschudden

     Na de geboorte van een kind komen de buren de nieuwe
     ouders bezoeken. De vrouwen bewonderen het kind en de
     mannen beoordelen het kind (als het een jongen is) op
     zijn fysieke kenmerken. De buren brengen een
     krentewegge mee. Dat is een krentebrood van ongeveer 1
     \Meter\ lengte. Natuurlijk wordt ook op het nieuwe kind
     geklonken.

\kop nabuurschap (naberschop)

     In de kleine gemeenschappen waren de mensen vroeger
     sterk op elkaar aangewezen. Leden van zo'n {\em
     nabuurschap} hielpen elkaar bij het oogsten,
     begrafenissen of tegenslagen die de gemeenschap te
     verwerken kreeg.

\kop Abraham / Sarah

     Als mensen 50 worden, wordt er van hen gezegd dat ze
     Abraham of Sarah zien. Het is gewoonte deze mensen een
     Abraham of Sarah van speculaas te geven.

\stopopsomming
\stopbuffer

\typebuffer

Het commando \type{\kop} kan worden ingesteld met
\type{\stelopsommingin}. In geval van een pagina||overgang
zal een nieuwe \type{\kop} altijd aan het begin van de
eerstvolgende pagina worden geplaatst.

Het eerdere ingevoerde voorbeeld over oude gebruiken komt er
na verwerking als volgt uit te zien:

\haalbuffer

De mogelijke instellingen van opsommingen zijn weergegeven in
\in{tabel}[tab:tabelinstellingen].

\plaatstabel
  [hier,forceer]
  [tab:tabelinstellingen]
  {Instellingen van opsommingen.}
\starttabel[|l|l|]
\HL
\NC \bf Optie   \NC \bf Betekenis                             \NC\SR
\HL
\NC standaard   \NC standaard instellingen                    \NC\FR
\NC opelkaar    \NC geen witruimte tussen onderdelen          \NC\MR
\NC aanelkaar   \NC geen wit voor en na de opsomming          \NC\MR
\NC aansluitend \NC weinig witruimte na de scheider           \NC\MR
\NC ruim        \NC extra witruimte na de scheider            \NC\MR
\NC inmarge     \NC scheider in de marge                      \NC\MR
\NC opmarge     \NC scheider op de marge                      \NC\MR
\NC afsluiter   \NC afsluiter na de scheider                  \NC\MR
\NC kolommen    \NC in kolommen                               \NC\MR
\NC intro       \NC geen pagina--overgang na introductieregel \NC\MR
\NC verder      \NC doornummeren                              \NC\LR
\HL
\stoptabel

U kunt lokale instellingen meegeven tussen de vierkante
haken direct achter het commando \type{\startopsomming},
maar voor de consistentie kunt u de voorkeuren ook voor het
gehele document instellen met \type{\stelopsommingin}.

De instelling \type{kolommen} wordt altijd gebruikt in
combinatie met een geschreven aantal. Indien u typt:

\startbuffer
\startopsomming[n,kolommen,vier,ruim]
\som Achter 't Werk
.
.
.
\som Justitiebastion
\stopopsomming
\stopbuffer

\typebuffer

Krijgt u:

\startbuffer
\startopsomming[n,kolommen,vier]
\som Achter 't Werk
\som Baangracht
\som Brouwersgracht
\som Eikenlaan
\som Eiland
\som Gasthuisstraat
\som Heerengracht
\som Hofstraat
\som Hoogstraat
\som Julianakade
\som Justitiebastion
\stopopsomming
\stopbuffer

\haalbuffer

Als u een opsomming na een kort intermezzo verder wilt laten
lopen dan kan dat. Als u bijvoorbeeld
\type{\startopsomming[verder,kolommen,drie,ruim]} intypt,
gaat de nummering verder in drie kolommen.

\startbuffer
\startopsomming[verder,kolommen,drie,ruim]
\som Kaai
\som Kalverstraat
\som Kastanjelaan
\som Keppelstraat
\som Markt
\som Meestersteeg
\som Prinsengracht
\som Raamstraat
\som Ridderstraat
\som Rosmolenstraat
\som Royenplein
\som Van Nahuijsweg
\som Vicariehof
\som Vissteeg
\som Watersteeg
\som Wilhelminalaan
\som Ziekenhuisstraat
\stopopsomming
\stopbuffer

\haalbuffer

De instelling \type{ruim} vergroot de horizontale witruimte
tussen scheider en itemtekst.

\shortsetup{stelopsommingin}

Een opsomming binnen een opsomming wordt automatisch op de
juiste wijze gezet. Als u bijvoorbeeld intypt:

\startbuffer
Steden kunnen zelf de hoogte van bepaalde belastingen
vaststellen. Hierdoor kunnen de kosten voor gemeentelijke
belastingen van stad tot stad verschillen. Die verschillen
lopen op tot 50\% in belastingen als:

\stelopsommingin[2][breedte=5em]
\startopsomming[n]

\som de onroerend goed belasting

     De onroerend goed belasting bestaat uit twee
     componenten:

     \startopsomming[a,opelkaar]
     \som het deel voor de eigenaar
     \som het deel voor de huurder / bewoner
     \stopopsomming

     In het geval dat een pand geen huurder heeft, betaalt
     de eigenaar beide componenten.

\som de hondenbelasting

     De eigenaar van een hond betaalt hondenbelasting. Als een
     hond wordt aangeschaft of doodgaat, dient dat te worden
     gemeld bij de gemeente.

\stopopsomming
\stopbuffer

\typebuffer

Dan wordt subopsomming automatisch geplaatst en de
horizontale witruimte op het tweede niveau ingesteld met
\type{\stelopsommingin[2][breedte=5em]}.

\start
\haalbuffer
\stop

\stoponderdeel
