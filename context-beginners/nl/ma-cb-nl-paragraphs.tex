\startonderdeel ma-cb-nl-paragraphs

\produkt ma-cb-nl

\hoofdstuk{Paragrafen en witruimte}

\paragraaf{Inleiding}

\index{paragraaf}

\Command{\tex{par}}
\Command{\tex{alinea}}

In \TEX\ en \CONTEXT\ is de belangerijkste eenheid van tekst
een paragraaf. Een nieuwe paragraaf wordt gestart met:

\startopsomming[opelkaar]
\som een lege regel
\som het \TEX||commando \type{\par}
\stopopsomming

In de \ASCII\ invoerfile worden lege regels gebruikt als
paragraafscheiders. Dit heeft als voordeel dat er een goed
leesbare tekst ontstaat, waarin fouten makkelijk kunnen
worden opgespoord.

Bij het gebruiken van commando's waarin paragrafen expliciet
moeten worden afgesloten, moet \type{\par} worden gebruikt.

\startbuffer
Tijdens een van de oorlogen die rond Hasselt werden
uitgevochten werd Hasselt belegerd. Na enige tijd ontstond er
een voedselprobleem en brak er een hongersnood uit in de
stad. Alles wat eetbaar was werd opgegeten. Op \'e\'en koe na.
Deze koe werd in leven gelaten en zelfs zeer goed
verzorgd.\par
E\'en keer per dag werd de koe over de verdedigingswerken van
Hasselt geleid en de bewoners zorgden ervoor dat de
belegeraars de koe goed konden zien. Zo leek het dat er
genoeg voedsel in de stad was en dat de belegering nog lang
kon duren. De belegeraars werden hierdoor zo ontmoedigd dat
ze het beleg opbraken.\par
In de Hoogstraat in Hasselt staat een huis met een gevelsteen
waarop een koe is afgebeeld. Deze steen herinnert aan de
belegering en de slimheid van de Hasselternaren.
\stopbuffer

\typebuffer

Deze tekst kan ook zonder \type{\par}s worden ingevoerd als
er met lege regels worden gewerkt.

\startbuffer
Tijdens een van de oorlogen die rond Hasselt werden
uitgevochten werd Hasselt belegerd. Na enige tijd ontstond er
een voedselprobleem en brak er een hongersnood uit in de
stad. Alles wat eetbaar was werd opgegeten. Op een koe na.
Deze koe werd in leven gelaten en zelfs zeer goed verzorgd.

Een keer per dag werd de koe over de verdedigingswerken van
Hasselt geleid en de bewoners zorgden ervoor dat de
belegeraars de koe goed konden zien. Zo leek het dat er
genoeg voedsel in de stad was en dat de belegering nog lang
kon duren. De belegeraars werden hierdoor zo ontmoedigd dat
ze de belegering opbraken.

In de Hoogstraat in Hasselt staat een huis met een gevelsteen
waarop een koe is afgebeeld. Deze steen herinnert aan de
belegering en de slimheid van de Hasseltenaren.
\stopbuffer

\typebuffer

\paragraaf{Witruimte tussen paragrafen}

\index{witruimte tussen paragrafen}

\Command{\tex{stelwitruimtein}}
\Command{\tex{geenwitruimte}}
\Command{\tex{witruimte}}
\Command{\tex{startregelcorrectie}}
\Command{\tex{blanko}}
\Command{\tex{stelblankoin}}
\Command{\tex{startopelkaar}}
\Command{\tex{startvanelkaar}}

De verticale witruimte tussen paragrafen
wordt ingesteld met:

\shortsetup{stelwitruimtein}

Dit document wordt geproduceerd met
\type{\stelwitruimtein[middel]}.

Wanneer de witruimte tussen paragrafen is ingesteld, zijn de
volgende commando's beschikbaar, hoewel ze zelden hoeven te
worden gebruikt:

\starttypen
\geenwitruimte
\witruimte
\stoptypen

Indien paragrafen lijnen bevatten dan verdient witruimte
extra aandacht, bijvoorbeeld bij:

\startbuffer
\omlijnd{Ridderstraat 27, 8061GH Hasselt}
\stopbuffer

\haalbuffer

moet een correctie worden uitgevoerd. Deze correctie kan
worden uitgevoerd met:

\shortsetup{startregelcorrectie}

Indien u zou intypen:

\startbuffer
\startregelcorrectie
\omlijnd{Ridderstraat 27, 8061GH Hasselt}
\stopregelcorrectie
\stopbuffer

\typebuffer

dan krijgt u een beter resultaat.

\haalbuffer

Een ander commando dat betrekking heeft op verticale
witruimte is:

\shortsetup{blanko}

Het hakenpaar is optioneel en u kunt tussen haken de
hoeveelheid witruimte opgeven. De mogelijke opties zijn
veelvouden van: \type{klein}, \type{middel} en \type{groot}
en zijn gerelateerd aan de korpsgrootte.

\startbuffer
In offici\"ele aanduidingen gaat de naam Hasselt altijd
vergezeld van de afkorting Ov. Dit is een afkorting van de
provincie Overijssel.
\blanko[2*groot]
Het grappige is dat er in Nederland geen tweede Hasselt is.
De toevoeging is daarom overbodig.
\blanko
De toevoeging is een overblijfsel uit de tijd dat Nederland
en Belgi\"e nog tot hetzelfde koninkrijk behoorden.
\blanko[2*groot]
Hasselt in Belgi\"e ligt in de provincie Limburg. Zouden de
Belgen hun brieven adresseren met Hasselt (Li)?
\stopbuffer

\typebuffer

Het commando \type{\blanko} zonder haken is de standaard
witruimte. Het voorbeeld komt er als volgt uit te zien:

\haalbuffer

De witruimte kan worden ingesteld met:

\shortsetup{stelblankoin}

Verticale witruimte kan worden onderdrukt met het
commando||paar:

\shortsetup{startopelkaar}

In deze handleiding is de witruimte ingesteld op
\type{middel}. In de volgende situatie wordt die instelling
genegeerd en worden de regels op elkaar geplaatst.

\startbuffer
\startopelkaar
Hasselt (Ov) ligt in Overijssel.

Hasselt (Li) ligt in Limburg.

Let wel: we praten over Belgisch Limburg want er is
ook een Nederlands Limburg.
\stopopelkaar
\stopbuffer

\typebuffer

Dit wordt:

\haalbuffer

De tegenhanger hiervan is:

\shortsetup{startvanelkaar}

Een verticale verplaatsing over een bepaalde afstand kan
worden afgedwongen met \type{\omlaag}. De verschuiving
wordt tussen de vierkante haken ingesteld.

\shortsetup{omlaag}

\paragraaf{Inspringen}

\index{inspringen}
\index{paragraaf+inspringen}

\Command{\tex{inspringen}}
\Command{\tex{nietinspringen}}
\Command{\tex{stelinspringenin}}

Een redelijke mate van inspringen wordt bereikt met:

\starttypen
\stelinspringenin[middel]
\stoptypen

Dit levert ingesprongen alineas. Standaard wordt na
witruimte, zoals afgedwongen door \type {\blanko}, niet
ingesprongen.

Tussendoor kunt u het inspringen be\"\i nvloeden met het
commando

\shortsetup{inspringen}

Tussen haken worden de voorkeuren aangegeven. Als u \type
{nooit} meegeeft, dan wordt vanaf dat moment niet meer
ingesprongen, terwijl met \type {geen} alleen de volgende
alinea wordt be\"{\i}nvloed.

Als inspringen aan staat, en u wilt {\em niet} inspringen,
dan kunt u ook het volgende commando gebruiken:

\starttypen
\nietinspringen
\stoptypen

\stoponderdeel
