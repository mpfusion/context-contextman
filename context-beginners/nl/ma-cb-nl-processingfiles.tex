\startonderdeel ma-cb-nl-processingfiles

\produkt ma-cb-nl

\hoofdstuk{Verwerkingsslagen}

\index[texutil]{\TEXUTIL}
\index[tuo]{{\tt tuo}--file}
\index{invoerfile+verwerken}

Tijdens de verwerking schrijft \CONTEXT\ informatie
naar de file \type{myfile.tui}. Deze informatie wordt
gebruikt in een volgende verwerkingsslag. Een deel van die
informatie wordt verwerkt door het programma \TEXUTIL.
Informatie over registers en lijsten worden opgeslagen in de
file \type{myfile.tuo}. Deze informatie wordt (indien nodig)
door \CONTEXT\ gefilterd en gebruikt.

\starttypen
texutil --references filename
\stoptypen

Als \CONTEXT\ een figuur niet kan vinden kan een hulpfile
worden aangemaakt met:

\starttypen
texutil --figures *.*
\stoptypen

Als \EPS\ illustraties moeten worden omgezet naar \PDF\ dan
is er het commando:

\starttypen
texutil --figures --epspage --epspdf
\stoptypen

Om \CONTEXT\ te runnen wordt het programma \TEXEXEC\
gebruikt:

\starttypen
texexec filename
\stoptypen

\TEXEXEC\ verwerkt de file zo vaak als nodig is om
referenties en lijsten correct te kunnen plaatsen. Diverse
instellingen van \CONTEXT\ kunnen op de commandoregel worden
ingevoerd. Een \PDF||file wordt bijvoorbeeld aangemaakt door
in te typen:

\starttypen
texexec --output=pdf filenaam
\stoptypen

Bij twijfel kan altijd \type{--help} worden ingetypt
voor meer informatie over de mogelijke opties. Een
afzonderlijke \TEXEXEC||handleiding is beschikbaar.

\stoponderdeel
