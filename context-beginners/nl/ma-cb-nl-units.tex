\startonderdeel ma-cb-nl-units

\produkt ma-cb-nl

\hoofdstuk[eenheden]{Eenheden}

\index{eenheden}
\index[sieenheden]{\kap{SI}--eenheden}

\Command{\tex{eenheid}}
\Command{\tex{promille}}
\Command{\tex{procent}}

Om consistentie in het gebruik van eenheden en dimensies in
een document af te dwingen kunt u uw eigen eenheden
defini\"eren. Eenheden worden gedefinieerd in het instelgebied
van uw invoerfile.

\CONTEXT\ heeft een eigen module waarin nagenoeg alle
\SI||eenheden zijn opgenomen. Als de module \type{eenheid}
met \type{\gebruikmodule[eenheid]} geladen is, kunnen
eenheden als volgt worden opgeroepen:

\startbuffer
\Meter \Per \Square \Meter
\Kubic \Meter \Per \Sec
\Square \Milli \Meter \Per \Inch
\Centi \Liter \Per \Sec
\Meter \Inverse \Sec
\Newton \Per \Square \Inch
\Newton \Times \Meter \Per \Square \Sec
\stopbuffer

\typebuffer

Dit lijkt veel typewerk maar zorgt wel voor een zeer
consistente weergave in uw document. Het commando eenheid
zorgt er tevens voor dat waarde en eenheid bij een
regelovergang niet van elkaar worden gescheiden. Het
resultaat ziet er als volgt uit:

\startsmaller
\startregels
\haalbuffer
\stopregels
\stopsmaller

U definieert uw eigen eenheden met:

\starttypen
\eenheid[Ounce]{oz}{}
\stoptypen

\eenheid[Ounce]{oz}{}

Indien dit commando in het instelgebied van uw invoerfile is
ingevoerd, kunt u vervolgens in uw document bijvoorbeeld
gebruik maken van het commando \type{\Ounce}. U typt
bijvoorbeeld \type{15.6 \Ounce} en krijgt 15.6 \Ounce.

Het commando \type{\eenheid} is een toepassing van het
commando \type{\synoniem}. Zie \in{hoofdstuk}[synoniemen]
voor meer informatie.

Om de eenheid percentage en promillage consistent te kunnen
gebruiken zijn de volgende commando's beschikbaar:

\type{\procent} \crlf
\type{\promille}

\stoponderdeel
