\startonderdeel ma-cb-nl-color

\produkt ma-cb-nl

\hoofdstuk{Kleur}

\index{kleur}

\Command{\tex{stelkleurenin}}
\Command{\tex{kleur}}
\Command{\tex{definieerkleur}}

Teksten kunnen in kleur worden gezet met:

\shortsetup{kleur}

Het gebruik van kleuren wordt geactiveerd door:

\starttypen
\stelkleurenin[status=start]
\stoptypen

Vanaf dat moment zijn de basiskleuren beschikbaar.
Basiskleuren zijn rood, groen en blauw.

\startbuffer
\startkleur[rood]
Hasselt is een \kleur[groen]{kleurrijke} stad.
\stopkleur
\stopbuffer

\typebuffer

\haalbuffer

Op een zwart||wit printer ziet u alleen maar grijswaarden.
In een elektronisch document verschijnen de kleuren zoals
bedoeld.

Het is ook mogelijk uw eigen kleuren te defini\"eren met:

\shortsetup{definieerkleur}

Bijvoorbeeld:

\startbuffer
\definieerkleur[donkerrood]  [r=.5,g=.0,b=.0]
\definieerkleur[donkergroen] [r=.0,g=.5,b=.0]
\stopbuffer

\typebuffer

Na de definitie zijn de kleuren donkerrood en donkergroen
beschikbaar als de commando's \type{\donkerrood} en
\type{\donkergroen}.

\stoponderdeel
