\startonderdeel ma-cb-nl-registers

\produkt ma-cb-nl

\hoofdstuk{Registers}

\index{register}

\Command{\tex{index}}
\Command{\tex{plaatsindex}}
\Command{\tex{volledigeindex}}
\Command{\tex{definieerregister}}
\Command{\tex{plaatsregister}}
\Command{\tex{volledigeregister}}
\Command{\tex{stelregistersin}}

Het is mogelijk om een of meer registers te genereren.
Standaard is hiervoor het commando \type{\index}
beschikbaar. Voor het toevoegen van een woord aan de lijst
typt u bijvoorbeeld:

\starttypen
\index{stadhuis}
\stoptypen

Het woord {\em stadhuis} zal als een ingang in het register
verschijnen. Een register wordt alfabetisch gesorteerd met
behulp van het programma \TEXUTIL. In sommige gevallen kan
een indexwoord niet worden gesorteerd. Dit is het geval bij
bepaalde symbolen. Dergelijke ingangen worden als volgt
gedefinieerd:

\starttypen
\index[minteken]{$-$}
\stoptypen

De ingang wordt nu gesorteerd op het woord tussen vierkante
haken {\em minteken}.

Soms zijn er sub- en  subsubingangen. Deze worden op een
vergelijkbare wijze gedefinieerd:

\starttypen
\index{stadhuis+locatie}
\index{stadhuis+architectuur}
\stoptypen

Het register zelf wordt (meestal aan het eind van een
document) gegenereerd met:

\starttypen
\plaatsindex
\stoptypen

of:

\starttypen
\volledigeindex
\stoptypen

Het commando \type{\index} is een voorgedefinieerd
\CONTEXT||commando, maar het is ook mogelijk uw eigen
registers te defini\"eren.

\shortsetup{definieerregister}

Een register gebaseerd op de straten van Hasselt kan als
volgt worden gedefinieerd.

\starttypen
\definieerregister[straat][straten]
\stoptypen

Op dit moment is er een registercommando \type{\straat}
beschikbaar. Een ingang wordt dan ingevoerd met
\type{\straat{Ridderstraat}}. Het register zelf wordt
opgeroepen met:

\starttypen
\plaatsstraat
\volledigestraat
\plaatsregister[straat]
\stoptypen

De weergave van de registers wordt ingesteld met:

\shortsetup{stelregisterin}

\stoponderdeel
