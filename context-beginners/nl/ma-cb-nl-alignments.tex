\startonderdeel ma-cb-nl-alignments

\produkt ma-cb-nl

\hoofdstuk{Uitlijnen}

\index{uitlijnen}

\Command{\tex{steluitlijnenin}}
\Command{\tex{steltolerantiein}}
\Command{\tex{regelrechts}}
\Command{\tex{regellinks}}
\Command{\tex{regelmidden}}

Horizontaal en verticaal uitlijnen wordt ingesteld met:

\shortsetup{steluitlijnenin}

Afzonderlijke regels kunnen worden uitgelijnd met:

\starttypen
\regelrechts{}
\regellinks{}
\regelmidden{}
\stoptypen

\startbuffer
\regellinks{Hasselt is gebouwd op een zandheuvel.}
\regelmidden{Hasselt ligt aan een kruising van twee rivieren.}
\regelrechts{Hasselt is vernoemd naar een hazelaar.}
\stopbuffer

\typebuffer

Na het verwerken ziet dit er als volgt uit:

\haalbuffer

Uitlijnen van een alinea wordt gedaan met het
commando||paar:

\shortsetup{startuitlijnen}

Bij het uitlijnen kan een tolerantie en de richting
(verticaal of horizontaal) worden ingesteld.
Normaal is de tolerantie \type{zeerstreng}. In kolommen
kan het uitlijnen soepeler worden ingesteld
\type{zeersoepel}. De uitlijntolerantie in deze handleiding
is als volgt ingesteld:

\starttypen
\steltolerantiein[horizontaal,zeer streng]
\stoptypen

\stoponderdeel
