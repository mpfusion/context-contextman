\startonderdeel ma-cb-nl-fonts

\produkt ma-cb-nl

\hoofdstuk{Fonts en fontovergangen}

\paragraaf{Introductie}

\index{Computer Modern Roman}
\index{Lucida Bright}
\index[ams]{\kap{AMS}}
\index{\kap{PS}--fonts}

De standaard font in \CONTEXT\ is {\em Computer Modern
Roman} (\type{cmr}). Bovendien is Lucida Bright (\type{lbr})
een volwaardig alternatief en zijn symbolen van de {\em
American Society} (\type{ams}) beschikbaar. Verder kunnen
PostScript fonts (\type{pos}) worden gebruikt.

\paragraaf{Fontstijl en grootte}

\index{font+stijl}
\index{font+grootte}

\Command{\tex{stelkorpsin}}
\Command{\tex{switchnaarkorps}}

Voorkeuren voor een fontfamilie, stijl en grootte worden
ingesteld met:

\shortsetup{stelkorpsin}

Wanneer u in het instelgebied typt
\type{\stelkorpsin[schreefloos,9pt]} {\switchnaarkorps[ss,9pt]
komt de tekst in uw document er ongeveer zo uit te zien.}

Veranderingen in de font op een willekeurige plaats in het
document kunnen worden gedaan met:

\shortsetup{switchnaarkorps}

\startbuffer
Op 10 november, een dag voor Sint Maarten, trekt de jeugd
van Hasselt van deur tot deur om een speciaal liedje te
zingen en zichzelf te begeleiden op de {\em foekepot}. Ze
gaan niet weg voordat ze wat geld of wat snoepgoed hebben
gekregen. Het liedje gaat als volgt:

\startsmaller
\switchnaarkorps[klein]
\startregels
Foekepotterij, foekepotterij,
Geef mij een centje dan ga'k voorbij.
Geef mij een alfje dan blijf ik staan,
'k Zal nog liever naar m'n arrenmoeder gaan.
Hier woont zo'n rieke man, die zo vulle g�ven kan.
G�f wat, old wat, g�f die arme stumpers wat,
'k Eb zo lange met de foekepot elopen.
'k Eb gien geld om brood te kopen.
Foekepotterij, foekepotterij,
Geef mij een centje dan ga'k voorbij.
\stopregels
\stopsmaller
\stopbuffer

\typebuffer

Hierbij wordt opgemerkt dat \type{\startsmaller} $\cdots$
\type{\stopsmaller} ook het begin en het einde aangeven van
de fontovergang. De functie van \type{\startregels} en
\type{\stopregels} in dit voorbeeld spreekt voor zich.

\start
\haalbuffer
\stop

Indien u een overzicht wilt van de fontfamilie kunt u het
volgende commando invoeren:

\startbuffer
\toonkorps[cmr]
\stopbuffer

\typebuffer

\haalbuffer

\paragraaf{Fontstijl- en grootte||overgang in commando's}

In enkele commando's kan men de \type{letter} instellen.
Bijvoorbeeld:

\startbuffer
\stelkopin[hoofdstuk][letter=\tfd]
\stopbuffer

\typebuffer

In dit geval wordt de fontgrootte voor het zetten van de
hoofdstukken aangegeven met het commando \type{\tfd}. In
plaats van een dergelijk commando kunnen ook de volgende
opties van het actuele font worden ingegeven:

\startbuffer
normaal  vet  schuin  vetschuin  type  mediaeval
klein  kleinvet  kleinschuin  kleinvetschuin  kleintype
kapitaal kap
\stopbuffer

\typebuffer

\paragraaf{Locale fontstijl- en fontgrootte||overgang}

\Command{\tex{rm}}
\Command{\tex{ss}}
\Command{\tex{tt}}
\Command{\tex{sl}}
\Command{\tex{bf}}
\Command{\tex{tfa}}
\Command{\tex{tfb}}
\Command{\tex{tfc}}
\Command{\tex{tfd}}

In de tekst kunt u de stijl veranderen in roman, sans serif
en teletype met \type{\rm}, \type{\ss} en \type{\tt}. De
lettertypen italic en boldface worden veranderd met
\type{\sl} en \type{\bf}. De grootte kan vari\"eren van 4pt
tot 12pt en wordt veranderd met \type{\switchnaarkorps}.

Het actuele font wordt steeds aangeduid met \type{\tf}.
Indien u naar een grotere letter wilt overgaan, kunt u
\type{\tfa}, \type{\tfb}, \type{\tfc} en \type{\tfd} typen.
In aanvulling op \type{a}, \type{b}, \type{c} en \type{d}
mag u ook \type{\sl}, \type{\it} en \type{\bf} gebruiken.

\startbuffer
{\tfc Muntslag}

In de periode van {\tt 1404} tot {\tt 1585} had Hasselt een
eigen muntatelier en mocht het zelf munten slaan. Dit recht
werd door andere steden aangevochten, maar de
{\switchnaarkorps[7pt] bisschop van Utrecht} ging niet in op
deze {\slb protesten}.
\stopbuffer

\typebuffer

de accolades geven het begin en eind van de fontovergangen
aan.

\haalbuffer

\paragraaf{Herdefini\"eren fontgrootte}

\Command{\tex{definieerkorps}}

Voor speciale toepassingen kunt u de fontgrootte
herdefini\"eren.

\shortsetup{definieerkorps}

Een definitie kan er als volgt uitzien:

\startbuffer
\definieerkorps[10pt][rm][tfe=Regular at 36pt]

{\tfe Hasselt!}
\stopbuffer

\typebuffer

Vervolgens produceert \type{\tfe} de 36pt grote letters:
\hbox{\haalbuffer}

\paragraaf{Klein kapitaal}

\index{klein kapitaal}

\Command{\tex{kap}}

Afkortingen als \PDF\ (\voluit{PDF}) worden gezet in pseudo
klein kapitaal. Een klein kapitaal is iets kleiner dan
kapitaal van het actuele font. Pseudo klein kapitaal wordt
gemaakt met:

\shortsetup{kap}

Als u \type{PDF}, \type{\kap{PDF}} en \type{\sc pdf} vergelijkt:

\regelmidden{PDF, \kap{PDF} en {\sc pdf}}

dan ziet u de verschillen. Het commando \type{\sc} toont een
'echte' klein kapitaal. De reden voor het gebruik van pseudo
klein kapitaal heeft te maken met persoonlijke voorkeuren.

\paragraaf{Benadrukken}

\index{benadrukken}

\Command{\tex{em}}

Om consistent tekstfragmenten te kunnen benadrukken bestaat
het commando:

\starttypen
\em
\stoptypen

Benadrukte woorden worden schuin gezet.

\startbuffer
Als u door Hasselt loopt, moet u uitkijken voor {\em
Amsterdammers}. Een {\em Amsterdammer} is {\bf \em geen}
inwoner van Amsterdam, maar een kleine stenen pilaar die
wordt gebruikt om trottoir en straat te scheiden. Wandelaars
zouden door die {\em Amsterdammers} beschermd moeten worden
tegen auto's e.d., maar heel vaak verwonden zij zich omdat ze
over de paaltjes struikelen.
\stopbuffer

\typebuffer

Dit wordt:

\haalbuffer

{\em Een benadrukt woord binnen een benadrukte zin wordt weer
{\em normaal} gedrukt en vet benadrukken zou er {\bf als
volgt uit moeten {\em zien}}}.

\paragraaf{Typeletters / verbatim}

\index{typeletters}
\index{verbatim}

\Command{\tex{starttypen}}
\Command{\tex{type}}
\Command{\tex{steltypenin}}
\Command{\tex{steltypein}}

Indien tekst in een typeletter moet worden weergegeven,
gebruikt u:

\shortsetup{starttypen}

In een tekst typt u:

\shortsetup{type}

De accolades omsluiten de tekst die in een typeletter moet
worden weergegeven. Een waarschuwing is op zijn plaats. Bij
het werken met \type{\type} moeten regelovergangen extra
worden gecontroleerd, omdat het afbreekmechanisme niet
werkt.

U kunt met betrekking tot typen het een en ander instellen
met:

\shortsetup{steltypenin}
\shortsetup{steltypein}

\pagina

\paragraaf{Encodings}

Deze paragraaf moet nog gevuld worden.

\stoponderdeel
