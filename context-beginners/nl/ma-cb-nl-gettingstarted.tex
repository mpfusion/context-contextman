\startonderdeel ma-cb-nl-gettingstarted

\produkt ma-cb-nl

\hoofdstuk{Het verwerken van een file}

\index{invoerfile+verwerken}
\index[dvifile]{\type{dvi}--file}

Als u een invoerfile wilt verwerken met \CONTEXT\ typt u na
de prompt:

\starttypen
context filenaam
\stoptypen

Het commando \type{context} kan per systeem verschillen. Vraag
zonodig uw systeembeheerder met welk commando u \CONTEXT\
opstart. Als uw filenaam \type{mijnfile.tex} is, dan kan dat
zijn:

\starttypen
context mijnfile
\stoptypen

of als \TEXEXEC\ goed is ge\"{\i}nstalleerd:

\starttypen
texexec --pdf mijnfile
\stoptypen


de extensie \type{.tex} hoeft niet te worden ingetypt.

Nadat \Enter\ is ingevoerd, wordt de verwerking gestart.
\CONTEXT\ geeft informatie over de verwerkingsstappen op het
beeldscherm. Als de verwerking succesvol is verlopen,
verschijnt de prompt en heeft \CONTEXT\ een
\type{dvi}||file of een \PDF||file aangemaakt.

Indien de verwerking niet goed verloopt, bijvoorbeeld omdat
u \type{\stptekst} in plaats van \type{\stoptekst} heeft
ingetypt, geeft \CONTEXT\ een \type{ ? }  op het beeldscherm
en geeft aan dat er een fout commando is verwerkt. Er wordt
een indicatie gegeven van het type fout en bovendien wordt
het regelnummer aangegeven.

Achter het door \CONTEXT\ geproduceerde vraagteken kunt u de
volgende invoer geven:

\starttabulatie[|||]
\NC \type{H} \NC voor helpinformatie over de fout \NC\NR
\NC \type{I} \NC voor het invoeren van het correcte \CONTEXT\ commando \NC\NR
\NC \type{Q} \NC voor het overgaan op batchverwerking \NC\NR
\NC \type{X} \NC voor het stoppen van de verwerking \NC\NR
\NC \Enter\  \NC voor het negeren van de fout \NC\NR
\stoptabulatie

Meestal is \Enter\ de beste optie en de verwerking zal gewoon
doorgaan. Op het moment dat de verwerking is afgelopen, kunt u
de fout herstellen met behulp van uw tekstverwerker.

Sommige fouten zijn aanleiding voor \CONTEXT\ om een
\type{ * } op het beeldscherm te genereren en de verwerking te
stoppen. Deze fout wordt veroorzaakt door een {\em fatal
error} in de invoerfile. Deze fout kan niet genegeerd worden
en de enige optie die u heeft is het typen van \type{\stop}
(in ernstige gevallen kunt u ook \type{Ctrl-Z} typen). Het
programma wordt gestopt en u kunt de fout herstellen.

Tijdens de verwerking informeert \CONTEXT\ de gebruiker over
de acties die op het document (invoerfile) worden
uitgevoerd. \CONTEXT\ toont bijvoorbeeld de paginanummers en
hoofdstuk- en paragraaftitels op het scherm. Bovendien
worden waarschuwingen gegeven. Waarschuwingen hebben meestal
een typografisch karakter en geven bijvoorbeeld aan dat het
afbreken van bepaalde woorden niet of niet goed verloopt.
Alle informatie over de verwerking wordt opgeslagen in een
\type{log}||file waarin de fouten en waarschuwingen en de
bijbehorende regelnummers nog eens kunnen worden
geraadpleegd.

Indien de verwerking succesvol is verlopen heeft \CONTEXT\
een nieuwe file aangemaakt \type{mijnfile.dvi} of
\type{mijnfile.pdf}. De afkorting \type{dvi} staat voor {\em
device indepent}. Dit betekent dat de file een min of meer
onafhankelijk formaat heeft en met behulp van geschikte
PostScript (\PS) printerdrivers kan worden omgezet naar een
\PS||file ten behoeve van afdrukken of bekijken. \PDF\ staat
voor Portable Document Format en is een platform
onafhankelijk formaat voor het afdrukken en bekijken van
de \PDF||files.

\stoponderdeel
