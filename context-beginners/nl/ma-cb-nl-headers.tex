\startonderdeel ma-cb-nl-headers

\produkt ma-cb-nl

\hoofdstuk{Hoofd- en voetteksten}

\index{hoofdteksten}
\index{voetteksten}

\Command{\tex{stelvoettekstenin}}
\Command{\tex{stelhoofdtekstenin}}
\Command{\tex{stelhoofdin}}
\Command{\tex{stelvoetin}}
\Command{\tex{geenhoofdenvoetteksten}}

Documenten hebben soms hoofd- en voetteksten die voor
allerlei doeleinden worden gebruikt. De commando's om hoofd-
en voetteksten te plaatsen zijn:

\shortsetup{stelvoettekstenin}
\shortsetup{stelhoofdtekstenin}

Het eerste paar haken is optioneel en bedoeld voor de
locatie van de voet- of hoofdtekst (\type{tekst},
\type{rand} enz.). De voet- en hoofdtekst zelf worden tussen
de overige vierkante haken geplaatst. In een enkelzijdig
document zijn alleen het tweede en derde paar vierkante
haken nodig. Het tweede paar van bijvoorbeeld
\type{\stelvoettekstenin} bevat de tekst linksonder en het
tweede paar de tekst die rechtsonder moet komen te staan. In
een dubbelzijdig document zijn nog twee paren beschikbaar
voor teksten in de voet van de linker pagina.

\startbuffer
\stelvoettekstenin[Handleiding][paragraaf]
\stopbuffer

\typebuffer

In dit voorbeeld verschijnt de tekst {\em Handleiding} in de
linker onderhoek van de pagina en titel van de actuele
paragraaf in de rechteronderhoek. Deze voettekst verandert
automatisch bij overgang naar een volgende paragraaf.

De hoofd- en voetteksten kunnen worden ingesteld met de
commando's:

\shortsetup{stelhoofdin}
\shortsetup{stelvoetin}

Als de hoofd- en/of voetteksten op een bepaalde pagina niet
nodig zijn, typt u:

\starttypen
\geenhoofdenvoetregels
\stoptypen

\stoponderdeel
