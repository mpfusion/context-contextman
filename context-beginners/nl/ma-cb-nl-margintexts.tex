\startonderdeel ma-cb-nl-margintexts

\produkt ma-cb-nl

\hoofdstuk{Margeteksten}

\index{margetekst}

\Command{\tex{inmarge}}
\Command{\tex{inlinker}}
\Command{\tex{inrechter}}
\Command{\tex{margetitel}}

Het is eenvoudig om teksten in de marge te plaatsen. Het
commando daarvoor is \type{\inmarge}.

\shortsetup{inmarge}

In een van de eerdere voorbeelden is al met \type{\inmarge}
gewerkt.

\typebuffer[margefiguur]

Dit resulteert in een figuur in de \paginareferentie
[margefiguur]\haalbuffer[margefiguur]marge. Een
figuur in de marge is natuurlijk erg smal en te klein om
goed te kunnen weergeven.

Hieronder worden enkele voorbeelden van margeteksten
gegeven.

\startbuffer
De Ridderstraat \inmarge{Ridderstraat} heeft een logische
naam. In de 14e en 15e eeuw woonde de adel en prominente
mensen in deze straat. Enkele van hun grote huizen staan er
nog en zijn later in gebruik genomen als armenhuis
\inrechter{armen-\\huis} en bejaardenhuis.

Tot aan \inlinker[laag]{\tfc 1940}1940 was er een synagoge
in de Ridderstraat. Ongeveer 40 joodse inwoners van Hasselt
vierden daar hun Sabbat. Tijdens de oorlog werden deze joden
naar Westerbork gestuurd en vandaar uit verder
getransporteerd naar de vernietigingskampen in Duitsland en
Polen. Geen van de joodse families keerde terug. De
synagoge werd in 1958 afgebroken.
\stopbuffer

\typebuffer

De commando's \type{\inmarge}, \type{\inlinker} en
\type{\inrechter} hebben allen een vergelijkbare functie. In
een tweezijdig document plaatst \type{\inmarge} de
margeteksten automatisch in de correcte marge. De \type{\\}
zijn om regelafbreking af te dwingen. Het voorbeeld komt er
als volgt uit te zien:

\haalbuffer

Margeteksten worden ingesteld met:

\starttypen
\stelinmargesin
\stoptypen

\stoponderdeel
