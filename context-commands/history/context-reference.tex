\environment e-context-reference
\starttext
\startstandardmakeup
  \setupalign[middle] \color[darkgreen] \bfd \setupinterlinespace \vfil
  \getvariable{pm}{title} \vfil
  \bfb \getvariable{pm}{author} \vfil
  \tfx \date[m=9, d=14, y=2010] \vfil\vfil\vfil
\stopstandardmakeup

\Topics

\chapter{Introduction}
The following slides present a possibility, to group all details about a
\ConTeXt\ command in a well structured and readable “command-description-file”.

The language of this file is Lua, because
\startitemize
\item it's a native language of \LuaTeX
\item and it's convenient to read and to edit by human beings (better than
  XML)
\stopitemize

\chapter{Motivation}
Main goal: getting one day the {\em Complete \ConTeXt\ Command Reference}.
With:
\startitemize
\item detailed descriptions of the macros and their arguments
\item classification in categories (chapters and sections)
\item index
\item cross-references
\item perhaps various output formats (PDF, HTML, various styles, ...)
\stopitemize

There are several hundreds of commands to describe, so in order to get there
in the most efficient way, we must be able to concentrate on the content:
\startitemize
\item ergonomic editing environment (using your favorite editor)
\item not too verbose structuring syntax (Lua is nice!)
\item avoidance of redundancy (description of inheritance)
\stopitemize

Furthermore, the command descriptions can be used for a future \ConTeXt\
syntax checker!

\chapter{Examples}
In the following examples, you will see, that there is no concept for optional
arguments.

This is so, because in \ConTeXt
\startitemize
\item sometimes you need to add argument 1 to make use of argument 2 (e.g.
  \tex{placefigure})
\item sometimes the meaning of arguments depends on the number of supplied
  arguments (e.g. \tex{setupheadertexts})
\item sometimes the position of an argument varies (e.g.
  \tex{externalfigure[ref][file]} and \tex{externalfigure[file][parent]})
\stopitemize

Instead, we use {\em variants} to describe the different possibilities to call
a macro.

\PrintExamples

\chapter{Perspectives for the future}
When this syntax is generally accepted, we can:
\startitemize
\item put these files on a public svn-server
\item integrate the content of the \type{cont-en.xml} file and the wiki
\item add a \LuaTeX\ script that generates a nice looking PDF (once per night
  for example)
\item and then add more and more content
\stopitemize

We hope, that this command description syntax will make it a pleasure to add
explanations of macros and their arguments.

\startalignment[flushright]
  \em Thanks for your attention!\crlf Taco and Peter
\stopalignment

\stoptext

% LocalWords:  darkgreen Lua PDF placefigure setupheadertexts externalfigure
% LocalWords:  svn xml wiki flushright
